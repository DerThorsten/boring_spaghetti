\documentclass[10pt,twocolumn,letterpaper]{article}

\usepackage{eso-pic}
\usepackage{cvpr}
\usepackage{times}
\usepackage{epsfig}
\usepackage{graphicx}
\usepackage{amsmath}
\usepackage{amssymb}




% Include other packages here, before hyperref.
\usepackage{framed}
\usepackage{caption}
\usepackage{subcaption}
%for tikz
\usepackage{tikz}
\usetikzlibrary{arrows,positioning,automata,shadows,fit,shapes}
\usetikzlibrary{arrows,petri,topaths}
\usepackage{tkz-berge}

\definecolor{shadecolor}{rgb}{0.01,0.199,0.1}

% If you comment hyperref and then uncomment it, you should delete
% egpaper.aux before re-running latex.  (Or just hit 'q' on the first latex
% run, let it finish, and you should be clear).
\usepackage[pagebackref=true,breaklinks=true,letterpaper=true,colorlinks,bookmarks=false]{hyperref}

% \cvprfinalcopy % *** Uncomment this line for the final submission

\def\cvprPaperID{****} % *** Enter the CVPR Paper ID here
\def\httilde{\mbox{\tt\raisebox{-.5ex}{\symbol{126}}}}

% Pages are numbered in submission mode, and unnumbered in camera-ready
\ifcvprfinal\pagestyle{empty}\fi
\begin{document}
%\GraphInit[vstyle=Normal] 
%%%%%%%%% TITLE
\title{FancyMc Moves \\ Fusion Moves for Multicut Objectives}

\author{First Author\\
Institution1\\
Institution1 address\\
{\tt\small firstauthor@i1.org}
% For a paper whose authors are all at the same institution,
% omit the following lines up until the closing ``}''.
% Additional authors and addresses can be added with ``\and'',
% just like the second author.
% To save space, use either the email address or home page, not both
\and
Second Author\\
Institution2\\
First line of institution2 address\\
{\tt\small secondauthor@i2.org}
}

\maketitle
%\thispagestyle{empty}

%%%%%%%%% ABSTRACT
\begin{abstract}
   Multicuts rule.
\end{abstract}
\section{Introduction}

The tale of the multicut

%-------------------------------------------------------------------------

\subsection{Related Work}

\subsubsection{Multicut}
   \begin{itemize}
   \item Andres \etal~\cite{andres_2011_iccv}
   \item Kappes \etal~\cite{kappes_2011_emmcvpr}
   \item Bagon and Galun~\cite{bagon_2011_arxiv}
   \item Yarkony \etal~\cite{yarkony_2012_eccv}
   \item Beier \etal~\cite{beier_2014_cvpr}
   \end{itemize}

\subsubsection{Fusion Moves}
Move making algorithms, in particular fusion moves, 
have become increasingly popular for energy minimization~\cite{???,kappes_2014_ws}.
For many large scale computer vision applications fusion moves lead to good approximations
with state of the art any time performance~\cite{kappes_2014_ws}.






\begin{figure}[t]
\begin{center}
\fbox{\rule{0pt}{2in} \rule{0.9\linewidth}{0pt}}
   %\includegraphics[width=0.8\linewidth]{egfigure.eps}
\end{center}
   \caption{Example of caption.  It is set in Roman so that mathematics
   (always set in Roman: $B \sin A = A \sin B$) may be included without an
   ugly clash.}
\label{fig:long}
\label{fig:onecol}
\end{figure}


.

%------------------------------------------------------------------------
\section{Name of My Method (Union Fusion Cut)}

Global optimal solvers for multicut do not scale beyond ??? \cite{???}.
Good approximate solvers for planar graphs exist \cite{beier_2014_cvpr,yarkony_2012_eccv} 
but have difficulties to find good solutions for non planar graphs \cite{beier_2014_cvpr}.



\input{move_graph.tex}



%-------------------------------------------------------------------------
\subsection{Proposal Generators}



%-------------------------------------------------------------------------
\subsection{Fusion Move Solver}


\section{Experiments}

\section{Conclusion}

    



{\small
\bibliographystyle{ieee}
\bibliography{egbib}
}

\end{document}
