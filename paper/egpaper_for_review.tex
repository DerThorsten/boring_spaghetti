\documentclass[10pt,twocolumn,letterpaper]{article}

\usepackage{eso-pic}
\usepackage{cvpr}
\usepackage{times}
\usepackage{epsfig}
\usepackage{graphicx}
\usepackage{amsmath}
\usepackage{amssymb}


% Include other packages here, before hyperref.
\usepackage{my_macros}
\usepackage{paralist}
%\usepackage{framed}
\usepackage{caption}
\usepackage{subcaption}
%for tikz
\usepackage{tikz,pgfplots}
\usetikzlibrary{arrows,positioning,automata,shadows,fit,shapes}
\usetikzlibrary{arrows,petri,topaths}
\usetikzlibrary{positioning,fit,calc}
\usetikzlibrary{shapes.arrows,chains,decorations.pathreplacing,fadings}
\usepackage{tkz-berge}
\DeclareMathOperator*{\argmin}{arg\,min}
\definecolor{shadecolor}{rgb}{0.01,0.199,0.1}

% If you comment hyperref and then uncomment it, you should delete
% egpaper.aux before re-running latex.  (Or just hit 'q' on the first latex
% run, let it finish, and you should be clear).
\usepackage[pagebackref=true,breaklinks=true,letterpaper=true,colorlinks,bookmarks=false]{hyperref}

% \cvprfinalcopy % *** Uncomment this line for the final submission

\def\cvprPaperID{****} % *** Enter the CVPR Paper ID here
\def\httilde{\mbox{\tt\raisebox{-.5ex}{\symbol{126}}}}

% Pages are numbered in submission mode, and unnumbered in camera-ready
\ifcvprfinal\pagestyle{empty}\fi
\begin{document}
%%%%%%%%% TITLE
%!TEX root = egpaper_for_review.tex
\newcommand{\Cut }{\mathcal{C}}
\title{FancyMc Moves \\ Fusion Moves for Multicut Objectives}

\author{First Author\\
Institution1\\
Institution1 address\\
{\tt\small firstauthor@i1.org}
% For a paper whose authors are all at the same institution,
% omit the following lines up until the closing ``}''.
% Additional authors and addresses can be added with ``\and'',
% just like the second author.
% To save space, use either the email address or home page, not both
\and
Second Author\\
Institution2\\
First line of institution2 address\\
{\tt\small secondauthor@i2.org}
}

\maketitle
%\thispagestyle{empty}

%%%%%%%%% ABSTRACT
\begin{abstract}
   Multicuts rule.
\end{abstract}
\section{Introduction}

The tale of the multicut

%-------------------------------------------------------------------------

\subsection{Related Work}


\subsubsection{Multicut Objective}

The multicut / correlation clustering objective 
can be formulated in different ways.



\paragraph{Edge Indicator Variables:}
\begin{center}
    \begin{eqnarray}
        y^* &=& \argmin_{y} \sum_{ e_{ij} \in E } w_{ij} \cdot y_{ij} \\
        s.t.:& & y \in \textit{Multicut Polytope} \nonumber
    \end{eqnarray}
\end{center}

\paragraph{Fully Connected Graph:}
\begin{center}
    \begin{eqnarray}
        y^*   & = & \argmin_{y} \sum_{ i<j \in V } w_{ij} \cdot y_{ij} \\
        s.t.: &  & x_{ij} + x_{jk} < x_{i,k} \quad \forall i, j, k   \nonumber
    \end{eqnarray}
\end{center}

\paragraph{Node Coloring:}
\begin{center}
    \begin{eqnarray}
        l^* &=& \argmin_{L} \sum_{ e_{ij} \in E } w_{ij} \cdot [l_{u} \neq l_{v}] \\
        y_{ij}^* &=& [l_{u} \neq l_{v}]  
    \end{eqnarray}
\end{center}

The multicut objective has been used for
\begin{inparaenum}[(i)]
    \item partitioning a superpixel region adjacency graph~\cite{andres_2011_iccv,kroeger_2012_eccv}
    \item with optional long range repulsive edges~\cite{yarknoy???}.
    \item Alush and Globerger showed how to average multiple segmentations with the multicut objective~\cite{alush_2012_pami}.
    \item Multicuts can also be used for interactive segmentation~\cite{bagon_2011_arxiv}
    \item and to cluster fully connected graphs~\cite{???}.
\end{inparaenum}


\subsubsection{Solver for the Multicut Objective}
   \begin{itemize}
   \item Multicut~\cite{kappes_2011_emmcvpr}
   \item Expand and Explorer~\cite{bagon_2011_arxiv}
   \item Fast Planar CC~\cite{yarkony_2012_eccv}
   \item Break and Conquer \cite{alush_2013_simbad}.
   \item Cut Glue And Cut~\cite{beier_2014_cvpr}
   \end{itemize}

\subsubsection{Fusion Moves}
Move making algorithms, in particular fusion moves, 
have become increasingly popular for energy minimization~\cite{???,kappes_2014_ws}.
For many large scale computer vision applications fusion moves lead to good approximations
with state of the art any time performance~\cite{kappes_2014_ws}.








%------------------------------------------------------------------------
\section{Name of My Method (Union Fusion Cut)}

Global optimal solvers for multicut do not scale beyond ??? \cite{???}.
Good approximate solvers for planar graphs exist \cite{beifiger_2014_cvpr,yarkony_2012_eccv} 
but have difficulties to find good solutions for non planar graphs \cite{beier_2014_cvpr}.



\input{inputs/fig_alg_graph.tex}

%!TEX root = ../egpaper_for_review.tex
\begin{figure*}
\tikzstyle{cedge}=[fill=white,dotted,font=\sffamily\tiny, opacity=0.5, text=gray]
\tikzstyle{aedge}=[fill=white,solid,font=\sffamily\tiny, text=black!70          ]
\tikzstyle{vert}=[circle,minimum size = 0.5cm,inner sep = 1pt,draw, font=\small,align=left]
\centering
%\begin{subfigure}[t]{1\linewidth}
   \resizebox{1.0\linewidth}{!}{
      \begin{tikzpicture}[scale=1.0,transform shape]
        \tikzstyle{vert}=[circle,minimum size = 0.5cm,inner sep = 0pt,draw, font=\small, node distance = 5cm]
        \draw (0*2,3*2) node[vert](1){$1$};
        \draw (1*2,3*2) node[vert](2){$2$};
        \draw (2*2,3*2) node[vert](3){$3$};
        \draw (3*2,3*2) node[vert](4){$4$};
        \draw (0*2,2*2) node[vert](5){$5$};
        \draw (1*2,2*2) node[vert](6){$6$};
        \draw (2*2,2*2) node[vert](7){$7$};
        \draw (3*2,2*2) node[vert](8){$8$};
        \draw (0*2,1*2) node[vert](9){$9$};
        \draw (1*2,1*2) node[vert](10){$10$};
        \draw (2*2,1*2) node[vert](11){$11$};
        \draw (3*2,1*2) node[vert](12){$12$};
        \draw (0*2,0*2) node[vert](13){$13$};
        \draw (1*2,0*2) node[vert](14){$14$};
        \draw (2*2,0*2) node[vert](15){$15$};
        \draw (3*2,0*2) node[vert](16){$16$};
        %
        \path[every node/.style={font=\sffamily\tiny, fill=white}]
            (1)     edge[aedge]     node{$w_{(1, 2)}$   }     (2)
            (2)     edge[aedge]     node{$w_{(2, 3)}$   }     (3)
            (3)     edge[aedge]     node{$w_{(3, 4)}$   }     (4)
            (5)     edge[aedge]     node{$w_{(5, 6)}$   }     (6)
            (6)     edge[aedge]     node{$w_{(6, 7)}$   }     (7)
            (7)     edge[aedge]     node{$w_{(7, 8)}$   }     (8)
            (9)     edge[aedge]     node{$w_{(9, 10)}$  }     (10)
            (10)    edge[aedge]     node{$w_{(10, 11)}$ }     (11)
            (11)    edge[aedge]     node{$w_{(11, 12)}$ }     (12)
            (13)    edge[aedge]     node{$w_{(13, 14)}$ }     (14)
            (14)    edge[aedge]     node{$w_{(14, 15)}$ }     (15)
            (15)    edge[aedge]     node{$w_{(15, 16)}$ }     (16)
            (1)     edge[aedge]     node{$w_{(1, 5)}$   }     (5)
            (5)     edge[aedge]     node{$w_{(5, 9)}$   }     (9)
            (9)     edge[aedge]     node{$w_{(9, 13)}$  }     (13)
            (2)     edge[aedge]     node{$w_{(2, 6)}$   }     (6)
            (6)     edge[aedge]     node{$w_{(6, 10)}$  }     (10)
            (10)    edge[aedge]     node{$w_{(10, 14)}$ }     (14)
            (3)     edge[aedge]     node{$w_{(3, 7)}$   }     (7)
            (7)     edge[aedge]     node{$w_{(7, 11)}$  }     (11)
            (11)    edge[aedge]     node{$w_{(11, 15)}$ }     (15)
            (4)     edge[aedge]     node{$w_{(4, 8)}$   }     (8)
            (8)     edge[aedge]     node{$w_{(8, 12)}$  }     (12)
            (12)    edge[aedge]     node{$w_{(12, 16)}$ }     (16)
        ;

        \draw (8+0*2,3*2) node[vert,fill=red!50](1){$1$};
        \draw (8+1*2,3*2) node[vert,fill=red!50](2){$2$};
        \draw (8+2*2,3*2) node[vert,fill=red!50](3){$3$};
        \draw (8+3*2,3*2) node[vert,fill=red!50](4){$4$};
        \draw (8+0*2,2*2) node[vert,fill=red!50](5){$5$};
        \draw (8+1*2,2*2) node[vert,fill=red!50](6){$6$};
        \draw (8+2*2,2*2) node[vert,fill=red!50](7){$7$};
        \draw (8+3*2,2*2) node[vert,fill=red!50](8){$8$};
        \draw (8+0*2,1*2) node[vert,fill=blue!50](9){$9$};
        \draw (8+1*2,1*2) node[vert,fill=blue!50](10){$10$};
        \draw (8+2*2,1*2) node[vert,fill=blue!50](11){$11$};
        \draw (8+3*2,1*2) node[vert,fill=blue!50](12){$12$};
        \draw (8+0*2,0*2) node[vert,fill=blue!50](13){$13$};
        \draw (8+1*2,0*2) node[vert,fill=blue!50](14){$14$};
        \draw (8+2*2,0*2) node[vert,fill=blue!50](15){$15$};
        \draw (8+3*2,0*2) node[vert,fill=blue!50](16){$16$};
        %
        \path[every node/.style={font=\sffamily\tiny, fill=white}]
            (1)     edge[aedge]     node{$w_{(1, 2)}$   }     (2)
            (2)     edge[aedge]     node{$w_{(2, 3)}$   }     (3)
            (3)     edge[aedge]     node{$w_{(3, 4)}$   }     (4)
            (5)     edge[aedge]     node{$w_{(5, 6)}$   }     (6)
            (6)     edge[aedge]     node{$w_{(6, 7)}$   }     (7)
            (7)     edge[aedge]     node{$w_{(7, 8)}$   }     (8)
            (9)     edge[aedge]     node{$w_{(9, 10)}$  }     (10)
            (10)    edge[aedge]     node{$w_{(10, 11)}$ }     (11)
            (11)    edge[aedge]     node{$w_{(11, 12)}$ }     (12)
            (13)    edge[aedge]     node{$w_{(13, 14)}$ }     (14)
            (14)    edge[aedge]     node{$w_{(14, 15)}$ }     (15)
            (15)    edge[aedge]     node{$w_{(15, 16)}$ }     (16)
            (1)     edge[aedge]     node{$w_{(1, 5)}$   }     (5)
            (5)     edge[cedge]     node{$w_{(5, 9)}$   }     (9)
            (9)     edge[aedge]     node{$w_{(9, 13)}$  }     (13)
            (2)     edge[aedge]     node{$w_{(2, 6)}$   }     (6)
            (6)     edge[cedge]     node{$w_{(6, 10)}$  }     (10)
            (10)    edge[aedge]     node{$w_{(10, 14)}$ }     (14)
            (3)     edge[aedge]     node{$w_{(3, 7)}$   }     (7)
            (7)     edge[cedge]     node{$w_{(7, 11)}$  }     (11)
            (11)    edge[aedge]     node{$w_{(11, 15)}$ }     (15)
            (4)     edge[aedge]     node{$w_{(4, 8)}$   }     (8)
            (8)     edge[cedge]     node{$w_{(8, 12)}$  }     (12)
            (12)    edge[aedge]     node{$w_{(12, 16)}$ }     (16)
        ;
        \draw (15+0*2,3*2-3) node[minimum size=2cm,font=\sffamily\Huge] (dummy){$\textbf{+}$};
        %\draw (15+0*2,3*2-4) node[minimum size=2cm,font=\sffamily\Huge] (dummy){$\rightarrow$};

        \draw (16+0*2,3*2) node[vert,fill=green!50](1){$1$};
        \draw (16+1*2,3*2) node[vert,fill=orange!50](2){$2$};
        \draw (16+2*2,3*2) node[vert,fill=orange!50](3){$3$};
        \draw (16+3*2,3*2) node[vert,fill=orange!50](4){$4$};
        \draw (16+0*2,2*2) node[vert,fill=yellow!=30](5){$5$};
        \draw (16+1*2,2*2) node[vert,fill=magenta!40](6){$6$};
        \draw (16+2*2,2*2) node[vert,fill=magenta!40](7){$7$};
        \draw (16+3*2,2*2) node[vert,fill=magenta!40](8){$8$};
        \draw (16+0*2,1*2) node[vert,fill=yellow!=30](9){$9$};
        \draw (16+1*2,1*2) node[vert,fill=magenta!40](10){$10$};
        \draw (16+2*2,1*2) node[vert,fill=magenta!40](11){$11$};
        \draw (16+3*2,1*2) node[vert,fill=magenta!40](12){$12$};
        \draw (16+0*2,0*2) node[vert,fill=yellow!=30](13){$13$};
        \draw (16+1*2,0*2) node[vert,fill=magenta!40](14){$14$};
        \draw (16+2*2,0*2) node[vert,fill=magenta!40](15){$15$};
        \draw (16+3*2,0*2) node[vert,fill=magenta!40](16){$16$};
        %
        \path[every node/.style={font=\sffamily\tiny, fill=white}]
            (1)     edge[cedge]     node{$w_{(1, 2)}$   }     (2)
            (2)     edge[aedge]     node{$w_{(2, 3)}$   }     (3)
            (3)     edge[aedge]     node{$w_{(3, 4)}$   }     (4)
            (5)     edge[cedge]     node{$w_{(5, 6)}$   }     (6)
            (6)     edge[aedge]     node{$w_{(6, 7)}$   }     (7)
            (7)     edge[aedge]     node{$w_{(7, 8)}$   }     (8)
            (9)     edge[cedge]     node{$w_{(9, 10)}$  }     (10)
            (10)    edge[aedge]     node{$w_{(10, 11)}$ }     (11)
            (11)    edge[aedge]     node{$w_{(11, 12)}$ }     (12)
            (13)    edge[cedge]     node{$w_{(13, 14)}$ }     (14)
            (14)    edge[aedge]     node{$w_{(14, 15)}$ }     (15)
            (15)    edge[aedge]     node{$w_{(15, 16)}$ }     (16)
            (1)     edge[cedge]     node{$w_{(1, 5)}$   }     (5)
            (5)     edge[aedge]     node{$w_{(5, 9)}$   }     (9)
            (9)     edge[aedge]     node{$w_{(9, 13)}$  }     (13)
            (2)     edge[cedge]     node{$w_{(2, 6)}$   }     (6)
            (6)     edge[aedge]     node{$w_{(6, 10)}$  }     (10)
            (10)    edge[aedge]     node{$w_{(10, 14)}$ }     (14)
            (3)     edge[cedge]     node{$w_{(3, 7)}$   }     (7)
            (7)     edge[aedge]     node{$w_{(7, 11)}$  }     (11)
            (11)    edge[aedge]     node{$w_{(11, 15)}$ }     (15)
            (4)     edge[cedge]     node{$w_{(4, 8)}$   }     (8)
            (8)     edge[aedge]     node{$w_{(8, 12)}$  }     (12)
            (12)    edge[aedge]     node{$w_{(12, 16)}$ }     (16)
        ;
        
        \draw (23+0*2,3*2-3) node[minimum size=2cm,font=\sffamily\Huge] (dummy){$\textbf{=}$};

        \draw (24+0*2,3*2) node[vert,fill=blue!50!green](1){$1$};
        \draw (24+1*2,3*2) node[vert,fill=brown!50](2){$2$};
        \draw (24+2*2,3*2) node[vert,fill=brown!50](3){$3$};
        \draw (24+3*2,3*2) node[vert,fill=brown!50](4){$4$};
        \draw (24+0*2,2*2) node[vert,fill=cyan!=30](5){$5$};
        \draw (24+1*2,2*2) node[vert,fill=red!50!lime](6){$6$};
        \draw (24+2*2,2*2) node[vert,fill=red!50!lime](7){$7$};
        \draw (24+3*2,2*2) node[vert,fill=red!50!lime](8){$8$};
        \draw (24+0*2,1*2) node[vert,fill=violet!60](9){$9$};
        \draw (24+1*2,1*2) node[vert,fill=lime!40](10){$10$};
        \draw (24+2*2,1*2) node[vert,fill=lime!40](11){$11$};
        \draw (24+3*2,1*2) node[vert,fill=lime!40](12){$12$};
        \draw (24+0*2,0*2) node[vert,fill=violet!60](13){$13$};
        \draw (24+1*2,0*2) node[vert,fill=lime!40](14){$14$};
        \draw (24+2*2,0*2) node[vert,fill=lime!40](15){$15$};
        \draw (24+3*2,0*2) node[vert,fill=lime!40](16){$16$};
        %
        \path[every node/.style={font=\sffamily\tiny, fill=white}]
            (1)     edge[cedge]     node{$w_{(1, 2)}$   }     (2)
            (2)     edge[aedge]     node{$w_{(2, 3)}$   }     (3)
            (3)     edge[aedge]     node{$w_{(3, 4)}$   }     (4)
            (5)     edge[cedge]     node{$w_{(5, 6)}$   }     (6)
            (6)     edge[aedge]     node{$w_{(6, 7)}$   }     (7)
            (7)     edge[aedge]     node{$w_{(7, 8)}$   }     (8)
            (9)     edge[cedge]     node{$w_{(9, 10)}$  }     (10)
            (10)    edge[aedge]     node{$w_{(10, 11)}$ }     (11)
            (11)    edge[aedge]     node{$w_{(11, 12)}$ }     (12)
            (13)    edge[cedge]     node{$w_{(13, 14)}$ }     (14)
            (14)    edge[aedge]     node{$w_{(14, 15)}$ }     (15)
            (15)    edge[aedge]     node{$w_{(15, 16)}$ }     (16)
            (1)     edge[cedge]     node{$w_{(1, 5)}$   }     (5)
            (5)     edge[cedge]     node{$w_{(5, 9)}$   }     (9)
            (9)     edge[aedge]     node{$w_{(9, 13)}$  }     (13)
            (2)     edge[cedge]     node{$w_{(2, 6)}$   }     (6)
            (6)     edge[cedge]     node{$w_{(6, 10)}$  }     (10)
            (10)    edge[aedge]     node{$w_{(10, 14)}$ }     (14)
            (3)     edge[cedge]     node{$w_{(3, 7)}$   }     (7)
            (7)     edge[cedge]     node{$w_{(7, 11)}$  }     (11)
            (11)    edge[aedge]     node{$w_{(11, 15)}$ }     (15)
            (4)     edge[cedge]     node{$w_{(4, 8)}$   }     (8)
            (8)     edge[cedge]     node{$w_{(8, 12)}$  }     (12)
            (12)    edge[aedge]     node{$w_{(12, 16)}$ }     (16)
        ;

        \draw (24+0*2,-1) node[minimum size=2cm,font=\sffamily\Huge] (dummy){$\downarrow$};
        \draw (27+0*2,-1) node[minimum size=2cm,font=\sffamily\huge] (dummy){$\textbf{contract}$};
        \draw (30   +0*2,-1) node[minimum size=2cm,font=\sffamily\Huge] (dummy){$\downarrow$};

        \draw (24+0*2,    -8.5+3*2) node[vert](1){$\{1\}$};
        \draw (24+2*2,    -8.5+3*2) node[vert](2){$\{2, 3, 4\}$};
        \draw (24+0*2,    -8.5+2*2) node[vert,](5){$\{5\}$};
        \draw (24+2*2,    -8.5+2*2) node[vert](6){$\{6, 7, 8\}$};
        \draw (24+0*2,    -8.5+1*2) node[vert](9){$\{9, 13\}$};
        \draw (24+2*2+0.5,-8.5+1*2-0.5) node[vert ](10){$\{10, 11, 12,14, 15, 16\}$};
        %
        \path[every node/.style={font=\sffamily\tiny, fill=white}]
            (1)     edge[aedge]     node{$w_{(1, 2)}$   }     (2)
            (1)     edge[aedge]     node{$w_{(1, 5)}$   }     (5)
            (1)     edge[aedge]     node{$w_{(1, 5)}$   }     (5)
            (5)     edge[aedge]     node{$w_{(5, 6)}$   }     (6)
            (5)     edge[aedge]     node{$w_{(5, 9)}$   }     (9)
            (9)     edge[aedge]     node{$w_{(9, 10)} + w_{(13, 14)}$   }     (10)
            (2)     edge[aedge]     node{$w_{(2, 6)} + w_{(3, 7)} + w_{(4, 8)} $   }     (6)
            (6)     edge[aedge]     node{$w_{(6, 10)} + w_{(7, 11)} + w_{(8, 12)} $   }     (10)
        ;

        \draw (22   +0*2,-2.5) node[minimum size=2cm,font=\sffamily\Huge] (dummy){$\longleftarrow$};
        \draw (22   +0*2,-4.5) node[minimum size=2cm,font=\sffamily\huge] (dummy){$\textbf{run CC}$};
        \draw (22   +0*2,-6.5) node[minimum size=2cm,font=\sffamily\Huge] (dummy){$\longleftarrow$};

        \draw (15+0*2,-8.5+3*2) node[vert,fill=blue!20!cyan](1){$\{1\}$};
        \draw (15+2*2,-8.5+3*2) node[vert,fill=green!30!cyan](2){$\{2, 3, 4\}$};
        \draw (15+0*2,-8.5+2*2) node[vert,fill=blue!20!cyan ](5){$\{5\}$};
        \draw (15+2*2,-8.5+2*2) node[vert,fill=blue!40!green](6){$\{6, 7, 8\}$};
        \draw (15+0*2,-8.5+1*2) node[vert,fill=blue!40!green](9){$\{9, 13\}$};
        \draw (15+2*2+0.5,-8.5+0.811*2-0.5) node[vert,fill=blue!40!green](10){$\{10, 11, 12,14, 15, 16\}$};
        %
        \path[every node/.style={font=\sffamily\tiny, fill=white}]
            (1)     edge[cedge]     node{$w_{(1, 2)}$   }     (2)
            (1)     edge[cedge]     node{$w_{(1, 5)}$   }     (5)
            (1)     edge[aedge]     node{$w_{(1, 5)}$   }     (5)
            (5)     edge[cedge]     node{$w_{(5, 6)}$   }     (6)
            (5)     edge[cedge]     node{$w_{(5, 9)}$   }     (9)
            (9)     edge[aedge]     node{$w_{(9, 10)} + w_{(13, 14)}$   }     (10)
            (2)     edge[cedge]     node{$w_{(2, 6)} + w_{(3, 7)} + w_{(4, 8)} $   }     (6)
            (6)     edge[aedge]     node{$w_{(6, 10)} + w_{(7, 11)} + w_{(8, 12)} $   }     (10)
        ;

        \draw (13   +0*2,-2.5) node[minimum size=2cm,font=\sffamily\Huge] (dummy){$\longleftarrow$};
        \draw (13   +0*2,-4.5) node[minimum size=2cm,font=\sffamily\huge] (dummy){$\textbf{project}$};
        \draw (13   +0*2,-6.5) node[minimum size=2cm,font=\sffamily\Huge] (dummy){$\longleftarrow$};


        \draw (5+0*2,-8.5+3*2) node[vert,fill=blue!20!cyan](1){$1$};
        \draw (5+1*2,-8.5+3*2) node[vert,fill=green!30!cyan](2){$2$};
        \draw (5+2*2,-8.5+3*2) node[vert,fill=green!30!cyan](3){$3$};
        \draw (5+3*2,-8.5+3*2) node[vert,fill=green!30!cyan](4){$4$};
        \draw (5+0*2,-8.5+2*2) node[vert,fill=blue!20!cyan](5){$5$};
        \draw (5+1*2,-8.5+2*2) node[vert,fill=blue!40!green](6){$6$};
        \draw (5+2*2,-8.5+2*2) node[vert,fill=blue!40!green](7){$7$};
        \draw (5+3*2,-8.5+2*2) node[vert,fill=blue!40!green](8){$8$};
        \draw (5+0*2,-8.5+1*2) node[vert,fill=blue!40!green](9){$9$};
        \draw (5+1*2,-8.5+1*2) node[vert,fill=blue!40!green](10){$10$};
        \draw (5+2*2,-8.5+1*2) node[vert,fill=blue!40!green](11){$11$};
        \draw (5+3*2,-8.5+1*2) node[vert,fill=blue!40!green](12){$12$};
        \draw (5+0*2,-8.5+0*2) node[vert,fill=blue!40!green](13){$13$};
        \draw (5+1*2,-8.5+0*2) node[vert,fill=blue!40!green](14){$14$};
        \draw (5+2*2,-8.5+0*2) node[vert,fill=blue!40!green](15){$15$};
        \draw (5+3*2,-8.5+0*2) node[vert,fill=blue!40!green](16){$16$};
        %
        \path[every node/.style={font=\sffamily\tiny, fill=white}]
            (1)     edge[cedge]     node{$w_{(1, 2)}$   }     (2)
            (2)     edge[aedge]     node{$w_{(2, 3)}$   }     (3)
            (3)     edge[aedge]     node{$w_{(3, 4)}$   }     (4)
            (5)     edge[cedge]     node{$w_{(5, 6)}$   }     (6)
            (6)     edge[aedge]     node{$w_{(6, 7)}$   }     (7)
            (7)     edge[aedge]     node{$w_{(7, 8)}$   }     (8)
            (9)     edge[aedge]     node{$w_{(9, 10)}$  }     (10)
            (10)    edge[aedge]     node{$w_{(10, 11)}$ }     (11)
            (11)    edge[aedge]     node{$w_{(11, 12)}$ }     (12)
            (13)    edge[aedge]     node{$w_{(13, 14)}$ }     (14)
            (14)    edge[aedge]     node{$w_{(14, 15)}$ }     (15)
            (15)    edge[aedge]     node{$w_{(15, 16)}$ }     (16)
            (1)     edge[aedge]     node{$w_{(1, 5)}$   }     (5)
            (5)     edge[cedge]     node{$w_{(5, 9)}$   }     (9)
            (9)     edge[aedge]     node{$w_{(9, 13)}$  }     (13)
            (2)     edge[cedge]     node{$w_{(2, 6)}$   }     (6)
            (6)     edge[aedge]     node{$w_{(6, 10)}$  }     (10)
            (10)    edge[aedge]     node{$w_{(10, 14)}$ }     (14)
            (3)     edge[cedge]     node{$w_{(3, 7)}$   }     (7)
            (7)     edge[aedge]     node{$w_{(7, 11)}$  }     (11)
            (11)    edge[aedge]     node{$w_{(11, 15)}$ }     (15)
            (4)     edge[cedge]     node{$w_{(4, 8)}$   }     (8)
            (8)     edge[aedge]     node{$w_{(8, 12)}$  }     (12)
            (12)    edge[aedge]     node{$w_{(12, 16)}$ }     (16)
        ;
      \end{tikzpicture}
   } 
%\end{subfigure}\quad\quad


\vspace{0.1cm}



\quad\quad\quad

\caption{
        Describe Method here
}
\end{figure*}




%-------------------------------------------------------------------------
\subsection{Proposal Generators}

\input{inputs/fig_hc.tex}


%-------------------------------------------------------------------------
\subsection{Fusion Move Solver}


\section{Experiments}
\subsection{Benchmark Models}
\subsection{Pixel-wise Multicuts}
%!TEX root = ../egpaper_for_review.tex

\newcounter{cX}
\newcounter{cY}


\newcounter{NPY}
\setcounter{NPY}{30}
\tikzstyle{pixel}=[opacity=1.0,thick,draw opacity=1.0, draw=black]
\tikzstyle{ln}=[opacity=0.6,fill=green,circle,draw, inner sep=0,font=\tiny,minimum size=0.15cm]
\tikzstyle{gn}=[opacity=0.6,fill=red,circle,draw, inner sep=0,font=\tiny,minimum size=0.15cm]
\tikzstyle{p}=[opacity=0.6,fill=white,circle,draw, inner sep=0,font=\tiny,minimum size=0.15cm]
\begin{figure}
\begin{center}
\begin{tikzpicture}[scale=1.0]
    \node[anchor=south west,inner sep=0] (image) at (0,0) 
        {\includegraphics[width=0.4\textwidth]{images/120093.jpg}};
    % shift scope to the image 
    \begin{scope}[x={(image.south east)},y={(image.north west)},xscale=100/481,yscale=100/321]   
        \draw[gray,xstep=3.21/\theNPY, ystep=3.21/\theNPY] (0,0) grid (4.81,3.21);
        % scope of the grid
        \begin{scope}[xscale=3.21/\theNPY,yscale=3.21/\theNPY]  

        \foreach \x\y in {22/18, 10/8} { 
        \setcounter{cX}{\x}
        \setcounter{cY}{\y}
        \draw (\thecX+0.5,\thecY+0.5) node[p]  (centerPixel){};
        \foreach \xx in {-4,0,4} { 
        \foreach \yy in {-4,0,4} { 
            \ifthenelse{\NOT 0 = \xx \OR \NOT 0 = \yy}{
                %\filldraw[green!40!white,pixel] 
                %(\thecX+\xx,\thecY+\yy) rectangle (\thecX+1+\xx,\thecY+1+\yy);
                \draw (\thecX+\xx+0.5,\thecY+\yy+0.5) node[gn](nonLocalPixel){};
                \path[]
                (centerPixel) edge[bend left=0*\xx*\yy ] (nonLocalPixel)
                ;
            }{
            }
        }
        }
        \foreach \xx/\yy/\pColor in {0/1/blue, 0/-1/blue, 1/0/blue, -1/0/blue} { 
            \draw (\thecX+\xx+0.5,\thecY+\yy+0.5) node[ln](nonLocalPixel){};
            (\thecX+\xx,\thecY+\yy) rectangle (\thecX+1+\xx,\thecY+1+\yy);
        }
        }
        % \filldraw[gray,pixel] (\thecX,\thecY) rectangle (\thecX+1,\thecY+1);
        

        \end{scope}
    \end{scope}


\end{tikzpicture}
\end{center}
\caption{
    Pixel Level Multicut:
    every pixel (white nodes) is connected
    to its 4 \emph{local} neighbors (edges between white and green nodes).
    Furthermore each pixel is connected to some \emph{non-local} neighbors within a 
    certain radius  (edges between white and green nodes).
    The local neighbors are connected with a \emph{positive}
    edge weight. If the edge indicator (as gradient magnitude)
    is very high, the local edge weight should be close to zero.
    If there is no evidence for a cut (low gradient magnitude for example)
    the local edge weight should be high.
    The \emph{non-local edge weights} are \emph{negative} to
    encourage label transitions.
    The weight of the non-local edge weights can
    be the negative value of the maximum gradient magnitude
    along a line between the red and white node.
    If there is evidence for a cut between red and white, 
    the weight should be strongly(?) negative.
}
\end{figure}







\section{Conclusion}

    



{\small
\bibliographystyle{ieee}
\bibliography{egbib}
}

\end{document}
