%!TEX root = ./egpaper_for_review.tex
\subsection{Datasets}\label{sec:datasets}
\textbf{Social Networks.}\label{sec:nets}
One important application for large scale correlation clustering are social networks.
We consider two of those networks from the Stanford Large Network Dataset Collection\footnote{\url{http://snap.stanford.edu/data/index.html}}.
Both networks are given by weighted directed graphs with edge weights $-1$ and $1$. 
%
The first network is called \emph{Epinions}. 
This is a who-trust-whom online social network of a general consumer review site. 
Each directed edge $a\to b$ indicates that user $a$ trusts  or does not trust user $b$ by a  positive or negative edge-weight, respectively.
The network contains $131828$ nodes and $841372$ edges from which $85.3\%$ are positively weighted.
%
The second network is called \emph{Slashdot}. 
Slashdot is a technology-related news website known for its specific user community. 
In 2002 Slashdot introduced the Slashdot Zoo feature which allows users to tag each other as friend or foe. 
The network was obtained in November 2008 and contains $77350$ nodes and $516575$ edges of which $76.73\%$ are positively weighted.

We consider the problem to cluster these graphs such that positively weighted edges ($E^+_{\to}$) link inside and negatively weighted edges ($E^-_{\to}$) between clusters.
In other words friends and people who trust each other should be in the same segment and foes and non-trusting people in different clusters.
% 
To compensate the high impact of nodes with high degree we can normalize the edge weights such that each person has the same impact on the overall network, by enforcing.
\begin{align}
  \sum_{i\to j \in E_{\to}} |w_{i\to j}| &= 1&\forall i\in V, deg^{\textrm{out}}(i)\geq 1 
\end{align}
We define the following energy function
\begin{align}
 J(y) &= \sum_{i\to j \in E^+_{\to}} y_{ij}\cdot w_{i \to j} +  \sum_{i\to j \in E^-_{\to}} (y_{ij}-1)\cdot w_{i \to j} \nonumber\\
      &= \sum_{ij \in E} y_{ij}\cdot \underbrace{(w_{i \to j}+w_{j \to i})}_{w_{ij}} + \textrm{const}
\end{align}
which is zero if the given partitioning does not violate any relation and larger otherwise.
We name these two datasets \emph{social nets} and \emph{normalized social nets}.

\textbf{Network Modularity Clustering.}
As another example for network clustering we use the \emph{modularity-clustering} models from~\cite{kappes_2014_benchmark_arxiv} which are small but fully connected.

\textbf{2D and 3D Image Segmentation}\label{sec:imseg}
To segment images or volumes into a previously
unknown number of clusters, correlation clustering
has been used~\cite{andres_2011_iccv,kroeger_2012_eccv}.

Starting from a super-pixel/-voxel segmentation,
correlation clustering finds the clustering with the lowest energy.
The energy is based on a likelihood of merging adjacent super-voxels.
Each edge has a probability to keep adjacent segments separate ($p(y_{ij} =1)$)
or to merge them ($p(y_{ij} = 0)$).
The energy function is
\begin{align}
 J(y)  &= \sum_{ij \in E} y_{ij}\cdot \underbrace{  log\left( \frac{p(y_{ij} =0)}{p(y_{ij} =1)}\right) + log \frac{1-\beta}{\beta}  }_{w_{ij}}
\end{align}
where $\beta$ is used as a prior~\cite{andres_2011_iccv}.

We use the publicly available benchmark instances from~\cite{kappes_2013_benchmark_cvpr,kappes_2014_benchmark_arxiv}.
For 2D images from the Berkeley Segmentation Database~\cite{martin_2001}, we took the segmentation problems called \emph{image-seg}~\cite{andres_2011_iccv,kappes_2013_benchmark_cvpr}.
For 3D volume segmentation we took the models \emph{knott-3d-150},\emph{-300} and \emph{-450} from~\cite{kroeger_2012_eccv,kappes_2014_benchmark_arxiv} as well as the large
instance from the \emph{3d-seg} model~\cite{andres_2011_iccv,kappes_2013_benchmark_cvpr}. These instances have underlying cube sizes of  $150^3$, $300^3$, $450^3$, and $900^3$, respectively.
We also requested larger instances from the authors of~\cite{kroeger_2012_eccv} who kindly provided us the dataset~\emph{knott-3d-550} with cube size  $550^3$.
%More information of the size of instances is given in Tab.~\ref{tab:instance_sizes}.
