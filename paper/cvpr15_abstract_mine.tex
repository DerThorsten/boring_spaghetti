% Use class option [extendedabs] to prepare the 1-page extended abstract.
\documentclass[extendedabs]{bmvc2k}
\usepackage[colorlinks = true,
            linkcolor = blue,
            urlcolor  = blue,
            citecolor = blue,
            anchorcolor = blue]{hyperref}

% Document starts here
\begin{document}


\title{Fusion Moves for Correlation Clustering}
\addauthor{
Thorsten Beier$^1$, 
Fred A. Hamprecht$^2$
J\"org H. Kappes$^3$
}{}{1}
\addinstitution{
$^1$thorsten.beier@iwr.uni-heidelberg.de 
$^2$fred.hamprecht@iwr.uni-heidelberg.de
$^3$kappes@math.uni-heidelberg.de
}
 
 
\maketitle
\let\thefootnote\relax\footnote{This is an extended abstract. The full paper is available at the \href{http://www.cv-foundation.org/openaccess/CVPR2015.py}{Computer Vision Foundation webpage}. }
\vspace{-0.2in}



% Extended abstract begins here.  In a one-page document, there is
% little need for section headers, but you may use \section etc if you
% wish.

\noindent
Correlation clustering, or multicut partitioning,
is widely used in image segmentation for
partitioning an undirected graph or image with positive and negative edge weights 
such that the sum of cut edge weights is minimized.
%
Due to its NP-hardness, exact solvers do not scale and approximative solvers often give unsatisfactory results.
We investigate scalable methods for correlation clustering.
To this end we define fusion moves for the correlation clustering problem.

Our algorithm iteratively fuses the current and a proposed partitioning which 
monotonously improves
the partitioning and maintains a valid partitioning at all times.
Furthermore, it scales to larger datasets, gives near optimal solutions, and at the same time shows
a good anytime performance.




Correlation clustering~\cite{Bansal-2002}, also known as the multicut problem~\cite{chopra_1993_mp} 
is a basic primitive in computer vision~\cite{andres_2011_iccv,kroeger_2012_eccv,yarkony_2012_eccv,alush_2013_simbad} and data mining~\cite{Arasu-2009,Sadikov-2010,Chen-2012,Chierichetti-2014}.
%See Sec.~\ref{sec:problem_formulation} for its formal definition of clustering the nodes of a graph.
 
Its merit is, firstly, that it accommodates both positive (attractive) \emph{and} negative (repulsive) edge weights.
This allows doing justice to evidence in the data that two nodes or pixels do not wish or do wish to end up in the same cluster or segment, respectively.
Secondly, it does not require a specification of the number of clusters beforehand.


In signed social networks, where positive and negative edges encode friend and foe relationships, respectively,
correlation clustering is a natural way to detect communities~\cite{Chen-2012,Chierichetti-2014}.
Correlation clustering can also be used to cluster query refinements in web search~\cite{Sadikov-2010}.
Because social and web-related networks are often huge, heuristic methods, \eg the PIVOT-algorithm~\cite{Ailon-2008},
are popular~\cite{Chierichetti-2014}.

In computer vision applications, unsupervised image segmentation algorithms often start with an over-segmentation
into superpixels (superregions), which are then clustered into ``perceptually meaningful''
regions by correlation clustering.
Such an approach has been shown to yield
state-of-the-art results on the Berkeley Segmentation Database
\cite{andres_2011_iccv,Kim-2011,yarkony_2012_eccv,alush_2013_simbad}.

While it has a clear mathematical formulation and nice properties,
correlation clustering suffers from NP-hardness. 
%
Consequently, partition problems on large scale data, \eg
huge volume images in computational neuroscience~\cite{kroeger_2012_eccv}
or social networks~\cite{Leskovec-2010}, 
are not tractable because reasonable solutions cannot be computed in acceptable time.


\vspace{0.1cm}
\noindent \textbf{Contribution.}
In this work we present novel approaches that are designed for large scale correlation clustering problems.
First, we define a novel energy based agglomerative clustering algorithm that monotonically increases the energy.
With this at hand we show how to improve the anytime performance of Cut, Clue \& Cut~\cite{beier_2014_cvpr}.
%
Second, we improve the anytime performance of polyhedral multicut methods~\cite{kappes_2013_arxiv} by more efficient separation procedures.
%
Third, we introduce cluster-fusion moves, which extend the original fusion moves~\cite{Lempitsky-2010} 
used in supervised segmentation to the unsupervised case and give a polyhedral interpretation of this algorithm.
Finally, we propose two versatile proposal generators, and evaluate the proposed methods on existing and new benchmark problems.
Experiments show that we can improve the computation time by one to two magnitudes without worsening the segmentation 
quality significantly.



\bibliography{egbib}

\end{document}
