\section{Introduction}
Correlation clustering~\cite{Bansal-2002} also known as the multicut problem~\cite{chopra_1993_mp} 
is a basic primitive in computer vision~\cite{andres_2011_iccv,kroeger_2012_eccv,yarkony_2012_eccv,alush_2013_simbad} and data mining~\cite{Chierichetti-2014,Arasu-2009,Sadikov-2010,Chen-2012},
see Sec.~\ref{sec:problem_formulation} for its formal definition of clustering the nodes of a graph..
 
Its value is, firstly, that it accommodates both positive (attractive) \emph{and} negative (repulsive) edge weights,
which allows doing justice to evidence in the data that two nodes or pixels do \emph{not} wish  or do wish to end up in the same cluster or segment, respectively.
Secondly, it does not require a specification of the number of clusters beforehand.


In signed social networks, where positive and negative edges encodes friend and foe relationships, respectively,
correlation clustering is a natural way to detect communities~\cite{Chierichetti-2014,Chen-2012}.
Correlation clustering can also been used to cluster query refinements in web search~\cite{Sadikov-2010}.
Because social and web-related networks are often very huge, heuristic methods, \eg the PIVOT-algorithm~\cite{Ailon-2008},
are very popular~\cite{Chierichetti-2014}.

In computer vision applications, unsupervised image segmentation algorithms usually start with an over-segmentation
into superpixels (superregions), which are then clustered into ``perceptually meaningful''
regions by correlation clustering on sparse graphs.
Such an approach has been shown to yield
state-of-the-art results on the Berkeley Segmentation Database
\cite{andres_2011_iccv,yarkony_2012_eccv,alush_2013_simbad}.

Despite the clear mathematical formulation and nice properties,
correlation clustering is known to be NP-hard. 
%
Consequently, partition problems on large scale data, \eg
huge volume images in computational neuroscience~\cite{kroeger_2012_eccv}
or social networks~\cite{Leskovec-2010}, 
are not tractable, because reasonable solutions cannot be computed in acceptable time.

% Importantly, this allows doing justice to evidence in the data that two nodes or pixels do \emph{not} wish to end up in the same cluster or segment. This is in contrast to the submodular potentials so popularized by the graph cut algorithm, which only allow two nodes to be attracted to each other, or at most be agnostic about membership in the same cluster. Secondly, the algorithm does not require a specification of the number of clusters. This is in contrast to methods such as normalized cut \cite{} that can only accommodate attractive interactions and hence need the number of clusters to be specified. 

% \paragraph{Contribution.} The evident usefulness of correlation clustering and its clean and compact formulation in terms of an optimization problem, together with its unfortunate NP-hardness, are an invitation to develop fast approximate solvers. The basic idea of the move making algorithm proposed here is to maintain, at all times, a best current partitioning; and to iteratively improve it (as we show, monotonously) by considering diverse and cheaply generated proposal partitionings. Any two nodes that are in one cluster in both the current and the proposal partitioning are contracted. Edges to contracted nodes are equally combined and reweighted, and the correlation clustering problem is then solved on this reduced graph. We offer a polyhedral characterization of this strategy, and evaluate it in conjunction with two versatile proposal generators.   We conduct experiments on a broad range of data sets ranging from 2D and 3D segmentation problems to clustering in signed social networks. The results suggest that this simple move making algorithm is the fastest technique known today, reaching close to globally optimal solutions in a tenth or a hundredth of the time required by the most efficient exact solvers. 
\vspace{0.1cm}
\noindent \textbf{Contribution.}
In the present work we present some novel approaches that are designed for large scale correlation clustering problems.
First, we define a novel energy based agglomerative clustering algorithm that monotonically increase the energy.
With this at hand we show how to improve the anytime performance of Cut, Clue \& Cut.
Second, we introduce cluster-fusion moves, which extend the original fusion moves~\cite{Lempitsky-2010} 
used in supervised segmentation to the unsupervised case and give a polyhedral interpretation of this algorithm.
We propose two versatile proposal generators, and evaluate the proposed methods on existing and new benchmark problems.
That experiments show that we can improve the computation time by one to two magnitudes without worsening the segmentation 
quality significantly.
 
\vspace{0.1cm}
\noindent \textbf{Related Work.}
A natural approach is to solve the integer linear programming problem directly. 
To this end, efficient separation procedures have been found~\cite{kappes_2011_emmcvpr,kappes_2013_arxiv} that allow to iteratively augment the set of constraints until a valid partitioning is found. 
Alternatively, it is possible to relax the integrality constraints of the integer linear programming formulation~\cite{kappes_2013_arxiv}. 
Such an outer relaxation can be iteratively tightened. However, intermediate solutions are fractional and so rounding is required to obtain a valid partitioning.
For the latter approach column generating methods exits, that works best on planar \cite{yarkony_2012_eccv} or almost planar \cite{yarkony-andres-2013} graphs. 

Another line of work uses move making algorithm 
to optimize correlation clustering~\cite{bansal_2004_ml,beier_2014_cvpr,Kernighan-1970}.
Starting with an initial segmentation, auxiliary max-cut problems are approximately solved,
such that the segmentation is strictly improved.
As shown by Beier \etal~\cite{beier_2014_cvpr} only CGC~\cite{beier_2014_cvpr} 
can deal with large scale problems, but can also suffer on very large auxiliary problems.

Outside computer vision greedy methods~\cite{Soon-2001,Ng-2002,Gionis-2007,Elsner-2008,Ailon-2008} has been suggested for correlation clustering problems, see \cite{Elsner-2009} for an overview.
The PIVOT Algorithm~\cite{Ailon-2008} iterate over all nodes in random order.
If the node is not assigned it construct a cluster containing the node and all its 
unassigned positively linked neighbors.  
%
A widely use post-processing method is  Best One Element Move (BOEM)~\cite{Gionis-2007}, which iteratively reassigned nodes to clusters.

For energy minimization problems fusion moves have become increasingly popular~\cite{Lempitsky-2010,kappes_2014_ws}.
For many large scale computer vision applications fusion moves lead to good approximations
with state of the art any time performance~\cite{kappes_2014_ws}.
Due to the ambiguity of a node-labeling, classical fusion moves~\cite{Lempitsky-2010} can not be applied directly for correlation clustering.
We will show how to overcome this point later.


\vspace{0.1cm}
\noindent \textbf{Outline:} 
In Sec.~\ref{sec:problem_formulation} we will give a 
detailed problem definition where we introduce 
the correlation clustering objective.
In Sec.~\ref{sec:cc_fm} we describe our proposed
method and show the properties of the algorithm.
In Sec.~\ref{sec:exp} we show an evaluation
of the proposed methods methods and state of the art
and conclude in Sec.~\ref{sec:future} and\ref{sec:conclusion}.
%-------------------------------------------------------------------------

\section{Notation and Problem Formulation}\label{sec:problem_formulation}
Let $G=(V,E, w)$ be a weighted graph of nodes $V$ and edges $E$.
%
The function $w : E \rightarrow \mathbb{R}$ assigns a weight to each edge.
We will use $w_e$ as a shorthand for $w(e)$.
A positive weight expresses the desire that two adjacent nodes should
be merged, whereas a negative weight indicates
that these nodes should be separated into two different regions.
%
%A \emph{subgraph} $G_A = \{A, E_A, w\}$ consists
%of nodes $A \subseteq V$ and edges $E_A := E\cap (A\times A)$.
%
A segmentation of the graph $G$ can by either given by an 
node labeling $l \in \mathbb{N}^{|V|}$
or an edge labeling $y \in\{0,1\}^{|E|}$, \cf Fig.~\ref{fig:notation} .
An edge labeling is only consistent if it contains no dangling edges~\cite{kappes_2013_arxiv}.
We denote the set of all consistent edge labelings by $P(G)\subset\{0,1\}^{|E|}$.
The convex hull of this set is known as the \emph{multicut polytope} $MC(G) = \textrm{conv}(P(G))$.

Given a weighted graph $G=(V,E,w)$ we consider the problem of segmenting $G$ such that the costs
of the edges between distinct segments is minimized. This can be formulated in the node domain
by assigning each node $v$ a label $l_v \in \mathbb{N}$
\begin{align}
  l^* &= \argmin_{L \in \mathbb{N}^{|V|}} \sum_{ (i,j) \in E } w_{ij} \cdot [l_{i} \neq l_{j}], \label{eq:nodeproblem}
\end{align} 
or in the edge domain, by label each edge $e$ as cut $y_e=1$ or uncut $y_e=0$ 
\begin{align}
  y^* &= \argmin_{y \in P(G)} \sum_{ (i,j) \in E } w_{ij} \cdot y_{ij} \label{eq:edgeproblem}.%\\ 
\end{align}
As shown in~\cite{kappes_2013_arxiv}  both problems are equivalent, but formulation \ref{eq:nodeproblem}
suffers from ambiguities in the formulation, \cf Fig.~\ref{fig:notation}. 

%!TEX root = ../egpaper_for_review.tex
%
%\tikzstyle{nS}=[circle,minimum size = 0.6cm,inner sep = 0pt,draw, font=\small,align=left]
%\tikzstyle{eNS}=[fill=white,minimum size = 0.5cm,inner sep = 0pt, font=\tiny,align=left]
% size = 0.5cm,inner sep = 0pt,draw=black, font=\small,align=left, draw=red!70!black, line width=1mm]
%
\begin{center}
\begin{figure}[t]
\begin{tiny}
\begin{subfigure}[t]{0.22\linewidth}
    \resizebox{\linewidth}{!}{
        \begin{tikzpicture}[scale=1]%[x=1 cm, y=1 cm]
                \draw (2,2) node[nS,fill=red!60] (n0) {$0$}; 
                \draw (4,2) node[nS,fill=red!60] (n1) {$1$};
                \draw (6,2) node[nS,fill=blue!60] (n2) {$2$};
                \draw (2,0) node[nS,fill=red!60] (n3) {$3$}; 
                \draw (4,0) node[nS,fill=red!60] (n4) {$4$};
                \draw (6,0) node[nS,fill=green!60] (n5) {$5$};
                \path
                    (n0) edge[eBlack] node[eNS]{$w_{01}$} (n1) 
                    (n1) edge[eBlack] node[eNS]{$w_{12}$} (n2) 
                    (n3) edge[eBlack] node[eNS]{$w_{34}$} (n4) 
                    (n4) edge[eBlack] node[eNS]{$w_{45}$} (n5)
                    (n0) edge[eBlack] node[eNS]{$w_{03}$} (n3)
                    (n1) edge[eBlack] node[eNS]{$w_{14}$} (n4)
                    (n2) edge[eBlack] node[eNS]{$w_{25}$} (n5)
                ;
        \end{tikzpicture}
    }
\caption{ \tiny{node labels}}
\label{fig:not_a}
\end{subfigure}
\hfill
\begin{subfigure}[t]{0.22\linewidth}
    \resizebox{\linewidth}{!}{
        \begin{tikzpicture}[scale=1]%[x=1 cm, y=1 cm]
                \draw (2,2) node[nS,fill=yellow!60] (n0) {$0$}; 
                \draw (4,2) node[nS,fill=yellow!60] (n1) {$1$};
                \draw (6,2) node[nS,fill=magenta!60] (n2) {$2$};
                \draw (2,0) node[nS,fill=yellow!60] (n3) {$3$}; 
                \draw (4,0) node[nS,fill=yellow!60] (n4) {$4$};
                \draw (6,0) node[nS,fill=cyan!60] (n5) {$5$};
                \path
                    (n0) edge[eBlack] node[eNS]{$w_{01}$} (n1) 
                    (n1) edge[eBlack] node[eNS]{$w_{12}$} (n2) 
                    (n3) edge[eBlack] node[eNS]{$w_{34}$} (n4) 
                    (n4) edge[eBlack] node[eNS]{$w_{45}$} (n5)
                    (n0) edge[eBlack] node[eNS]{$w_{03}$} (n3)
                    (n1) edge[eBlack] node[eNS]{$w_{14}$} (n4)
                    (n2) edge[eBlack] node[eNS]{$w_{25}$} (n5)
                ;
        \end{tikzpicture}
    }
\caption{ \tiny{node labels}}
\label{fig:not_b}
\end{subfigure}
\hfill
\begin{subfigure}[t]{0.22\linewidth}
    \resizebox{\linewidth}{!}{
        \begin{tikzpicture}[scale=1]%[x=1 cm, y=1 cm]
                \draw (2,2) node[nS] (n0) {$0$}; 
                \draw (4,2) node[nS] (n1) {$1$};
                \draw (6,2) node[nS] (n2) {$2$};
                \draw (2,0) node[nS] (n3) {$3$}; 
                \draw (4,0) node[nS] (n4) {$4$};
                \draw (6,0) node[nS] (n5) {$5$};
                \path
                    (n0) edge[eNotCut] node[eNS]{$w_{01}$} (n1) 
                    (n1) edge[eCut] node[eNS]{$w_{12}$} (n2) 
                    (n3) edge[eNotCut] node[eNS]{$w_{34}$} (n4) 
                    (n4) edge[eCut] node[eNS]{$w_{45}$} (n5)
                    (n0) edge[eNotCut] node[eNS]{$w_{03}$} (n3)
                    (n1) edge[eNotCut] node[eNS]{$w_{14}$} (n4)
                    (n2) edge[eCut] node[eNS]{$w_{25}$} (n5)
                ;
        \end{tikzpicture}
    }
\caption{ \tiny{edge labels}}
\label{fig:not_c}
\end{subfigure}
\hfill
\begin{subfigure}[t]{0.22\linewidth}
    \resizebox{\linewidth}{!}{
        \begin{tikzpicture}[scale=1]%[x=1 cm, y=1 cm]
                \draw (2,2) node[nS] (n0) {$0$}; 
                \draw (4,2) node[nS] (n1) {$1$};
                \draw (6,2) node[nS] (n2) {$2$};
                \draw (2,0) node[nS] (n3) {$3$}; 
                \draw (4,0) node[nS] (n4) {$4$};
                \draw (6,0) node[nS] (n5) {$5$};
                \path
                    (n0) edge[eNotCut] node[eNS]{$w_{01}$} (n1) 
                    (n1) edge[eCut] node[eNS]{$w_{12}$} (n2) 
                    (n3) edge[eNotCut] node[eNS]{$w_{34}$} (n4) 
                    (n4) edge[eCut] node[eNS]{$w_{45}$} (n5)
                    (n0) edge[eNotCut] node[eNS]{$w_{03}$} (n3)
                    (n1) edge[eCut] node[eNS]{$w_{14}$} (n4)
                    (n2) edge[eCut] node[eNS]{$w_{25}$} (n5)
                ;
        \end{tikzpicture}
    }
\caption{ \tiny{dangling edge}}
\label{fig:not_d}
\end{subfigure}
\end{tiny}
\vspace{-0.1cm}
\caption{
For correlation clustering node labels suffers from ambiguities.
\ref{fig:not_a} and~\ref{fig:not_b} encode the same partition.
Edge labels as in~\ref{fig:not_c} do not suffer from these 
ambiguities, but can have \emph{dangling edges} as in~\ref{fig:not_d}.
Node $1$ and $4$
are in the same connected component,
and at the same time $e_{14}$ is cut.
%Therefore~\ref{fig:not_d} 
This is not a valid partition.
}\label{fig:notation}
\end{figure}
\end{center}


