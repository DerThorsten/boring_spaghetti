\documentclass[10pt,letterpaper]{article}

\usepackage{eso-pic}
\usepackage{cvpr}
\usepackage{times}
\usepackage{epsfig}
\usepackage{graphicx}
\usepackage{amsmath}
\usepackage{amssymb}
\usepackage{xspace}
\usepackage{booktabs}
\usepackage{placeins}


% Include other packages here, before hyperref.
\usepackage{my_macros}
\usepackage{paralist}
\usepackage{amsthm}
\usepackage{dirtytalk}
%\usepackage{framed}
\usepackage{caption}
\usepackage{subcaption}
\usepackage{algorithm}
%\usepackage{algorithmic}
\usepackage{algpseudocode}
\usepackage{setspace}
%for tikz
\usepackage{tikz,pgfplots}
\usepackage{pgfplotstable}
\usepackage{filecontents}
%\usepackage{scrextend}
\usepackage[pagebackref=true,breaklinks=true,letterpaper=true,colorlinks,bookmarks=false]{hyperref}

\newcommand{\footlabel}[2]{%
    \addtocounter{footnote}{1}%
    \footnotetext[\thefootnote]{%
        \addtocounter{footnote}{-1}%
        \refstepcounter{footnote}\label{#1}%
        #2%
    }%
    $^{\ref{#1}}$%
}

\newcommand{\footref}[1]{%
    $^{\ref{#1}}$%
}

\usetikzlibrary{arrows,positioning,automata,shadows,fit,shapes}
\usetikzlibrary{arrows,petri,topaths}
\usetikzlibrary{positioning,fit,calc}
\usetikzlibrary{shapes.arrows,chains,decorations.pathreplacing,fadings}
\usetikzlibrary{calc, matrix, backgrounds}

\pgfplotsset{every axis/.append style={
  every axis y label/.style = {at={(ticklabel cs:0.5)}, rotate=90, anchor=south},
  axis x line = {bottom},
  axis y line = {left},
  tick align = outside,
  ymajorgrids = true,
%  legend style = {draw=none, at={(1.05, 0.5)}, anchor=west, font=\small},
  legend style = {font=\tiny},
  legend columns = 1,
  every axis plot/.append style = {line width=1pt},
  label style = {font=\small},
  tick label style={font=\small},
  scaled ticks = false,
}}

% Two Colored Circle Split 
\makeatletter
\tikzset{circle split part fill/.style  args={#1,#2}{%
 alias=tmp@name, 
  postaction={%
    insert path={
     \pgfextra{% 
     \pgfpointdiff{\pgfpointanchor{\pgf@node@name}{center}}%
                  {\pgfpointanchor{\pgf@node@name}{east}}%            
     \pgfmathsetmacro\insiderad{\pgf@x}
      \fill[#1] (\pgf@node@name.base) ([xshift=-\pgflinewidth]\pgf@node@name.east) arc
                          (0:180:\insiderad-\pgflinewidth)--cycle;
      \fill[#2] (\pgf@node@name.base) ([xshift=\pgflinewidth]\pgf@node@name.west)  arc
                           (180:360:\insiderad-\pgflinewidth)--cycle;            
         }}}}}  
 \makeatother  

%\usepackage{tkz-berge}
\DeclareMathOperator*{\argmin}{arg\,min}
\DeclareMathOperator*{\argmax}{arg\,max}
\definecolor{shadecolor}{rgb}{0.01,0.199,0.1}
\usepackage{xargs} 
\newtheorem{theorem}{Theorem}
\newtheorem{remark}{Remark}

% If you comment hyperref and then uncomment it, you should delete
% egpaper.aux before re-running latex.  (Or just hit 'q' on the first latex
% run, let it finish, and you should be clear).
\usepackage{bm}% ändert \boldsymbol
\def\arraystretch{0.8}
\renewcommand{\tabcolsep}{2pt}
\newcommand{\thickline}{2pt}
\newcommand{\scatterplotpath}{./scatterplots/}

\usepackage[colorinlistoftodos,prependcaption,textsize=tiny]{todonotes}
\newcommandx{\unsure}[2][1=]{\todo[linecolor=red,backgroundcolor=red!25,bordercolor=red,#1]{#2}}
\newcommandx{\change}[2][1=]{\todo[linecolor=blue,backgroundcolor=blue!25,bordercolor=blue,#1]{#2}}
\newcommandx{\info}[2][1=]{\todo[linecolor=OliveGreen,backgroundcolor=OliveGreen!25,bordercolor=OliveGreen,#1]{#2}}
\newcommandx{\improvement}[2][1=]{\todo[linecolor=Plum,backgroundcolor=Plum!25,bordercolor=Plum,#1]{#2}}
\newcommandx{\thiswillnotshow}[2][1=]{\todo[disable,#1]{#2}}
\newcommand{\OR}{\textrm{ or }}

\cvprfinalcopy % *** Uncomment this line for the final submission


\def\cvprPaperID{1881} % *** Enter the CVPR Paper ID here
\def\httilde{\mbox{\tt\raisebox{-.5ex}{\symbol{126}}}}

\newcommand{\realplot}[1]{
\begin{tikzpicture}
    \begin{axis}
        \addlegendimage{empty legend}\addlegendentry{Matrix #1}
        \addplot {0};
        \addplot {1};
        \addplot {2};
        \addplot {3};
    \end{axis}
\end{tikzpicture}}  

% \newcommand{\anytimeplot}[1]{
%   \addplot[color=brown,mark=x] table[x=time,y=HC]{data3/knott-3d-300.data};              \addlegendentry{HC}
%   \addplot[color=black,mark=x] table[x=time,y=HC-CGC]{data3/knott-3d-300.data};          \addlegendentry{HC-CGC}
%   \addplot[color=red,mark=x] table[x=time,y=CGC]{data3/knott-3d-300.data};               \addlegendentry{CGC}
%   \addplot[color=gray,mark=o] table[x=time, y=ogm-KL]{data3/knott-3d-300.data};          \addlegendentry{KL}
%   \addplot[color=purple,mark=x] table[x=time, y=MCR-CCFDB]{data3/knott-3d-300.data};     \addlegendentry{MC-R}
%   \addplot[color=blue,mark=x] table[x=time, y=MCI-CCIFD]{data3/knott-3d-300.data};       \addlegendentry{MC-I}
%   \addplot[color=green,mark=x] table[x=time, y=DYNCC-HC-MC]{data3/knott-3d-300.data};    \addlegendentry{Fusion-HC-MC}
%   \addplot[color=cyan,mark=x] table[x=time, y=DYNCC-HC-CGC]{data3/knott-3d-300.data};    \addlegendentry{Fusion-HC-CGC}
%   \addplot[color=orange,mark=x] table[x=time, y=DYNCC-WS-MC]{data3/knott-3d-300.data};  \addlegendentry{Fusion-WS-MC}
%   \addplot[color=orange,mark=x] table[x=time, y=DYNCC-WS-CGC]{data3/knott-3d-300.data};  \addlegendentry{Fusion-WS-CGC}
% }
\newcommand{\anytimeplot}[1]{
  \addplot[color=brown,mark=x] table[x=time,y=HC]{data3/#1.data};              \addlegendentry{HC}
  \addplot[color=black,mark=square] table[x=time,y=HC-CGC]{data3/#1.data};          \addlegendentry{HC-CGC}
  \addplot[color=red,mark=x] table[x=time,y=CGC]{data3/#1.data};               \addlegendentry{CGC}
  \addplot[color=gray,mark=o] table[x=time, y=ogm-KL]{data3/#1.data};          \addlegendentry{KL} 
  \addplot[color=yellow!50!black,mark=square] table[x=time, y=ogm-mcfusion-HC-CF*]{data3/#1.data};  \addlegendentry{Fusion}
  \addplot[color=purple,mark=o] table[x=time, y=MCR-CCFDB]{data3/#1.data};     \addlegendentry{MC-R}
  \addplot[color=blue,mark=o] table[x=time, y=MCI-CCIFD]{data3/#1.data};       \addlegendentry{MC-I}
  \addplot[color=green,mark=o] table[x=time, y=DYNCC-HC-MC]{data3/#1.data};    \addlegendentry{CC-Fusion-HC-MC}
  \addplot[color=cyan,mark=x] table[x=time, y=DYNCC-HC-CGC]{data3/#1.data};    \addlegendentry{CC-Fusion-HC-CGC}
  \addplot[color=orange,mark=o] table[x=time, y=DYNCC-WS-MC]{data3/#1.data};  \addlegendentry{CC-Fusion-WS-MC}
  \addplot[color=pink,mark=x] table[x=time, y=DYNCC-WS-CGC]{data3/#1.data};  \addlegendentry{CC-Fusion-WS-CGC}
}
%  \addplot[color=yellow!50!black,mark=o] table[x=time, y=ogm-mcfusion-HC-CF*]{data3/#1.data};  \addlegendentry{Fusion}
  %\addplot[color=green!50,mark=o] table[x=time, y=PIVOT*]{data3/#1.data};          \addlegendentry{PIVOT-BOEM}

% Pages are numbered in submission mode, and unnumbered in camera-ready
\ifcvprfinal\pagestyle{empty}\fi
\begin{document}
%%%%%%%%% TITLE
%!TEX root = egpaper_for_review.tex
\newcommand{\Cut }{\mathcal{C}}
%\title{Correlation Clustering with Dynamic Super-Nodes (DySNCC)}
\title{Fusion Moves for Correlation Clustering\\(Supplementary Material)}

\author{Thorsten Beier\\
%Institution1\\
%Institution1 address\\
{\tt\small thorsten.beier@iwr.uni-heidelberg.de}
% For a paper whose authors are all at the same institution,
% omit the following lines up until the closing ``}''.
% Additional authors and addresses can be added with ``\and'',
% just like the second author.
% To save space, use either the email address or home page, not both
\and
Fred A. Hamprecht\\
%Institution2\\
%First line of institution2 address\\
{\tt\small fred.hamprecht@iwr.uni-heidelberg.de}
\and
J\"org H. Kappes \\
%Institution2\\
%First line of institution2 address\\
{\tt\small kappes@math.uni-heidelberg.de}
}

\maketitle
%\thispagestyle{empty}

 \tableofcontents
%%%%%%%%% ABSTRACT
%\begin{abstract}
%\end{abstract}
\newpage
\section{Anytime Plots (Full)}
%!TEX root = egpaper_for_review.tex
%%%%%%%%%%%%%%%%%%%%%%%%%%%%%%%%%%%%%%%%%%%%%%%%%%%%%%%%%%%%%%%%%%%%%%%%%%%%%%%%%%%%%%%%%%%%
\pgfplotsset{every axis legend/.append style={
at={(1.0,1.2)},
anchor=north east}} 
\begin{figure*}[h]
  \begin{subfigure}[b]{0.33\textwidth}
  \centering
  \begin{tikzpicture}
  \begin{semilogxaxis}[  mark size=1pt,
  %restrict y to domain=0:4620,
  xlabel = {runtime},
  xmin = 0,
  xmax = 4100,
  width = 1.0\columnwidth,
  scaled ticks = false,
  every axis legend/.code={\let\addlegendentry\relax} 
  ]
  \addplot[color=yellow,mark=o] table[x=time, y=PIVOT*]{data3/image-seg.data};    \addlegendentry{PIVOT-BOEM}  
  \anytimeplot{image-seg}
  \end{semilogxaxis}
  \end{tikzpicture}
  
  \caption{image-seg}\label{fig:at:image-seg}
  \end{subfigure}
  %%%%%%%%%%%%%%%%%%%%%%%%%%%%%%%%%%%%%%%%%%%%%%%%%%%%%%%%%%%%%%%%%%%%%%%%% 
  \begin{subfigure}[b]{0.33\textwidth}
    \centering
    \begin{tikzpicture}
    \begin{semilogxaxis}[  mark size=1pt,
   % restrict y to domain=-6000:-4400,
    xlabel = {runtime},
    xmin = 0,
    xmax = 100,
    width = 1.0\columnwidth,
    scaled ticks = false,
    every axis legend/.code={\let\addlegendentry\relax} 
    ] 
    \addplot[color=yellow,mark=o] table[x=time, y=PIVOT*]{data3/knott-3d-150.data};          \addlegendentry{PIVOT-BOEM}
    \anytimeplot{knott-3d-150}
  
    \end{semilogxaxis}
    \end{tikzpicture}
    \caption{knott-3d-150}\label{fig:at:knott-150}
  \end{subfigure}
  %%%%%%%%%%%%%%%%%%%%%%%%%%%%%%%%%%%%%%%%%%%%%%%%%%%%%%%%%%%%%%%%%%%%%%%%%%%%%%%%%%%%%%%%%%%%%
  \begin{subfigure}[b]{0.33\textwidth}
  \centering
  \begin{tikzpicture}
  \begin{semilogxaxis}[  mark size=1pt,
  %restrict y to domain=-40000:-24000,
  xlabel = {runtime},
  xmin = 0,
  xmax = 4100,
  width = 1.0\columnwidth,
  scaled ticks = false,
  %legend to name = ledgendPosition,
  legend columns=6,
  every axis legend/.code={\let\addlegendentry\relax} 
  ]  
  \anytimeplot{knott-3d-300}
  \end{semilogxaxis}
  \end{tikzpicture}
  \caption{knott-3d-300}\label{fig:at:knott-300}
  \end{subfigure}
  \newline
  %%%%%%%%%%%%%%%%%%%%%%%%%%%%%%%%%%%%%%%%%%%%%%%%%%%%%%%%%%%%%%%%%%%%%%%%%%%%%%%%%%%%%%%%%%%%%
  \begin{subfigure}[b]{0.33\textwidth}
  \centering
  \begin{tikzpicture}
  \begin{semilogxaxis}[  mark size=1pt,
  %restrict y to domain=-80000:-60000,
  xlabel = {runtime},
  xmin = 0,
  xmax = 4100,
  width = 1.0\columnwidth,
  scaled ticks = false,
  every axis legend/.code={\let\addlegendentry\relax} 
  ] 
  \anytimeplot{knott-3d-450} 
  \end{semilogxaxis}
  \end{tikzpicture}
  \caption{knott-3d-450}\label{fig:at:knott-450}
  \end{subfigure}
  %%%%%%%%%%%%%%%%%%%%%%%%%%%%%%%%%%%%%%%%%%%%%%%%%%%%%%%%%%%%%%%%%%%%%%%%%%%%%%%%%%%%%%%%%%%%%
  \begin{subfigure}[b]{0.33\textwidth}
  \centering
  \begin{tikzpicture}
  \begin{semilogxaxis}[  mark size=1pt,
  %restrict y to domain=-10000000:-100000,
  xlabel = {runtime},
  xmin = 0,
  xmax = 4100,
  width = 1.0\columnwidth,
  scaled ticks = false,
  every axis legend/.code={\let\addlegendentry\relax} 
  ]
  \anytimeplot{knott-3d-550} 
  \end{semilogxaxis}
  \end{tikzpicture}
  \caption{knott-3d-550}\label{fig:at:knott-550}
  \end{subfigure}
  %%%%%%%%%%%%%%%%%%%%%%%%%%%%%%%%%%%%%%%%%%%%%%%%%%%%%%%%%%%%%%%%%%%%%%%%%%%%%%%%%%%%%%%%%%%%%\begin{subfigure}[H]
  \begin{subfigure}[b]{0.33\textwidth}
  \centering
  \begin{tikzpicture}
  \begin{semilogxaxis}[  mark size=1pt,
  %restrict y to domain=-10000000:-100000,
  %restrict y to domain=-10000000:1200000,
  xlabel = {runtime},
  xmin = 0,
  xmax = 4100,
  width = 1.0\columnwidth,
  scaled ticks = false,
  every axis legend/.code={\let\addlegendentry\relax} 
  ] 
  \anytimeplot{seg-3d} 
  \end{semilogxaxis}
  \end{tikzpicture}
  \caption{seg-3d}\label{fig:at:seg3d}
  \end{subfigure}
 %%%%%%%%%%%%%%%%%%%%%%%%%%%%%%%%%%%%%%%%%%%%%%%%%%%%%%%%%%%%%%%%%%%%%%%%%%%%%%%%%%%%%%%%%%%%%
  \begin{subfigure}[b]{0.33\textwidth}
  \centering
  \begin{tikzpicture}
  \begin{semilogxaxis}[  mark size=1pt,
  %restrict y to domain=-10000000:-100000,
  %restrict y to domain=-10000000:100000,
  xlabel = {runtime},
  xmin = 0,
  xmax = 4100,
  width = 1.0\columnwidth,
  scaled ticks = false,
  every axis legend/.code={\let\addlegendentry\relax} 
  ]  
  \anytimeplot{socialnets} 
  \end{semilogxaxis}
  \end{tikzpicture}
  \caption{social nets}\label{fig:at:socialnets}
  \end{subfigure}
 %%%%%%%%%%%%%%%%%%%%%%%%%%%%%%%%%%%%%%%%%%%%%%%%%%%%%%%%%%%%%%%%%%%%%%%%%%%%%%%%%%%%%%%%%%%%%
  \begin{subfigure}[b]{0.33\textwidth}
  \centering
  \begin{tikzpicture}
  \begin{semilogxaxis}[  mark size=1pt,
  %restrict y to domain=-10000000:-100000,
  %restrict y to domain=-10000000:100000,
  xlabel = {runtime},
  xmin = 0,
  xmax = 4100,
  width = 1.0\columnwidth,
  scaled ticks = false,
  every axis legend/.code={\let\addlegendentry\relax} 
  ] 
  \anytimeplot{normalizedsocialnets}
  \end{semilogxaxis}
  \end{tikzpicture}
  \caption{normalized social nets}\label{fig:at:nsocialnets}
  \end{subfigure}
 %%%%%%%%%%%%%%%%%%%%%%%%%%%%%%%%%%%%%%%%%%%%%%%%%%%%%%%%%%%%%%%%%%%%%%%%%%%%%%%%%%%%%%%%%%%%%
  \begin{subfigure}[b]{0.33\textwidth}
  \centering
  \begin{tikzpicture}
  \begin{semilogxaxis}[  mark size=1pt,
  %restrict y to domain=-10000000:-100000,
  %restrict y to domain=-10000000:100000,
  xlabel = {runtime},
  xmin = 0,
  xmax = 4100,
  width = 1.0\columnwidth,
  legend to name = ledgendPosition,
  legend columns=6,
  scaled ticks = false%,
  %every axis legend/.code={\let\addlegendentry\relax} 
  ]
  \addplot[color=yellow,mark=o] table[x=time, y=PIVOT*]{data3/modularity-clustering.data};    \addlegendentry{PIVOT-BOEM} 
  \anytimeplot{modularity-clustering}
  \end{semilogxaxis}
  \end{tikzpicture}
  \caption{modularity clustering}\label{fig:at:modularity}
  \end{subfigure}
 %%%%%%%%%%%%%%%%%%%%%%%%%%%%%%%%%%%%%%%%%%%%%%%%%%%%%%%%%%%%%%%%%%%%%%%%%%%%%%%%%%%%%%%%%%%%%

  \begin{center}
  \hypersetup{linkcolor = black}
  \ref{ledgendPosition}
  \hypersetup{linkcolor = red}
  \end{center}

\end{figure*}
%%%%%%%%%%%%%%%%%%%%%%%%%%%%%%%%%%%%%%%%%%%%%%%%%%%%%%%%%%%%%%%%%%%%%%%%%%%%%%%%%%%%%%%%%%%%%

\newpage
\section{Anytime Plots (Zommed)}
%!TEX root = egpaper_for_review.tex
%%%%%%%%%%%%%%%%%%%%%%%%%%%%%%%%%%%%%%%%%%%%%%%%%%%%%%%%%%%%%%%%%%%%%%%%%%%%%%%%%%%%%%%%%%%%
\pgfplotsset{every axis legend/.append style={
at={(1.0,1.2)},
anchor=north east}} 
\begin{figure*}[h]
  \begin{subfigure}[b]{0.33\textwidth}
  \centering
  \begin{tikzpicture}
  \begin{semilogxaxis}[  mark size=1pt,
  restrict y to domain=0:4520,
  xlabel = {runtime},
  xmin = 0,
  xmax = 4100,
  width = 1.0\columnwidth,
  scaled ticks = false,
  every axis legend/.code={\let\addlegendentry\relax} 
  ]
  \addplot[color=yellow,mark=o] table[x=time, y=PIVOT*]{data3/image-seg.data};    \addlegendentry{PIVOT-BOEM}  
  \anytimeplot{image-seg}
  \end{semilogxaxis}
  \end{tikzpicture}
  
  \caption{image-seg}\label{fig:at:image-seg2}
  \end{subfigure}
  %%%%%%%%%%%%%%%%%%%%%%%%%%%%%%%%%%%%%%%%%%%%%%%%%%%%%%%%%%%%%%%%%%%%%%%%% 
  \begin{subfigure}[b]{0.33\textwidth}
    \centering
    \begin{tikzpicture}
    \begin{semilogxaxis}[  mark size=1pt,
    restrict y to domain=-6000:-4550,
    xlabel = {runtime},
    xmin = 0,
    xmax = 100,
    width = 1.0\columnwidth,
    scaled ticks = false,
    every axis legend/.code={\let\addlegendentry\relax} 
    ] 
    \addplot[color=yellow,mark=o] table[x=time, y=PIVOT*]{data3/knott-3d-150.data};          \addlegendentry{PIVOT-BOEM}
    \anytimeplot{knott-3d-150}
  
    \end{semilogxaxis}
    \end{tikzpicture}
    \caption{knott-3d-150}\label{fig:at:knott-1502}
  \end{subfigure}
  %%%%%%%%%%%%%%%%%%%%%%%%%%%%%%%%%%%%%%%%%%%%%%%%%%%%%%%%%%%%%%%%%%%%%%%%%%%%%%%%%%%%%%%%%%%%%
  \begin{subfigure}[b]{0.33\textwidth}
  \centering
  \begin{tikzpicture}
  \begin{semilogxaxis}[  mark size=1pt,
  restrict y to domain=-40000:-27000,
  xlabel = {runtime},
  xmin = 0,
  xmax = 4100,
  width = 1.0\columnwidth,
  scaled ticks = false,
  %legend to name = ledgendPosition,
  legend columns=6,
  every axis legend/.code={\let\addlegendentry\relax} 
  ]  
  \anytimeplot{knott-3d-300}
  \end{semilogxaxis}
  \end{tikzpicture}
  \caption{knott-3d-300}\label{fig:at:knott-3002}
  \end{subfigure}
  \newline
  %%%%%%%%%%%%%%%%%%%%%%%%%%%%%%%%%%%%%%%%%%%%%%%%%%%%%%%%%%%%%%%%%%%%%%%%%%%%%%%%%%%%%%%%%%%%%
  \begin{subfigure}[b]{0.33\textwidth}
  \centering
  \begin{tikzpicture}
  \begin{semilogxaxis}[  mark size=1pt,
  restrict y to domain=-80000:-75000,
  xlabel = {runtime},
  xmin = 0,
  xmax = 4100,
  width = 1.0\columnwidth,
  scaled ticks = false,
  every axis legend/.code={\let\addlegendentry\relax} 
  ] 
  \anytimeplot{knott-3d-450} 
  \end{semilogxaxis}
  \end{tikzpicture}
  \caption{knott-3d-450}\label{fig:at:knott-4502}
  \end{subfigure}
  %%%%%%%%%%%%%%%%%%%%%%%%%%%%%%%%%%%%%%%%%%%%%%%%%%%%%%%%%%%%%%%%%%%%%%%%%%%%%%%%%%%%%%%%%%%%%
  \begin{subfigure}[b]{0.33\textwidth}
  \centering
  \begin{tikzpicture}
  \begin{semilogxaxis}[  mark size=1pt,
  restrict y to domain=-10000000:-130000,
  xlabel = {runtime},
  xmin = 0,
  xmax = 4100,
  width = 1.0\columnwidth,
  scaled ticks = false,
  every axis legend/.code={\let\addlegendentry\relax} 
  ]
  \anytimeplot{knott-3d-550} 
  \end{semilogxaxis}
  \end{tikzpicture}
  \caption{knott-3d-550}\label{fig:at:knott-5502}
  \end{subfigure}
  %%%%%%%%%%%%%%%%%%%%%%%%%%%%%%%%%%%%%%%%%%%%%%%%%%%%%%%%%%%%%%%%%%%%%%%%%%%%%%%%%%%%%%%%%%%%%\begin{subfigure}[H]
  \begin{subfigure}[b]{0.33\textwidth}
  \centering
  \begin{tikzpicture}
  \begin{semilogxaxis}[  mark size=1pt,
  %restrict y to domain=-10000000:-100000,
  restrict y to domain=-10000000:850000,
  xlabel = {runtime},
  xmin = 0,
  xmax = 4100,
  width = 1.0\columnwidth,
  scaled ticks = false,
  every axis legend/.code={\let\addlegendentry\relax} 
  ] 
  \anytimeplot{seg-3d} 
  \end{semilogxaxis}
  \end{tikzpicture}
  \caption{seg-3d}\label{fig:at:seg3d2}
  \end{subfigure}
 %%%%%%%%%%%%%%%%%%%%%%%%%%%%%%%%%%%%%%%%%%%%%%%%%%%%%%%%%%%%%%%%%%%%%%%%%%%%%%%%%%%%%%%%%%%%%
  \begin{subfigure}[b]{0.33\textwidth}
  \centering
  \begin{tikzpicture}
  \begin{semilogxaxis}[  mark size=1pt,
  %restrict y to domain=-10000000:-100000,
  restrict y to domain=-10000000:80000,
  xlabel = {runtime},
  xmin = 0,
  xmax = 4100,
  width = 1.0\columnwidth,
  scaled ticks = false,
  every axis legend/.code={\let\addlegendentry\relax} 
  ]  
  \anytimeplot{socialnets} 
  \end{semilogxaxis}
  \end{tikzpicture}
  \caption{social nets}\label{fig:at:socialnets2}
  \end{subfigure}
 %%%%%%%%%%%%%%%%%%%%%%%%%%%%%%%%%%%%%%%%%%%%%%%%%%%%%%%%%%%%%%%%%%%%%%%%%%%%%%%%%%%%%%%%%%%%%
  \begin{subfigure}[b]{0.33\textwidth}
  \centering
  \begin{tikzpicture}
  \begin{semilogxaxis}[  mark size=1pt,
  %restrict y to domain=-10000000:-100000,
  restrict y to domain=-10000000:4000,
  xlabel = {runtime},
  xmin = 0,
  xmax = 4100,
  width = 1.0\columnwidth,
  scaled ticks = false,
  every axis legend/.code={\let\addlegendentry\relax} 
  ] 
  \anytimeplot{normalizedsocialnets}
  \end{semilogxaxis}
  \end{tikzpicture}
  \caption{normalized social nets}\label{fig:at:nsocialnets2}
  \end{subfigure}
 %%%%%%%%%%%%%%%%%%%%%%%%%%%%%%%%%%%%%%%%%%%%%%%%%%%%%%%%%%%%%%%%%%%%%%%%%%%%%%%%%%%%%%%%%%%%%
  \begin{subfigure}[b]{0.33\textwidth}
  \centering
  \begin{tikzpicture}
  \begin{semilogxaxis}[  mark size=1pt,
  %restrict y to domain=-10000000:-100000,
  restrict y to domain=-10000000:100000,
  xlabel = {runtime},
  xmin = 0,
  xmax = 4100,
  width = 1.0\columnwidth,
  legend to name = ledgendPosition2,
  legend columns=6,
  scaled ticks = false%,
  %every axis legend/.code={\let\addlegendentry\relax} 
  ]
  \addplot[color=yellow,mark=o] table[x=time, y=PIVOT*]{data3/modularity-clustering.data};    \addlegendentry{PIVOT-BOEM} 
  \anytimeplot{modularity-clustering}
  \end{semilogxaxis}
  \end{tikzpicture}
  \caption{modularity clustering}\label{fig:at:modularity2}
  \end{subfigure}
 %%%%%%%%%%%%%%%%%%%%%%%%%%%%%%%%%%%%%%%%%%%%%%%%%%%%%%%%%%%%%%%%%%%%%%%%%%%%%%%%%%%%%%%%%%%%%

  \begin{center}
  \hypersetup{linkcolor = black}
  \ref{ledgendPosition2}
  \hypersetup{linkcolor = red}
  \end{center}

\end{figure*}
%%%%%%%%%%%%%%%%%%%%%%%%%%%%%%%%%%%%%%%%%%%%%%%%%%%%%%%%%%%%%%%%%%%%%%%%%%%%%%%%%%%%%%%%%%%%%


\newpage
\section{Anytime Tables (per Dataset)}
\begin{table}[H]
\tiny
\centering
\caption{image-seg (100 instances)}
\label{tab:smalltable-image-seg}
\begin{tabular}{lrrrrrr}
\toprule
           algorithm &       runtime     &         value &         bound &           mem &     best &      opt   \\ \midrule 
     ogm-CGC-planar* & $         0.28$ sec & $      4445.22$ & $      4136.83$ & $         0.02$ GB & $      28$ & $       0$ \\ 
ogm-mcfusion-HC-BASE* & $         0.12$ sec & $      5462.60$ & $-\infty$ & $         0.01$ GB & $       0$ & $       0$ \\ 
ogm-mcfusion-HC-BASE-CGCF* & $         0.12$ sec & $      5085.80$ & $-\infty$ & $         0.01$ GB & $       0$ & $       0$ \\ 
ogm-mcfusion-HC-CGC* & $         1.23$ sec & $      4444.56$ & $-\infty$ & $         0.01$ GB & $      40$ & $       0$ \\ 
ogm-mcfusion-HC-CGC-CGCF* & $         1.22$ sec & $      4444.75$ & $-\infty$ & $         0.02$ GB & $      40$ & $       0$ \\ 
 ogm-mcfusion-HC-MC* & $         5.12$ sec & $      4443.61$ & $-\infty$ & $         0.04$ GB & $      73$ & $       0$ \\ 
ogm-mcfusion-HC-MC-CGCF* & $         5.14$ sec & $      4443.61$ & $-\infty$ & $         0.05$ GB & $      73$ & $       0$ \\ 
ogm-mcfusion-WS-BASE* & $         0.05$ sec & $      6622.98$ & $-\infty$ & $         0.01$ GB & $       0$ & $       0$ \\ 
ogm-mcfusion-WS-BASE-CGCF* & $         1.77$ sec & $      4460.12$ & $-\infty$ & $         0.01$ GB & $       3$ & $       0$ \\ 
ogm-mcfusion-WS-CGC* & $         1.44$ sec & $      4445.57$ & $-\infty$ & $         0.01$ GB & $      16$ & $       0$ \\ 
ogm-mcfusion-WS-CGC-CGCF* & $         1.75$ sec & $      4444.59$ & $-\infty$ & $         0.01$ GB & $      27$ & $       0$ \\ 
 ogm-mcfusion-WS-MC* & $         6.00$ sec & $      4444.58$ & $-\infty$ & $         0.03$ GB & $      22$ & $       0$ \\ 
ogm-mcfusion-WS-MC-CGCF* & $         6.31$ sec & $      4443.79$ & $-\infty$ & $         0.03$ GB & $      46$ & $       0$ \\ 
\bottomrule
\end{tabular}
\end{table}
\begin{table}[H]
\scriptsize
\centering
\caption{knott-3d-150 (8 instances)}
\label{tab:smalltable-knott-3d-150}
\begin{tabular}{lrrrrrr}
\toprule
           algorithm &       runtime     &         value &         bound &           mem &     best &      opt   \\ \midrule 
            ogm-CGC* & $         0.08$ sec & $     -4566.41$ & $     -4855.18$ & $         0.01$ GB & $       5$ & $       0$ \\ 
ogm-mcfusion-HC-BASE* & $         0.12$ sec & $         0.00$ & $-\infty$ & $         0.01$ GB & $       0$ & $       0$ \\ 
ogm-mcfusion-HC-BASE-CGCF* & $         0.16$ sec & $     -4160.88$ & $-\infty$ & $         0.01$ GB & $       0$ & $       0$ \\ 
ogm-mcfusion-HC-CGC* & $         0.36$ sec & $     -4559.96$ & $-\infty$ & $         0.01$ GB & $       6$ & $       0$ \\ 
ogm-mcfusion-HC-CGC-CGCF* & $         0.36$ sec & $     -4565.61$ & $-\infty$ & $         0.01$ GB & $       6$ & $       0$ \\ 
 ogm-mcfusion-HC-MC* & $         0.88$ sec & $     -4559.96$ & $-\infty$ & $         0.03$ GB & $       6$ & $       0$ \\ 
ogm-mcfusion-HC-MC-CGCF* & $         0.88$ sec & $     -4565.61$ & $-\infty$ & $         0.03$ GB & $       6$ & $       0$ \\ 
ogm-mcfusion-WS-BASE* & $         0.03$ sec & $         0.00$ & $-\infty$ & $         0.01$ GB & $       0$ & $       0$ \\ 
ogm-mcfusion-WS-BASE-CGCF* & $         0.12$ sec & $     -4565.95$ & $-\infty$ & $         0.01$ GB & $       5$ & $       0$ \\ 
ogm-mcfusion-WS-CGC* & $          NaN$ sec & $          NaN$ & $          NaN$ & $         0.01$ GB & $       4$ & $       0$ \\ 
ogm-mcfusion-WS-CGC-CGCF* & $         0.37$ sec & $     -4571.22$ & $-\infty$ & $         0.01$ GB & $       7$ & $       0$ \\ 
 ogm-mcfusion-WS-MC* & $         1.28$ sec & $     -4570.83$ & $-\infty$ & $         0.03$ GB & $       5$ & $       0$ \\ 
ogm-mcfusion-WS-MC-CGCF* & $         1.31$ sec & $     -4571.22$ & $-\infty$ & $         0.03$ GB & $       7$ & $       0$ \\ 
\bottomrule
\end{tabular}
\end{table}
\begin{table}[H]
\tiny
\centering
\caption{knott-3d-300 (8 instances)}
\label{tab:smalltable-knott-3d-300}
\begin{tabular}{lrrrrrr}
\toprule
           algorithm &       runtime     &         value &         bound &           mem &     best &      opt   \\ \midrule 
             ogm-ICM & $        84.37$ sec & $    -25196.51$ & $-\infty$ & $         0.01$ GB & $       0$ & $       0$ \\ 
              ogm-KL & $        13.16$ sec & $    -25556.93$ & $-\infty$ & $         0.01$ GB & $       0$ & $       0$ \\ 
            ogm-LF-1 & $        29.08$ sec & $    -25243.76$ & $-\infty$ & $         0.02$ GB & $       0$ & $       0$ \\ 
\cmidrule{1-1} 
              MCR-CC & $      3423.65$ sec & $    -26161.81$ & $    -27434.30$ & $         0.57$ GB & $       1$ & $       1$ \\ 
           MCR-CCFDB & $      1338.99$ sec & $    -27276.12$ & $    -27307.22$ & $         0.15$ GB & $       1$ & $       1$ \\ 
       MCR-CCFDB-OWC & $      1367.03$ sec & $    -27287.23$ & $    -27309.62$ & $         0.15$ GB & $       6$ & $       6$ \\ 
\cmidrule{1-1} 
     MCI-CCFDB-CCIFD & $      1261.99$ sec & $    -26826.57$ & $    -27308.19$ & $         0.37$ GB & $       6$ & $       6$ \\ 
             MCI-CCI & $       220.30$ sec & $    -27302.78$ & $    -27305.02$ & $         0.28$ GB & $       8$ & $       7$ \\ 
           MCI-CCIFD & $       104.55$ sec & $    -27302.78$ & $    -27302.78$ & $         0.16$ GB & $       8$ & $       8$ \\ 
\cmidrule{1-1} 
            ogm-CGC* & $         4.53$ sec & $    -27251.42$ & $    -28901.58$ & $         0.02$ GB & $       0$ & $       0$ \\ 
ogm-mcfusion-HC-BASE* & $         1.19$ sec & $         0.00$ & $-\infty$ & $         0.02$ GB & $       0$ & $       0$ \\ 
ogm-mcfusion-HC-BASE-CGCF* & $         2.38$ sec & $    -24707.81$ & $-\infty$ & $         0.04$ GB & $       0$ & $       0$ \\ 
ogm-mcfusion-HC-CGC* & $         5.60$ sec & $    -27282.58$ & $-\infty$ & $         0.03$ GB & $       2$ & $       0$ \\ 
ogm-mcfusion-HC-CGC-CGCF* & $         5.61$ sec & $    -27283.82$ & $-\infty$ & $         0.04$ GB & $       3$ & $       0$ \\ 
 ogm-mcfusion-HC-MC* & $         8.87$ sec & $    -27283.36$ & $-\infty$ & $         0.05$ GB & $       3$ & $       0$ \\ 
ogm-mcfusion-HC-MC-CGCF* & $         8.92$ sec & $    -27284.60$ & $-\infty$ & $         0.06$ GB & $       4$ & $       0$ \\ 
ogm-mcfusion-WS-BASE* & $         0.24$ sec & $         0.00$ & $-\infty$ & $         0.01$ GB & $       0$ & $       0$ \\ 
ogm-mcfusion-WS-BASE-CGCF* & $         4.91$ sec & $    -27253.30$ & $-\infty$ & $         0.03$ GB & $       0$ & $       0$ \\ 
ogm-mcfusion-WS-CGC* & $        22.99$ sec & $    -27273.39$ & $-\infty$ & $         0.01$ GB & $       1$ & $       0$ \\ 
ogm-mcfusion-WS-CGC-CGCF* & $        23.91$ sec & $    -27286.88$ & $-\infty$ & $         0.03$ GB & $       4$ & $       0$ \\ 
 ogm-mcfusion-WS-MC* & $        36.85$ sec & $    -27280.59$ & $-\infty$ & $         0.05$ GB & $       2$ & $       0$ \\ 
ogm-mcfusion-WS-MC-CGCF* & $        37.77$ sec & $    -27293.33$ & $-\infty$ & $         0.06$ GB & $       6$ & $       0$ \\ 
\bottomrule
\end{tabular}
\end{table}
\begin{table}[H]
\tiny
\centering
\caption{knott-3d-450 (8 instances)}
\label{tab:anytimetable-knott-3d-450}
\begin{tabular}{lrrrrrrr}
\toprule
           algorithm &                                   \multicolumn{6}{c}{value} & \multicolumn{1}{c}{time}   \\  
\cmidrule(lr){2-7}\cmidrule(lr){8-8}  
                     & \multicolumn{1}{c}{(1 sec)} & \multicolumn{1}{c}{(10 sec)} & \multicolumn{1}{c}{(60 sec)} & \multicolumn{1}{c}{(600 sec)} & \multicolumn{1}{c}{(1800 sec)} & \multicolumn{1}{c}{(end)} & \multicolumn{1}{c}{(end)}   \\ \midrule 
                 CGC & $         0.00$ & $         0.00$ & $    -77155.45$ & $    -78234.10$ & $    -78234.10$ & $    -78234.10$ & $       105.19$ sec   \\ 
                  HC & $    -67700.01$ & $    -67700.01$ & $    -67700.01$ & $    -67700.01$ & $    -67700.01$ & $    -67700.01$ & $         0.33$ sec   \\ 
              HC-CGC & $    -74452.50$ & $    -74453.05$ & $    -74453.05$ & $    -74453.05$ & $    -74453.05$ & $    -74453.05$ & $         0.71$ sec   \\ 
             ogm-ICM & $         0.00$ & $         0.00$ & $         0.00$ & $         0.00$ & $    -72464.54$ & $    -72464.54$ & $       876.97$ sec   \\ 
              ogm-KL & $     -4892.36$ & $     -4892.36$ & $    -72145.23$ & $    -73188.82$ & $    -73188.82$ & $    -73188.82$ & $       185.03$ sec   \\ 
            ogm-LF-1 & $         0.00$ & $         0.00$ & $         0.00$ & $    -72479.60$ & $    -72479.60$ & $    -72479.60$ & $       316.54$ sec   \\ 
          DYNCC-HC-B & $    -58581.91$ & $    -60435.27$ & $    -60536.35$ & $    -60536.35$ & $    -60536.35$ & $    -60536.35$ & $        12.48$ sec   \\ 
      DYNCC-HC-B-CGC & $    -58783.34$ & $    -62896.41$ & $    -67554.82$ & $    -67554.82$ & $    -67554.82$ & $    -67554.82$ & $        12.73$ sec   \\ 
        DYNCC-HC-CGC & $    -63567.96$ & $    -78226.60$ & $    -78415.84$ & $    -78418.51$ & $    -78418.51$ & $    -78418.51$ & $        65.96$ sec   \\ 
    DYNCC-HC-CGC-CGC & $    -64572.90$ & $    -78225.77$ & $    -78418.71$ & $    -78420.78$ & $    -78420.78$ & $    -78420.78$ & $        65.85$ sec   \\ 
         DYNCC-HC-MC & $    -54707.69$ & $    -77849.03$ & $    -78375.92$ & $    -78414.40$ & $    -78414.40$ & $    -78414.40$ & $       147.10$ sec   \\ 
     DYNCC-HC-MC-CGC & $    -54707.69$ & $    -77843.28$ & $    -78377.49$ & $    -78418.71$ & $    -78418.71$ & $    -78418.71$ & $       141.48$ sec   \\ 
          DYNCC-WS-B & $     -3492.29$ & $     -3492.29$ & $     -3492.29$ & $     -3492.29$ & $     -3492.29$ & $     -3492.29$ & $         2.60$ sec   \\ 
      DYNCC-WS-B-CGC & $         0.00$ & $         0.00$ & $         0.00$ & $    -78360.13$ & $    -78360.13$ & $    -78360.13$ & $       126.75$ sec   \\ 
        DYNCC-WS-CGC & $      -326.22$ & $     -9304.90$ & $    -78276.36$ & $    -78422.41$ & $    -78423.04$ & $    -78423.04$ & $       545.74$ sec   \\ 
    DYNCC-WS-CGC-CGC & $      -326.22$ & $    -18146.49$ & $    -78270.62$ & $    -78439.45$ & $    -78444.99$ & $    -78444.99$ & $       564.43$ sec   \\ 
         DYNCC-WS-MC & $      -326.22$ & $      -326.22$ & $    -77886.34$ & $    -78447.83$ & $    -78449.64$ & $    -78449.64$ & $       714.53$ sec   \\ 
     DYNCC-WS-MC-CGC & $      -326.22$ & $      -326.22$ & $    -77870.42$ & $    -78449.95$ & $    -78452.38$ & $    -78452.38$ & $       710.28$ sec   \\ 
\cmidrule{1-1} 
              MCR-CC & $     -4892.36$ & $     -4892.36$ & $     -4892.36$ & $    -24088.54$ & $    -59663.09$ & $    -69409.00$ & $      3601.00$ sec   \\ 
           MCR-CCFDB & $     -4892.36$ & $     -4892.36$ & $     -4892.36$ & $    -21100.37$ & $    -61124.13$ & $    -78071.77$ & $      3215.70$ sec   \\ 
       MCR-CCFDB-OWC & $     -4892.36$ & $     -4892.36$ & $     -4892.36$ & $    -22579.23$ & $    -62670.36$ & $    -77872.70$ & $      3216.71$ sec   \\ 
\cmidrule{1-1} 
     MCI-CCFDB-CCIFD & $     -4892.36$ & $     -4892.36$ & $     -4892.36$ & $    -25080.75$ & $    -65765.54$ & $    -77903.12$ & $      3071.35$ sec   \\ 
             MCI-CCI & $     -4892.36$ & $     -4892.36$ & $    -24623.32$ & $    -78261.91$ & $    -78378.83$ & $    -78378.83$ & $      1350.73$ sec   \\ 
           MCI-CCIFD & $     -4892.36$ & $     -4892.36$ & $    -19495.47$ & $    -78390.56$ & $    -78412.27$ & $    -78422.67$ & $      1169.63$ sec   \\ 
\bottomrule
\end{tabular}
\end{table}

\begin{table}[H]
\scriptsize
\centering
\caption{knott-3d-550 (8 instances)}
\label{tab:anytimetable-knott-3d-550}
\begin{tabular}{lrrrrrrrrr}
\toprule
           algorithm &                                   \multicolumn{8}{c}{value} & \multicolumn{1}{c}{time}   \\  
\cmidrule(lr){2-9}\cmidrule(lr){10-10}   
                     & \multicolumn{1}{c}{(0.5 sec)} & \multicolumn{1}{c}{(1 sec)} & \multicolumn{1}{c}{(10 sec)} & \multicolumn{1}{c}{(60 sec)} & \multicolumn{1}{c}{(300 sec)} & \multicolumn{1}{c}{(600 sec)} & \multicolumn{1}{c}{(1800 sec)} & \multicolumn{1}{c}{(end)} & \multicolumn{1}{c}{(end)}   \\ \midrule 
                 CGC & $     -2794.65$ & $     -2794.65$ & $     -2794.65$ & $    -14877.47$ & $   -134088.54$ & $   -136010.62$ & $   -136188.55$ & $   -136188.55$ & $       642.88$ sec   \\ 
                  HC & $   -119817.00$ & $   -119817.00$ & $   -119817.00$ & $   -119817.00$ & $   -119817.00$ & $   -119817.00$ & $   -119817.00$ & $   -119817.00$ & $         0.76$ sec   \\ 
              HC-CGC & $   -119817.00$ & $   -121635.46$ & $   -129458.56$ & $   -132967.67$ & $   -136034.77$ & $   -136208.48$ & $   -136216.88$ & $   -136216.88$ & $       500.80$ sec   \\ 
              ogm-KL & $     -8187.14$ & $     -8187.14$ & $     -8187.14$ & $     -8187.14$ & $   -125886.60$ & $   -127027.40$ & $   -127032.70$ & $   -127032.70$ & $       654.69$ sec   \\ 
    CC-Fusion-HC-CGC & $    -50653.90$ & $   -113407.57$ & $   -130789.76$ & $   -132194.34$ & $   -133017.00$ & $   -133017.00$ & $   -133017.00$ & $   -133017.00$ & $       240.05$ sec   \\ 
     CC-Fusion-HC-MC & $     -8187.14$ & $    -23928.53$ & $   -135401.38$ & $   -136391.50$ & $   -136457.29$ & $   -136457.29$ & $   -136457.29$ & $   -136457.29$ & $       327.30$ sec   \\ 
    CC-Fusion-WS-CGC & $     -8187.14$ & $    -48261.86$ & $   -124659.83$ & $   -128978.18$ & $   -130050.37$ & $   -130181.61$ & $   -130181.61$ & $   -130181.61$ & $       720.61$ sec   \\ 
     CC-Fusion-WS-MC & $     -8187.14$ & $     -8187.14$ & $   -101507.47$ & $   -135964.83$ & $   -136421.13$ & $   -136476.32$ & $   -136508.95$ & $   -136508.95$ & $      1653.98$ sec   \\ 
\cmidrule{1-1} 
           MCR-CCFDB & $     -8187.14$ & $     -8187.14$ & $     -8187.14$ & $     -8187.14$ & $     -8187.14$ & $    -12604.86$ & $    -36297.50$ & $    -36297.50$ & $      2009.80$ sec   \\ 
\cmidrule{1-1} 
           MCI-CCIFD & $     -8187.14$ & $     -8187.14$ & $     -8187.14$ & $     -8187.14$ & $    -51626.33$ & $    -92513.86$ & $   -136198.25$ & $   -136198.25$ & $      1530.04$ sec   \\ 
\bottomrule
\end{tabular}
\end{table}
\begin{table}[H]
\scriptsize
\centering
\caption{seg-3d (1 instances)}
\label{tab:anytimetable-seg-3d}
\begin{tabular}{lrrrrrrrrrrr}
\toprule
           algorithm &                                   \multicolumn{8}{c}{value} & \multicolumn{1}{c}{time}    & \multicolumn{1}{c}{VI}  & \multicolumn{1}{c}{RI} \\  
\cmidrule(lr){2-9}\cmidrule(lr){10-10} \cmidrule(lr){11-11} \cmidrule(lr){12-12}   
                     & \multicolumn{1}{c}{(0.5 sec)} & \multicolumn{1}{c}{(1 sec)} & \multicolumn{1}{c}{(10 sec)} & \multicolumn{1}{c}{(60 sec)} & \multicolumn{1}{c}{(300 sec)} & \multicolumn{1}{c}{(600 sec)} & \multicolumn{1}{c}{(1800 sec)} & \multicolumn{1}{c}{(end)} & \multicolumn{1}{c}{(end)}    & \multicolumn{1}{c}{(end)}   & \multicolumn{1}{c}{(end)}  \\ \midrule 
                 CGC & $   1281549.37$ & $   1281549.37$ & $   1281549.37$ & $   1281549.37$ & $   1281549.37$ & $   1281549.37$ & $    949757.11$ & $    882747.02$ & $      1995.76$ sec    & $       6.8908$  & $       0.6024$ \\ 
                  HC & $   1281549.37$ & $   1281549.37$ & $    820184.98$ & $    820184.98$ & $    820184.98$ & $    820184.98$ & $    820184.98$ & $    820184.98$ & $         2.03$ sec    & $       2.8395$  & $       0.9651$ \\ 
              HC-CGC & $   1281549.37$ & $   1281549.37$ & $    802315.77$ & $    800636.51$ & $    796844.12$ & $    795020.78$ & $    787846.06$ & $    787836.19$ & $      1802.81$ sec    & $       1.7603$  & $       0.9861$ \\ 
              ogm-KL & $   1258675.53$ & $   1258675.53$ & $   1258675.53$ & $   1258675.53$ & $   1258675.53$ & $   1258675.53$ & $   1258675.53$ & $    839974.05$ & $      4543.73$ sec    & $       7.1057$  & $       0.5849$ \\ 
    CC-Fusion-HC-CGC & $   1265334.24$ & $   1265334.24$ & $    861280.16$ & $    818529.13$ & $    809569.07$ & $    809569.07$ & $    804313.81$ & $    804313.81$ & $      1801.80$ sec    & $       2.1347$  & $       0.9775$ \\ 
     CC-Fusion-HC-MC & $   1265334.24$ & $   1265334.24$ & $    829064.68$ & $    779585.33$ & $    779014.36$ & $    778968.07$ & $    778958.97$ & $    778958.97$ & $      1801.20$ sec    & $       1.3347$  & $       0.9906$ \\ 
    CC-Fusion-WS-CGC & $   1265334.24$ & $   1265334.24$ & $   1265334.24$ & $    884727.49$ & $    824624.32$ & $    817956.21$ & $    815461.43$ & $    815461.43$ & $      1892.94$ sec    & $       3.3514$  & $       0.8895$ \\ 
     CC-Fusion-WS-MC & $   1265334.24$ & $   1265334.24$ & $   1265334.24$ & $   1265334.24$ & $    781378.89$ & $    779662.36$ & $    779056.84$ & $    779026.51$ & $      1808.10$ sec    & $       1.3334$  & $       0.9906$ \\ 
\cmidrule{1-1} 
           MCR-CCFDB & $   1253637.12$ & $   1253637.12$ & $   1253637.12$ & $   1253637.12$ & $   1253637.12$ & $   1253637.12$ & $   1253637.12$ & $   1253637.12$ & $      8291.80$ sec    & $       6.5058$  & $       0.0432$ \\ 
\cmidrule{1-1} 
           MCI-CCIFD & $   1253637.12$ & $   1253637.12$ & $   1253637.12$ & $   1253637.12$ & $   1165149.49$ & $   1165149.49$ & $    963034.43$ & $    963034.43$ & $      2358.10$ sec    & $       4.3319$  & $       0.5461$ \\ 
\bottomrule
\end{tabular}
\end{table}
\begin{table}[H]
\scriptsize
\centering
\caption{socialnets (2 instances)}
\label{tab:anytimetable-socialnets}
\begin{tabular}{lrrrrrrrrr}
\toprule
           algorithm &                                   \multicolumn{8}{c}{value} & \multicolumn{1}{c}{time}   \\  
\cmidrule(lr){2-9}\cmidrule(lr){10-10}   
                     & \multicolumn{1}{c}{(0.5 sec)} & \multicolumn{1}{c}{(1 sec)} & \multicolumn{1}{c}{(10 sec)} & \multicolumn{1}{c}{(60 sec)} & \multicolumn{1}{c}{(300 sec)} & \multicolumn{1}{c}{(600 sec)} & \multicolumn{1}{c}{(1800 sec)} & \multicolumn{1}{c}{(end)} & \multicolumn{1}{c}{(end)}   \\ \midrule 
                 CGC & $    119638.00$ & $    119638.00$ & $    119638.00$ & $    119638.00$ & $    119638.00$ & $     64244.50$ & $     60883.50$ & $     60882.50$ & $      3673.15$ sec   \\ 
                  HC & $    119638.00$ & $    119638.00$ & $    119638.00$ & $    119638.00$ & $    119638.00$ & $    119638.00$ & $    119638.00$ & $    119638.00$ & $        20.42$ sec   \\ 
              HC-CGC & $    119638.00$ & $    119638.00$ & $    119638.00$ & $    119309.50$ & $    119309.50$ & $     69550.00$ & $     60916.00$ & $     60915.00$ & $      4708.54$ sec   \\ 
              ogm-KL & $\infty$ & $\infty$ & $\infty$ & $\infty$ & $\infty$ & $\infty$ & $\infty$ & $          NaN$ & $          NaN$ sec   \\ 
    CC-Fusion-HC-CGC & $     94861.00$ & $     94861.00$ & $     93700.50$ & $     83766.00$ & $     72063.00$ & $     68472.00$ & $     65204.50$ & $     65202.00$ & $      1803.23$ sec   \\ 
     CC-Fusion-HC-MC & $     94861.00$ & $     94861.00$ & $     93700.50$ & $     83768.00$ & $     71898.50$ & $     68023.00$ & $     64625.00$ & $     64622.00$ & $      1804.59$ sec   \\ 
    CC-Fusion-WS-CGC & $     94861.00$ & $     94861.00$ & $     94861.00$ & $     94861.00$ & $     72480.00$ & $     66223.00$ & $     62596.00$ & $     62353.50$ & $      1946.90$ sec   \\ 
     CC-Fusion-WS-MC & $\infty$ & $\infty$ & $\infty$ & $\infty$ & $\infty$ & $\infty$ & $\infty$ & $          NaN$ & $          NaN$ sec   \\ 
\cmidrule{1-1} 
           MCR-CCFDB & $     94861.00$ & $     94861.00$ & $     94861.00$ & $     94861.00$ & $     94861.00$ & $     94861.00$ & $     94861.00$ & $     94861.00$ & $      3536.21$ sec   \\ 
\cmidrule{1-1} 
           MCI-CCIFD & $     94861.00$ & $     94861.00$ & $     94861.00$ & $     94861.00$ & $     94861.00$ & $     94861.00$ & $     94861.00$ & $     94861.00$ & $      2968.81$ sec   \\ 
\bottomrule
\end{tabular}
\end{table}
\begin{table}[H]
\scriptsize
\centering
\caption{normalizedsocialnets (2 instances)}
\label{tab:anytimetable-normalizedsocialnets}
\begin{tabular}{lrrrrrrrrr}
\toprule
           algorithm &                                   \multicolumn{8}{c}{value} & \multicolumn{1}{c}{time}   \\  
\cmidrule(lr){2-9}\cmidrule(lr){10-10}   
                     & \multicolumn{1}{c}{(0.5 sec)} & \multicolumn{1}{c}{(1 sec)} & \multicolumn{1}{c}{(10 sec)} & \multicolumn{1}{c}{(60 sec)} & \multicolumn{1}{c}{(300 sec)} & \multicolumn{1}{c}{(600 sec)} & \multicolumn{1}{c}{(1800 sec)} & \multicolumn{1}{c}{(end)} & \multicolumn{1}{c}{(end)}   \\ \midrule 
                 CGC & $      8304.07$ & $      8304.07$ & $      8304.07$ & $      8304.07$ & $      8304.07$ & $      8304.07$ & $      4129.02$ & $      2547.10$ & $      2771.35$ sec   \\ 
                  HC & $      8304.07$ & $      8304.07$ & $      7375.42$ & $      7375.42$ & $      7375.42$ & $      7375.42$ & $      7375.42$ & $      7375.42$ & $        10.17$ sec   \\ 
              HC-CGC & $\infty$ & $\infty$ & $\infty$ & $\infty$ & $\infty$ & $\infty$ & $\infty$ & $          NaN$ & $          NaN$ sec   \\ 
              ogm-KL & $\infty$ & $\infty$ & $\infty$ & $\infty$ & $\infty$ & $\infty$ & $\infty$ & $          NaN$ & $          NaN$ sec   \\ 
    CC-Fusion-HC-CGC & $      4132.22$ & $      4132.22$ & $      4058.67$ & $      3655.71$ & $      3207.35$ & $      3057.04$ & $      2910.77$ & $      2910.77$ & $      1804.26$ sec   \\ 
     CC-Fusion-HC-MC & $      4132.22$ & $      4132.22$ & $      4032.30$ & $      3643.70$ & $      3187.09$ & $      3033.03$ & $      2875.74$ & $      2875.65$ & $      1804.24$ sec   \\ 
    CC-Fusion-WS-CGC & $      4132.22$ & $      4132.22$ & $      4132.22$ & $      4132.22$ & $      4132.22$ & $      4045.33$ & $      3672.84$ & $      3392.00$ & $      2018.43$ sec   \\ 
     CC-Fusion-WS-MC & $\infty$ & $\infty$ & $\infty$ & $\infty$ & $\infty$ & $\infty$ & $\infty$ & $          NaN$ & $          NaN$ sec   \\ 
\cmidrule{1-1} 
           MCR-CCFDB & $      4132.22$ & $      4132.22$ & $      4132.22$ & $      4132.22$ & $      4132.22$ & $      4132.22$ & $      4132.22$ & $      4132.22$ & $      4320.10$ sec   \\ 
\cmidrule{1-1} 
           MCI-CCIFD & $      4132.22$ & $      4132.22$ & $      4132.22$ & $      4132.22$ & $      4132.22$ & $      4132.22$ & $      4132.22$ & $      4132.22$ & $      3087.45$ sec   \\ 
\bottomrule
\end{tabular}
\end{table}
\begin{tikzpicture}
  \begin{loglogaxis}[
    xlabel = {runtime},
    ylabel = {distance to optimum/best},
    xlabel style={name=xlabel},
    mark size=1pt,
    every axis/.append style={font=\tiny}, 
    width = 0.95\textwidth, 
    height= 0.95\textwidth, 
    legend style = { at={(xlabel.south)}, yshift=-1ex, anchor=north,legend cell align=left, font=\tiny},  
    legend columns = 4 
   ]
  \addplot[only marks, mark=*,red, line width=\thickline,opacity=0.8] table[x=Togm_CGC, y=Vogm_CGC]{\scatterplotpath modularity-clustering.data}; 
  \addlegendentry{CGC};   
  \addplot[only marks, mark=*,green, line width=\thickline,opacity=0.8] table[x=Togm_icm, y=Vogm_icm]{\scatterplotpath modularity-clustering.data}; 
  \addlegendentry{ogm-ICM};   
  \addplot[only marks, mark=*,orange!50, line width=\thickline,opacity=0.8] table[x=Togm_kl, y=Vogm_kl]{\scatterplotpath modularity-clustering.data}; 
  \addlegendentry{ogm-KL};   
  \addplot[only marks, mark=*,cyan, line width=\thickline,opacity=0.8] table[x=Togm_lf1, y=Vogm_lf1]{\scatterplotpath modularity-clustering.data}; 
  \addlegendentry{ogm-LF-1};   
  \addplot[only marks, mark=*,magenta, line width=\thickline,opacity=0.8] table[x=Togm_mcfusion_HC_BASE, y=Vogm_mcfusion_HC_BASE]{\scatterplotpath modularity-clustering.data}; 
  \addlegendentry{DYNCC-HC-B};   
  \addplot[only marks, mark=*,black, line width=\thickline,opacity=0.8] table[x=Togm_mcfusion_HC_BASE_CGCF, y=Vogm_mcfusion_HC_BASE_CGCF]{\scatterplotpath modularity-clustering.data}; 
  \addlegendentry{DYNCC-HC-B-CGC};   
  \addplot[only marks, mark=*,brown, line width=\thickline,opacity=0.8] table[x=Togm_mcfusion_HC_CGC, y=Vogm_mcfusion_HC_CGC]{\scatterplotpath modularity-clustering.data}; 
  \addlegendentry{DYNCC-HC-CGC};   
  \addplot[only marks, mark=*,yellow!70!black, line width=\thickline,opacity=0.8] table[x=Togm_mcfusion_HC_CGC_CGCF, y=Vogm_mcfusion_HC_CGC_CGCF]{\scatterplotpath modularity-clustering.data}; 
  \addlegendentry{DYNCC-HC-CGC-CGC};   
  \addplot[only marks, mark=*,blue!40, line width=\thickline,opacity=0.8] table[x=Togm_mcfusion_HC_MC, y=Vogm_mcfusion_HC_MC]{\scatterplotpath modularity-clustering.data}; 
  \addlegendentry{DYNCC-HC-MC};   
  \addplot[only marks, mark=*,blue, line width=\thickline,opacity=0.8] table[x=Togm_mcfusion_HC_MC_CGCF, y=Vogm_mcfusion_HC_MC_CGCF]{\scatterplotpath modularity-clustering.data}; 
  \addlegendentry{DYNCC-HC-MC-CGC};   
  \addplot[only marks, mark=x,red, line width=\thickline,opacity=0.8] table[x=Togm_mcfusion_WS_BASE, y=Vogm_mcfusion_WS_BASE]{\scatterplotpath modularity-clustering.data}; 
  \addlegendentry{DYNCC-WS-B};   
  \addplot[only marks, mark=x,green, line width=\thickline,opacity=0.8] table[x=Togm_mcfusion_WS_BASE_CGCF, y=Vogm_mcfusion_WS_BASE_CGCF]{\scatterplotpath modularity-clustering.data}; 
  \addlegendentry{DYNCC-WS-B-CGC};   
  \addplot[only marks, mark=x,orange!50, line width=\thickline,opacity=0.8] table[x=Togm_mcfusion_WS_CGC, y=Vogm_mcfusion_WS_CGC]{\scatterplotpath modularity-clustering.data}; 
  \addlegendentry{DYNCC-WS-CGC};   
  \addplot[only marks, mark=x,cyan, line width=\thickline,opacity=0.8] table[x=Togm_mcfusion_WS_CGC_CGCF, y=Vogm_mcfusion_WS_CGC_CGCF]{\scatterplotpath modularity-clustering.data}; 
  \addlegendentry{DYNCC-WS-CGC-CGC};   
  \addplot[only marks, mark=x,magenta, line width=\thickline,opacity=0.8] table[x=Togm_mcfusion_WS_MC, y=Vogm_mcfusion_WS_MC]{\scatterplotpath modularity-clustering.data}; 
  \addlegendentry{DYNCC-WS-MC};   
  \addplot[only marks, mark=x,black, line width=\thickline,opacity=0.8] table[x=Togm_mcfusion_WS_MC_CGCF, y=Vogm_mcfusion_WS_MC_CGCF]{\scatterplotpath modularity-clustering.data}; 
  \addlegendentry{DYNCC-WS-MC-CGC};   
  \addplot[only marks, mark=x,brown, line width=\thickline,opacity=0.8] table[x=TMC_CC, y=VMC_CC]{\scatterplotpath modularity-clustering.data}; 
  \addlegendentry{MCR-CC};   
  \addplot[only marks, mark=x,yellow!70!black, line width=\thickline,opacity=0.8] table[x=TMC_CCFDB, y=VMC_CCFDB]{\scatterplotpath modularity-clustering.data}; 
  \addlegendentry{MCR-CCFDB};   
  \addplot[only marks, mark=x,blue!40, line width=\thickline,opacity=0.8] table[x=TMC_CCFDB_OWC, y=VMC_CCFDB_OWC]{\scatterplotpath modularity-clustering.data}; 
  \addlegendentry{MCR-CCFDB-OWC};   
  \addplot[only marks, mark=x,blue, line width=\thickline,opacity=0.8] table[x=TMC_CCFDB_CCIFD, y=VMC_CCFDB_CCIFD]{\scatterplotpath modularity-clustering.data}; 
  \addlegendentry{MCI-CCFDB-CCIFD};   
  \addplot[only marks, mark=square*,red, line width=\thickline,opacity=0.8] table[x=TMC_CCI, y=VMC_CCI]{\scatterplotpath modularity-clustering.data}; 
  \addlegendentry{MCI-CCI};   
  \addplot[only marks, mark=square*,green, line width=\thickline,opacity=0.8] table[x=TMC_CCIFD, y=VMC_CCIFD]{\scatterplotpath modularity-clustering.data}; 
  \addlegendentry{MCI-CCIFD};   
  \end{loglogaxis} 
\end{tikzpicture} 


\newpage
\section{Anytime Tables (per Instance)}
\subsection{image-seg}
\begin{table}[H]
\scriptsize
\centering
\caption{image-seg (101085.bmp)}
\label{tab:anytimetable-image-seg-101085.bmp}
\begin{tabular}{lrrrrrrrrrrr}
\toprule
           algorithm &                                   \multicolumn{8}{c}{value} & \multicolumn{1}{c}{time}    & \multicolumn{1}{c}{VI}  & \multicolumn{1}{c}{RI} \\  
\cmidrule(lr){2-9}\cmidrule(lr){10-10} \cmidrule(lr){11-11} \cmidrule(lr){12-12}   
                     & \multicolumn{1}{c}{(0.5 sec)} & \multicolumn{1}{c}{(1 sec)} & \multicolumn{1}{c}{(10 sec)} & \multicolumn{1}{c}{(60 sec)} & \multicolumn{1}{c}{(300 sec)} & \multicolumn{1}{c}{(600 sec)} & \multicolumn{1}{c}{(1800 sec)} & \multicolumn{1}{c}{(end)} & \multicolumn{1}{c}{(end)}    & \multicolumn{1}{c}{(end)}   & \multicolumn{1}{c}{(end)}  \\ \midrule 
          PIVIT-BOEM & $\infty$ & $\infty$ & $\infty$ & $      7469.86$ & $      7469.86$ & $      7469.86$ & $      7469.86$ & $      7469.86$ & $        43.12$ sec    & $       4.9512$  & $       0.8794$ \\ 
                 CGC & $      5226.49$ & $      5226.49$ & $      5226.49$ & $      5226.49$ & $      5226.49$ & $      5226.49$ & $      5226.49$ & $      5226.49$ & $         0.13$ sec    & $       2.2471$  & $       0.9237$ \\ 
                  HC & $      5746.53$ & $      5746.53$ & $      5746.53$ & $      5746.53$ & $      5746.53$ & $      5746.53$ & $      5746.53$ & $      5746.53$ & $         0.01$ sec    & $       2.3440$  & $       0.9148$ \\ 
              HC-CGC & $      5223.37$ & $      5223.37$ & $      5223.37$ & $      5223.37$ & $      5223.37$ & $      5223.37$ & $      5223.37$ & $      5223.37$ & $         0.09$ sec    & $       2.3197$  & $       0.9157$ \\ 
              ogm-KL & $      5503.36$ & $      5503.35$ & $      5503.35$ & $      5503.35$ & $      5503.35$ & $      5503.35$ & $      5503.35$ & $      5503.35$ & $         0.83$ sec    & $       3.2597$  & $       0.7314$ \\ 
    CC-Fusion-HC-CGC & $      5214.32$ & $      5212.18$ & $      5212.03$ & $      5212.03$ & $      5212.03$ & $      5212.03$ & $      5212.03$ & $      5212.03$ & $         1.60$ sec    & $       2.2368$  & $       0.9188$ \\ 
     CC-Fusion-HC-MC & $      5210.88$ & $      5208.44$ & $      5207.50$ & $      5207.50$ & $      5207.50$ & $      5207.50$ & $      5207.50$ & $      5207.50$ & $         3.51$ sec    & $       2.3488$  & $       0.9146$ \\ 
    CC-Fusion-WS-CGC & $      5232.73$ & $      5232.73$ & $      5232.73$ & $      5232.73$ & $      5232.73$ & $      5232.73$ & $      5232.73$ & $      5232.73$ & $         0.62$ sec    & $       2.2764$  & $       0.9176$ \\ 
     CC-Fusion-WS-MC & $      5248.12$ & $      5212.81$ & $      5209.17$ & $      5209.17$ & $      5209.17$ & $      5209.17$ & $      5209.17$ & $      5209.17$ & $         3.60$ sec    & $       2.3369$  & $       0.9146$ \\ 
\cmidrule{1-1} 
           MCR-CCFDB & $      5207.50$ & $      5207.50$ & $      5207.50$ & $      5207.50$ & $      5207.50$ & $      5207.50$ & $      5207.50$ & $      5207.50$ & $         0.15$ sec    & $       2.3488$  & $       0.9146$ \\ 
\cmidrule{1-1} 
           MCI-CCIFD & $      5243.96$ & $      5207.50$ & $      5207.50$ & $      5207.50$ & $      5207.50$ & $      5207.50$ & $      5207.50$ & $      5207.50$ & $         0.90$ sec    & $       2.3488$  & $       0.9146$ \\ 
\bottomrule
\end{tabular}
\end{table}

\begin{table}[H]
\scriptsize
\centering
\caption{image-seg (101087.bmp)}
\label{tab:anytimetable-image-seg-101087.bmp}
\begin{tabular}{lrrrrrrrrrrr}
\toprule
           algorithm &                                   \multicolumn{8}{c}{value} & \multicolumn{1}{c}{time}    & \multicolumn{1}{c}{VI}  & \multicolumn{1}{c}{RI} \\  
\cmidrule(lr){2-9}\cmidrule(lr){10-10} \cmidrule(lr){11-11} \cmidrule(lr){12-12}   
                     & \multicolumn{1}{c}{(0.5 sec)} & \multicolumn{1}{c}{(1 sec)} & \multicolumn{1}{c}{(10 sec)} & \multicolumn{1}{c}{(60 sec)} & \multicolumn{1}{c}{(300 sec)} & \multicolumn{1}{c}{(600 sec)} & \multicolumn{1}{c}{(1800 sec)} & \multicolumn{1}{c}{(end)} & \multicolumn{1}{c}{(end)}    & \multicolumn{1}{c}{(end)}   & \multicolumn{1}{c}{(end)}  \\ \midrule 
          PIVIT-BOEM & $\infty$ & $\infty$ & $      3902.82$ & $      3902.82$ & $      3902.82$ & $      3902.82$ & $      3902.82$ & $      3902.82$ & $         5.77$ sec    & $       2.5928$  & $       0.9321$ \\ 
                 CGC & $      2803.03$ & $      2803.03$ & $      2803.03$ & $      2803.03$ & $      2803.03$ & $      2803.03$ & $      2803.03$ & $      2803.03$ & $         0.06$ sec    & $       1.5848$  & $       0.9339$ \\ 
                  HC & $      2985.75$ & $      2985.75$ & $      2985.75$ & $      2985.75$ & $      2985.75$ & $      2985.75$ & $      2985.75$ & $      2985.75$ & $         0.00$ sec    & $       1.6375$  & $       0.9286$ \\ 
              HC-CGC & $      2803.52$ & $      2803.52$ & $      2803.52$ & $      2803.52$ & $      2803.52$ & $      2803.52$ & $      2803.52$ & $      2803.52$ & $         0.08$ sec    & $       1.5771$  & $       0.9339$ \\ 
              ogm-KL & $      2869.47$ & $      2869.47$ & $      2869.47$ & $      2869.47$ & $      2869.47$ & $      2869.47$ & $      2869.47$ & $      2869.47$ & $         0.12$ sec    & $       2.7070$  & $       0.7349$ \\ 
    CC-Fusion-HC-CGC & $      2790.01$ & $      2790.01$ & $      2790.01$ & $      2790.01$ & $      2790.01$ & $      2790.01$ & $      2790.01$ & $      2790.01$ & $         0.32$ sec    & $       1.5245$  & $       0.9404$ \\ 
     CC-Fusion-HC-MC & $      2789.90$ & $      2789.90$ & $      2789.90$ & $      2789.90$ & $      2789.90$ & $      2789.90$ & $      2789.90$ & $      2789.90$ & $         1.70$ sec    & $       1.5220$  & $       0.9404$ \\ 
    CC-Fusion-WS-CGC & $      2793.22$ & $      2793.22$ & $      2793.22$ & $      2793.22$ & $      2793.22$ & $      2793.22$ & $      2793.22$ & $      2793.22$ & $         0.45$ sec    & $       1.5463$  & $       0.9417$ \\ 
     CC-Fusion-WS-MC & $      2793.54$ & $      2789.90$ & $      2789.90$ & $      2789.90$ & $      2789.90$ & $      2789.90$ & $      2789.90$ & $      2789.90$ & $         2.43$ sec    & $       1.5220$  & $       0.9404$ \\ 
\cmidrule{1-1} 
           MCR-CCFDB & $      2790.78$ & $      2790.78$ & $      2790.78$ & $      2790.78$ & $      2790.78$ & $      2790.78$ & $      2790.78$ & $      2790.78$ & $         0.06$ sec    & $       1.5224$  & $       0.9404$ \\ 
\cmidrule{1-1} 
           MCI-CCIFD & $      2791.61$ & $      2789.90$ & $      2789.90$ & $      2789.90$ & $      2789.90$ & $      2789.90$ & $      2789.90$ & $      2789.90$ & $         0.66$ sec    & $       1.5220$  & $       0.9404$ \\ 
\bottomrule
\end{tabular}
\end{table}

\begin{table}[H]
\scriptsize
\centering
\caption{image-seg (102061.bmp)}
\label{tab:anytimetable-image-seg-102061.bmp}
\begin{tabular}{lrrrrrrrrrrr}
\toprule
           algorithm &                                   \multicolumn{8}{c}{value} & \multicolumn{1}{c}{time}    & \multicolumn{1}{c}{VI}  & \multicolumn{1}{c}{RI} \\  
\cmidrule(lr){2-9}\cmidrule(lr){10-10} \cmidrule(lr){11-11} \cmidrule(lr){12-12}   
                     & \multicolumn{1}{c}{(0.5 sec)} & \multicolumn{1}{c}{(1 sec)} & \multicolumn{1}{c}{(10 sec)} & \multicolumn{1}{c}{(60 sec)} & \multicolumn{1}{c}{(300 sec)} & \multicolumn{1}{c}{(600 sec)} & \multicolumn{1}{c}{(1800 sec)} & \multicolumn{1}{c}{(end)} & \multicolumn{1}{c}{(end)}    & \multicolumn{1}{c}{(end)}   & \multicolumn{1}{c}{(end)}  \\ \midrule 
          PIVIT-BOEM & $\infty$ & $\infty$ & $      3739.06$ & $      3739.06$ & $      3739.06$ & $      3739.06$ & $      3739.06$ & $      3739.06$ & $         6.08$ sec    & $       3.8093$  & $       0.7968$ \\ 
                 CGC & $      2955.99$ & $      2955.99$ & $      2955.99$ & $      2955.99$ & $      2955.99$ & $      2955.99$ & $      2955.99$ & $      2955.99$ & $         0.15$ sec    & $       2.2416$  & $       0.8264$ \\ 
                  HC & $      3168.30$ & $      3168.30$ & $      3168.30$ & $      3168.30$ & $      3168.30$ & $      3168.30$ & $      3168.30$ & $      3168.30$ & $         0.00$ sec    & $       2.1557$  & $       0.8332$ \\ 
              HC-CGC & $      2954.52$ & $      2954.52$ & $      2954.52$ & $      2954.52$ & $      2954.52$ & $      2954.52$ & $      2954.52$ & $      2954.52$ & $         0.12$ sec    & $       2.1844$  & $       0.8283$ \\ 
              ogm-KL & $      3077.77$ & $      3077.77$ & $      3077.77$ & $      3077.77$ & $      3077.77$ & $      3077.77$ & $      3077.77$ & $      3077.77$ & $         0.49$ sec    & $       2.4854$  & $       0.6706$ \\ 
    CC-Fusion-HC-CGC & $      2953.93$ & $      2953.93$ & $      2953.93$ & $      2953.93$ & $      2953.93$ & $      2953.93$ & $      2953.93$ & $      2953.93$ & $         0.51$ sec    & $       2.2270$  & $       0.8306$ \\ 
     CC-Fusion-HC-MC & $      2945.33$ & $      2944.79$ & $      2944.79$ & $      2944.79$ & $      2944.79$ & $      2944.79$ & $      2944.79$ & $      2944.79$ & $         3.19$ sec    & $       2.2577$  & $       0.8274$ \\ 
    CC-Fusion-WS-CGC & $      2958.05$ & $      2958.05$ & $      2958.05$ & $      2958.05$ & $      2958.05$ & $      2958.05$ & $      2958.05$ & $      2958.05$ & $         0.45$ sec    & $       2.1394$  & $       0.8442$ \\ 
     CC-Fusion-WS-MC & $      2950.94$ & $      2945.07$ & $      2943.77$ & $      2943.77$ & $      2943.77$ & $      2943.77$ & $      2943.77$ & $      2943.77$ & $         7.03$ sec    & $       2.2150$  & $       0.8320$ \\ 
\cmidrule{1-1} 
           MCR-CCFDB & $      2947.74$ & $      2947.74$ & $      2947.74$ & $      2947.74$ & $      2947.74$ & $      2947.74$ & $      2947.74$ & $      2947.74$ & $         0.11$ sec    & $       2.2232$  & $       0.8320$ \\ 
\cmidrule{1-1} 
           MCI-CCIFD & $      2976.60$ & $      2976.05$ & $      2943.77$ & $      2943.77$ & $      2943.77$ & $      2943.77$ & $      2943.77$ & $      2943.77$ & $         1.11$ sec    & $       2.2150$  & $       0.8320$ \\ 
\bottomrule
\end{tabular}
\end{table}

\begin{table}[H]
\scriptsize
\centering
\caption{image-seg (103070.bmp)}
\label{tab:anytimetable-image-seg-103070.bmp}
\begin{tabular}{lrrrrrrrrrrr}
\toprule
           algorithm &                                   \multicolumn{8}{c}{value} & \multicolumn{1}{c}{time}    & \multicolumn{1}{c}{VI}  & \multicolumn{1}{c}{RI} \\  
\cmidrule(lr){2-9}\cmidrule(lr){10-10} \cmidrule(lr){11-11} \cmidrule(lr){12-12}   
                     & \multicolumn{1}{c}{(0.5 sec)} & \multicolumn{1}{c}{(1 sec)} & \multicolumn{1}{c}{(10 sec)} & \multicolumn{1}{c}{(60 sec)} & \multicolumn{1}{c}{(300 sec)} & \multicolumn{1}{c}{(600 sec)} & \multicolumn{1}{c}{(1800 sec)} & \multicolumn{1}{c}{(end)} & \multicolumn{1}{c}{(end)}    & \multicolumn{1}{c}{(end)}   & \multicolumn{1}{c}{(end)}  \\ \midrule 
          PIVIT-BOEM & $\infty$ & $\infty$ & $\infty$ & $      5862.71$ & $      5862.71$ & $      5862.71$ & $      5862.71$ & $      5862.71$ & $        19.10$ sec    & $       5.2157$  & $       0.8095$ \\ 
                 CGC & $      4380.72$ & $      4290.48$ & $      4239.74$ & $      4239.74$ & $      4239.74$ & $      4239.74$ & $      4239.74$ & $      4239.74$ & $         2.12$ sec    & $       3.0829$  & $       0.6658$ \\ 
                  HC & $      4691.54$ & $      4691.54$ & $      4691.54$ & $      4691.54$ & $      4691.54$ & $      4691.54$ & $      4691.54$ & $      4691.54$ & $         0.00$ sec    & $       3.4197$  & $       0.7221$ \\ 
              HC-CGC & $      4233.18$ & $      4228.66$ & $      4228.66$ & $      4228.66$ & $      4228.66$ & $      4228.66$ & $      4228.66$ & $      4228.66$ & $         1.16$ sec    & $       3.1020$  & $       0.7567$ \\ 
              ogm-KL & $      4446.22$ & $      4445.75$ & $      4445.75$ & $      4445.75$ & $      4445.75$ & $      4445.75$ & $      4445.75$ & $      4445.75$ & $         0.84$ sec    & $       3.5892$  & $       0.4534$ \\ 
    CC-Fusion-HC-CGC & $      4253.73$ & $      4253.73$ & $      4253.73$ & $      4253.73$ & $      4253.73$ & $      4253.73$ & $      4253.73$ & $      4253.73$ & $         0.78$ sec    & $       3.2242$  & $       0.7238$ \\ 
     CC-Fusion-HC-MC & $      4228.41$ & $      4212.32$ & $      4199.58$ & $      4199.58$ & $      4199.58$ & $      4199.58$ & $      4199.58$ & $      4199.58$ & $         5.49$ sec    & $       2.8686$  & $       0.8123$ \\ 
    CC-Fusion-WS-CGC & $      4277.12$ & $      4275.56$ & $      4245.11$ & $      4245.11$ & $      4245.11$ & $      4245.11$ & $      4245.11$ & $      4245.11$ & $         1.84$ sec    & $       2.9942$  & $       0.7702$ \\ 
     CC-Fusion-WS-MC & $      4410.36$ & $      4207.24$ & $      4199.58$ & $      4199.58$ & $      4199.58$ & $      4199.58$ & $      4199.58$ & $      4199.58$ & $         7.28$ sec    & $       2.8686$  & $       0.8123$ \\ 
\cmidrule{1-1} 
           MCR-CCFDB & $      4361.98$ & $      4199.38$ & $      4199.38$ & $      4199.38$ & $      4199.38$ & $      4199.38$ & $      4199.38$ & $      4199.38$ & $         0.59$ sec    & $       2.8892$  & $       0.8118$ \\ 
\cmidrule{1-1} 
           MCI-CCIFD & $      4356.27$ & $      4212.04$ & $      4199.38$ & $      4199.38$ & $      4199.38$ & $      4199.38$ & $      4199.38$ & $      4199.38$ & $         1.00$ sec    & $       2.8892$  & $       0.8118$ \\ 
\bottomrule
\end{tabular}
\end{table}

\begin{table}[H]
\scriptsize
\centering
\caption{image-seg (105025.bmp)}
\label{tab:anytimetable-image-seg-105025.bmp}
\begin{tabular}{lrrrrrrrrrrr}
\toprule
           algorithm &                                   \multicolumn{8}{c}{value} & \multicolumn{1}{c}{time}    & \multicolumn{1}{c}{VI}  & \multicolumn{1}{c}{RI} \\  
\cmidrule(lr){2-9}\cmidrule(lr){10-10} \cmidrule(lr){11-11} \cmidrule(lr){12-12}   
                     & \multicolumn{1}{c}{(0.5 sec)} & \multicolumn{1}{c}{(1 sec)} & \multicolumn{1}{c}{(10 sec)} & \multicolumn{1}{c}{(60 sec)} & \multicolumn{1}{c}{(300 sec)} & \multicolumn{1}{c}{(600 sec)} & \multicolumn{1}{c}{(1800 sec)} & \multicolumn{1}{c}{(end)} & \multicolumn{1}{c}{(end)}    & \multicolumn{1}{c}{(end)}   & \multicolumn{1}{c}{(end)}  \\ \midrule 
          PIVIT-BOEM & $\infty$ & $\infty$ & $\infty$ & $      8378.28$ & $      8378.28$ & $      8378.28$ & $      8378.28$ & $      8378.28$ & $        58.95$ sec    & $       5.8627$  & $       0.8016$ \\ 
                 CGC & $      6222.98$ & $      6135.88$ & $      6093.55$ & $      6093.55$ & $      6093.55$ & $      6093.55$ & $      6093.55$ & $      6093.55$ & $         3.23$ sec    & $       2.6426$  & $       0.8211$ \\ 
                  HC & $      6713.14$ & $      6713.14$ & $      6713.14$ & $      6713.14$ & $      6713.14$ & $      6713.14$ & $      6713.14$ & $      6713.14$ & $         0.01$ sec    & $       2.7860$  & $       0.8057$ \\ 
              HC-CGC & $      6149.01$ & $      6110.87$ & $      6106.14$ & $      6106.14$ & $      6106.14$ & $      6106.14$ & $      6106.14$ & $      6106.14$ & $         1.20$ sec    & $       2.5595$  & $       0.7952$ \\ 
              ogm-KL & $      6323.96$ & $      6306.79$ & $      6306.79$ & $      6306.79$ & $      6306.79$ & $      6306.79$ & $      6306.79$ & $      6306.79$ & $         1.17$ sec    & $       3.2042$  & $       0.4955$ \\ 
    CC-Fusion-HC-CGC & $      6133.28$ & $      6131.30$ & $      6117.43$ & $      6117.43$ & $      6117.43$ & $      6117.43$ & $      6117.43$ & $      6117.43$ & $         1.76$ sec    & $       2.6041$  & $       0.7781$ \\ 
     CC-Fusion-HC-MC & $      6136.91$ & $      6087.05$ & $      6070.91$ & $      6070.91$ & $      6070.91$ & $      6070.91$ & $      6070.91$ & $      6070.91$ & $         8.85$ sec    & $       2.8538$  & $       0.7765$ \\ 
    CC-Fusion-WS-CGC & $      6156.14$ & $      6156.14$ & $      6156.14$ & $      6156.14$ & $      6156.14$ & $      6156.14$ & $      6156.14$ & $      6156.14$ & $         1.03$ sec    & $       2.9371$  & $       0.7482$ \\ 
     CC-Fusion-WS-MC & $      6180.85$ & $      6114.11$ & $      6068.28$ & $      6068.28$ & $      6068.28$ & $      6068.28$ & $      6068.28$ & $      6068.28$ & $        16.06$ sec    & $       2.6990$  & $       0.8406$ \\ 
\cmidrule{1-1} 
           MCR-CCFDB & $      6211.10$ & $      6073.28$ & $      6073.28$ & $      6073.28$ & $      6073.28$ & $      6073.28$ & $      6073.28$ & $      6073.28$ & $         0.70$ sec    & $       2.8464$  & $       0.8170$ \\ 
\cmidrule{1-1} 
           MCI-CCIFD & $      6181.02$ & $      6148.73$ & $      6055.33$ & $      6055.33$ & $      6055.33$ & $      6055.33$ & $      6055.33$ & $      6055.33$ & $         2.83$ sec    & $       2.8160$  & $       0.8174$ \\ 
\bottomrule
\end{tabular}
\end{table}

\begin{table}[H]
\scriptsize
\centering
\caption{image-seg (106024.bmp)}
\label{tab:anytimetable-image-seg-106024.bmp}
\begin{tabular}{lrrrrrrrrrrr}
\toprule
           algorithm &                                   \multicolumn{8}{c}{value} & \multicolumn{1}{c}{time}    & \multicolumn{1}{c}{VI}  & \multicolumn{1}{c}{RI} \\  
\cmidrule(lr){2-9}\cmidrule(lr){10-10} \cmidrule(lr){11-11} \cmidrule(lr){12-12}   
                     & \multicolumn{1}{c}{(0.5 sec)} & \multicolumn{1}{c}{(1 sec)} & \multicolumn{1}{c}{(10 sec)} & \multicolumn{1}{c}{(60 sec)} & \multicolumn{1}{c}{(300 sec)} & \multicolumn{1}{c}{(600 sec)} & \multicolumn{1}{c}{(1800 sec)} & \multicolumn{1}{c}{(end)} & \multicolumn{1}{c}{(end)}    & \multicolumn{1}{c}{(end)}   & \multicolumn{1}{c}{(end)}  \\ \midrule 
          PIVIT-BOEM & $\infty$ & $      2509.12$ & $      2509.12$ & $      2509.12$ & $      2509.12$ & $      2509.12$ & $      2509.12$ & $      2509.12$ & $         0.89$ sec    & $       3.5488$  & $       0.6997$ \\ 
                 CGC & $      1600.36$ & $      1600.36$ & $      1600.36$ & $      1600.36$ & $      1600.36$ & $      1600.36$ & $      1600.36$ & $      1600.36$ & $         0.07$ sec    & $       1.9753$  & $       0.5861$ \\ 
                  HC & $      1769.75$ & $      1769.75$ & $      1769.75$ & $      1769.75$ & $      1769.75$ & $      1769.75$ & $      1769.75$ & $      1769.75$ & $         0.00$ sec    & $       1.8411$  & $       0.6615$ \\ 
              HC-CGC & $      1607.46$ & $      1607.46$ & $      1607.46$ & $      1607.46$ & $      1607.46$ & $      1607.46$ & $      1607.46$ & $      1607.46$ & $         0.10$ sec    & $       2.1485$  & $       0.5106$ \\ 
              ogm-KL & $      1626.41$ & $      1626.41$ & $      1626.41$ & $      1626.41$ & $      1626.41$ & $      1626.41$ & $      1626.41$ & $      1626.41$ & $         0.03$ sec    & $       2.1427$  & $       0.4911$ \\ 
    CC-Fusion-HC-CGC & $      1599.29$ & $      1599.29$ & $      1599.29$ & $      1599.29$ & $      1599.29$ & $      1599.29$ & $      1599.29$ & $      1599.29$ & $         0.31$ sec    & $       1.9706$  & $       0.5866$ \\ 
     CC-Fusion-HC-MC & $      1599.29$ & $      1599.29$ & $      1599.29$ & $      1599.29$ & $      1599.29$ & $      1599.29$ & $      1599.29$ & $      1599.29$ & $         0.88$ sec    & $       1.9706$  & $       0.5866$ \\ 
    CC-Fusion-WS-CGC & $      1603.92$ & $      1603.92$ & $      1603.92$ & $      1603.92$ & $      1603.92$ & $      1603.92$ & $      1603.92$ & $      1603.92$ & $         0.19$ sec    & $       2.0837$  & $       0.5816$ \\ 
     CC-Fusion-WS-MC & $      1599.29$ & $      1599.29$ & $      1599.29$ & $      1599.29$ & $      1599.29$ & $      1599.29$ & $      1599.29$ & $      1599.29$ & $         1.16$ sec    & $       1.9706$  & $       0.5866$ \\ 
\cmidrule{1-1} 
           MCR-CCFDB & $      1604.09$ & $      1604.09$ & $      1604.09$ & $      1604.09$ & $      1604.09$ & $      1604.09$ & $      1604.09$ & $      1604.09$ & $         0.07$ sec    & $       2.0129$  & $       0.5855$ \\ 
\cmidrule{1-1} 
           MCI-CCIFD & $      1600.57$ & $      1599.25$ & $      1599.25$ & $      1599.25$ & $      1599.25$ & $      1599.25$ & $      1599.25$ & $      1599.25$ & $         0.69$ sec    & $       2.0092$  & $       0.5855$ \\ 
\bottomrule
\end{tabular}
\end{table}

\begin{table}[H]
\scriptsize
\centering
\caption{image-seg (108005.bmp)}
\label{tab:anytimetable-image-seg-108005.bmp}
\begin{tabular}{lrrrrrrrrrrr}
\toprule
           algorithm &                                   \multicolumn{8}{c}{value} & \multicolumn{1}{c}{time}    & \multicolumn{1}{c}{VI}  & \multicolumn{1}{c}{RI} \\  
\cmidrule(lr){2-9}\cmidrule(lr){10-10} \cmidrule(lr){11-11} \cmidrule(lr){12-12}   
                     & \multicolumn{1}{c}{(0.5 sec)} & \multicolumn{1}{c}{(1 sec)} & \multicolumn{1}{c}{(10 sec)} & \multicolumn{1}{c}{(60 sec)} & \multicolumn{1}{c}{(300 sec)} & \multicolumn{1}{c}{(600 sec)} & \multicolumn{1}{c}{(1800 sec)} & \multicolumn{1}{c}{(end)} & \multicolumn{1}{c}{(end)}    & \multicolumn{1}{c}{(end)}   & \multicolumn{1}{c}{(end)}  \\ \midrule 
          PIVIT-BOEM & $\infty$ & $\infty$ & $\infty$ & $\infty$ & $      8625.85$ & $      8625.85$ & $      8625.85$ & $      8625.85$ & $        88.73$ sec    & $       6.7715$  & $       0.7133$ \\ 
                 CGC & $      6804.65$ & $      6709.02$ & $      6617.36$ & $      6617.36$ & $      6617.36$ & $      6617.36$ & $      6617.36$ & $      6617.36$ & $         5.11$ sec    & $       4.0642$  & $       0.6163$ \\ 
                  HC & $      7121.43$ & $      7121.43$ & $      7121.43$ & $      7121.43$ & $      7121.43$ & $      7121.43$ & $      7121.43$ & $      7121.43$ & $         0.01$ sec    & $       4.2541$  & $       0.6338$ \\ 
              HC-CGC & $      6655.68$ & $      6614.60$ & $      6599.14$ & $      6599.14$ & $      6599.14$ & $      6599.14$ & $      6599.14$ & $      6599.14$ & $         1.55$ sec    & $       4.3886$  & $       0.6775$ \\ 
              ogm-KL & $     10588.54$ & $     10588.54$ & $      6912.42$ & $      6912.42$ & $      6912.42$ & $      6912.42$ & $      6912.42$ & $      6912.42$ & $         4.31$ sec    & $       3.4629$  & $       0.4791$ \\ 
    CC-Fusion-HC-CGC & $      6647.86$ & $      6631.03$ & $      6605.37$ & $      6605.37$ & $      6605.37$ & $      6605.37$ & $      6605.37$ & $      6605.37$ & $         3.12$ sec    & $       4.3134$  & $       0.6937$ \\ 
     CC-Fusion-HC-MC & $      6645.11$ & $      6585.52$ & $      6578.03$ & $      6578.03$ & $      6578.03$ & $      6578.03$ & $      6578.03$ & $      6578.03$ & $         6.09$ sec    & $       4.3639$  & $       0.7019$ \\ 
    CC-Fusion-WS-CGC & $      6654.39$ & $      6650.95$ & $      6643.86$ & $      6643.86$ & $      6643.86$ & $      6643.86$ & $      6643.86$ & $      6643.86$ & $         1.98$ sec    & $       4.2804$  & $       0.7050$ \\ 
     CC-Fusion-WS-MC & $      6752.41$ & $      6678.81$ & $      6582.74$ & $      6582.74$ & $      6582.74$ & $      6582.74$ & $      6582.74$ & $      6582.74$ & $        10.11$ sec    & $       4.3308$  & $       0.6947$ \\ 
\cmidrule{1-1} 
           MCR-CCFDB & $      7906.62$ & $      6581.96$ & $      6581.96$ & $      6581.96$ & $      6581.96$ & $      6581.96$ & $      6581.96$ & $      6581.96$ & $         0.70$ sec    & $       4.3685$  & $       0.7019$ \\ 
\cmidrule{1-1} 
           MCI-CCIFD & $      7250.79$ & $      6773.75$ & $      6578.03$ & $      6578.03$ & $      6578.03$ & $      6578.03$ & $      6578.03$ & $      6578.03$ & $         1.64$ sec    & $       4.3639$  & $       0.7019$ \\ 
\bottomrule
\end{tabular}
\end{table}

\begin{table}[H]
\scriptsize
\centering
\caption{image-seg (108070.bmp)}
\label{tab:anytimetable-image-seg-108070.bmp}
\begin{tabular}{lrrrrrrrrrrr}
\toprule
           algorithm &                                   \multicolumn{8}{c}{value} & \multicolumn{1}{c}{time}    & \multicolumn{1}{c}{VI}  & \multicolumn{1}{c}{RI} \\  
\cmidrule(lr){2-9}\cmidrule(lr){10-10} \cmidrule(lr){11-11} \cmidrule(lr){12-12}   
                     & \multicolumn{1}{c}{(0.5 sec)} & \multicolumn{1}{c}{(1 sec)} & \multicolumn{1}{c}{(10 sec)} & \multicolumn{1}{c}{(60 sec)} & \multicolumn{1}{c}{(300 sec)} & \multicolumn{1}{c}{(600 sec)} & \multicolumn{1}{c}{(1800 sec)} & \multicolumn{1}{c}{(end)} & \multicolumn{1}{c}{(end)}    & \multicolumn{1}{c}{(end)}   & \multicolumn{1}{c}{(end)}  \\ \midrule 
          PIVIT-BOEM & $\infty$ & $\infty$ & $\infty$ & $\infty$ & $     11947.12$ & $     11947.12$ & $     11947.12$ & $     11947.12$ & $       171.04$ sec    & $       7.7419$  & $       0.5215$ \\ 
                 CGC & $      8611.13$ & $      8600.20$ & $      8459.19$ & $      8441.67$ & $      8441.67$ & $      8441.67$ & $      8441.67$ & $      8441.67$ & $        21.65$ sec    & $       3.2492$  & $       0.5549$ \\ 
                  HC & $      9041.17$ & $      9041.17$ & $      9041.17$ & $      9041.17$ & $      9041.17$ & $      9041.17$ & $      9041.17$ & $      9041.17$ & $         0.01$ sec    & $       3.5225$  & $       0.6097$ \\ 
              HC-CGC & $      8596.76$ & $      8565.77$ & $      8440.20$ & $      8436.35$ & $      8436.35$ & $      8436.35$ & $      8436.35$ & $      8436.35$ & $        17.41$ sec    & $       3.2074$  & $       0.5618$ \\ 
              ogm-KL & $     11215.59$ & $     11215.59$ & $      8636.17$ & $      8636.17$ & $      8636.17$ & $      8636.17$ & $      8636.17$ & $      8636.17$ & $         3.33$ sec    & $       2.5094$  & $       0.5387$ \\ 
    CC-Fusion-HC-CGC & $      8493.42$ & $      8485.06$ & $      8456.98$ & $      8456.98$ & $      8456.98$ & $      8456.98$ & $      8456.98$ & $      8456.98$ & $         4.72$ sec    & $       3.2574$  & $       0.5553$ \\ 
     CC-Fusion-HC-MC & $      8515.21$ & $      8443.22$ & $      8425.09$ & $      8425.09$ & $      8425.09$ & $      8425.09$ & $      8425.09$ & $      8425.09$ & $        15.44$ sec    & $       3.4284$  & $       0.6020$ \\ 
    CC-Fusion-WS-CGC & $      8534.52$ & $      8515.59$ & $      8476.32$ & $      8476.32$ & $      8476.32$ & $      8476.32$ & $      8476.32$ & $      8476.32$ & $         2.76$ sec    & $       3.1948$  & $       0.5578$ \\ 
     CC-Fusion-WS-MC & $      9216.25$ & $      8617.53$ & $      8425.44$ & $      8425.09$ & $      8425.09$ & $      8425.09$ & $      8425.09$ & $      8425.09$ & $        25.55$ sec    & $       3.4284$  & $       0.6020$ \\ 
\cmidrule{1-1} 
           MCR-CCFDB & $     10108.62$ & $      8760.98$ & $      8426.08$ & $      8426.08$ & $      8426.08$ & $      8426.08$ & $      8426.08$ & $      8426.08$ & $         1.65$ sec    & $       3.3905$  & $       0.6305$ \\ 
\cmidrule{1-1} 
           MCI-CCIFD & $      8822.24$ & $      8602.64$ & $      8422.24$ & $      8422.24$ & $      8422.24$ & $      8422.24$ & $      8422.24$ & $      8422.24$ & $         2.14$ sec    & $       3.3875$  & $       0.6304$ \\ 
\bottomrule
\end{tabular}
\end{table}

\begin{table}[H]
\scriptsize
\centering
\caption{image-seg (108082.bmp)}
\label{tab:anytimetable-image-seg-108082.bmp}
\begin{tabular}{lrrrrrrrrrrr}
\toprule
           algorithm &                                   \multicolumn{8}{c}{value} & \multicolumn{1}{c}{time}    & \multicolumn{1}{c}{VI}  & \multicolumn{1}{c}{RI} \\  
\cmidrule(lr){2-9}\cmidrule(lr){10-10} \cmidrule(lr){11-11} \cmidrule(lr){12-12}   
                     & \multicolumn{1}{c}{(0.5 sec)} & \multicolumn{1}{c}{(1 sec)} & \multicolumn{1}{c}{(10 sec)} & \multicolumn{1}{c}{(60 sec)} & \multicolumn{1}{c}{(300 sec)} & \multicolumn{1}{c}{(600 sec)} & \multicolumn{1}{c}{(1800 sec)} & \multicolumn{1}{c}{(end)} & \multicolumn{1}{c}{(end)}    & \multicolumn{1}{c}{(end)}   & \multicolumn{1}{c}{(end)}  \\ \midrule 
          PIVIT-BOEM & $\infty$ & $\infty$ & $\infty$ & $      6332.15$ & $      6332.15$ & $      6332.15$ & $      6332.15$ & $      6332.15$ & $        31.97$ sec    & $       6.1989$  & $       0.6320$ \\ 
                 CGC & $      4890.98$ & $      4842.39$ & $      4835.78$ & $      4835.78$ & $      4835.78$ & $      4835.78$ & $      4835.78$ & $      4835.78$ & $         1.99$ sec    & $       3.9599$  & $       0.5957$ \\ 
                  HC & $      5330.47$ & $      5330.47$ & $      5330.47$ & $      5330.47$ & $      5330.47$ & $      5330.47$ & $      5330.47$ & $      5330.47$ & $         0.00$ sec    & $       3.9840$  & $       0.5914$ \\ 
              HC-CGC & $      4837.38$ & $      4812.66$ & $      4811.33$ & $      4811.33$ & $      4811.33$ & $      4811.33$ & $      4811.33$ & $      4811.33$ & $         1.19$ sec    & $       3.9844$  & $       0.6160$ \\ 
              ogm-KL & $      5113.17$ & $      5045.49$ & $      5037.94$ & $      5037.94$ & $      5037.94$ & $      5037.94$ & $      5037.94$ & $      5037.94$ & $         1.54$ sec    & $       2.8742$  & $       0.5656$ \\ 
    CC-Fusion-HC-CGC & $      4833.99$ & $      4824.63$ & $      4823.33$ & $      4823.33$ & $      4823.33$ & $      4823.33$ & $      4823.33$ & $      4823.33$ & $         1.65$ sec    & $       3.9881$  & $       0.6161$ \\ 
     CC-Fusion-HC-MC & $      4814.80$ & $      4804.23$ & $      4801.22$ & $      4801.22$ & $      4801.22$ & $      4801.22$ & $      4801.22$ & $      4801.22$ & $         7.07$ sec    & $       4.0023$  & $       0.6376$ \\ 
    CC-Fusion-WS-CGC & $      4846.49$ & $      4846.49$ & $      4846.49$ & $      4846.49$ & $      4846.49$ & $      4846.49$ & $      4846.49$ & $      4846.49$ & $         0.72$ sec    & $       3.8971$  & $       0.6378$ \\ 
     CC-Fusion-WS-MC & $      5242.84$ & $      4885.83$ & $      4802.49$ & $      4802.49$ & $      4802.49$ & $      4802.49$ & $      4802.49$ & $      4802.49$ & $        10.20$ sec    & $       3.9886$  & $       0.6299$ \\ 
\cmidrule{1-1} 
           MCR-CCFDB & $      4828.26$ & $      4800.15$ & $      4800.15$ & $      4800.15$ & $      4800.15$ & $      4800.15$ & $      4800.15$ & $      4800.15$ & $         0.54$ sec    & $       3.9890$  & $       0.6327$ \\ 
\cmidrule{1-1} 
           MCI-CCIFD & $      4952.90$ & $      4826.12$ & $      4800.15$ & $      4800.15$ & $      4800.15$ & $      4800.15$ & $      4800.15$ & $      4800.15$ & $         1.83$ sec    & $       3.9890$  & $       0.6327$ \\ 
\bottomrule
\end{tabular}
\end{table}

\begin{table}[H]
\scriptsize
\centering
\caption{image-seg (109053.bmp)}
\label{tab:anytimetable-image-seg-109053.bmp}
\begin{tabular}{lrrrrrrrrrrr}
\toprule
           algorithm &                                   \multicolumn{8}{c}{value} & \multicolumn{1}{c}{time}    & \multicolumn{1}{c}{VI}  & \multicolumn{1}{c}{RI} \\  
\cmidrule(lr){2-9}\cmidrule(lr){10-10} \cmidrule(lr){11-11} \cmidrule(lr){12-12}   
                     & \multicolumn{1}{c}{(0.5 sec)} & \multicolumn{1}{c}{(1 sec)} & \multicolumn{1}{c}{(10 sec)} & \multicolumn{1}{c}{(60 sec)} & \multicolumn{1}{c}{(300 sec)} & \multicolumn{1}{c}{(600 sec)} & \multicolumn{1}{c}{(1800 sec)} & \multicolumn{1}{c}{(end)} & \multicolumn{1}{c}{(end)}    & \multicolumn{1}{c}{(end)}   & \multicolumn{1}{c}{(end)}  \\ \midrule 
          PIVIT-BOEM & $\infty$ & $\infty$ & $\infty$ & $      5880.78$ & $      5880.78$ & $      5880.78$ & $      5880.78$ & $      5880.78$ & $        22.47$ sec    & $       5.4642$  & $       0.7216$ \\ 
                 CGC & $      4594.80$ & $      4565.57$ & $      4472.80$ & $      4472.80$ & $      4472.80$ & $      4472.80$ & $      4472.80$ & $      4472.80$ & $         4.59$ sec    & $       3.2634$  & $       0.4657$ \\ 
                  HC & $      5014.88$ & $      5014.88$ & $      5014.88$ & $      5014.88$ & $      5014.88$ & $      5014.88$ & $      5014.88$ & $      5014.88$ & $         0.00$ sec    & $       3.5099$  & $       0.5099$ \\ 
              HC-CGC & $      4484.28$ & $      4460.87$ & $      4442.56$ & $      4442.56$ & $      4442.56$ & $      4442.56$ & $      4442.56$ & $      4442.56$ & $         5.58$ sec    & $       3.3472$  & $       0.4716$ \\ 
              ogm-KL & $      4628.24$ & $      4608.27$ & $      4606.41$ & $      4606.41$ & $      4606.41$ & $      4606.41$ & $      4606.41$ & $      4606.41$ & $         1.43$ sec    & $       2.8857$  & $       0.4083$ \\ 
    CC-Fusion-HC-CGC & $      4448.45$ & $      4448.45$ & $      4448.45$ & $      4448.45$ & $      4448.45$ & $      4448.45$ & $      4448.45$ & $      4448.45$ & $         0.83$ sec    & $       3.2949$  & $       0.4942$ \\ 
     CC-Fusion-HC-MC & $      4421.20$ & $      4421.13$ & $      4421.13$ & $      4421.13$ & $      4421.13$ & $      4421.13$ & $      4421.13$ & $      4421.13$ & $         3.03$ sec    & $       3.3533$  & $       0.5328$ \\ 
    CC-Fusion-WS-CGC & $      4497.48$ & $      4484.98$ & $      4484.98$ & $      4484.98$ & $      4484.98$ & $      4484.98$ & $      4484.98$ & $      4484.98$ & $         1.19$ sec    & $       3.3959$  & $       0.4951$ \\ 
     CC-Fusion-WS-MC & $      4455.97$ & $      4433.67$ & $      4421.83$ & $      4421.83$ & $      4421.83$ & $      4421.83$ & $      4421.83$ & $      4421.83$ & $         5.67$ sec    & $       3.2410$  & $       0.6103$ \\ 
\cmidrule{1-1} 
           MCR-CCFDB & $      4453.23$ & $      4425.56$ & $      4425.56$ & $      4425.56$ & $      4425.56$ & $      4425.56$ & $      4425.56$ & $      4425.56$ & $         0.62$ sec    & $       3.3633$  & $       0.5343$ \\ 
\cmidrule{1-1} 
           MCI-CCIFD & $      4519.74$ & $      4431.77$ & $      4421.13$ & $      4421.13$ & $      4421.13$ & $      4421.13$ & $      4421.13$ & $      4421.13$ & $         1.06$ sec    & $       3.3533$  & $       0.5328$ \\ 
\bottomrule
\end{tabular}
\end{table}

\begin{table}[H]
\scriptsize
\centering
\caption{image-seg (119082.bmp)}
\label{tab:anytimetable-image-seg-119082.bmp}
\begin{tabular}{lrrrrrrrrrrr}
\toprule
           algorithm &                                   \multicolumn{8}{c}{value} & \multicolumn{1}{c}{time}    & \multicolumn{1}{c}{VI}  & \multicolumn{1}{c}{RI} \\  
\cmidrule(lr){2-9}\cmidrule(lr){10-10} \cmidrule(lr){11-11} \cmidrule(lr){12-12}   
                     & \multicolumn{1}{c}{(0.5 sec)} & \multicolumn{1}{c}{(1 sec)} & \multicolumn{1}{c}{(10 sec)} & \multicolumn{1}{c}{(60 sec)} & \multicolumn{1}{c}{(300 sec)} & \multicolumn{1}{c}{(600 sec)} & \multicolumn{1}{c}{(1800 sec)} & \multicolumn{1}{c}{(end)} & \multicolumn{1}{c}{(end)}    & \multicolumn{1}{c}{(end)}   & \multicolumn{1}{c}{(end)}  \\ \midrule 
          PIVIT-BOEM & $\infty$ & $\infty$ & $\infty$ & $      5586.29$ & $      5586.29$ & $      5586.29$ & $      5586.29$ & $      5586.29$ & $        27.73$ sec    & $       5.1062$  & $       0.8905$ \\ 
                 CGC & $      4543.93$ & $      4543.93$ & $      4543.93$ & $      4543.93$ & $      4543.93$ & $      4543.93$ & $      4543.93$ & $      4543.93$ & $         0.17$ sec    & $       3.4900$  & $       0.8708$ \\ 
                  HC & $      4837.30$ & $      4837.30$ & $      4837.30$ & $      4837.30$ & $      4837.30$ & $      4837.30$ & $      4837.30$ & $      4837.30$ & $         0.00$ sec    & $       3.4814$  & $       0.8545$ \\ 
              HC-CGC & $      4541.05$ & $      4541.05$ & $      4541.05$ & $      4541.05$ & $      4541.05$ & $      4541.05$ & $      4541.05$ & $      4541.05$ & $         0.13$ sec    & $       3.4636$  & $       0.8740$ \\ 
              ogm-KL & $      4722.36$ & $      4705.51$ & $      4705.51$ & $      4705.51$ & $      4705.51$ & $      4705.51$ & $      4705.51$ & $      4705.51$ & $         1.19$ sec    & $       4.0665$  & $       0.7653$ \\ 
    CC-Fusion-HC-CGC & $      4532.52$ & $      4532.51$ & $      4532.51$ & $      4532.51$ & $      4532.51$ & $      4532.51$ & $      4532.51$ & $      4532.51$ & $         0.99$ sec    & $       3.1669$  & $       0.9135$ \\ 
     CC-Fusion-HC-MC & $      4531.76$ & $      4530.71$ & $      4530.71$ & $      4530.71$ & $      4530.71$ & $      4530.71$ & $      4530.71$ & $      4530.71$ & $         2.15$ sec    & $       3.2507$  & $       0.9100$ \\ 
    CC-Fusion-WS-CGC & $      4536.13$ & $      4534.47$ & $      4534.29$ & $      4534.29$ & $      4534.29$ & $      4534.29$ & $      4534.29$ & $      4534.29$ & $         1.26$ sec    & $       3.2112$  & $       0.9120$ \\ 
     CC-Fusion-WS-MC & $      4536.41$ & $      4534.16$ & $      4530.71$ & $      4530.71$ & $      4530.71$ & $      4530.71$ & $      4530.71$ & $      4530.71$ & $         4.51$ sec    & $       3.2507$  & $       0.9100$ \\ 
\cmidrule{1-1} 
           MCR-CCFDB & $      4530.71$ & $      4530.71$ & $      4530.71$ & $      4530.71$ & $      4530.71$ & $      4530.71$ & $      4530.71$ & $      4530.71$ & $         0.07$ sec    & $       3.2507$  & $       0.9100$ \\ 
\cmidrule{1-1} 
           MCI-CCIFD & $      4530.71$ & $      4530.71$ & $      4530.71$ & $      4530.71$ & $      4530.71$ & $      4530.71$ & $      4530.71$ & $      4530.71$ & $         0.19$ sec    & $       3.2507$  & $       0.9100$ \\ 
\bottomrule
\end{tabular}
\end{table}

\begin{table}[H]
\scriptsize
\centering
\caption{image-seg (12084.bmp)}
\label{tab:anytimetable-image-seg-12084.bmp}
\begin{tabular}{lrrrrrrrrrrr}
\toprule
           algorithm &                                   \multicolumn{8}{c}{value} & \multicolumn{1}{c}{time}    & \multicolumn{1}{c}{VI}  & \multicolumn{1}{c}{RI} \\  
\cmidrule(lr){2-9}\cmidrule(lr){10-10} \cmidrule(lr){11-11} \cmidrule(lr){12-12}   
                     & \multicolumn{1}{c}{(0.5 sec)} & \multicolumn{1}{c}{(1 sec)} & \multicolumn{1}{c}{(10 sec)} & \multicolumn{1}{c}{(60 sec)} & \multicolumn{1}{c}{(300 sec)} & \multicolumn{1}{c}{(600 sec)} & \multicolumn{1}{c}{(1800 sec)} & \multicolumn{1}{c}{(end)} & \multicolumn{1}{c}{(end)}    & \multicolumn{1}{c}{(end)}   & \multicolumn{1}{c}{(end)}  \\ \midrule 
          PIVIT-BOEM & $\infty$ & $\infty$ & $\infty$ & $\infty$ & $      8902.80$ & $      8902.80$ & $      8902.80$ & $      8902.80$ & $       118.42$ sec    & $       8.0619$  & $       0.5089$ \\ 
                 CGC & $      7391.84$ & $      7376.93$ & $      7301.42$ & $      7301.42$ & $      7301.42$ & $      7301.42$ & $      7301.42$ & $      7301.42$ & $         8.54$ sec    & $       5.4724$  & $       0.4626$ \\ 
                  HC & $      7742.81$ & $      7742.81$ & $      7742.81$ & $      7742.81$ & $      7742.81$ & $      7742.81$ & $      7742.81$ & $      7742.81$ & $         0.01$ sec    & $       5.4058$  & $       0.4890$ \\ 
              HC-CGC & $      7327.67$ & $      7301.35$ & $      7293.87$ & $      7293.87$ & $      7293.87$ & $      7293.87$ & $      7293.87$ & $      7293.87$ & $         3.73$ sec    & $       5.5457$  & $       0.4880$ \\ 
              ogm-KL & $      9328.84$ & $      9328.84$ & $      7459.40$ & $      7456.34$ & $      7456.34$ & $      7456.34$ & $      7456.34$ & $      7456.34$ & $        12.95$ sec    & $       4.1543$  & $       0.4703$ \\ 
    CC-Fusion-HC-CGC & $      7301.25$ & $      7295.29$ & $      7290.16$ & $      7290.16$ & $      7290.16$ & $      7290.16$ & $      7290.16$ & $      7290.16$ & $         2.71$ sec    & $       5.5907$  & $       0.4984$ \\ 
     CC-Fusion-HC-MC & $      7312.97$ & $      7296.42$ & $      7287.68$ & $      7287.68$ & $      7287.68$ & $      7287.68$ & $      7287.68$ & $      7287.68$ & $         7.54$ sec    & $       5.6152$  & $       0.4940$ \\ 
    CC-Fusion-WS-CGC & $      7324.13$ & $      7314.66$ & $      7306.02$ & $      7306.02$ & $      7306.02$ & $      7306.02$ & $      7306.02$ & $      7306.02$ & $         2.66$ sec    & $       5.5513$  & $       0.4964$ \\ 
     CC-Fusion-WS-MC & $      7755.99$ & $      7373.70$ & $      7284.45$ & $      7284.45$ & $      7284.45$ & $      7284.45$ & $      7284.45$ & $      7284.45$ & $        13.38$ sec    & $       5.7016$  & $       0.5074$ \\ 
\cmidrule{1-1} 
           MCR-CCFDB & $      7288.30$ & $      7288.30$ & $      7288.30$ & $      7288.30$ & $      7288.30$ & $      7288.30$ & $      7288.30$ & $      7288.30$ & $         0.44$ sec    & $       5.7131$  & $       0.5074$ \\ 
\cmidrule{1-1} 
           MCI-CCIFD & $      7385.90$ & $      7313.23$ & $      7284.45$ & $      7284.45$ & $      7284.45$ & $      7284.45$ & $      7284.45$ & $      7284.45$ & $         1.07$ sec    & $       5.7016$  & $       0.5074$ \\ 
\bottomrule
\end{tabular}
\end{table}

\begin{table}[H]
\scriptsize
\centering
\caption{image-seg (123074.bmp)}
\label{tab:anytimetable-image-seg-123074.bmp}
\begin{tabular}{lrrrrrrrrrrr}
\toprule
           algorithm &                                   \multicolumn{8}{c}{value} & \multicolumn{1}{c}{time}    & \multicolumn{1}{c}{VI}  & \multicolumn{1}{c}{RI} \\  
\cmidrule(lr){2-9}\cmidrule(lr){10-10} \cmidrule(lr){11-11} \cmidrule(lr){12-12}   
                     & \multicolumn{1}{c}{(0.5 sec)} & \multicolumn{1}{c}{(1 sec)} & \multicolumn{1}{c}{(10 sec)} & \multicolumn{1}{c}{(60 sec)} & \multicolumn{1}{c}{(300 sec)} & \multicolumn{1}{c}{(600 sec)} & \multicolumn{1}{c}{(1800 sec)} & \multicolumn{1}{c}{(end)} & \multicolumn{1}{c}{(end)}    & \multicolumn{1}{c}{(end)}   & \multicolumn{1}{c}{(end)}  \\ \midrule 
          PIVIT-BOEM & $\infty$ & $\infty$ & $\infty$ & $      5974.05$ & $      5974.05$ & $      5974.05$ & $      5974.05$ & $      5974.05$ & $        13.51$ sec    & $       5.1189$  & $       0.8013$ \\ 
                 CGC & $      3979.04$ & $      3904.66$ & $      3880.29$ & $      3880.29$ & $      3880.29$ & $      3880.29$ & $      3880.29$ & $      3880.29$ & $         1.73$ sec    & $       2.6904$  & $       0.4909$ \\ 
                  HC & $      4310.37$ & $      4310.37$ & $      4310.37$ & $      4310.37$ & $      4310.37$ & $      4310.37$ & $      4310.37$ & $      4310.37$ & $         0.00$ sec    & $       3.0191$  & $       0.6645$ \\ 
              HC-CGC & $      3856.11$ & $      3856.11$ & $      3856.11$ & $      3856.11$ & $      3856.11$ & $      3856.11$ & $      3856.11$ & $      3856.11$ & $         0.53$ sec    & $       2.6860$  & $       0.6508$ \\ 
              ogm-KL & $      4059.26$ & $      4059.26$ & $      4059.26$ & $      4059.26$ & $      4059.26$ & $      4059.26$ & $      4059.26$ & $      4059.26$ & $         0.22$ sec    & $       2.9356$  & $       0.3470$ \\ 
    CC-Fusion-HC-CGC & $      3869.89$ & $      3869.89$ & $      3869.89$ & $      3869.89$ & $      3869.89$ & $      3869.89$ & $      3869.89$ & $      3869.89$ & $         0.53$ sec    & $       2.6554$  & $       0.6171$ \\ 
     CC-Fusion-HC-MC & $      3847.83$ & $      3847.59$ & $      3842.74$ & $      3842.74$ & $      3842.74$ & $      3842.74$ & $      3842.74$ & $      3842.74$ & $         4.22$ sec    & $       2.6999$  & $       0.6741$ \\ 
    CC-Fusion-WS-CGC & $      3901.18$ & $      3888.91$ & $      3878.90$ & $      3878.90$ & $      3878.90$ & $      3878.90$ & $      3878.90$ & $      3878.90$ & $         1.50$ sec    & $       2.7619$  & $       0.5513$ \\ 
     CC-Fusion-WS-MC & $      3905.76$ & $      3848.05$ & $      3842.74$ & $      3842.74$ & $      3842.74$ & $      3842.74$ & $      3842.74$ & $      3842.74$ & $         5.55$ sec    & $       2.6999$  & $       0.6741$ \\ 
\cmidrule{1-1} 
           MCR-CCFDB & $      4200.09$ & $      3842.74$ & $      3842.74$ & $      3842.74$ & $      3842.74$ & $      3842.74$ & $      3842.74$ & $      3842.74$ & $         0.80$ sec    & $       2.6999$  & $       0.6741$ \\ 
\cmidrule{1-1} 
           MCI-CCIFD & $      3968.12$ & $      3877.22$ & $      3842.74$ & $      3842.74$ & $      3842.74$ & $      3842.74$ & $      3842.74$ & $      3842.74$ & $         2.87$ sec    & $       2.6999$  & $       0.6741$ \\ 
\bottomrule
\end{tabular}
\end{table}

\begin{table}[H]
\scriptsize
\centering
\caption{image-seg (126007.bmp)}
\label{tab:anytimetable-image-seg-126007.bmp}
\begin{tabular}{lrrrrrrrrrrr}
\toprule
           algorithm &                                   \multicolumn{8}{c}{value} & \multicolumn{1}{c}{time}    & \multicolumn{1}{c}{VI}  & \multicolumn{1}{c}{RI} \\  
\cmidrule(lr){2-9}\cmidrule(lr){10-10} \cmidrule(lr){11-11} \cmidrule(lr){12-12}   
                     & \multicolumn{1}{c}{(0.5 sec)} & \multicolumn{1}{c}{(1 sec)} & \multicolumn{1}{c}{(10 sec)} & \multicolumn{1}{c}{(60 sec)} & \multicolumn{1}{c}{(300 sec)} & \multicolumn{1}{c}{(600 sec)} & \multicolumn{1}{c}{(1800 sec)} & \multicolumn{1}{c}{(end)} & \multicolumn{1}{c}{(end)}    & \multicolumn{1}{c}{(end)}   & \multicolumn{1}{c}{(end)}  \\ \midrule 
          PIVIT-BOEM & $\infty$ & $\infty$ & $      3461.57$ & $      3461.57$ & $      3461.57$ & $      3461.57$ & $      3461.57$ & $      3461.57$ & $         4.66$ sec    & $       2.9698$  & $       0.9085$ \\ 
                 CGC & $      2692.98$ & $      2692.98$ & $      2692.98$ & $      2692.98$ & $      2692.98$ & $      2692.98$ & $      2692.98$ & $      2692.98$ & $         0.05$ sec    & $       1.5914$  & $       0.9438$ \\ 
                  HC & $      2898.24$ & $      2898.24$ & $      2898.24$ & $      2898.24$ & $      2898.24$ & $      2898.24$ & $      2898.24$ & $      2898.24$ & $         0.00$ sec    & $       1.9315$  & $       0.9027$ \\ 
              HC-CGC & $      2688.59$ & $      2688.59$ & $      2688.59$ & $      2688.59$ & $      2688.59$ & $      2688.59$ & $      2688.59$ & $      2688.59$ & $         0.04$ sec    & $       1.6463$  & $       0.9415$ \\ 
              ogm-KL & $      2791.58$ & $      2791.58$ & $      2791.58$ & $      2791.58$ & $      2791.58$ & $      2791.58$ & $      2791.58$ & $      2791.58$ & $         0.19$ sec    & $       2.0876$  & $       0.8715$ \\ 
    CC-Fusion-HC-CGC & $      2685.26$ & $      2685.26$ & $      2685.26$ & $      2685.26$ & $      2685.26$ & $      2685.26$ & $      2685.26$ & $      2685.26$ & $         0.53$ sec    & $       1.6525$  & $       0.9417$ \\ 
     CC-Fusion-HC-MC & $      2685.03$ & $      2684.83$ & $      2684.83$ & $      2684.83$ & $      2684.83$ & $      2684.83$ & $      2684.83$ & $      2684.83$ & $         1.58$ sec    & $       1.5870$  & $       0.9443$ \\ 
    CC-Fusion-WS-CGC & $      2688.73$ & $      2688.73$ & $      2688.73$ & $      2688.73$ & $      2688.73$ & $      2688.73$ & $      2688.73$ & $      2688.73$ & $         0.65$ sec    & $       1.5721$  & $       0.9445$ \\ 
     CC-Fusion-WS-MC & $      2695.17$ & $      2684.83$ & $      2684.83$ & $      2684.83$ & $      2684.83$ & $      2684.83$ & $      2684.83$ & $      2684.83$ & $         2.42$ sec    & $       1.5870$  & $       0.9443$ \\ 
\cmidrule{1-1} 
           MCR-CCFDB & $      2686.11$ & $      2686.11$ & $      2686.11$ & $      2686.11$ & $      2686.11$ & $      2686.11$ & $      2686.11$ & $      2686.11$ & $         0.05$ sec    & $       1.5858$  & $       0.9443$ \\ 
\cmidrule{1-1} 
           MCI-CCIFD & $      2684.83$ & $      2684.83$ & $      2684.83$ & $      2684.83$ & $      2684.83$ & $      2684.83$ & $      2684.83$ & $      2684.83$ & $         0.32$ sec    & $       1.5866$  & $       0.9443$ \\ 
\bottomrule
\end{tabular}
\end{table}

\begin{table}[H]
\scriptsize
\centering
\caption{image-seg (130026.bmp)}
\label{tab:anytimetable-image-seg-130026.bmp}
\begin{tabular}{lrrrrrrrrrrr}
\toprule
           algorithm &                                   \multicolumn{8}{c}{value} & \multicolumn{1}{c}{time}    & \multicolumn{1}{c}{VI}  & \multicolumn{1}{c}{RI} \\  
\cmidrule(lr){2-9}\cmidrule(lr){10-10} \cmidrule(lr){11-11} \cmidrule(lr){12-12}   
                     & \multicolumn{1}{c}{(0.5 sec)} & \multicolumn{1}{c}{(1 sec)} & \multicolumn{1}{c}{(10 sec)} & \multicolumn{1}{c}{(60 sec)} & \multicolumn{1}{c}{(300 sec)} & \multicolumn{1}{c}{(600 sec)} & \multicolumn{1}{c}{(1800 sec)} & \multicolumn{1}{c}{(end)} & \multicolumn{1}{c}{(end)}    & \multicolumn{1}{c}{(end)}   & \multicolumn{1}{c}{(end)}  \\ \midrule 
          PIVIT-BOEM & $\infty$ & $\infty$ & $\infty$ & $      8034.43$ & $      8034.43$ & $      8034.43$ & $      8034.43$ & $      8034.43$ & $        47.80$ sec    & $       7.0019$  & $       0.4619$ \\ 
                 CGC & $      5573.42$ & $      5531.49$ & $      5392.59$ & $      5392.59$ & $      5392.59$ & $      5392.59$ & $      5392.59$ & $      5392.59$ & $         4.99$ sec    & $       2.2961$  & $       0.6529$ \\ 
                  HC & $      6255.27$ & $      6255.27$ & $      6255.27$ & $      6255.27$ & $      6255.27$ & $      6255.27$ & $      6255.27$ & $      6255.27$ & $         0.01$ sec    & $       3.0404$  & $       0.5186$ \\ 
              HC-CGC & $      5508.38$ & $      5438.60$ & $      5401.18$ & $      5401.18$ & $      5401.18$ & $      5401.18$ & $      5401.18$ & $      5401.18$ & $         4.73$ sec    & $       2.3782$  & $       0.6118$ \\ 
              ogm-KL & $      5620.22$ & $      5597.67$ & $      5594.57$ & $      5594.57$ & $      5594.57$ & $      5594.57$ & $      5594.57$ & $      5594.57$ & $         1.55$ sec    & $       1.8612$  & $       0.6020$ \\ 
    CC-Fusion-HC-CGC & $      5383.41$ & $      5377.79$ & $      5367.14$ & $      5367.14$ & $      5367.14$ & $      5367.14$ & $      5367.14$ & $      5367.14$ & $         2.33$ sec    & $       2.3708$  & $       0.6596$ \\ 
     CC-Fusion-HC-MC & $      5365.86$ & $      5355.93$ & $      5351.15$ & $      5351.15$ & $      5351.15$ & $      5351.15$ & $      5351.15$ & $      5351.15$ & $         6.99$ sec    & $       2.4069$  & $       0.6601$ \\ 
    CC-Fusion-WS-CGC & $      5479.09$ & $      5479.09$ & $      5479.09$ & $      5479.09$ & $      5479.09$ & $      5479.09$ & $      5479.09$ & $      5479.09$ & $         0.68$ sec    & $       2.5547$  & $       0.6312$ \\ 
     CC-Fusion-WS-MC & $      5889.88$ & $      5454.16$ & $      5351.15$ & $      5351.15$ & $      5351.15$ & $      5351.15$ & $      5351.15$ & $      5351.15$ & $        10.55$ sec    & $       2.4069$  & $       0.6601$ \\ 
\cmidrule{1-1} 
           MCR-CCFDB & $      6383.91$ & $      5569.26$ & $      5366.20$ & $      5366.20$ & $      5366.20$ & $      5366.20$ & $      5366.20$ & $      5366.20$ & $         1.20$ sec    & $       2.4523$  & $       0.6137$ \\ 
\cmidrule{1-1} 
           MCI-CCIFD & $      5858.00$ & $      5722.56$ & $      5350.83$ & $      5350.83$ & $      5350.83$ & $      5350.83$ & $      5350.83$ & $      5350.83$ & $         2.65$ sec    & $       2.3888$  & $       0.6136$ \\ 
\bottomrule
\end{tabular}
\end{table}

\begin{table}[H]
\scriptsize
\centering
\caption{image-seg (134035.bmp)}
\label{tab:anytimetable-image-seg-134035.bmp}
\begin{tabular}{lrrrrrrrrrrr}
\toprule
           algorithm &                                   \multicolumn{8}{c}{value} & \multicolumn{1}{c}{time}    & \multicolumn{1}{c}{VI}  & \multicolumn{1}{c}{RI} \\  
\cmidrule(lr){2-9}\cmidrule(lr){10-10} \cmidrule(lr){11-11} \cmidrule(lr){12-12}   
                     & \multicolumn{1}{c}{(0.5 sec)} & \multicolumn{1}{c}{(1 sec)} & \multicolumn{1}{c}{(10 sec)} & \multicolumn{1}{c}{(60 sec)} & \multicolumn{1}{c}{(300 sec)} & \multicolumn{1}{c}{(600 sec)} & \multicolumn{1}{c}{(1800 sec)} & \multicolumn{1}{c}{(end)} & \multicolumn{1}{c}{(end)}    & \multicolumn{1}{c}{(end)}   & \multicolumn{1}{c}{(end)}  \\ \midrule 
          PIVIT-BOEM & $\infty$ & $\infty$ & $\infty$ & $\infty$ & $      8833.43$ & $      8833.43$ & $      8833.43$ & $      8833.43$ & $        90.58$ sec    & $       6.9579$  & $       0.5864$ \\ 
                 CGC & $      6678.11$ & $      6674.28$ & $      6602.29$ & $      6590.58$ & $      6590.58$ & $      6590.58$ & $      6590.58$ & $      6590.58$ & $        29.29$ sec    & $       3.9122$  & $       0.5387$ \\ 
                  HC & $      7355.83$ & $      7355.83$ & $      7355.83$ & $      7355.83$ & $      7355.83$ & $      7355.83$ & $      7355.83$ & $      7355.83$ & $         0.01$ sec    & $       3.7491$  & $       0.5253$ \\ 
              HC-CGC & $      6709.05$ & $      6707.17$ & $      6601.89$ & $      6598.31$ & $      6598.31$ & $      6598.31$ & $      6598.31$ & $      6598.31$ & $        22.38$ sec    & $       3.9165$  & $       0.5395$ \\ 
              ogm-KL & $      9136.90$ & $      9136.90$ & $      6678.92$ & $      6678.92$ & $      6678.92$ & $      6678.92$ & $      6678.92$ & $      6678.92$ & $         8.47$ sec    & $       3.4611$  & $       0.5319$ \\ 
    CC-Fusion-HC-CGC & $      6586.92$ & $      6581.98$ & $      6581.38$ & $      6581.38$ & $      6581.38$ & $      6581.38$ & $      6581.38$ & $      6581.38$ & $         2.00$ sec    & $       3.9380$  & $       0.5405$ \\ 
     CC-Fusion-HC-MC & $      6610.55$ & $      6583.05$ & $      6579.13$ & $      6579.13$ & $      6579.13$ & $      6579.13$ & $      6579.13$ & $      6579.13$ & $         5.06$ sec    & $       3.9367$  & $       0.5419$ \\ 
    CC-Fusion-WS-CGC & $      6609.94$ & $      6604.73$ & $      6596.94$ & $      6596.94$ & $      6596.94$ & $      6596.94$ & $      6596.94$ & $      6596.94$ & $         2.19$ sec    & $       3.9030$  & $       0.5417$ \\ 
     CC-Fusion-WS-MC & $      6692.87$ & $      6619.96$ & $      6581.55$ & $      6581.55$ & $      6581.55$ & $      6581.55$ & $      6581.55$ & $      6581.55$ & $         9.05$ sec    & $       3.9369$  & $       0.5416$ \\ 
\cmidrule{1-1} 
           MCR-CCFDB & $      6699.59$ & $      6590.30$ & $      6590.30$ & $      6590.30$ & $      6590.30$ & $      6590.30$ & $      6590.30$ & $      6590.30$ & $         0.94$ sec    & $       3.9567$  & $       0.5420$ \\ 
\cmidrule{1-1} 
           MCI-CCIFD & $      6803.74$ & $      6638.86$ & $      6578.98$ & $      6578.98$ & $      6578.98$ & $      6578.98$ & $      6578.98$ & $      6578.98$ & $         4.24$ sec    & $       3.9131$  & $       0.5412$ \\ 
\bottomrule
\end{tabular}
\end{table}

\begin{table}[H]
\scriptsize
\centering
\caption{image-seg (14037.bmp)}
\label{tab:anytimetable-image-seg-14037.bmp}
\begin{tabular}{lrrrrrrrrrrr}
\toprule
           algorithm &                                   \multicolumn{8}{c}{value} & \multicolumn{1}{c}{time}    & \multicolumn{1}{c}{VI}  & \multicolumn{1}{c}{RI} \\  
\cmidrule(lr){2-9}\cmidrule(lr){10-10} \cmidrule(lr){11-11} \cmidrule(lr){12-12}   
                     & \multicolumn{1}{c}{(0.5 sec)} & \multicolumn{1}{c}{(1 sec)} & \multicolumn{1}{c}{(10 sec)} & \multicolumn{1}{c}{(60 sec)} & \multicolumn{1}{c}{(300 sec)} & \multicolumn{1}{c}{(600 sec)} & \multicolumn{1}{c}{(1800 sec)} & \multicolumn{1}{c}{(end)} & \multicolumn{1}{c}{(end)}    & \multicolumn{1}{c}{(end)}   & \multicolumn{1}{c}{(end)}  \\ \midrule 
          PIVIT-BOEM & $\infty$ & $      2049.04$ & $      2049.04$ & $      2049.04$ & $      2049.04$ & $      2049.04$ & $      2049.04$ & $      2049.04$ & $         0.57$ sec    & $       3.0931$  & $       0.8347$ \\ 
                 CGC & $      1385.23$ & $      1385.23$ & $      1385.23$ & $      1385.23$ & $      1385.23$ & $      1385.23$ & $      1385.23$ & $      1385.23$ & $         0.02$ sec    & $       1.2705$  & $       0.8898$ \\ 
                  HC & $      1527.15$ & $      1527.15$ & $      1527.15$ & $      1527.15$ & $      1527.15$ & $      1527.15$ & $      1527.15$ & $      1527.15$ & $         0.00$ sec    & $       1.2240$  & $       0.8952$ \\ 
              HC-CGC & $      1389.67$ & $      1389.67$ & $      1389.67$ & $      1389.67$ & $      1389.67$ & $      1389.67$ & $      1389.67$ & $      1389.67$ & $         0.02$ sec    & $       1.2535$  & $       0.8917$ \\ 
              ogm-KL & $      1444.67$ & $      1444.67$ & $      1444.67$ & $      1444.67$ & $      1444.67$ & $      1444.67$ & $      1444.67$ & $      1444.67$ & $         0.02$ sec    & $       2.2659$  & $       0.6622$ \\ 
    CC-Fusion-HC-CGC & $      1383.14$ & $      1383.14$ & $      1383.14$ & $      1383.14$ & $      1383.14$ & $      1383.14$ & $      1383.14$ & $      1383.14$ & $         0.13$ sec    & $       1.2867$  & $       0.8895$ \\ 
     CC-Fusion-HC-MC & $      1383.14$ & $      1383.14$ & $      1383.14$ & $      1383.14$ & $      1383.14$ & $      1383.14$ & $      1383.14$ & $      1383.14$ & $         0.88$ sec    & $       1.2867$  & $       0.8895$ \\ 
    CC-Fusion-WS-CGC & $      1383.14$ & $      1383.14$ & $      1383.14$ & $      1383.14$ & $      1383.14$ & $      1383.14$ & $      1383.14$ & $      1383.14$ & $         0.11$ sec    & $       1.2867$  & $       0.8895$ \\ 
     CC-Fusion-WS-MC & $      1383.14$ & $      1383.14$ & $      1383.14$ & $      1383.14$ & $      1383.14$ & $      1383.14$ & $      1383.14$ & $      1383.14$ & $         1.11$ sec    & $       1.2867$  & $       0.8895$ \\ 
\cmidrule{1-1} 
           MCR-CCFDB & $      1383.14$ & $      1383.14$ & $      1383.14$ & $      1383.14$ & $      1383.14$ & $      1383.14$ & $      1383.14$ & $      1383.14$ & $         0.06$ sec    & $       1.2867$  & $       0.8895$ \\ 
\cmidrule{1-1} 
           MCI-CCIFD & $      1383.14$ & $      1383.14$ & $      1383.14$ & $      1383.14$ & $      1383.14$ & $      1383.14$ & $      1383.14$ & $      1383.14$ & $         0.06$ sec    & $       1.2867$  & $       0.8895$ \\ 
\bottomrule
\end{tabular}
\end{table}

\begin{table}[H]
\scriptsize
\centering
\caption{image-seg (143090.bmp)}
\label{tab:anytimetable-image-seg-143090.bmp}
\begin{tabular}{lrrrrrrrrrrr}
\toprule
           algorithm &                                   \multicolumn{8}{c}{value} & \multicolumn{1}{c}{time}    & \multicolumn{1}{c}{VI}  & \multicolumn{1}{c}{RI} \\  
\cmidrule(lr){2-9}\cmidrule(lr){10-10} \cmidrule(lr){11-11} \cmidrule(lr){12-12}   
                     & \multicolumn{1}{c}{(0.5 sec)} & \multicolumn{1}{c}{(1 sec)} & \multicolumn{1}{c}{(10 sec)} & \multicolumn{1}{c}{(60 sec)} & \multicolumn{1}{c}{(300 sec)} & \multicolumn{1}{c}{(600 sec)} & \multicolumn{1}{c}{(1800 sec)} & \multicolumn{1}{c}{(end)} & \multicolumn{1}{c}{(end)}    & \multicolumn{1}{c}{(end)}   & \multicolumn{1}{c}{(end)}  \\ \midrule 
          PIVIT-BOEM & $\infty$ & $\infty$ & $      2357.83$ & $      2357.83$ & $      2357.83$ & $      2357.83$ & $      2357.83$ & $      2357.83$ & $         1.03$ sec    & $       2.9212$  & $       0.8294$ \\ 
                 CGC & $      1722.91$ & $      1722.91$ & $      1722.91$ & $      1722.91$ & $      1722.91$ & $      1722.91$ & $      1722.91$ & $      1722.91$ & $         0.05$ sec    & $       1.2508$  & $       0.8906$ \\ 
                  HC & $      1945.72$ & $      1945.72$ & $      1945.72$ & $      1945.72$ & $      1945.72$ & $      1945.72$ & $      1945.72$ & $      1945.72$ & $         0.00$ sec    & $       1.1706$  & $       0.8943$ \\ 
              HC-CGC & $      1723.71$ & $      1723.71$ & $      1723.71$ & $      1723.71$ & $      1723.71$ & $      1723.71$ & $      1723.71$ & $      1723.71$ & $         0.05$ sec    & $       1.2774$  & $       0.8893$ \\ 
              ogm-KL & $      1784.06$ & $      1784.06$ & $      1784.06$ & $      1784.06$ & $      1784.06$ & $      1784.06$ & $      1784.06$ & $      1784.06$ & $         0.04$ sec    & $       2.0059$  & $       0.6847$ \\ 
    CC-Fusion-HC-CGC & $      1720.50$ & $      1720.50$ & $      1720.50$ & $      1720.50$ & $      1720.50$ & $      1720.50$ & $      1720.50$ & $      1720.50$ & $         0.21$ sec    & $       1.2404$  & $       0.8901$ \\ 
     CC-Fusion-HC-MC & $      1720.30$ & $      1714.38$ & $      1714.38$ & $      1714.38$ & $      1714.38$ & $      1714.38$ & $      1714.38$ & $      1714.38$ & $         1.62$ sec    & $       1.3168$  & $       0.8878$ \\ 
    CC-Fusion-WS-CGC & $      1717.16$ & $      1714.54$ & $      1714.54$ & $      1714.54$ & $      1714.54$ & $      1714.54$ & $      1714.54$ & $      1714.54$ & $         0.65$ sec    & $       1.2963$  & $       0.8879$ \\ 
     CC-Fusion-WS-MC & $      1719.98$ & $      1714.38$ & $      1714.38$ & $      1714.38$ & $      1714.38$ & $      1714.38$ & $      1714.38$ & $      1714.38$ & $         1.49$ sec    & $       1.3168$  & $       0.8878$ \\ 
\cmidrule{1-1} 
           MCR-CCFDB & $      1714.38$ & $      1714.38$ & $      1714.38$ & $      1714.38$ & $      1714.38$ & $      1714.38$ & $      1714.38$ & $      1714.38$ & $         0.04$ sec    & $       1.3168$  & $       0.8878$ \\ 
\cmidrule{1-1} 
           MCI-CCIFD & $      1714.38$ & $      1714.38$ & $      1714.38$ & $      1714.38$ & $      1714.38$ & $      1714.38$ & $      1714.38$ & $      1714.38$ & $         0.24$ sec    & $       1.3168$  & $       0.8878$ \\ 
\bottomrule
\end{tabular}
\end{table}

\begin{table}[H]
\scriptsize
\centering
\caption{image-seg (145086.bmp)}
\label{tab:anytimetable-image-seg-145086.bmp}
\begin{tabular}{lrrrrrrrrrrr}
\toprule
           algorithm &                                   \multicolumn{8}{c}{value} & \multicolumn{1}{c}{time}    & \multicolumn{1}{c}{VI}  & \multicolumn{1}{c}{RI} \\  
\cmidrule(lr){2-9}\cmidrule(lr){10-10} \cmidrule(lr){11-11} \cmidrule(lr){12-12}   
                     & \multicolumn{1}{c}{(0.5 sec)} & \multicolumn{1}{c}{(1 sec)} & \multicolumn{1}{c}{(10 sec)} & \multicolumn{1}{c}{(60 sec)} & \multicolumn{1}{c}{(300 sec)} & \multicolumn{1}{c}{(600 sec)} & \multicolumn{1}{c}{(1800 sec)} & \multicolumn{1}{c}{(end)} & \multicolumn{1}{c}{(end)}    & \multicolumn{1}{c}{(end)}   & \multicolumn{1}{c}{(end)}  \\ \midrule 
          PIVIT-BOEM & $\infty$ & $\infty$ & $\infty$ & $      5192.72$ & $      5192.72$ & $      5192.72$ & $      5192.72$ & $      5192.72$ & $        12.16$ sec    & $       4.1520$  & $       0.8125$ \\ 
                 CGC & $      3337.96$ & $      3337.96$ & $      3337.96$ & $      3337.96$ & $      3337.96$ & $      3337.96$ & $      3337.96$ & $      3337.96$ & $         0.04$ sec    & $       1.3195$  & $       0.9109$ \\ 
                  HC & $      3446.29$ & $      3446.29$ & $      3446.29$ & $      3446.29$ & $      3446.29$ & $      3446.29$ & $      3446.29$ & $      3446.29$ & $         0.00$ sec    & $       1.6429$  & $       0.8911$ \\ 
              HC-CGC & $      3330.38$ & $      3330.38$ & $      3330.38$ & $      3330.38$ & $      3330.38$ & $      3330.38$ & $      3330.38$ & $      3330.38$ & $         0.03$ sec    & $       1.2445$  & $       0.9172$ \\ 
              ogm-KL & $      3393.28$ & $      3393.28$ & $      3393.28$ & $      3393.28$ & $      3393.28$ & $      3393.28$ & $      3393.28$ & $      3393.28$ & $         0.22$ sec    & $       1.8228$  & $       0.8599$ \\ 
    CC-Fusion-HC-CGC & $      3323.53$ & $      3323.53$ & $      3323.53$ & $      3323.53$ & $      3323.53$ & $      3323.53$ & $      3323.53$ & $      3323.53$ & $         0.67$ sec    & $       1.4601$  & $       0.9016$ \\ 
     CC-Fusion-HC-MC & $      3323.00$ & $      3322.51$ & $      3322.51$ & $      3322.51$ & $      3322.51$ & $      3322.51$ & $      3322.51$ & $      3322.51$ & $         1.73$ sec    & $       1.4835$  & $       0.9010$ \\ 
    CC-Fusion-WS-CGC & $      3325.68$ & $      3325.68$ & $      3325.68$ & $      3325.68$ & $      3325.68$ & $      3325.68$ & $      3325.68$ & $      3325.68$ & $         0.49$ sec    & $       1.4822$  & $       0.8977$ \\ 
     CC-Fusion-WS-MC & $      3322.51$ & $      3322.51$ & $      3322.21$ & $      3322.21$ & $      3322.21$ & $      3322.21$ & $      3322.21$ & $      3322.21$ & $         2.51$ sec    & $       1.5486$  & $       0.8971$ \\ 
\cmidrule{1-1} 
           MCR-CCFDB & $      3322.21$ & $      3322.21$ & $      3322.21$ & $      3322.21$ & $      3322.21$ & $      3322.21$ & $      3322.21$ & $      3322.21$ & $         0.04$ sec    & $       1.5486$  & $       0.8971$ \\ 
\cmidrule{1-1} 
           MCI-CCIFD & $      3322.21$ & $      3322.21$ & $      3322.21$ & $      3322.21$ & $      3322.21$ & $      3322.21$ & $      3322.21$ & $      3322.21$ & $         0.22$ sec    & $       1.5486$  & $       0.8971$ \\ 
\bottomrule
\end{tabular}
\end{table}

\begin{table}[H]
\scriptsize
\centering
\caption{image-seg (147091.bmp)}
\label{tab:anytimetable-image-seg-147091.bmp}
\begin{tabular}{lrrrrrrrrrrr}
\toprule
           algorithm &                                   \multicolumn{8}{c}{value} & \multicolumn{1}{c}{time}    & \multicolumn{1}{c}{VI}  & \multicolumn{1}{c}{RI} \\  
\cmidrule(lr){2-9}\cmidrule(lr){10-10} \cmidrule(lr){11-11} \cmidrule(lr){12-12}   
                     & \multicolumn{1}{c}{(0.5 sec)} & \multicolumn{1}{c}{(1 sec)} & \multicolumn{1}{c}{(10 sec)} & \multicolumn{1}{c}{(60 sec)} & \multicolumn{1}{c}{(300 sec)} & \multicolumn{1}{c}{(600 sec)} & \multicolumn{1}{c}{(1800 sec)} & \multicolumn{1}{c}{(end)} & \multicolumn{1}{c}{(end)}    & \multicolumn{1}{c}{(end)}   & \multicolumn{1}{c}{(end)}  \\ \midrule 
          PIVIT-BOEM & $\infty$ & $\infty$ & $\infty$ & $      5311.98$ & $      5311.98$ & $      5311.98$ & $      5311.98$ & $      5311.98$ & $        15.46$ sec    & $       4.5198$  & $       0.6600$ \\ 
                 CGC & $      3995.30$ & $      3995.30$ & $      3995.30$ & $      3995.30$ & $      3995.30$ & $      3995.30$ & $      3995.30$ & $      3995.30$ & $         0.32$ sec    & $       1.4691$  & $       0.9028$ \\ 
                  HC & $      4267.63$ & $      4267.63$ & $      4267.63$ & $      4267.63$ & $      4267.63$ & $      4267.63$ & $      4267.63$ & $      4267.63$ & $         0.00$ sec    & $       1.5907$  & $       0.8931$ \\ 
              HC-CGC & $      3992.55$ & $      3987.68$ & $      3987.68$ & $      3987.68$ & $      3987.68$ & $      3987.68$ & $      3987.68$ & $      3987.68$ & $         1.04$ sec    & $       1.6124$  & $       0.8824$ \\ 
              ogm-KL & $      4085.05$ & $      4085.05$ & $      4085.05$ & $      4085.05$ & $      4085.05$ & $      4085.05$ & $      4085.05$ & $      4085.05$ & $         0.65$ sec    & $       1.8671$  & $       0.7421$ \\ 
    CC-Fusion-HC-CGC & $      3988.71$ & $      3988.71$ & $      3988.71$ & $      3988.71$ & $      3988.71$ & $      3988.71$ & $      3988.71$ & $      3988.71$ & $         0.50$ sec    & $       1.5529$  & $       0.8834$ \\ 
     CC-Fusion-HC-MC & $      3978.39$ & $      3978.39$ & $      3975.15$ & $      3975.15$ & $      3975.15$ & $      3975.15$ & $      3975.15$ & $      3975.15$ & $         3.92$ sec    & $       1.5645$  & $       0.8944$ \\ 
    CC-Fusion-WS-CGC & $      3999.51$ & $      3993.72$ & $      3993.72$ & $      3993.72$ & $      3993.72$ & $      3993.72$ & $      3993.72$ & $      3993.72$ & $         0.86$ sec    & $       1.5447$  & $       0.8947$ \\ 
     CC-Fusion-WS-MC & $      4009.08$ & $      3984.31$ & $      3973.71$ & $      3973.71$ & $      3973.71$ & $      3973.71$ & $      3973.71$ & $      3973.71$ & $         5.42$ sec    & $       1.6036$  & $       0.8933$ \\ 
\cmidrule{1-1} 
           MCR-CCFDB & $      3976.61$ & $      3976.61$ & $      3976.61$ & $      3976.61$ & $      3976.61$ & $      3976.61$ & $      3976.61$ & $      3976.61$ & $         0.22$ sec    & $       1.6149$  & $       0.8932$ \\ 
\cmidrule{1-1} 
           MCI-CCIFD & $      4007.45$ & $      3973.71$ & $      3973.71$ & $      3973.71$ & $      3973.71$ & $      3973.71$ & $      3973.71$ & $      3973.71$ & $         0.89$ sec    & $       1.6036$  & $       0.8933$ \\ 
\bottomrule
\end{tabular}
\end{table}

\begin{table}[H]
\scriptsize
\centering
\caption{image-seg (148026.bmp)}
\label{tab:anytimetable-image-seg-148026.bmp}
\begin{tabular}{lrrrrrrrrrrr}
\toprule
           algorithm &                                   \multicolumn{8}{c}{value} & \multicolumn{1}{c}{time}    & \multicolumn{1}{c}{VI}  & \multicolumn{1}{c}{RI} \\  
\cmidrule(lr){2-9}\cmidrule(lr){10-10} \cmidrule(lr){11-11} \cmidrule(lr){12-12}   
                     & \multicolumn{1}{c}{(0.5 sec)} & \multicolumn{1}{c}{(1 sec)} & \multicolumn{1}{c}{(10 sec)} & \multicolumn{1}{c}{(60 sec)} & \multicolumn{1}{c}{(300 sec)} & \multicolumn{1}{c}{(600 sec)} & \multicolumn{1}{c}{(1800 sec)} & \multicolumn{1}{c}{(end)} & \multicolumn{1}{c}{(end)}    & \multicolumn{1}{c}{(end)}   & \multicolumn{1}{c}{(end)}  \\ \midrule 
          PIVIT-BOEM & $\infty$ & $\infty$ & $\infty$ & $\infty$ & $     10358.57$ & $     10358.57$ & $     10358.57$ & $     10358.57$ & $       156.50$ sec    & $       6.4461$  & $       0.7616$ \\ 
                 CGC & $      8233.45$ & $      8233.45$ & $      8233.45$ & $      8233.45$ & $      8233.45$ & $      8233.45$ & $      8233.45$ & $      8233.45$ & $         0.48$ sec    & $       3.7507$  & $       0.7976$ \\ 
                  HC & $      8708.30$ & $      8708.30$ & $      8708.30$ & $      8708.30$ & $      8708.30$ & $      8708.30$ & $      8708.30$ & $      8708.30$ & $         0.01$ sec    & $       3.7371$  & $       0.7926$ \\ 
              HC-CGC & $      8225.98$ & $      8225.98$ & $      8225.98$ & $      8225.98$ & $      8225.98$ & $      8225.98$ & $      8225.98$ & $      8225.98$ & $         0.30$ sec    & $       3.7763$  & $       0.7967$ \\ 
              ogm-KL & $     11005.86$ & $      8484.22$ & $      8463.81$ & $      8463.81$ & $      8463.81$ & $      8463.81$ & $      8463.81$ & $      8463.81$ & $         1.66$ sec    & $       3.3414$  & $       0.7640$ \\ 
    CC-Fusion-HC-CGC & $      8219.45$ & $      8217.53$ & $      8217.53$ & $      8217.53$ & $      8217.53$ & $      8217.53$ & $      8217.53$ & $      8217.53$ & $         1.56$ sec    & $       3.7010$  & $       0.8063$ \\ 
     CC-Fusion-HC-MC & $      8275.54$ & $      8226.19$ & $      8205.98$ & $      8205.98$ & $      8205.98$ & $      8205.98$ & $      8205.98$ & $      8205.98$ & $         7.12$ sec    & $       3.7380$  & $       0.8065$ \\ 
    CC-Fusion-WS-CGC & $      8260.58$ & $      8249.21$ & $      8247.72$ & $      8247.72$ & $      8247.72$ & $      8247.72$ & $      8247.72$ & $      8247.72$ & $         1.84$ sec    & $       3.7103$  & $       0.8064$ \\ 
     CC-Fusion-WS-MC & $      8391.06$ & $      8274.49$ & $      8206.03$ & $      8206.03$ & $      8206.03$ & $      8206.03$ & $      8206.03$ & $      8206.03$ & $        12.08$ sec    & $       3.7371$  & $       0.8065$ \\ 
\cmidrule{1-1} 
           MCR-CCFDB & $      8208.02$ & $      8208.02$ & $      8208.02$ & $      8208.02$ & $      8208.02$ & $      8208.02$ & $      8208.02$ & $      8208.02$ & $         0.24$ sec    & $       3.7392$  & $       0.8065$ \\ 
\cmidrule{1-1} 
           MCI-CCIFD & $      8212.83$ & $      8205.98$ & $      8205.98$ & $      8205.98$ & $      8205.98$ & $      8205.98$ & $      8205.98$ & $      8205.98$ & $         0.86$ sec    & $       3.7380$  & $       0.8065$ \\ 
\bottomrule
\end{tabular}
\end{table}

\begin{table}[H]
\scriptsize
\centering
\caption{image-seg (148089.bmp)}
\label{tab:anytimetable-image-seg-148089.bmp}
\begin{tabular}{lrrrrrrrrrrr}
\toprule
           algorithm &                                   \multicolumn{8}{c}{value} & \multicolumn{1}{c}{time}    & \multicolumn{1}{c}{VI}  & \multicolumn{1}{c}{RI} \\  
\cmidrule(lr){2-9}\cmidrule(lr){10-10} \cmidrule(lr){11-11} \cmidrule(lr){12-12}   
                     & \multicolumn{1}{c}{(0.5 sec)} & \multicolumn{1}{c}{(1 sec)} & \multicolumn{1}{c}{(10 sec)} & \multicolumn{1}{c}{(60 sec)} & \multicolumn{1}{c}{(300 sec)} & \multicolumn{1}{c}{(600 sec)} & \multicolumn{1}{c}{(1800 sec)} & \multicolumn{1}{c}{(end)} & \multicolumn{1}{c}{(end)}    & \multicolumn{1}{c}{(end)}   & \multicolumn{1}{c}{(end)}  \\ \midrule 
          PIVIT-BOEM & $\infty$ & $\infty$ & $\infty$ & $\infty$ & $      8630.87$ & $      8630.87$ & $      8630.87$ & $      8630.87$ & $        75.76$ sec    & $       6.4171$  & $       0.7751$ \\ 
                 CGC & $      6583.57$ & $      6466.63$ & $      6457.36$ & $      6457.36$ & $      6457.36$ & $      6457.36$ & $      6457.36$ & $      6457.36$ & $         2.14$ sec    & $       3.9737$  & $       0.7717$ \\ 
                  HC & $      7061.69$ & $      7061.69$ & $      7061.69$ & $      7061.69$ & $      7061.69$ & $      7061.69$ & $      7061.69$ & $      7061.69$ & $         0.01$ sec    & $       4.1278$  & $       0.7845$ \\ 
              HC-CGC & $      6458.50$ & $      6455.85$ & $      6455.85$ & $      6455.85$ & $      6455.85$ & $      6455.85$ & $      6455.85$ & $      6455.85$ & $         1.02$ sec    & $       3.9547$  & $       0.7769$ \\ 
              ogm-KL & $      8715.56$ & $      8715.56$ & $      6691.00$ & $      6691.00$ & $      6691.00$ & $      6691.00$ & $      6691.00$ & $      6691.00$ & $         4.57$ sec    & $       3.9201$  & $       0.5529$ \\ 
    CC-Fusion-HC-CGC & $      6448.72$ & $      6445.96$ & $      6444.72$ & $      6444.72$ & $      6444.72$ & $      6444.72$ & $      6444.72$ & $      6444.72$ & $         1.76$ sec    & $       3.7969$  & $       0.8016$ \\ 
     CC-Fusion-HC-MC & $      6444.64$ & $      6439.58$ & $      6439.58$ & $      6439.58$ & $      6439.58$ & $      6439.58$ & $      6439.58$ & $      6439.58$ & $         3.57$ sec    & $       3.8431$  & $       0.8008$ \\ 
    CC-Fusion-WS-CGC & $      6474.84$ & $      6467.71$ & $      6467.71$ & $      6467.71$ & $      6467.71$ & $      6467.71$ & $      6467.71$ & $      6467.71$ & $         1.27$ sec    & $       3.8231$  & $       0.7997$ \\ 
     CC-Fusion-WS-MC & $      6670.96$ & $      6501.31$ & $      6440.12$ & $      6440.12$ & $      6440.12$ & $      6440.12$ & $      6440.12$ & $      6440.12$ & $        10.80$ sec    & $       3.8418$  & $       0.8008$ \\ 
\cmidrule{1-1} 
           MCR-CCFDB & $      6442.79$ & $      6442.57$ & $      6442.57$ & $      6442.57$ & $      6442.57$ & $      6442.57$ & $      6442.57$ & $      6442.57$ & $         0.51$ sec    & $       3.8412$  & $       0.8008$ \\ 
\cmidrule{1-1} 
           MCI-CCIFD & $      6645.42$ & $      6478.51$ & $      6439.58$ & $      6439.58$ & $      6439.58$ & $      6439.58$ & $      6439.58$ & $      6439.58$ & $         1.98$ sec    & $       3.8431$  & $       0.8008$ \\ 
\bottomrule
\end{tabular}
\end{table}

\begin{table}[H]
\scriptsize
\centering
\caption{image-seg (156065.bmp)}
\label{tab:anytimetable-image-seg-156065.bmp}
\begin{tabular}{lrrrrrrrrrrr}
\toprule
           algorithm &                                   \multicolumn{8}{c}{value} & \multicolumn{1}{c}{time}    & \multicolumn{1}{c}{VI}  & \multicolumn{1}{c}{RI} \\  
\cmidrule(lr){2-9}\cmidrule(lr){10-10} \cmidrule(lr){11-11} \cmidrule(lr){12-12}   
                     & \multicolumn{1}{c}{(0.5 sec)} & \multicolumn{1}{c}{(1 sec)} & \multicolumn{1}{c}{(10 sec)} & \multicolumn{1}{c}{(60 sec)} & \multicolumn{1}{c}{(300 sec)} & \multicolumn{1}{c}{(600 sec)} & \multicolumn{1}{c}{(1800 sec)} & \multicolumn{1}{c}{(end)} & \multicolumn{1}{c}{(end)}    & \multicolumn{1}{c}{(end)}   & \multicolumn{1}{c}{(end)}  \\ \midrule 
          PIVIT-BOEM & $\infty$ & $\infty$ & $\infty$ & $      7980.57$ & $      7980.57$ & $      7980.57$ & $      7980.57$ & $      7980.57$ & $        45.34$ sec    & $       6.4751$  & $       0.5609$ \\ 
                 CGC & $      5347.35$ & $      5298.28$ & $      5294.51$ & $      5294.51$ & $      5294.51$ & $      5294.51$ & $      5294.51$ & $      5294.51$ & $         2.28$ sec    & $       2.4257$  & $       0.6898$ \\ 
                  HC & $      5706.16$ & $      5706.16$ & $      5706.16$ & $      5706.16$ & $      5706.16$ & $      5706.16$ & $      5706.16$ & $      5706.16$ & $         0.01$ sec    & $       3.1844$  & $       0.6115$ \\ 
              HC-CGC & $      5303.51$ & $      5293.58$ & $      5285.52$ & $      5285.52$ & $      5285.52$ & $      5285.52$ & $      5285.52$ & $      5285.52$ & $         2.60$ sec    & $       2.4888$  & $       0.6705$ \\ 
              ogm-KL & $      6806.07$ & $      5439.05$ & $      5418.66$ & $      5418.66$ & $      5418.66$ & $      5418.66$ & $      5418.66$ & $      5418.66$ & $         2.49$ sec    & $       2.4145$  & $       0.5389$ \\ 
    CC-Fusion-HC-CGC & $      5268.73$ & $      5253.04$ & $      5253.04$ & $      5253.04$ & $      5253.04$ & $      5253.04$ & $      5253.04$ & $      5253.04$ & $         1.11$ sec    & $       2.7943$  & $       0.6454$ \\ 
     CC-Fusion-HC-MC & $      5243.29$ & $      5241.11$ & $      5234.15$ & $      5234.15$ & $      5234.15$ & $      5234.15$ & $      5234.15$ & $      5234.15$ & $         4.49$ sec    & $       2.8637$  & $       0.6428$ \\ 
    CC-Fusion-WS-CGC & $      5291.45$ & $      5291.45$ & $      5291.45$ & $      5291.45$ & $      5291.45$ & $      5291.45$ & $      5291.45$ & $      5291.45$ & $         0.83$ sec    & $       3.0040$  & $       0.6363$ \\ 
     CC-Fusion-WS-MC & $      5361.37$ & $      5275.39$ & $      5234.15$ & $      5234.15$ & $      5234.15$ & $      5234.15$ & $      5234.15$ & $      5234.15$ & $         7.52$ sec    & $       2.8637$  & $       0.6428$ \\ 
\cmidrule{1-1} 
           MCR-CCFDB & $      5552.39$ & $      5236.79$ & $      5236.79$ & $      5236.79$ & $      5236.79$ & $      5236.79$ & $      5236.79$ & $      5236.79$ & $         0.69$ sec    & $       2.8650$  & $       0.6428$ \\ 
\cmidrule{1-1} 
           MCI-CCIFD & $      5252.30$ & $      5239.36$ & $      5234.15$ & $      5234.15$ & $      5234.15$ & $      5234.15$ & $      5234.15$ & $      5234.15$ & $         3.17$ sec    & $       2.8637$  & $       0.6428$ \\ 
\bottomrule
\end{tabular}
\end{table}

\begin{table}[H]
\scriptsize
\centering
\caption{image-seg (157055.bmp)}
\label{tab:anytimetable-image-seg-157055.bmp}
\begin{tabular}{lrrrrrrrrrrr}
\toprule
           algorithm &                                   \multicolumn{8}{c}{value} & \multicolumn{1}{c}{time}    & \multicolumn{1}{c}{VI}  & \multicolumn{1}{c}{RI} \\  
\cmidrule(lr){2-9}\cmidrule(lr){10-10} \cmidrule(lr){11-11} \cmidrule(lr){12-12}   
                     & \multicolumn{1}{c}{(0.5 sec)} & \multicolumn{1}{c}{(1 sec)} & \multicolumn{1}{c}{(10 sec)} & \multicolumn{1}{c}{(60 sec)} & \multicolumn{1}{c}{(300 sec)} & \multicolumn{1}{c}{(600 sec)} & \multicolumn{1}{c}{(1800 sec)} & \multicolumn{1}{c}{(end)} & \multicolumn{1}{c}{(end)}    & \multicolumn{1}{c}{(end)}   & \multicolumn{1}{c}{(end)}  \\ \midrule 
          PIVIT-BOEM & $\infty$ & $\infty$ & $\infty$ & $      6009.12$ & $      6009.12$ & $      6009.12$ & $      6009.12$ & $      6009.12$ & $        27.57$ sec    & $       4.6000$  & $       0.8641$ \\ 
                 CGC & $      4694.94$ & $      4694.94$ & $      4694.94$ & $      4694.94$ & $      4694.94$ & $      4694.94$ & $      4694.94$ & $      4694.94$ & $         0.13$ sec    & $       3.0130$  & $       0.8709$ \\ 
                  HC & $      4997.22$ & $      4997.22$ & $      4997.22$ & $      4997.22$ & $      4997.22$ & $      4997.22$ & $      4997.22$ & $      4997.22$ & $         0.00$ sec    & $       2.9006$  & $       0.8841$ \\ 
              HC-CGC & $      4698.96$ & $      4698.96$ & $      4698.96$ & $      4698.96$ & $      4698.96$ & $      4698.96$ & $      4698.96$ & $      4698.96$ & $         0.08$ sec    & $       2.9241$  & $       0.8840$ \\ 
              ogm-KL & $      4798.78$ & $      4798.78$ & $      4798.78$ & $      4798.78$ & $      4798.78$ & $      4798.78$ & $      4798.78$ & $      4798.78$ & $         0.66$ sec    & $       3.1407$  & $       0.8354$ \\ 
    CC-Fusion-HC-CGC & $      4686.58$ & $      4685.17$ & $      4685.17$ & $      4685.17$ & $      4685.17$ & $      4685.17$ & $      4685.17$ & $      4685.17$ & $         1.06$ sec    & $       2.8787$  & $       0.8882$ \\ 
     CC-Fusion-HC-MC & $      4686.99$ & $      4686.43$ & $      4685.17$ & $      4685.17$ & $      4685.17$ & $      4685.17$ & $      4685.17$ & $      4685.17$ & $         2.47$ sec    & $       2.8834$  & $       0.8882$ \\ 
    CC-Fusion-WS-CGC & $      4687.26$ & $      4685.71$ & $      4685.24$ & $      4685.24$ & $      4685.24$ & $      4685.24$ & $      4685.24$ & $      4685.24$ & $         1.33$ sec    & $       2.8844$  & $       0.8882$ \\ 
     CC-Fusion-WS-MC & $      4689.03$ & $      4685.17$ & $      4685.17$ & $      4685.17$ & $      4685.17$ & $      4685.17$ & $      4685.17$ & $      4685.17$ & $         2.46$ sec    & $       2.8834$  & $       0.8882$ \\ 
\cmidrule{1-1} 
           MCR-CCFDB & $      4686.20$ & $      4686.20$ & $      4686.20$ & $      4686.20$ & $      4686.20$ & $      4686.20$ & $      4686.20$ & $      4686.20$ & $         0.07$ sec    & $       2.8884$  & $       0.8882$ \\ 
\cmidrule{1-1} 
           MCI-CCIFD & $      4685.21$ & $      4685.17$ & $      4685.17$ & $      4685.17$ & $      4685.17$ & $      4685.17$ & $      4685.17$ & $      4685.17$ & $         0.62$ sec    & $       2.8834$  & $       0.8882$ \\ 
\bottomrule
\end{tabular}
\end{table}

\begin{table}[H]
\scriptsize
\centering
\caption{image-seg (159008.bmp)}
\label{tab:anytimetable-image-seg-159008.bmp}
\begin{tabular}{lrrrrrrrrrrr}
\toprule
           algorithm &                                   \multicolumn{8}{c}{value} & \multicolumn{1}{c}{time}    & \multicolumn{1}{c}{VI}  & \multicolumn{1}{c}{RI} \\  
\cmidrule(lr){2-9}\cmidrule(lr){10-10} \cmidrule(lr){11-11} \cmidrule(lr){12-12}   
                     & \multicolumn{1}{c}{(0.5 sec)} & \multicolumn{1}{c}{(1 sec)} & \multicolumn{1}{c}{(10 sec)} & \multicolumn{1}{c}{(60 sec)} & \multicolumn{1}{c}{(300 sec)} & \multicolumn{1}{c}{(600 sec)} & \multicolumn{1}{c}{(1800 sec)} & \multicolumn{1}{c}{(end)} & \multicolumn{1}{c}{(end)}    & \multicolumn{1}{c}{(end)}   & \multicolumn{1}{c}{(end)}  \\ \midrule 
          PIVIT-BOEM & $\infty$ & $\infty$ & $\infty$ & $      5873.46$ & $      5873.46$ & $      5873.46$ & $      5873.46$ & $      5873.46$ & $        29.68$ sec    & $       5.8549$  & $       0.7164$ \\ 
                 CGC & $      4556.59$ & $      4556.59$ & $      4556.59$ & $      4556.59$ & $      4556.59$ & $      4556.59$ & $      4556.59$ & $      4556.59$ & $         0.24$ sec    & $       3.6424$  & $       0.7338$ \\ 
                  HC & $      4973.51$ & $      4973.51$ & $      4973.51$ & $      4973.51$ & $      4973.51$ & $      4973.51$ & $      4973.51$ & $      4973.51$ & $         0.00$ sec    & $       3.6939$  & $       0.7297$ \\ 
              HC-CGC & $      4554.56$ & $      4554.56$ & $      4554.56$ & $      4554.56$ & $      4554.56$ & $      4554.56$ & $      4554.56$ & $      4554.56$ & $         0.13$ sec    & $       3.6638$  & $       0.7311$ \\ 
              ogm-KL & $      4841.43$ & $      4798.82$ & $      4793.11$ & $      4793.11$ & $      4793.11$ & $      4793.11$ & $      4793.11$ & $      4793.11$ & $         2.17$ sec    & $       3.6426$  & $       0.6126$ \\ 
    CC-Fusion-HC-CGC & $      4547.83$ & $      4547.05$ & $      4547.05$ & $      4547.05$ & $      4547.05$ & $      4547.05$ & $      4547.05$ & $      4547.05$ & $         1.06$ sec    & $       3.7101$  & $       0.7310$ \\ 
     CC-Fusion-HC-MC & $      4555.24$ & $      4542.63$ & $      4540.87$ & $      4540.87$ & $      4540.87$ & $      4540.87$ & $      4540.87$ & $      4540.87$ & $         4.27$ sec    & $       3.7434$  & $       0.7342$ \\ 
    CC-Fusion-WS-CGC & $      4588.69$ & $      4579.72$ & $      4579.72$ & $      4579.72$ & $      4579.72$ & $      4579.72$ & $      4579.72$ & $      4579.72$ & $         0.91$ sec    & $       3.5606$  & $       0.7350$ \\ 
     CC-Fusion-WS-MC & $      4554.84$ & $      4543.74$ & $      4542.52$ & $      4542.52$ & $      4542.52$ & $      4542.52$ & $      4542.52$ & $      4542.52$ & $         4.64$ sec    & $       3.7406$  & $       0.7342$ \\ 
\cmidrule{1-1} 
           MCR-CCFDB & $      4545.66$ & $      4545.66$ & $      4545.66$ & $      4545.66$ & $      4545.66$ & $      4545.66$ & $      4545.66$ & $      4545.66$ & $         0.17$ sec    & $       3.7414$  & $       0.7342$ \\ 
\cmidrule{1-1} 
           MCI-CCIFD & $      4656.16$ & $      4544.28$ & $      4540.87$ & $      4540.87$ & $      4540.87$ & $      4540.87$ & $      4540.87$ & $      4540.87$ & $         1.45$ sec    & $       3.7434$  & $       0.7342$ \\ 
\bottomrule
\end{tabular}
\end{table}

\begin{table}[H]
\scriptsize
\centering
\caption{image-seg (160068.bmp)}
\label{tab:anytimetable-image-seg-160068.bmp}
\begin{tabular}{lrrrrrrrrrrr}
\toprule
           algorithm &                                   \multicolumn{8}{c}{value} & \multicolumn{1}{c}{time}    & \multicolumn{1}{c}{VI}  & \multicolumn{1}{c}{RI} \\  
\cmidrule(lr){2-9}\cmidrule(lr){10-10} \cmidrule(lr){11-11} \cmidrule(lr){12-12}   
                     & \multicolumn{1}{c}{(0.5 sec)} & \multicolumn{1}{c}{(1 sec)} & \multicolumn{1}{c}{(10 sec)} & \multicolumn{1}{c}{(60 sec)} & \multicolumn{1}{c}{(300 sec)} & \multicolumn{1}{c}{(600 sec)} & \multicolumn{1}{c}{(1800 sec)} & \multicolumn{1}{c}{(end)} & \multicolumn{1}{c}{(end)}    & \multicolumn{1}{c}{(end)}   & \multicolumn{1}{c}{(end)}  \\ \midrule 
          PIVIT-BOEM & $\infty$ & $\infty$ & $      4182.67$ & $      4182.67$ & $      4182.67$ & $      4182.67$ & $      4182.67$ & $      4182.67$ & $         6.63$ sec    & $       3.2572$  & $       0.8895$ \\ 
                 CGC & $      3100.79$ & $      3100.79$ & $      3100.79$ & $      3100.79$ & $      3100.79$ & $      3100.79$ & $      3100.79$ & $      3100.79$ & $         0.32$ sec    & $       1.8613$  & $       0.9005$ \\ 
                  HC & $      3444.01$ & $      3444.01$ & $      3444.01$ & $      3444.01$ & $      3444.01$ & $      3444.01$ & $      3444.01$ & $      3444.01$ & $         0.00$ sec    & $       1.8729$  & $       0.9088$ \\ 
              HC-CGC & $      3092.72$ & $      3092.72$ & $      3092.72$ & $      3092.72$ & $      3092.72$ & $      3092.72$ & $      3092.72$ & $      3092.72$ & $         0.26$ sec    & $       1.8870$  & $       0.9009$ \\ 
              ogm-KL & $      3216.91$ & $      3216.91$ & $      3216.91$ & $      3216.91$ & $      3216.91$ & $      3216.91$ & $      3216.91$ & $      3216.91$ & $         0.31$ sec    & $       2.5831$  & $       0.6388$ \\ 
    CC-Fusion-HC-CGC & $      3097.42$ & $      3097.42$ & $      3097.42$ & $      3097.42$ & $      3097.42$ & $      3097.42$ & $      3097.42$ & $      3097.42$ & $         0.36$ sec    & $       1.9039$  & $       0.9047$ \\ 
     CC-Fusion-HC-MC & $      3091.38$ & $      3091.38$ & $      3089.32$ & $      3089.32$ & $      3089.32$ & $      3089.32$ & $      3089.32$ & $      3089.32$ & $         2.47$ sec    & $       1.8716$  & $       0.9009$ \\ 
    CC-Fusion-WS-CGC & $      3102.83$ & $      3102.83$ & $      3102.83$ & $      3102.83$ & $      3102.83$ & $      3102.83$ & $      3102.83$ & $      3102.83$ & $         0.49$ sec    & $       1.8899$  & $       0.9007$ \\ 
     CC-Fusion-WS-MC & $      3097.07$ & $      3091.48$ & $      3091.48$ & $      3091.48$ & $      3091.48$ & $      3091.48$ & $      3091.48$ & $      3091.48$ & $         2.43$ sec    & $       1.9086$  & $       0.9046$ \\ 
\cmidrule{1-1} 
           MCR-CCFDB & $      3091.15$ & $      3091.15$ & $      3091.15$ & $      3091.15$ & $      3091.15$ & $      3091.15$ & $      3091.15$ & $      3091.15$ & $         0.16$ sec    & $       1.8829$  & $       0.9008$ \\ 
\cmidrule{1-1} 
           MCI-CCIFD & $      3089.32$ & $      3089.32$ & $      3089.32$ & $      3089.32$ & $      3089.32$ & $      3089.32$ & $      3089.32$ & $      3089.32$ & $         0.42$ sec    & $       1.8716$  & $       0.9009$ \\ 
\bottomrule
\end{tabular}
\end{table}

\begin{table}[H]
\scriptsize
\centering
\caption{image-seg (16077.bmp)}
\label{tab:anytimetable-image-seg-16077.bmp}
\begin{tabular}{lrrrrrrrrrrr}
\toprule
           algorithm &                                   \multicolumn{8}{c}{value} & \multicolumn{1}{c}{time}    & \multicolumn{1}{c}{VI}  & \multicolumn{1}{c}{RI} \\  
\cmidrule(lr){2-9}\cmidrule(lr){10-10} \cmidrule(lr){11-11} \cmidrule(lr){12-12}   
                     & \multicolumn{1}{c}{(0.5 sec)} & \multicolumn{1}{c}{(1 sec)} & \multicolumn{1}{c}{(10 sec)} & \multicolumn{1}{c}{(60 sec)} & \multicolumn{1}{c}{(300 sec)} & \multicolumn{1}{c}{(600 sec)} & \multicolumn{1}{c}{(1800 sec)} & \multicolumn{1}{c}{(end)} & \multicolumn{1}{c}{(end)}    & \multicolumn{1}{c}{(end)}   & \multicolumn{1}{c}{(end)}  \\ \midrule 
          PIVIT-BOEM & $\infty$ & $\infty$ & $\infty$ & $      5523.56$ & $      5523.56$ & $      5523.56$ & $      5523.56$ & $      5523.56$ & $        20.50$ sec    & $       5.5947$  & $       0.7775$ \\ 
                 CGC & $      4255.85$ & $      4255.85$ & $      4255.85$ & $      4255.85$ & $      4255.85$ & $      4255.85$ & $      4255.85$ & $      4255.85$ & $         0.25$ sec    & $       3.5034$  & $       0.7610$ \\ 
                  HC & $      4646.12$ & $      4646.12$ & $      4646.12$ & $      4646.12$ & $      4646.12$ & $      4646.12$ & $      4646.12$ & $      4646.12$ & $         0.00$ sec    & $       3.4908$  & $       0.7634$ \\ 
              HC-CGC & $      4244.79$ & $      4244.79$ & $      4244.79$ & $      4244.79$ & $      4244.79$ & $      4244.79$ & $      4244.79$ & $      4244.79$ & $         0.21$ sec    & $       3.6675$  & $       0.7547$ \\ 
              ogm-KL & $      4383.98$ & $      4382.90$ & $      4382.90$ & $      4382.90$ & $      4382.90$ & $      4382.90$ & $      4382.90$ & $      4382.90$ & $         0.74$ sec    & $       3.7870$  & $       0.5834$ \\ 
    CC-Fusion-HC-CGC & $      4233.06$ & $      4232.10$ & $      4232.10$ & $      4232.10$ & $      4232.10$ & $      4232.10$ & $      4232.10$ & $      4232.10$ & $         1.10$ sec    & $       3.6248$  & $       0.7521$ \\ 
     CC-Fusion-HC-MC & $      4231.67$ & $      4229.73$ & $      4227.88$ & $      4227.88$ & $      4227.88$ & $      4227.88$ & $      4227.88$ & $      4227.88$ & $         4.25$ sec    & $       3.6559$  & $       0.7523$ \\ 
    CC-Fusion-WS-CGC & $      4271.88$ & $      4265.35$ & $      4261.58$ & $      4261.58$ & $      4261.58$ & $      4261.58$ & $      4261.58$ & $      4261.58$ & $         1.68$ sec    & $       3.6230$  & $       0.7886$ \\ 
     CC-Fusion-WS-MC & $      4263.66$ & $      4232.64$ & $      4228.35$ & $      4228.35$ & $      4228.35$ & $      4228.35$ & $      4228.35$ & $      4228.35$ & $         3.19$ sec    & $       3.6602$  & $       0.7519$ \\ 
\cmidrule{1-1} 
           MCR-CCFDB & $      4230.49$ & $      4230.49$ & $      4230.49$ & $      4230.49$ & $      4230.49$ & $      4230.49$ & $      4230.49$ & $      4230.49$ & $         0.20$ sec    & $       3.6581$  & $       0.7522$ \\ 
\cmidrule{1-1} 
           MCI-CCIFD & $      4271.07$ & $      4227.88$ & $      4227.88$ & $      4227.88$ & $      4227.88$ & $      4227.88$ & $      4227.88$ & $      4227.88$ & $         0.62$ sec    & $       3.6559$  & $       0.7523$ \\ 
\bottomrule
\end{tabular}
\end{table}

\begin{table}[H]
\scriptsize
\centering
\caption{image-seg (163085.bmp)}
\label{tab:anytimetable-image-seg-163085.bmp}
\begin{tabular}{lrrrrrrrrrrr}
\toprule
           algorithm &                                   \multicolumn{8}{c}{value} & \multicolumn{1}{c}{time}    & \multicolumn{1}{c}{VI}  & \multicolumn{1}{c}{RI} \\  
\cmidrule(lr){2-9}\cmidrule(lr){10-10} \cmidrule(lr){11-11} \cmidrule(lr){12-12}   
                     & \multicolumn{1}{c}{(0.5 sec)} & \multicolumn{1}{c}{(1 sec)} & \multicolumn{1}{c}{(10 sec)} & \multicolumn{1}{c}{(60 sec)} & \multicolumn{1}{c}{(300 sec)} & \multicolumn{1}{c}{(600 sec)} & \multicolumn{1}{c}{(1800 sec)} & \multicolumn{1}{c}{(end)} & \multicolumn{1}{c}{(end)}    & \multicolumn{1}{c}{(end)}   & \multicolumn{1}{c}{(end)}  \\ \midrule 
          PIVIT-BOEM & $\infty$ & $\infty$ & $\infty$ & $      5707.74$ & $      5707.74$ & $      5707.74$ & $      5707.74$ & $      5707.74$ & $        18.31$ sec    & $       5.7992$  & $       0.7478$ \\ 
                 CGC & $      4533.51$ & $      4468.67$ & $      4415.74$ & $      4415.74$ & $      4415.74$ & $      4415.74$ & $      4415.74$ & $      4415.74$ & $         4.37$ sec    & $       3.5168$  & $       0.5235$ \\ 
                  HC & $      4862.02$ & $      4862.02$ & $      4862.02$ & $      4862.02$ & $      4862.02$ & $      4862.02$ & $      4862.02$ & $      4862.02$ & $         0.00$ sec    & $       3.2824$  & $       0.5849$ \\ 
              HC-CGC & $      4441.73$ & $      4425.20$ & $      4425.20$ & $      4425.20$ & $      4425.20$ & $      4425.20$ & $      4425.20$ & $      4425.20$ & $         0.99$ sec    & $       3.4789$  & $       0.5813$ \\ 
              ogm-KL & $      4562.73$ & $      4558.56$ & $      4558.56$ & $      4558.56$ & $      4558.56$ & $      4558.56$ & $      4558.56$ & $      4558.56$ & $         0.74$ sec    & $       3.0211$  & $       0.4362$ \\ 
    CC-Fusion-HC-CGC & $      4417.90$ & $      4417.90$ & $      4417.90$ & $      4417.90$ & $      4417.90$ & $      4417.90$ & $      4417.90$ & $      4417.90$ & $         0.92$ sec    & $       2.9835$  & $       0.7280$ \\ 
     CC-Fusion-HC-MC & $      4425.53$ & $      4390.79$ & $      4381.52$ & $      4381.52$ & $      4381.52$ & $      4381.52$ & $      4381.52$ & $      4381.52$ & $        12.96$ sec    & $       3.0889$  & $       0.7198$ \\ 
    CC-Fusion-WS-CGC & $      4487.11$ & $      4460.90$ & $      4460.90$ & $      4460.90$ & $      4460.90$ & $      4460.90$ & $      4460.90$ & $      4460.90$ & $         1.26$ sec    & $       3.3805$  & $       0.6567$ \\ 
     CC-Fusion-WS-MC & $      4579.44$ & $      4503.81$ & $      4381.21$ & $      4381.13$ & $      4381.13$ & $      4381.13$ & $      4381.13$ & $      4381.13$ & $        22.85$ sec    & $       3.1678$  & $       0.7179$ \\ 
\cmidrule{1-1} 
           MCR-CCFDB & $      4569.48$ & $      4392.93$ & $      4392.93$ & $      4392.93$ & $      4392.93$ & $      4392.93$ & $      4392.93$ & $      4392.93$ & $         0.65$ sec    & $       3.1989$  & $       0.7178$ \\ 
\cmidrule{1-1} 
           MCI-CCIFD & $      4545.13$ & $      4465.31$ & $      4381.13$ & $      4381.13$ & $      4381.13$ & $      4381.13$ & $      4381.13$ & $      4381.13$ & $         1.08$ sec    & $       3.1678$  & $       0.7179$ \\ 
\bottomrule
\end{tabular}
\end{table}

\begin{table}[H]
\scriptsize
\centering
\caption{image-seg (167062.bmp)}
\label{tab:anytimetable-image-seg-167062.bmp}
\begin{tabular}{lrrrrrrrrrrr}
\toprule
           algorithm &                                   \multicolumn{8}{c}{value} & \multicolumn{1}{c}{time}    & \multicolumn{1}{c}{VI}  & \multicolumn{1}{c}{RI} \\  
\cmidrule(lr){2-9}\cmidrule(lr){10-10} \cmidrule(lr){11-11} \cmidrule(lr){12-12}   
                     & \multicolumn{1}{c}{(0.5 sec)} & \multicolumn{1}{c}{(1 sec)} & \multicolumn{1}{c}{(10 sec)} & \multicolumn{1}{c}{(60 sec)} & \multicolumn{1}{c}{(300 sec)} & \multicolumn{1}{c}{(600 sec)} & \multicolumn{1}{c}{(1800 sec)} & \multicolumn{1}{c}{(end)} & \multicolumn{1}{c}{(end)}    & \multicolumn{1}{c}{(end)}   & \multicolumn{1}{c}{(end)}  \\ \midrule 
          PIVIT-BOEM & $\infty$ & $      2249.67$ & $      2249.67$ & $      2249.67$ & $      2249.67$ & $      2249.67$ & $      2249.67$ & $      2249.67$ & $         0.69$ sec    & $       2.8103$  & $       0.6598$ \\ 
                 CGC & $      1273.93$ & $      1273.93$ & $      1273.93$ & $      1273.93$ & $      1273.93$ & $      1273.93$ & $      1273.93$ & $      1273.93$ & $         0.02$ sec    & $       0.2529$  & $       0.9792$ \\ 
                  HC & $      1316.63$ & $      1316.63$ & $      1316.63$ & $      1316.63$ & $      1316.63$ & $      1316.63$ & $      1316.63$ & $      1316.63$ & $         0.00$ sec    & $       0.2481$  & $       0.9787$ \\ 
              HC-CGC & $      1273.78$ & $      1273.78$ & $      1273.78$ & $      1273.78$ & $      1273.78$ & $      1273.78$ & $      1273.78$ & $      1273.78$ & $         0.03$ sec    & $       0.2503$  & $       0.9792$ \\ 
              ogm-KL & $      1274.78$ & $      1274.78$ & $      1274.78$ & $      1274.78$ & $      1274.78$ & $      1274.78$ & $      1274.78$ & $      1274.78$ & $         0.04$ sec    & $       0.2947$  & $       0.9709$ \\ 
    CC-Fusion-HC-CGC & $      1273.72$ & $      1273.72$ & $      1273.72$ & $      1273.72$ & $      1273.72$ & $      1273.72$ & $      1273.72$ & $      1273.72$ & $         0.13$ sec    & $       0.2546$  & $       0.9787$ \\ 
     CC-Fusion-HC-MC & $      1273.72$ & $      1273.72$ & $      1273.72$ & $      1273.72$ & $      1273.72$ & $      1273.72$ & $      1273.72$ & $      1273.72$ & $         0.99$ sec    & $       0.2546$  & $       0.9787$ \\ 
    CC-Fusion-WS-CGC & $      1273.72$ & $      1273.72$ & $      1273.72$ & $      1273.72$ & $      1273.72$ & $      1273.72$ & $      1273.72$ & $      1273.72$ & $         0.08$ sec    & $       0.2546$  & $       0.9787$ \\ 
     CC-Fusion-WS-MC & $      1273.72$ & $      1273.72$ & $      1273.72$ & $      1273.72$ & $      1273.72$ & $      1273.72$ & $      1273.72$ & $      1273.72$ & $         1.23$ sec    & $       0.2546$  & $       0.9787$ \\ 
\cmidrule{1-1} 
           MCR-CCFDB & $      1274.19$ & $      1274.19$ & $      1274.19$ & $      1274.19$ & $      1274.19$ & $      1274.19$ & $      1274.19$ & $      1274.19$ & $         0.01$ sec    & $       0.2552$  & $       0.9787$ \\ 
\cmidrule{1-1} 
           MCI-CCIFD & $      1273.72$ & $      1273.72$ & $      1273.72$ & $      1273.72$ & $      1273.72$ & $      1273.72$ & $      1273.72$ & $      1273.72$ & $         0.17$ sec    & $       0.2546$  & $       0.9787$ \\ 
\bottomrule
\end{tabular}
\end{table}

\begin{table}[H]
\scriptsize
\centering
\caption{image-seg (167083.bmp)}
\label{tab:anytimetable-image-seg-167083.bmp}
\begin{tabular}{lrrrrrrrrrrr}
\toprule
           algorithm &                                   \multicolumn{8}{c}{value} & \multicolumn{1}{c}{time}    & \multicolumn{1}{c}{VI}  & \multicolumn{1}{c}{RI} \\  
\cmidrule(lr){2-9}\cmidrule(lr){10-10} \cmidrule(lr){11-11} \cmidrule(lr){12-12}   
                     & \multicolumn{1}{c}{(0.5 sec)} & \multicolumn{1}{c}{(1 sec)} & \multicolumn{1}{c}{(10 sec)} & \multicolumn{1}{c}{(60 sec)} & \multicolumn{1}{c}{(300 sec)} & \multicolumn{1}{c}{(600 sec)} & \multicolumn{1}{c}{(1800 sec)} & \multicolumn{1}{c}{(end)} & \multicolumn{1}{c}{(end)}    & \multicolumn{1}{c}{(end)}   & \multicolumn{1}{c}{(end)}  \\ \midrule 
          PIVIT-BOEM & $\infty$ & $\infty$ & $\infty$ & $\infty$ & $     12262.87$ & $     12262.87$ & $     12262.87$ & $     12262.87$ & $       167.32$ sec    & $       6.3080$  & $       0.7532$ \\ 
                 CGC & $      8503.85$ & $      8472.52$ & $      8360.25$ & $      8360.25$ & $      8360.25$ & $      8360.25$ & $      8360.25$ & $      8360.25$ & $         3.50$ sec    & $       2.7648$  & $       0.7724$ \\ 
                  HC & $      8977.05$ & $      8977.05$ & $      8977.05$ & $      8977.05$ & $      8977.05$ & $      8977.05$ & $      8977.05$ & $      8977.05$ & $         0.01$ sec    & $       2.5980$  & $       0.8199$ \\ 
              HC-CGC & $      8354.70$ & $      8346.31$ & $      8344.30$ & $      8344.30$ & $      8344.30$ & $      8344.30$ & $      8344.30$ & $      8344.30$ & $         1.38$ sec    & $       2.9703$  & $       0.7714$ \\ 
              ogm-KL & $     10993.43$ & $     10993.43$ & $      8572.76$ & $      8572.76$ & $      8572.76$ & $      8572.76$ & $      8572.76$ & $      8572.76$ & $         3.48$ sec    & $       2.6473$  & $       0.6127$ \\ 
    CC-Fusion-HC-CGC & $      8367.21$ & $      8365.94$ & $      8361.93$ & $      8361.93$ & $      8361.93$ & $      8361.93$ & $      8361.93$ & $      8361.93$ & $         2.35$ sec    & $       2.8402$  & $       0.7788$ \\ 
     CC-Fusion-HC-MC & $      8390.17$ & $      8353.66$ & $      8331.63$ & $      8331.63$ & $      8331.63$ & $      8331.63$ & $      8331.63$ & $      8331.63$ & $        11.07$ sec    & $       2.7377$  & $       0.8137$ \\ 
    CC-Fusion-WS-CGC & $      8419.79$ & $      8419.79$ & $      8419.79$ & $      8419.79$ & $      8419.79$ & $      8419.79$ & $      8419.79$ & $      8419.79$ & $         1.28$ sec    & $       3.0286$  & $       0.7642$ \\ 
     CC-Fusion-WS-MC & $      8497.80$ & $      8437.79$ & $      8332.10$ & $      8331.63$ & $      8331.63$ & $      8331.63$ & $      8331.63$ & $      8331.63$ & $        26.46$ sec    & $       2.7377$  & $       0.8137$ \\ 
\cmidrule{1-1} 
           MCR-CCFDB & $      9520.92$ & $      8333.88$ & $      8333.88$ & $      8333.88$ & $      8333.88$ & $      8333.88$ & $      8333.88$ & $      8333.88$ & $         0.85$ sec    & $       2.7379$  & $       0.8137$ \\ 
\cmidrule{1-1} 
           MCI-CCIFD & $      8555.05$ & $      8423.16$ & $      8331.63$ & $      8331.63$ & $      8331.63$ & $      8331.63$ & $      8331.63$ & $      8331.63$ & $         1.38$ sec    & $       2.7365$  & $       0.8137$ \\ 
\bottomrule
\end{tabular}
\end{table}

\begin{table}[H]
\scriptsize
\centering
\caption{image-seg (170057.bmp)}
\label{tab:anytimetable-image-seg-170057.bmp}
\begin{tabular}{lrrrrrrrrrrr}
\toprule
           algorithm &                                   \multicolumn{8}{c}{value} & \multicolumn{1}{c}{time}    & \multicolumn{1}{c}{VI}  & \multicolumn{1}{c}{RI} \\  
\cmidrule(lr){2-9}\cmidrule(lr){10-10} \cmidrule(lr){11-11} \cmidrule(lr){12-12}   
                     & \multicolumn{1}{c}{(0.5 sec)} & \multicolumn{1}{c}{(1 sec)} & \multicolumn{1}{c}{(10 sec)} & \multicolumn{1}{c}{(60 sec)} & \multicolumn{1}{c}{(300 sec)} & \multicolumn{1}{c}{(600 sec)} & \multicolumn{1}{c}{(1800 sec)} & \multicolumn{1}{c}{(end)} & \multicolumn{1}{c}{(end)}    & \multicolumn{1}{c}{(end)}   & \multicolumn{1}{c}{(end)}  \\ \midrule 
          PIVIT-BOEM & $\infty$ & $\infty$ & $      5025.86$ & $      5025.86$ & $      5025.86$ & $      5025.86$ & $      5025.86$ & $      5025.86$ & $         8.59$ sec    & $       4.6952$  & $       0.8418$ \\ 
                 CGC & $      3319.84$ & $      3305.67$ & $      3296.05$ & $      3296.05$ & $      3296.05$ & $      3296.05$ & $      3296.05$ & $      3296.05$ & $         2.57$ sec    & $       3.1630$  & $       0.3510$ \\ 
                  HC & $      3635.86$ & $      3635.86$ & $      3635.86$ & $      3635.86$ & $      3635.86$ & $      3635.86$ & $      3635.86$ & $      3635.86$ & $         0.00$ sec    & $       2.9926$  & $       0.3907$ \\ 
              HC-CGC & $      3321.50$ & $      3288.11$ & $      3272.34$ & $      3272.34$ & $      3272.34$ & $      3272.34$ & $      3272.34$ & $      3272.34$ & $         1.65$ sec    & $       2.8835$  & $       0.4367$ \\ 
              ogm-KL & $      3345.02$ & $      3345.02$ & $      3345.02$ & $      3345.02$ & $      3345.02$ & $      3345.02$ & $      3345.02$ & $      3345.02$ & $         0.55$ sec    & $       3.2455$  & $       0.3154$ \\ 
    CC-Fusion-HC-CGC & $      3279.23$ & $      3271.82$ & $      3271.82$ & $      3271.82$ & $      3271.82$ & $      3271.82$ & $      3271.82$ & $      3271.82$ & $         1.19$ sec    & $       2.8443$  & $       0.4440$ \\ 
     CC-Fusion-HC-MC & $      3269.41$ & $      3268.63$ & $      3266.73$ & $      3266.73$ & $      3266.73$ & $      3266.73$ & $      3266.73$ & $      3266.73$ & $         4.11$ sec    & $       2.8527$  & $       0.4495$ \\ 
    CC-Fusion-WS-CGC & $      3283.59$ & $      3283.59$ & $      3283.59$ & $      3283.59$ & $      3283.59$ & $      3283.59$ & $      3283.59$ & $      3283.59$ & $         0.70$ sec    & $       2.9616$  & $       0.4283$ \\ 
     CC-Fusion-WS-MC & $      3282.95$ & $      3275.51$ & $      3267.14$ & $      3266.58$ & $      3266.58$ & $      3266.58$ & $      3266.58$ & $      3266.58$ & $        20.52$ sec    & $       2.8967$  & $       0.5439$ \\ 
\cmidrule{1-1} 
           MCR-CCFDB & $      3287.66$ & $      3287.66$ & $      3287.66$ & $      3287.66$ & $      3287.66$ & $      3287.66$ & $      3287.66$ & $      3287.66$ & $         0.59$ sec    & $       2.9356$  & $       0.5592$ \\ 
\cmidrule{1-1} 
           MCI-CCIFD & $      3291.66$ & $      3291.66$ & $      3266.17$ & $      3266.17$ & $      3266.17$ & $      3266.17$ & $      3266.17$ & $      3266.17$ & $         2.95$ sec    & $       2.8955$  & $       0.5438$ \\ 
\bottomrule
\end{tabular}
\end{table}

\begin{table}[H]
\scriptsize
\centering
\caption{image-seg (175032.bmp)}
\label{tab:anytimetable-image-seg-175032.bmp}
\begin{tabular}{lrrrrrrrrrrr}
\toprule
           algorithm &                                   \multicolumn{8}{c}{value} & \multicolumn{1}{c}{time}    & \multicolumn{1}{c}{VI}  & \multicolumn{1}{c}{RI} \\  
\cmidrule(lr){2-9}\cmidrule(lr){10-10} \cmidrule(lr){11-11} \cmidrule(lr){12-12}   
                     & \multicolumn{1}{c}{(0.5 sec)} & \multicolumn{1}{c}{(1 sec)} & \multicolumn{1}{c}{(10 sec)} & \multicolumn{1}{c}{(60 sec)} & \multicolumn{1}{c}{(300 sec)} & \multicolumn{1}{c}{(600 sec)} & \multicolumn{1}{c}{(1800 sec)} & \multicolumn{1}{c}{(end)} & \multicolumn{1}{c}{(end)}    & \multicolumn{1}{c}{(end)}   & \multicolumn{1}{c}{(end)}  \\ \midrule 
          PIVIT-BOEM & $\infty$ & $\infty$ & $\infty$ & $\infty$ & $\infty$ & $     15402.59$ & $     15402.59$ & $     15402.59$ & $       402.56$ sec    & $       7.5624$  & $       0.6992$ \\ 
                 CGC & $     15859.14$ & $     11926.15$ & $     11816.51$ & $     11619.91$ & $     11609.38$ & $     11609.38$ & $     11609.38$ & $     11609.38$ & $        86.73$ sec    & $       3.8452$  & $       0.5634$ \\ 
                  HC & $     12687.80$ & $     12687.80$ & $     12687.80$ & $     12687.80$ & $     12687.80$ & $     12687.80$ & $     12687.80$ & $     12687.80$ & $         0.01$ sec    & $       3.9944$  & $       0.6858$ \\ 
              HC-CGC & $     11938.70$ & $     11899.82$ & $     11640.31$ & $     11605.03$ & $     11605.03$ & $     11605.03$ & $     11605.03$ & $     11605.03$ & $        38.93$ sec    & $       4.0880$  & $       0.5823$ \\ 
              ogm-KL & $     15464.52$ & $     15464.52$ & $     11888.55$ & $     11888.55$ & $     11888.55$ & $     11888.55$ & $     11888.55$ & $     11888.55$ & $         3.85$ sec    & $       3.1694$  & $       0.4315$ \\ 
    CC-Fusion-HC-CGC & $     11800.48$ & $     11732.36$ & $     11673.16$ & $     11673.16$ & $     11673.16$ & $     11673.16$ & $     11673.16$ & $     11673.16$ & $         8.16$ sec    & $       4.0444$  & $       0.5863$ \\ 
     CC-Fusion-HC-MC & $     12318.12$ & $     11782.70$ & $     11553.73$ & $     11544.34$ & $     11544.34$ & $     11544.34$ & $     11544.34$ & $     11544.34$ & $        54.86$ sec    & $       4.0924$  & $       0.6994$ \\ 
    CC-Fusion-WS-CGC & $     11834.24$ & $     11798.04$ & $     11750.88$ & $     11750.88$ & $     11750.88$ & $     11750.88$ & $     11750.88$ & $     11750.88$ & $         3.81$ sec    & $       3.9001$  & $       0.5659$ \\ 
     CC-Fusion-WS-MC & $     13761.10$ & $     12389.89$ & $     11561.60$ & $     11543.61$ & $     11543.61$ & $     11543.61$ & $     11543.61$ & $     11543.61$ & $        83.59$ sec    & $       4.0962$  & $       0.6993$ \\ 
\cmidrule{1-1} 
           MCR-CCFDB & $     15464.52$ & $     14588.17$ & $     11574.52$ & $     11574.52$ & $     11574.52$ & $     11574.52$ & $     11574.52$ & $     11574.52$ & $         5.55$ sec    & $       4.1655$  & $       0.6992$ \\ 
\cmidrule{1-1} 
           MCI-CCIFD & $     13510.38$ & $     12232.23$ & $     11566.32$ & $     11542.63$ & $     11542.63$ & $     11542.63$ & $     11542.63$ & $     11542.63$ & $        36.18$ sec    & $       4.1087$  & $       0.6992$ \\ 
\bottomrule
\end{tabular}
\end{table}

\begin{table}[H]
\scriptsize
\centering
\caption{image-seg (175043.bmp)}
\label{tab:anytimetable-image-seg-175043.bmp}
\begin{tabular}{lrrrrrrrrrrr}
\toprule
           algorithm &                                   \multicolumn{8}{c}{value} & \multicolumn{1}{c}{time}    & \multicolumn{1}{c}{VI}  & \multicolumn{1}{c}{RI} \\  
\cmidrule(lr){2-9}\cmidrule(lr){10-10} \cmidrule(lr){11-11} \cmidrule(lr){12-12}   
                     & \multicolumn{1}{c}{(0.5 sec)} & \multicolumn{1}{c}{(1 sec)} & \multicolumn{1}{c}{(10 sec)} & \multicolumn{1}{c}{(60 sec)} & \multicolumn{1}{c}{(300 sec)} & \multicolumn{1}{c}{(600 sec)} & \multicolumn{1}{c}{(1800 sec)} & \multicolumn{1}{c}{(end)} & \multicolumn{1}{c}{(end)}    & \multicolumn{1}{c}{(end)}   & \multicolumn{1}{c}{(end)}  \\ \midrule 
          PIVIT-BOEM & $\infty$ & $\infty$ & $\infty$ & $\infty$ & $      9837.76$ & $      9837.76$ & $      9837.76$ & $      9837.76$ & $       128.42$ sec    & $       8.3795$  & $       0.2924$ \\ 
                 CGC & $      8098.10$ & $      7888.65$ & $      7882.39$ & $      7882.39$ & $      7882.39$ & $      7882.39$ & $      7882.39$ & $      7882.39$ & $         1.12$ sec    & $       5.5365$  & $       0.3235$ \\ 
                  HC & $      8536.56$ & $      8536.56$ & $      8536.56$ & $      8536.56$ & $      8536.56$ & $      8536.56$ & $      8536.56$ & $      8536.56$ & $         0.01$ sec    & $       5.6158$  & $       0.3120$ \\ 
              HC-CGC & $      7853.79$ & $      7853.79$ & $      7853.79$ & $      7853.79$ & $      7853.79$ & $      7853.79$ & $      7853.79$ & $      7853.79$ & $         0.24$ sec    & $       5.7623$  & $       0.3138$ \\ 
              ogm-KL & $     12318.40$ & $     12318.40$ & $      8260.16$ & $      8260.16$ & $      8260.16$ & $      8260.16$ & $      8260.16$ & $      8260.16$ & $         4.43$ sec    & $       2.9531$  & $       0.4981$ \\ 
    CC-Fusion-HC-CGC & $      7859.65$ & $      7842.03$ & $      7840.83$ & $      7840.83$ & $      7840.83$ & $      7840.83$ & $      7840.83$ & $      7840.83$ & $         1.87$ sec    & $       5.7208$  & $       0.3141$ \\ 
     CC-Fusion-HC-MC & $      7870.73$ & $      7822.96$ & $      7820.21$ & $      7820.21$ & $      7820.21$ & $      7820.21$ & $      7820.21$ & $      7820.21$ & $         4.37$ sec    & $       5.8466$  & $       0.3094$ \\ 
    CC-Fusion-WS-CGC & $      7936.18$ & $      7904.21$ & $      7904.21$ & $      7904.21$ & $      7904.21$ & $      7904.21$ & $      7904.21$ & $      7904.21$ & $         1.34$ sec    & $       5.7228$  & $       0.3130$ \\ 
     CC-Fusion-WS-MC & $      8324.29$ & $      7971.72$ & $      7817.30$ & $      7817.30$ & $      7817.30$ & $      7817.30$ & $      7817.30$ & $      7817.30$ & $        11.85$ sec    & $       5.8333$  & $       0.3096$ \\ 
\cmidrule{1-1} 
           MCR-CCFDB & $      8667.37$ & $      7816.92$ & $      7816.92$ & $      7816.92$ & $      7816.92$ & $      7816.92$ & $      7816.92$ & $      7816.92$ & $         0.61$ sec    & $       5.8411$  & $       0.3091$ \\ 
\cmidrule{1-1} 
           MCI-CCIFD & $      8422.56$ & $      7900.02$ & $      7816.92$ & $      7816.92$ & $      7816.92$ & $      7816.92$ & $      7816.92$ & $      7816.92$ & $         1.82$ sec    & $       5.8411$  & $       0.3091$ \\ 
\bottomrule
\end{tabular}
\end{table}

\begin{table}[H]
\scriptsize
\centering
\caption{image-seg (182053.bmp)}
\label{tab:anytimetable-image-seg-182053.bmp}
\begin{tabular}{lrrrrrrrrrrr}
\toprule
           algorithm &                                   \multicolumn{8}{c}{value} & \multicolumn{1}{c}{time}    & \multicolumn{1}{c}{VI}  & \multicolumn{1}{c}{RI} \\  
\cmidrule(lr){2-9}\cmidrule(lr){10-10} \cmidrule(lr){11-11} \cmidrule(lr){12-12}   
                     & \multicolumn{1}{c}{(0.5 sec)} & \multicolumn{1}{c}{(1 sec)} & \multicolumn{1}{c}{(10 sec)} & \multicolumn{1}{c}{(60 sec)} & \multicolumn{1}{c}{(300 sec)} & \multicolumn{1}{c}{(600 sec)} & \multicolumn{1}{c}{(1800 sec)} & \multicolumn{1}{c}{(end)} & \multicolumn{1}{c}{(end)}    & \multicolumn{1}{c}{(end)}   & \multicolumn{1}{c}{(end)}  \\ \midrule 
          PIVIT-BOEM & $\infty$ & $\infty$ & $\infty$ & $      4908.61$ & $      4908.61$ & $      4908.61$ & $      4908.61$ & $      4908.61$ & $        11.27$ sec    & $       3.4775$  & $       0.9228$ \\ 
                 CGC & $      3596.52$ & $      3596.52$ & $      3596.52$ & $      3596.52$ & $      3596.52$ & $      3596.52$ & $      3596.52$ & $      3596.52$ & $         0.27$ sec    & $       2.3543$  & $       0.8689$ \\ 
                  HC & $      3957.81$ & $      3957.81$ & $      3957.81$ & $      3957.81$ & $      3957.81$ & $      3957.81$ & $      3957.81$ & $      3957.81$ & $         0.00$ sec    & $       2.3121$  & $       0.8988$ \\ 
              HC-CGC & $      3588.44$ & $      3588.44$ & $      3588.44$ & $      3588.44$ & $      3588.44$ & $      3588.44$ & $      3588.44$ & $      3588.44$ & $         0.36$ sec    & $       2.4027$  & $       0.8851$ \\ 
              ogm-KL & $      3751.47$ & $      3751.47$ & $      3751.47$ & $      3751.47$ & $      3751.47$ & $      3751.47$ & $      3751.47$ & $      3751.47$ & $         0.44$ sec    & $       2.6442$  & $       0.7688$ \\ 
    CC-Fusion-HC-CGC & $      3595.59$ & $      3589.97$ & $      3584.57$ & $      3584.57$ & $      3584.57$ & $      3584.57$ & $      3584.57$ & $      3584.57$ & $         1.87$ sec    & $       2.2317$  & $       0.9147$ \\ 
     CC-Fusion-HC-MC & $      3586.15$ & $      3580.57$ & $      3579.24$ & $      3579.24$ & $      3579.24$ & $      3579.24$ & $      3579.24$ & $      3579.24$ & $         7.95$ sec    & $       2.3030$  & $       0.9056$ \\ 
    CC-Fusion-WS-CGC & $      3603.53$ & $      3603.53$ & $      3603.53$ & $      3603.53$ & $      3603.53$ & $      3603.53$ & $      3603.53$ & $      3603.53$ & $         0.58$ sec    & $       2.3667$  & $       0.8730$ \\ 
     CC-Fusion-WS-MC & $      3605.77$ & $      3593.19$ & $      3579.49$ & $      3579.49$ & $      3579.49$ & $      3579.49$ & $      3579.49$ & $      3579.49$ & $        11.81$ sec    & $       2.2895$  & $       0.9059$ \\ 
\cmidrule{1-1} 
           MCR-CCFDB & $      3581.74$ & $      3581.74$ & $      3581.74$ & $      3581.74$ & $      3581.74$ & $      3581.74$ & $      3581.74$ & $      3581.74$ & $         0.30$ sec    & $       2.3058$  & $       0.9055$ \\ 
\cmidrule{1-1} 
           MCI-CCIFD & $      3649.67$ & $      3590.79$ & $      3579.24$ & $      3579.24$ & $      3579.24$ & $      3579.24$ & $      3579.24$ & $      3579.24$ & $         2.42$ sec    & $       2.3030$  & $       0.9056$ \\ 
\bottomrule
\end{tabular}
\end{table}

\begin{table}[H]
\scriptsize
\centering
\caption{image-seg (189080.bmp)}
\label{tab:anytimetable-image-seg-189080.bmp}
\begin{tabular}{lrrrrrrrrrrr}
\toprule
           algorithm &                                   \multicolumn{8}{c}{value} & \multicolumn{1}{c}{time}    & \multicolumn{1}{c}{VI}  & \multicolumn{1}{c}{RI} \\  
\cmidrule(lr){2-9}\cmidrule(lr){10-10} \cmidrule(lr){11-11} \cmidrule(lr){12-12}   
                     & \multicolumn{1}{c}{(0.5 sec)} & \multicolumn{1}{c}{(1 sec)} & \multicolumn{1}{c}{(10 sec)} & \multicolumn{1}{c}{(60 sec)} & \multicolumn{1}{c}{(300 sec)} & \multicolumn{1}{c}{(600 sec)} & \multicolumn{1}{c}{(1800 sec)} & \multicolumn{1}{c}{(end)} & \multicolumn{1}{c}{(end)}    & \multicolumn{1}{c}{(end)}   & \multicolumn{1}{c}{(end)}  \\ \midrule 
          PIVIT-BOEM & $      1558.24$ & $      1558.24$ & $      1558.24$ & $      1558.24$ & $      1558.24$ & $      1558.24$ & $      1558.24$ & $      1558.24$ & $         0.23$ sec    & $       2.6569$  & $       0.8448$ \\ 
                 CGC & $      1092.58$ & $      1092.58$ & $      1092.58$ & $      1092.58$ & $      1092.58$ & $      1092.58$ & $      1092.58$ & $      1092.58$ & $         0.01$ sec    & $       1.3029$  & $       0.8789$ \\ 
                  HC & $      1147.86$ & $      1147.86$ & $      1147.86$ & $      1147.86$ & $      1147.86$ & $      1147.86$ & $      1147.86$ & $      1147.86$ & $         0.00$ sec    & $       1.0224$  & $       0.9251$ \\ 
              HC-CGC & $      1090.77$ & $      1090.77$ & $      1090.77$ & $      1090.77$ & $      1090.77$ & $      1090.77$ & $      1090.77$ & $      1090.77$ & $         0.01$ sec    & $       1.2274$  & $       0.9055$ \\ 
              ogm-KL & $      1103.64$ & $      1103.64$ & $      1103.64$ & $      1103.64$ & $      1103.64$ & $      1103.64$ & $      1103.64$ & $      1103.64$ & $         0.01$ sec    & $       1.2925$  & $       0.8703$ \\ 
    CC-Fusion-HC-CGC & $      1077.47$ & $      1077.47$ & $      1077.47$ & $      1077.47$ & $      1077.47$ & $      1077.47$ & $      1077.47$ & $      1077.47$ & $         0.10$ sec    & $       1.2663$  & $       0.9053$ \\ 
     CC-Fusion-HC-MC & $      1077.47$ & $      1077.47$ & $      1077.47$ & $      1077.47$ & $      1077.47$ & $      1077.47$ & $      1077.47$ & $      1077.47$ & $         0.94$ sec    & $       1.2663$  & $       0.9053$ \\ 
    CC-Fusion-WS-CGC & $      1078.41$ & $      1078.41$ & $      1078.41$ & $      1078.41$ & $      1078.41$ & $      1078.41$ & $      1078.41$ & $      1078.41$ & $         0.11$ sec    & $       1.2788$  & $       0.8988$ \\ 
     CC-Fusion-WS-MC & $      1078.41$ & $      1077.47$ & $      1077.47$ & $      1077.47$ & $      1077.47$ & $      1077.47$ & $      1077.47$ & $      1077.47$ & $         1.70$ sec    & $       1.2663$  & $       0.9053$ \\ 
\cmidrule{1-1} 
           MCR-CCFDB & $      1080.02$ & $      1080.02$ & $      1080.02$ & $      1080.02$ & $      1080.02$ & $      1080.02$ & $      1080.02$ & $      1080.02$ & $         0.01$ sec    & $       1.2690$  & $       0.9053$ \\ 
\cmidrule{1-1} 
           MCI-CCIFD & $      1077.47$ & $      1077.47$ & $      1077.47$ & $      1077.47$ & $      1077.47$ & $      1077.47$ & $      1077.47$ & $      1077.47$ & $         0.13$ sec    & $       1.2663$  & $       0.9053$ \\ 
\bottomrule
\end{tabular}
\end{table}

\begin{table}[H]
\scriptsize
\centering
\caption{image-seg (19021.bmp)}
\label{tab:anytimetable-image-seg-19021.bmp}
\begin{tabular}{lrrrrrrrrrrr}
\toprule
           algorithm &                                   \multicolumn{8}{c}{value} & \multicolumn{1}{c}{time}    & \multicolumn{1}{c}{VI}  & \multicolumn{1}{c}{RI} \\  
\cmidrule(lr){2-9}\cmidrule(lr){10-10} \cmidrule(lr){11-11} \cmidrule(lr){12-12}   
                     & \multicolumn{1}{c}{(0.5 sec)} & \multicolumn{1}{c}{(1 sec)} & \multicolumn{1}{c}{(10 sec)} & \multicolumn{1}{c}{(60 sec)} & \multicolumn{1}{c}{(300 sec)} & \multicolumn{1}{c}{(600 sec)} & \multicolumn{1}{c}{(1800 sec)} & \multicolumn{1}{c}{(end)} & \multicolumn{1}{c}{(end)}    & \multicolumn{1}{c}{(end)}   & \multicolumn{1}{c}{(end)}  \\ \midrule 
          PIVIT-BOEM & $\infty$ & $\infty$ & $\infty$ & $      6101.70$ & $      6101.70$ & $      6101.70$ & $      6101.70$ & $      6101.70$ & $        25.37$ sec    & $       4.7765$  & $       0.8602$ \\ 
                 CGC & $      4601.22$ & $      4542.47$ & $      4531.82$ & $      4531.82$ & $      4531.82$ & $      4531.82$ & $      4531.82$ & $      4531.82$ & $         1.25$ sec    & $       2.4066$  & $       0.8738$ \\ 
                  HC & $      4979.08$ & $      4979.08$ & $      4979.08$ & $      4979.08$ & $      4979.08$ & $      4979.08$ & $      4979.08$ & $      4979.08$ & $         0.00$ sec    & $       3.1780$  & $       0.6736$ \\ 
              HC-CGC & $      4576.06$ & $      4533.42$ & $      4526.64$ & $      4526.64$ & $      4526.64$ & $      4526.64$ & $      4526.64$ & $      4526.64$ & $         2.25$ sec    & $       2.9031$  & $       0.7445$ \\ 
              ogm-KL & $      4625.23$ & $      4616.42$ & $      4608.54$ & $      4608.54$ & $      4608.54$ & $      4608.54$ & $      4608.54$ & $      4608.54$ & $         2.08$ sec    & $       3.1923$  & $       0.6468$ \\ 
    CC-Fusion-HC-CGC & $      4523.08$ & $      4522.07$ & $      4522.07$ & $      4522.07$ & $      4522.07$ & $      4522.07$ & $      4522.07$ & $      4522.07$ & $         1.01$ sec    & $       2.3474$  & $       0.8843$ \\ 
     CC-Fusion-HC-MC & $      4524.16$ & $      4516.34$ & $      4515.08$ & $      4515.08$ & $      4515.08$ & $      4515.08$ & $      4515.08$ & $      4515.08$ & $         5.19$ sec    & $       2.4479$  & $       0.8822$ \\ 
    CC-Fusion-WS-CGC & $      4549.94$ & $      4534.21$ & $      4534.21$ & $      4534.21$ & $      4534.21$ & $      4534.21$ & $      4534.21$ & $      4534.21$ & $         1.10$ sec    & $       2.6594$  & $       0.8338$ \\ 
     CC-Fusion-WS-MC & $      4539.99$ & $      4530.19$ & $      4515.08$ & $      4515.08$ & $      4515.08$ & $      4515.08$ & $      4515.08$ & $      4515.08$ & $         8.40$ sec    & $       2.4502$  & $       0.8822$ \\ 
\cmidrule{1-1} 
           MCR-CCFDB & $      4520.06$ & $      4520.06$ & $      4520.06$ & $      4520.06$ & $      4520.06$ & $      4520.06$ & $      4520.06$ & $      4520.06$ & $         0.34$ sec    & $       2.4515$  & $       0.8822$ \\ 
\cmidrule{1-1} 
           MCI-CCIFD & $      4581.98$ & $      4515.08$ & $      4515.08$ & $      4515.08$ & $      4515.08$ & $      4515.08$ & $      4515.08$ & $      4515.08$ & $         1.09$ sec    & $       2.4479$  & $       0.8822$ \\ 
\bottomrule
\end{tabular}
\end{table}

\begin{table}[H]
\scriptsize
\centering
\caption{image-seg (196073.bmp)}
\label{tab:anytimetable-image-seg-196073.bmp}
\begin{tabular}{lrrrrrrrrrrr}
\toprule
           algorithm &                                   \multicolumn{8}{c}{value} & \multicolumn{1}{c}{time}    & \multicolumn{1}{c}{VI}  & \multicolumn{1}{c}{RI} \\  
\cmidrule(lr){2-9}\cmidrule(lr){10-10} \cmidrule(lr){11-11} \cmidrule(lr){12-12}   
                     & \multicolumn{1}{c}{(0.5 sec)} & \multicolumn{1}{c}{(1 sec)} & \multicolumn{1}{c}{(10 sec)} & \multicolumn{1}{c}{(60 sec)} & \multicolumn{1}{c}{(300 sec)} & \multicolumn{1}{c}{(600 sec)} & \multicolumn{1}{c}{(1800 sec)} & \multicolumn{1}{c}{(end)} & \multicolumn{1}{c}{(end)}    & \multicolumn{1}{c}{(end)}   & \multicolumn{1}{c}{(end)}  \\ \midrule 
          PIVIT-BOEM & $       802.45$ & $       802.45$ & $       802.45$ & $       802.45$ & $       802.45$ & $       802.45$ & $       802.45$ & $       802.45$ & $         0.08$ sec    & $       0.6855$  & $       0.9074$ \\ 
                 CGC & $       545.53$ & $       545.53$ & $       545.53$ & $       545.53$ & $       545.53$ & $       545.53$ & $       545.53$ & $       545.53$ & $         0.00$ sec    & $       0.2684$  & $       0.9700$ \\ 
                  HC & $       596.13$ & $       596.13$ & $       596.13$ & $       596.13$ & $       596.13$ & $       596.13$ & $       596.13$ & $       596.13$ & $         0.00$ sec    & $       0.3810$  & $       0.9222$ \\ 
              HC-CGC & $       547.55$ & $       547.55$ & $       547.55$ & $       547.55$ & $       547.55$ & $       547.55$ & $       547.55$ & $       547.55$ & $         0.01$ sec    & $       0.2507$  & $       0.9696$ \\ 
              ogm-KL & $       554.88$ & $       554.88$ & $       554.88$ & $       554.88$ & $       554.88$ & $       554.88$ & $       554.88$ & $       554.88$ & $         0.00$ sec    & $       0.3511$  & $       0.9230$ \\ 
    CC-Fusion-HC-CGC & $       545.47$ & $       545.47$ & $       545.47$ & $       545.47$ & $       545.47$ & $       545.47$ & $       545.47$ & $       545.47$ & $         0.06$ sec    & $       0.2459$  & $       0.9702$ \\ 
     CC-Fusion-HC-MC & $       545.47$ & $       545.47$ & $       545.47$ & $       545.47$ & $       545.47$ & $       545.47$ & $       545.47$ & $       545.47$ & $         0.65$ sec    & $       0.2459$  & $       0.9702$ \\ 
    CC-Fusion-WS-CGC & $       545.47$ & $       545.47$ & $       545.47$ & $       545.47$ & $       545.47$ & $       545.47$ & $       545.47$ & $       545.47$ & $         0.04$ sec    & $       0.2459$  & $       0.9702$ \\ 
     CC-Fusion-WS-MC & $       545.47$ & $       545.47$ & $       545.47$ & $       545.47$ & $       545.47$ & $       545.47$ & $       545.47$ & $       545.47$ & $         0.86$ sec    & $       0.2459$  & $       0.9702$ \\ 
\cmidrule{1-1} 
           MCR-CCFDB & $       545.47$ & $       545.47$ & $       545.47$ & $       545.47$ & $       545.47$ & $       545.47$ & $       545.47$ & $       545.47$ & $         0.01$ sec    & $       0.2459$  & $       0.9702$ \\ 
\cmidrule{1-1} 
           MCI-CCIFD & $       545.47$ & $       545.47$ & $       545.47$ & $       545.47$ & $       545.47$ & $       545.47$ & $       545.47$ & $       545.47$ & $         0.06$ sec    & $       0.2459$  & $       0.9702$ \\ 
\bottomrule
\end{tabular}
\end{table}

\begin{table}[H]
\scriptsize
\centering
\caption{image-seg (197017.bmp)}
\label{tab:anytimetable-image-seg-197017.bmp}
\begin{tabular}{lrrrrrrrrrrr}
\toprule
           algorithm &                                   \multicolumn{8}{c}{value} & \multicolumn{1}{c}{time}    & \multicolumn{1}{c}{VI}  & \multicolumn{1}{c}{RI} \\  
\cmidrule(lr){2-9}\cmidrule(lr){10-10} \cmidrule(lr){11-11} \cmidrule(lr){12-12}   
                     & \multicolumn{1}{c}{(0.5 sec)} & \multicolumn{1}{c}{(1 sec)} & \multicolumn{1}{c}{(10 sec)} & \multicolumn{1}{c}{(60 sec)} & \multicolumn{1}{c}{(300 sec)} & \multicolumn{1}{c}{(600 sec)} & \multicolumn{1}{c}{(1800 sec)} & \multicolumn{1}{c}{(end)} & \multicolumn{1}{c}{(end)}    & \multicolumn{1}{c}{(end)}   & \multicolumn{1}{c}{(end)}  \\ \midrule 
          PIVIT-BOEM & $\infty$ & $\infty$ & $      4505.51$ & $      4505.51$ & $      4505.51$ & $      4505.51$ & $      4505.51$ & $      4505.51$ & $         6.78$ sec    & $       5.0621$  & $       0.6852$ \\ 
                 CGC & $      2800.33$ & $      2800.33$ & $      2800.33$ & $      2800.33$ & $      2800.33$ & $      2800.33$ & $      2800.33$ & $      2800.33$ & $         0.06$ sec    & $       1.3674$  & $       0.9147$ \\ 
                  HC & $      2947.36$ & $      2947.36$ & $      2947.36$ & $      2947.36$ & $      2947.36$ & $      2947.36$ & $      2947.36$ & $      2947.36$ & $         0.00$ sec    & $       1.4555$  & $       0.9088$ \\ 
              HC-CGC & $      2802.47$ & $      2802.47$ & $      2802.47$ & $      2802.47$ & $      2802.47$ & $      2802.47$ & $      2802.47$ & $      2802.47$ & $         0.07$ sec    & $       1.3638$  & $       0.9148$ \\ 
              ogm-KL & $      2917.21$ & $      2917.21$ & $      2917.21$ & $      2917.21$ & $      2917.21$ & $      2917.21$ & $      2917.21$ & $      2917.21$ & $         0.34$ sec    & $       2.0348$  & $       0.7939$ \\ 
    CC-Fusion-HC-CGC & $      2798.95$ & $      2798.95$ & $      2798.95$ & $      2798.95$ & $      2798.95$ & $      2798.95$ & $      2798.95$ & $      2798.95$ & $         0.30$ sec    & $       1.3770$  & $       0.9135$ \\ 
     CC-Fusion-HC-MC & $      2798.77$ & $      2798.77$ & $      2798.77$ & $      2798.77$ & $      2798.77$ & $      2798.77$ & $      2798.77$ & $      2798.77$ & $         1.09$ sec    & $       1.3787$  & $       0.9135$ \\ 
    CC-Fusion-WS-CGC & $      2799.40$ & $      2799.40$ & $      2799.40$ & $      2799.40$ & $      2799.40$ & $      2799.40$ & $      2799.40$ & $      2799.40$ & $         0.23$ sec    & $       1.3751$  & $       0.9135$ \\ 
     CC-Fusion-WS-MC & $      2798.84$ & $      2798.77$ & $      2798.77$ & $      2798.77$ & $      2798.77$ & $      2798.77$ & $      2798.77$ & $      2798.77$ & $         1.94$ sec    & $       1.3787$  & $       0.9135$ \\ 
\cmidrule{1-1} 
           MCR-CCFDB & $      2798.77$ & $      2798.77$ & $      2798.77$ & $      2798.77$ & $      2798.77$ & $      2798.77$ & $      2798.77$ & $      2798.77$ & $         0.04$ sec    & $       1.3787$  & $       0.9135$ \\ 
\cmidrule{1-1} 
           MCI-CCIFD & $      2798.77$ & $      2798.77$ & $      2798.77$ & $      2798.77$ & $      2798.77$ & $      2798.77$ & $      2798.77$ & $      2798.77$ & $         0.24$ sec    & $       1.3787$  & $       0.9135$ \\ 
\bottomrule
\end{tabular}
\end{table}

\begin{table}[H]
\scriptsize
\centering
\caption{image-seg (208001.bmp)}
\label{tab:anytimetable-image-seg-208001.bmp}
\begin{tabular}{lrrrrrrrrrrr}
\toprule
           algorithm &                                   \multicolumn{8}{c}{value} & \multicolumn{1}{c}{time}    & \multicolumn{1}{c}{VI}  & \multicolumn{1}{c}{RI} \\  
\cmidrule(lr){2-9}\cmidrule(lr){10-10} \cmidrule(lr){11-11} \cmidrule(lr){12-12}   
                     & \multicolumn{1}{c}{(0.5 sec)} & \multicolumn{1}{c}{(1 sec)} & \multicolumn{1}{c}{(10 sec)} & \multicolumn{1}{c}{(60 sec)} & \multicolumn{1}{c}{(300 sec)} & \multicolumn{1}{c}{(600 sec)} & \multicolumn{1}{c}{(1800 sec)} & \multicolumn{1}{c}{(end)} & \multicolumn{1}{c}{(end)}    & \multicolumn{1}{c}{(end)}   & \multicolumn{1}{c}{(end)}  \\ \midrule 
          PIVIT-BOEM & $\infty$ & $\infty$ & $\infty$ & $\infty$ & $      8759.13$ & $      8759.13$ & $      8759.13$ & $      8759.13$ & $        62.97$ sec    & $       5.4229$  & $       0.8212$ \\ 
                 CGC & $      6331.28$ & $      6317.09$ & $      6317.09$ & $      6317.09$ & $      6317.09$ & $      6317.09$ & $      6317.09$ & $      6317.09$ & $         0.80$ sec    & $       2.7689$  & $       0.8306$ \\ 
                  HC & $      6869.30$ & $      6869.30$ & $      6869.30$ & $      6869.30$ & $      6869.30$ & $      6869.30$ & $      6869.30$ & $      6869.30$ & $         0.01$ sec    & $       2.8688$  & $       0.8246$ \\ 
              HC-CGC & $      6317.52$ & $      6316.75$ & $      6316.75$ & $      6316.75$ & $      6316.75$ & $      6316.75$ & $      6316.75$ & $      6316.75$ & $         0.59$ sec    & $       2.7870$  & $       0.8322$ \\ 
              ogm-KL & $      6617.39$ & $      6609.41$ & $      6609.41$ & $      6609.41$ & $      6609.41$ & $      6609.41$ & $      6609.41$ & $      6609.41$ & $         0.99$ sec    & $       3.9545$  & $       0.4900$ \\ 
    CC-Fusion-HC-CGC & $      6306.93$ & $      6305.32$ & $      6305.32$ & $      6305.32$ & $      6305.32$ & $      6305.32$ & $      6305.32$ & $      6305.32$ & $         1.30$ sec    & $       2.6688$  & $       0.8468$ \\ 
     CC-Fusion-HC-MC & $      6312.71$ & $      6285.11$ & $      6275.39$ & $      6275.39$ & $      6275.39$ & $      6275.39$ & $      6275.39$ & $      6275.39$ & $        10.36$ sec    & $       2.6910$  & $       0.8500$ \\ 
    CC-Fusion-WS-CGC & $      6348.84$ & $      6327.73$ & $      6327.73$ & $      6327.73$ & $      6327.73$ & $      6327.73$ & $      6327.73$ & $      6327.73$ & $         1.03$ sec    & $       2.6530$  & $       0.8496$ \\ 
     CC-Fusion-WS-MC & $      6394.46$ & $      6310.06$ & $      6275.10$ & $      6275.10$ & $      6275.10$ & $      6275.10$ & $      6275.10$ & $      6275.10$ & $        13.34$ sec    & $       2.6642$  & $       0.8502$ \\ 
\cmidrule{1-1} 
           MCR-CCFDB & $      6401.11$ & $      6276.25$ & $      6276.25$ & $      6276.25$ & $      6276.25$ & $      6276.25$ & $      6276.25$ & $      6276.25$ & $         0.77$ sec    & $       2.7692$  & $       0.8437$ \\ 
\cmidrule{1-1} 
           MCI-CCIFD & $      6450.72$ & $      6364.41$ & $      6272.68$ & $      6272.68$ & $      6272.68$ & $      6272.68$ & $      6272.68$ & $      6272.68$ & $         8.39$ sec    & $       2.7630$  & $       0.8433$ \\ 
\bottomrule
\end{tabular}
\end{table}

\begin{table}[H]
\scriptsize
\centering
\caption{image-seg (210088.bmp)}
\label{tab:anytimetable-image-seg-210088.bmp}
\begin{tabular}{lrrrrrrrrrrr}
\toprule
           algorithm &                                   \multicolumn{8}{c}{value} & \multicolumn{1}{c}{time}    & \multicolumn{1}{c}{VI}  & \multicolumn{1}{c}{RI} \\  
\cmidrule(lr){2-9}\cmidrule(lr){10-10} \cmidrule(lr){11-11} \cmidrule(lr){12-12}   
                     & \multicolumn{1}{c}{(0.5 sec)} & \multicolumn{1}{c}{(1 sec)} & \multicolumn{1}{c}{(10 sec)} & \multicolumn{1}{c}{(60 sec)} & \multicolumn{1}{c}{(300 sec)} & \multicolumn{1}{c}{(600 sec)} & \multicolumn{1}{c}{(1800 sec)} & \multicolumn{1}{c}{(end)} & \multicolumn{1}{c}{(end)}    & \multicolumn{1}{c}{(end)}   & \multicolumn{1}{c}{(end)}  \\ \midrule 
          PIVIT-BOEM & $\infty$ & $\infty$ & $      2314.80$ & $      2314.80$ & $      2314.80$ & $      2314.80$ & $      2314.80$ & $      2314.80$ & $         1.18$ sec    & $       6.3004$  & $       0.3043$ \\ 
                 CGC & $      1907.78$ & $      1907.78$ & $      1907.78$ & $      1907.78$ & $      1907.78$ & $      1907.78$ & $      1907.78$ & $      1907.78$ & $         0.04$ sec    & $       4.2902$  & $       0.3490$ \\ 
                  HC & $      2109.40$ & $      2109.40$ & $      2109.40$ & $      2109.40$ & $      2109.40$ & $      2109.40$ & $      2109.40$ & $      2109.40$ & $         0.00$ sec    & $       3.8516$  & $       0.3803$ \\ 
              HC-CGC & $      1904.32$ & $      1904.32$ & $      1904.32$ & $      1904.32$ & $      1904.32$ & $      1904.32$ & $      1904.32$ & $      1904.32$ & $         0.02$ sec    & $       4.4182$  & $       0.3390$ \\ 
              ogm-KL & $      2016.88$ & $      2016.88$ & $      2016.88$ & $      2016.88$ & $      2016.88$ & $      2016.88$ & $      2016.88$ & $      2016.88$ & $         0.08$ sec    & $       2.7117$  & $       0.4774$ \\ 
    CC-Fusion-HC-CGC & $      1899.80$ & $      1898.13$ & $      1898.13$ & $      1898.13$ & $      1898.13$ & $      1898.13$ & $      1898.13$ & $      1898.13$ & $         0.93$ sec    & $       4.5710$  & $       0.3299$ \\ 
     CC-Fusion-HC-MC & $      1895.44$ & $      1895.44$ & $      1895.44$ & $      1895.44$ & $      1895.44$ & $      1895.44$ & $      1895.44$ & $      1895.44$ & $         1.19$ sec    & $       4.3517$  & $       0.3426$ \\ 
    CC-Fusion-WS-CGC & $      1908.90$ & $      1908.90$ & $      1908.90$ & $      1908.90$ & $      1908.90$ & $      1908.90$ & $      1908.90$ & $      1908.90$ & $         0.29$ sec    & $       4.7390$  & $       0.3244$ \\ 
     CC-Fusion-WS-MC & $      1897.37$ & $      1896.22$ & $      1895.44$ & $      1895.44$ & $      1895.44$ & $      1895.44$ & $      1895.44$ & $      1895.44$ & $         2.13$ sec    & $       4.3517$  & $       0.3426$ \\ 
\cmidrule{1-1} 
           MCR-CCFDB & $      1895.44$ & $      1895.44$ & $      1895.44$ & $      1895.44$ & $      1895.44$ & $      1895.44$ & $      1895.44$ & $      1895.44$ & $         0.03$ sec    & $       4.3517$  & $       0.3426$ \\ 
\cmidrule{1-1} 
           MCI-CCIFD & $      1895.44$ & $      1895.44$ & $      1895.44$ & $      1895.44$ & $      1895.44$ & $      1895.44$ & $      1895.44$ & $      1895.44$ & $         0.28$ sec    & $       4.3517$  & $       0.3426$ \\ 
\bottomrule
\end{tabular}
\end{table}

\begin{table}[H]
\scriptsize
\centering
\caption{image-seg (21077.bmp)}
\label{tab:anytimetable-image-seg-21077.bmp}
\begin{tabular}{lrrrrrrrrrrr}
\toprule
           algorithm &                                   \multicolumn{8}{c}{value} & \multicolumn{1}{c}{time}    & \multicolumn{1}{c}{VI}  & \multicolumn{1}{c}{RI} \\  
\cmidrule(lr){2-9}\cmidrule(lr){10-10} \cmidrule(lr){11-11} \cmidrule(lr){12-12}   
                     & \multicolumn{1}{c}{(0.5 sec)} & \multicolumn{1}{c}{(1 sec)} & \multicolumn{1}{c}{(10 sec)} & \multicolumn{1}{c}{(60 sec)} & \multicolumn{1}{c}{(300 sec)} & \multicolumn{1}{c}{(600 sec)} & \multicolumn{1}{c}{(1800 sec)} & \multicolumn{1}{c}{(end)} & \multicolumn{1}{c}{(end)}    & \multicolumn{1}{c}{(end)}   & \multicolumn{1}{c}{(end)}  \\ \midrule 
          PIVIT-BOEM & $\infty$ & $\infty$ & $      3913.82$ & $      3913.82$ & $      3913.82$ & $      3913.82$ & $      3913.82$ & $      3913.82$ & $         6.42$ sec    & $       4.2707$  & $       0.7088$ \\ 
                 CGC & $      2949.63$ & $      2949.63$ & $      2949.63$ & $      2949.63$ & $      2949.63$ & $      2949.63$ & $      2949.63$ & $      2949.63$ & $         0.03$ sec    & $       2.6694$  & $       0.7481$ \\ 
                  HC & $      3130.37$ & $      3130.37$ & $      3130.37$ & $      3130.37$ & $      3130.37$ & $      3130.37$ & $      3130.37$ & $      3130.37$ & $         0.00$ sec    & $       2.7688$  & $       0.7411$ \\ 
              HC-CGC & $      2950.76$ & $      2950.76$ & $      2950.76$ & $      2950.76$ & $      2950.76$ & $      2950.76$ & $      2950.76$ & $      2950.76$ & $         0.03$ sec    & $       2.6620$  & $       0.7481$ \\ 
              ogm-KL & $      2993.34$ & $      2993.34$ & $      2993.34$ & $      2993.34$ & $      2993.34$ & $      2993.34$ & $      2993.34$ & $      2993.34$ & $         0.08$ sec    & $       2.6913$  & $       0.7484$ \\ 
    CC-Fusion-HC-CGC & $      2946.71$ & $      2946.71$ & $      2946.71$ & $      2946.71$ & $      2946.71$ & $      2946.71$ & $      2946.71$ & $      2946.71$ & $         0.45$ sec    & $       2.7148$  & $       0.7473$ \\ 
     CC-Fusion-HC-MC & $      2948.03$ & $      2946.71$ & $      2946.71$ & $      2946.71$ & $      2946.71$ & $      2946.71$ & $      2946.71$ & $      2946.71$ & $         1.96$ sec    & $       2.7148$  & $       0.7473$ \\ 
    CC-Fusion-WS-CGC & $      2946.71$ & $      2946.71$ & $      2946.71$ & $      2946.71$ & $      2946.71$ & $      2946.71$ & $      2946.71$ & $      2946.71$ & $         0.36$ sec    & $       2.7148$  & $       0.7473$ \\ 
     CC-Fusion-WS-MC & $      2952.40$ & $      2947.35$ & $      2946.71$ & $      2946.71$ & $      2946.71$ & $      2946.71$ & $      2946.71$ & $      2946.71$ & $         2.41$ sec    & $       2.7148$  & $       0.7473$ \\ 
\cmidrule{1-1} 
           MCR-CCFDB & $      2946.71$ & $      2946.71$ & $      2946.71$ & $      2946.71$ & $      2946.71$ & $      2946.71$ & $      2946.71$ & $      2946.71$ & $         0.04$ sec    & $       2.7148$  & $       0.7473$ \\ 
\cmidrule{1-1} 
           MCI-CCIFD & $      2954.77$ & $      2946.71$ & $      2946.71$ & $      2946.71$ & $      2946.71$ & $      2946.71$ & $      2946.71$ & $      2946.71$ & $         0.88$ sec    & $       2.7148$  & $       0.7473$ \\ 
\bottomrule
\end{tabular}
\end{table}

\begin{table}[H]
\scriptsize
\centering
\caption{image-seg (216081.bmp)}
\label{tab:anytimetable-image-seg-216081.bmp}
\begin{tabular}{lrrrrrrrrrrr}
\toprule
           algorithm &                                   \multicolumn{8}{c}{value} & \multicolumn{1}{c}{time}    & \multicolumn{1}{c}{VI}  & \multicolumn{1}{c}{RI} \\  
\cmidrule(lr){2-9}\cmidrule(lr){10-10} \cmidrule(lr){11-11} \cmidrule(lr){12-12}   
                     & \multicolumn{1}{c}{(0.5 sec)} & \multicolumn{1}{c}{(1 sec)} & \multicolumn{1}{c}{(10 sec)} & \multicolumn{1}{c}{(60 sec)} & \multicolumn{1}{c}{(300 sec)} & \multicolumn{1}{c}{(600 sec)} & \multicolumn{1}{c}{(1800 sec)} & \multicolumn{1}{c}{(end)} & \multicolumn{1}{c}{(end)}    & \multicolumn{1}{c}{(end)}   & \multicolumn{1}{c}{(end)}  \\ \midrule 
          PIVIT-BOEM & $\infty$ & $\infty$ & $\infty$ & $      5709.21$ & $      5709.21$ & $      5709.21$ & $      5709.21$ & $      5709.21$ & $        21.32$ sec    & $       4.0093$  & $       0.9016$ \\ 
                 CGC & $      4166.86$ & $      4166.86$ & $      4166.86$ & $      4166.86$ & $      4166.86$ & $      4166.86$ & $      4166.86$ & $      4166.86$ & $         0.06$ sec    & $       2.2544$  & $       0.9224$ \\ 
                  HC & $      4447.97$ & $      4447.97$ & $      4447.97$ & $      4447.97$ & $      4447.97$ & $      4447.97$ & $      4447.97$ & $      4447.97$ & $         0.00$ sec    & $       2.2734$  & $       0.9212$ \\ 
              HC-CGC & $      4163.11$ & $      4163.11$ & $      4163.11$ & $      4163.11$ & $      4163.11$ & $      4163.11$ & $      4163.11$ & $      4163.11$ & $         0.04$ sec    & $       2.2990$  & $       0.9222$ \\ 
              ogm-KL & $      4263.48$ & $      4263.48$ & $      4263.48$ & $      4263.48$ & $      4263.48$ & $      4263.48$ & $      4263.48$ & $      4263.48$ & $         0.53$ sec    & $       2.9393$  & $       0.8620$ \\ 
    CC-Fusion-HC-CGC & $      4158.73$ & $      4158.73$ & $      4158.73$ & $      4158.73$ & $      4158.73$ & $      4158.73$ & $      4158.73$ & $      4158.73$ & $         0.67$ sec    & $       2.2304$  & $       0.9244$ \\ 
     CC-Fusion-HC-MC & $      4158.73$ & $      4158.73$ & $      4158.73$ & $      4158.73$ & $      4158.73$ & $      4158.73$ & $      4158.73$ & $      4158.73$ & $         2.18$ sec    & $       2.2304$  & $       0.9244$ \\ 
    CC-Fusion-WS-CGC & $      4170.28$ & $      4170.28$ & $      4170.28$ & $      4170.28$ & $      4170.28$ & $      4170.28$ & $      4170.28$ & $      4170.28$ & $         0.66$ sec    & $       2.2623$  & $       0.9233$ \\ 
     CC-Fusion-WS-MC & $      4175.99$ & $      4165.39$ & $      4158.73$ & $      4158.73$ & $      4158.73$ & $      4158.73$ & $      4158.73$ & $      4158.73$ & $         3.95$ sec    & $       2.2304$  & $       0.9244$ \\ 
\cmidrule{1-1} 
           MCR-CCFDB & $      4159.53$ & $      4159.53$ & $      4159.53$ & $      4159.53$ & $      4159.53$ & $      4159.53$ & $      4159.53$ & $      4159.53$ & $         0.05$ sec    & $       2.2304$  & $       0.9244$ \\ 
\cmidrule{1-1} 
           MCI-CCIFD & $      4159.27$ & $      4158.73$ & $      4158.73$ & $      4158.73$ & $      4158.73$ & $      4158.73$ & $      4158.73$ & $      4158.73$ & $         0.61$ sec    & $       2.2304$  & $       0.9244$ \\ 
\bottomrule
\end{tabular}
\end{table}

\begin{table}[H]
\scriptsize
\centering
\caption{image-seg (219090.bmp)}
\label{tab:anytimetable-image-seg-219090.bmp}
\begin{tabular}{lrrrrrrrrrrr}
\toprule
           algorithm &                                   \multicolumn{8}{c}{value} & \multicolumn{1}{c}{time}    & \multicolumn{1}{c}{VI}  & \multicolumn{1}{c}{RI} \\  
\cmidrule(lr){2-9}\cmidrule(lr){10-10} \cmidrule(lr){11-11} \cmidrule(lr){12-12}   
                     & \multicolumn{1}{c}{(0.5 sec)} & \multicolumn{1}{c}{(1 sec)} & \multicolumn{1}{c}{(10 sec)} & \multicolumn{1}{c}{(60 sec)} & \multicolumn{1}{c}{(300 sec)} & \multicolumn{1}{c}{(600 sec)} & \multicolumn{1}{c}{(1800 sec)} & \multicolumn{1}{c}{(end)} & \multicolumn{1}{c}{(end)}    & \multicolumn{1}{c}{(end)}   & \multicolumn{1}{c}{(end)}  \\ \midrule 
          PIVIT-BOEM & $\infty$ & $\infty$ & $      3485.32$ & $      3485.32$ & $      3485.32$ & $      3485.32$ & $      3485.32$ & $      3485.32$ & $         3.99$ sec    & $       3.8640$  & $       0.8106$ \\ 
                 CGC & $      2502.92$ & $      2502.92$ & $      2502.92$ & $      2502.92$ & $      2502.92$ & $      2502.92$ & $      2502.92$ & $      2502.92$ & $         0.06$ sec    & $       1.3574$  & $       0.9492$ \\ 
                  HC & $      2661.40$ & $      2661.40$ & $      2661.40$ & $      2661.40$ & $      2661.40$ & $      2661.40$ & $      2661.40$ & $      2661.40$ & $         0.00$ sec    & $       1.3574$  & $       0.9477$ \\ 
              HC-CGC & $      2502.46$ & $      2502.46$ & $      2502.46$ & $      2502.46$ & $      2502.46$ & $      2502.46$ & $      2502.46$ & $      2502.46$ & $         0.03$ sec    & $       1.3586$  & $       0.9494$ \\ 
              ogm-KL & $      2576.46$ & $      2576.46$ & $      2576.46$ & $      2576.46$ & $      2576.46$ & $      2576.46$ & $      2576.46$ & $      2576.46$ & $         0.28$ sec    & $       2.0018$  & $       0.7534$ \\ 
    CC-Fusion-HC-CGC & $      2501.27$ & $      2501.27$ & $      2501.27$ & $      2501.27$ & $      2501.27$ & $      2501.27$ & $      2501.27$ & $      2501.27$ & $         0.32$ sec    & $       1.3548$  & $       0.9494$ \\ 
     CC-Fusion-HC-MC & $      2501.27$ & $      2501.27$ & $      2501.27$ & $      2501.27$ & $      2501.27$ & $      2501.27$ & $      2501.27$ & $      2501.27$ & $         1.15$ sec    & $       1.3548$  & $       0.9494$ \\ 
    CC-Fusion-WS-CGC & $      2501.52$ & $      2501.52$ & $      2501.52$ & $      2501.52$ & $      2501.52$ & $      2501.52$ & $      2501.52$ & $      2501.52$ & $         0.42$ sec    & $       1.3514$  & $       0.9494$ \\ 
     CC-Fusion-WS-MC & $      2502.08$ & $      2501.27$ & $      2501.27$ & $      2501.27$ & $      2501.27$ & $      2501.27$ & $      2501.27$ & $      2501.27$ & $         1.83$ sec    & $       1.3548$  & $       0.9494$ \\ 
\cmidrule{1-1} 
           MCR-CCFDB & $      2501.27$ & $      2501.27$ & $      2501.27$ & $      2501.27$ & $      2501.27$ & $      2501.27$ & $      2501.27$ & $      2501.27$ & $         0.03$ sec    & $       1.3548$  & $       0.9494$ \\ 
\cmidrule{1-1} 
           MCI-CCIFD & $      2501.27$ & $      2501.27$ & $      2501.27$ & $      2501.27$ & $      2501.27$ & $      2501.27$ & $      2501.27$ & $      2501.27$ & $         0.05$ sec    & $       1.3548$  & $       0.9494$ \\ 
\bottomrule
\end{tabular}
\end{table}

\begin{table}[H]
\scriptsize
\centering
\caption{image-seg (220075.bmp)}
\label{tab:anytimetable-image-seg-220075.bmp}
\begin{tabular}{lrrrrrrrrrrr}
\toprule
           algorithm &                                   \multicolumn{8}{c}{value} & \multicolumn{1}{c}{time}    & \multicolumn{1}{c}{VI}  & \multicolumn{1}{c}{RI} \\  
\cmidrule(lr){2-9}\cmidrule(lr){10-10} \cmidrule(lr){11-11} \cmidrule(lr){12-12}   
                     & \multicolumn{1}{c}{(0.5 sec)} & \multicolumn{1}{c}{(1 sec)} & \multicolumn{1}{c}{(10 sec)} & \multicolumn{1}{c}{(60 sec)} & \multicolumn{1}{c}{(300 sec)} & \multicolumn{1}{c}{(600 sec)} & \multicolumn{1}{c}{(1800 sec)} & \multicolumn{1}{c}{(end)} & \multicolumn{1}{c}{(end)}    & \multicolumn{1}{c}{(end)}   & \multicolumn{1}{c}{(end)}  \\ \midrule 
          PIVIT-BOEM & $\infty$ & $\infty$ & $      3901.26$ & $      3901.26$ & $      3901.26$ & $      3901.26$ & $      3901.26$ & $      3901.26$ & $         8.53$ sec    & $       5.0626$  & $       0.7871$ \\ 
                 CGC & $      3127.17$ & $      3127.17$ & $      3127.17$ & $      3127.17$ & $      3127.17$ & $      3127.17$ & $      3127.17$ & $      3127.17$ & $         0.04$ sec    & $       3.3135$  & $       0.7940$ \\ 
                  HC & $      3313.61$ & $      3313.61$ & $      3313.61$ & $      3313.61$ & $      3313.61$ & $      3313.61$ & $      3313.61$ & $      3313.61$ & $         0.00$ sec    & $       3.3330$  & $       0.7991$ \\ 
              HC-CGC & $      3124.32$ & $      3124.32$ & $      3124.32$ & $      3124.32$ & $      3124.32$ & $      3124.32$ & $      3124.32$ & $      3124.32$ & $         0.04$ sec    & $       3.2815$  & $       0.7989$ \\ 
              ogm-KL & $      3154.56$ & $      3154.56$ & $      3154.56$ & $      3154.56$ & $      3154.56$ & $      3154.56$ & $      3154.56$ & $      3154.56$ & $         0.12$ sec    & $       2.9903$  & $       0.7897$ \\ 
    CC-Fusion-HC-CGC & $      3117.29$ & $      3117.29$ & $      3117.29$ & $      3117.29$ & $      3117.29$ & $      3117.29$ & $      3117.29$ & $      3117.29$ & $         0.43$ sec    & $       3.3184$  & $       0.7962$ \\ 
     CC-Fusion-HC-MC & $      3119.68$ & $      3117.29$ & $      3115.95$ & $      3115.95$ & $      3115.95$ & $      3115.95$ & $      3115.95$ & $      3115.95$ & $         3.21$ sec    & $       3.2603$  & $       0.7969$ \\ 
    CC-Fusion-WS-CGC & $      3116.68$ & $      3116.11$ & $      3116.11$ & $      3116.11$ & $      3116.11$ & $      3116.11$ & $      3116.11$ & $      3116.11$ & $         1.03$ sec    & $       3.2549$  & $       0.7970$ \\ 
     CC-Fusion-WS-MC & $      3117.11$ & $      3116.30$ & $      3115.95$ & $      3115.95$ & $      3115.95$ & $      3115.95$ & $      3115.95$ & $      3115.95$ & $         2.89$ sec    & $       3.2603$  & $       0.7969$ \\ 
\cmidrule{1-1} 
           MCR-CCFDB & $      3115.95$ & $      3115.95$ & $      3115.95$ & $      3115.95$ & $      3115.95$ & $      3115.95$ & $      3115.95$ & $      3115.95$ & $         0.02$ sec    & $       3.2603$  & $       0.7969$ \\ 
\cmidrule{1-1} 
           MCI-CCIFD & $      3115.95$ & $      3115.95$ & $      3115.95$ & $      3115.95$ & $      3115.95$ & $      3115.95$ & $      3115.95$ & $      3115.95$ & $         0.12$ sec    & $       3.2603$  & $       0.7969$ \\ 
\bottomrule
\end{tabular}
\end{table}

\begin{table}[H]
\scriptsize
\centering
\caption{image-seg (223061.bmp)}
\label{tab:anytimetable-image-seg-223061.bmp}
\begin{tabular}{lrrrrrrrrrrr}
\toprule
           algorithm &                                   \multicolumn{8}{c}{value} & \multicolumn{1}{c}{time}    & \multicolumn{1}{c}{VI}  & \multicolumn{1}{c}{RI} \\  
\cmidrule(lr){2-9}\cmidrule(lr){10-10} \cmidrule(lr){11-11} \cmidrule(lr){12-12}   
                     & \multicolumn{1}{c}{(0.5 sec)} & \multicolumn{1}{c}{(1 sec)} & \multicolumn{1}{c}{(10 sec)} & \multicolumn{1}{c}{(60 sec)} & \multicolumn{1}{c}{(300 sec)} & \multicolumn{1}{c}{(600 sec)} & \multicolumn{1}{c}{(1800 sec)} & \multicolumn{1}{c}{(end)} & \multicolumn{1}{c}{(end)}    & \multicolumn{1}{c}{(end)}   & \multicolumn{1}{c}{(end)}  \\ \midrule 
          PIVIT-BOEM & $\infty$ & $\infty$ & $\infty$ & $\infty$ & $      9155.86$ & $      9155.86$ & $      9155.86$ & $      9155.86$ & $        66.84$ sec    & $       5.2630$  & $       0.7366$ \\ 
                 CGC & $      6717.33$ & $      6714.01$ & $      6645.67$ & $      6614.35$ & $      6614.35$ & $      6614.35$ & $      6614.35$ & $      6614.35$ & $        27.32$ sec    & $       2.4830$  & $       0.7330$ \\ 
                  HC & $      7157.96$ & $      7157.96$ & $      7157.96$ & $      7157.96$ & $      7157.96$ & $      7157.96$ & $      7157.96$ & $      7157.96$ & $         0.01$ sec    & $       2.8465$  & $       0.7424$ \\ 
              HC-CGC & $      6759.27$ & $      6650.76$ & $      6605.00$ & $      6605.00$ & $      6605.00$ & $      6605.00$ & $      6605.00$ & $      6605.00$ & $         5.46$ sec    & $       2.3223$  & $       0.8007$ \\ 
              ogm-KL & $      6823.53$ & $      6789.08$ & $      6789.08$ & $      6789.08$ & $      6789.08$ & $      6789.08$ & $      6789.08$ & $      6789.08$ & $         1.08$ sec    & $       2.8707$  & $       0.4413$ \\ 
    CC-Fusion-HC-CGC & $      6675.35$ & $      6654.09$ & $      6640.56$ & $      6640.56$ & $      6640.56$ & $      6640.56$ & $      6640.56$ & $      6640.56$ & $         2.14$ sec    & $       2.3010$  & $       0.7823$ \\ 
     CC-Fusion-HC-MC & $      6667.42$ & $      6620.43$ & $      6584.23$ & $      6580.38$ & $      6580.38$ & $      6580.38$ & $      6580.38$ & $      6580.38$ & $        28.95$ sec    & $       2.2809$  & $       0.8084$ \\ 
    CC-Fusion-WS-CGC & $      6681.38$ & $      6680.10$ & $      6652.80$ & $      6652.80$ & $      6652.80$ & $      6652.80$ & $      6652.80$ & $      6652.80$ & $         2.55$ sec    & $       2.2396$  & $       0.7917$ \\ 
     CC-Fusion-WS-MC & $      7130.21$ & $      6779.34$ & $      6588.76$ & $      6584.14$ & $      6584.14$ & $      6584.14$ & $      6584.14$ & $      6584.14$ & $        27.68$ sec    & $       2.3161$  & $       0.7914$ \\ 
\cmidrule{1-1} 
           MCR-CCFDB & $      7564.80$ & $      6989.93$ & $      6585.28$ & $      6585.28$ & $      6585.28$ & $      6585.28$ & $      6585.28$ & $      6585.28$ & $         2.54$ sec    & $       2.3196$  & $       0.8070$ \\ 
\cmidrule{1-1} 
           MCI-CCIFD & $      7027.70$ & $      6883.68$ & $      6576.83$ & $      6576.83$ & $      6576.83$ & $      6576.83$ & $      6576.83$ & $      6576.83$ & $         4.81$ sec    & $       2.2995$  & $       0.8072$ \\ 
\bottomrule
\end{tabular}
\end{table}

\begin{table}[H]
\scriptsize
\centering
\caption{image-seg (227092.bmp)}
\label{tab:anytimetable-image-seg-227092.bmp}
\begin{tabular}{lrrrrrrrrrrr}
\toprule
           algorithm &                                   \multicolumn{8}{c}{value} & \multicolumn{1}{c}{time}    & \multicolumn{1}{c}{VI}  & \multicolumn{1}{c}{RI} \\  
\cmidrule(lr){2-9}\cmidrule(lr){10-10} \cmidrule(lr){11-11} \cmidrule(lr){12-12}   
                     & \multicolumn{1}{c}{(0.5 sec)} & \multicolumn{1}{c}{(1 sec)} & \multicolumn{1}{c}{(10 sec)} & \multicolumn{1}{c}{(60 sec)} & \multicolumn{1}{c}{(300 sec)} & \multicolumn{1}{c}{(600 sec)} & \multicolumn{1}{c}{(1800 sec)} & \multicolumn{1}{c}{(end)} & \multicolumn{1}{c}{(end)}    & \multicolumn{1}{c}{(end)}   & \multicolumn{1}{c}{(end)}  \\ \midrule 
          PIVIT-BOEM & $\infty$ & $\infty$ & $      2761.68$ & $      2761.68$ & $      2761.68$ & $      2761.68$ & $      2761.68$ & $      2761.68$ & $         1.28$ sec    & $       2.5505$  & $       0.8820$ \\ 
                 CGC & $      2014.62$ & $      2014.62$ & $      2014.62$ & $      2014.62$ & $      2014.62$ & $      2014.62$ & $      2014.62$ & $      2014.62$ & $         0.26$ sec    & $       1.3719$  & $       0.8888$ \\ 
                  HC & $      2134.47$ & $      2134.47$ & $      2134.47$ & $      2134.47$ & $      2134.47$ & $      2134.47$ & $      2134.47$ & $      2134.47$ & $         0.00$ sec    & $       1.5525$  & $       0.8856$ \\ 
              HC-CGC & $      2015.89$ & $      2015.89$ & $      2015.89$ & $      2015.89$ & $      2015.89$ & $      2015.89$ & $      2015.89$ & $      2015.89$ & $         0.31$ sec    & $       1.4132$  & $       0.8872$ \\ 
              ogm-KL & $      2071.11$ & $      2071.11$ & $      2071.11$ & $      2071.11$ & $      2071.11$ & $      2071.11$ & $      2071.11$ & $      2071.11$ & $         0.03$ sec    & $       1.9407$  & $       0.7325$ \\ 
    CC-Fusion-HC-CGC & $      2001.14$ & $      2001.14$ & $      2001.14$ & $      2001.14$ & $      2001.14$ & $      2001.14$ & $      2001.14$ & $      2001.14$ & $         0.52$ sec    & $       1.5152$  & $       0.8824$ \\ 
     CC-Fusion-HC-MC & $      2000.89$ & $      1999.16$ & $      1998.46$ & $      1998.46$ & $      1998.46$ & $      1998.46$ & $      1998.46$ & $      1998.46$ & $         4.13$ sec    & $       1.5330$  & $       0.8824$ \\ 
    CC-Fusion-WS-CGC & $      2003.52$ & $      2003.52$ & $      2003.52$ & $      2003.52$ & $      2003.52$ & $      2003.52$ & $      2003.52$ & $      2003.52$ & $         0.48$ sec    & $       1.4798$  & $       0.8830$ \\ 
     CC-Fusion-WS-MC & $      2002.93$ & $      2000.81$ & $      1998.46$ & $      1998.46$ & $      1998.46$ & $      1998.46$ & $      1998.46$ & $      1998.46$ & $        10.49$ sec    & $       1.5330$  & $       0.8824$ \\ 
\cmidrule{1-1} 
           MCR-CCFDB & $      2004.79$ & $      2004.79$ & $      2004.79$ & $      2004.79$ & $      2004.79$ & $      2004.79$ & $      2004.79$ & $      2004.79$ & $         0.16$ sec    & $       1.5412$  & $       0.8846$ \\ 
\cmidrule{1-1} 
           MCI-CCIFD & $      2002.79$ & $      1998.46$ & $      1998.46$ & $      1998.46$ & $      1998.46$ & $      1998.46$ & $      1998.46$ & $      1998.46$ & $         0.52$ sec    & $       1.5330$  & $       0.8824$ \\ 
\bottomrule
\end{tabular}
\end{table}

\begin{table}[H]
\scriptsize
\centering
\caption{image-seg (229036.bmp)}
\label{tab:anytimetable-image-seg-229036.bmp}
\begin{tabular}{lrrrrrrrrrrr}
\toprule
           algorithm &                                   \multicolumn{8}{c}{value} & \multicolumn{1}{c}{time}    & \multicolumn{1}{c}{VI}  & \multicolumn{1}{c}{RI} \\  
\cmidrule(lr){2-9}\cmidrule(lr){10-10} \cmidrule(lr){11-11} \cmidrule(lr){12-12}   
                     & \multicolumn{1}{c}{(0.5 sec)} & \multicolumn{1}{c}{(1 sec)} & \multicolumn{1}{c}{(10 sec)} & \multicolumn{1}{c}{(60 sec)} & \multicolumn{1}{c}{(300 sec)} & \multicolumn{1}{c}{(600 sec)} & \multicolumn{1}{c}{(1800 sec)} & \multicolumn{1}{c}{(end)} & \multicolumn{1}{c}{(end)}    & \multicolumn{1}{c}{(end)}   & \multicolumn{1}{c}{(end)}  \\ \midrule 
          PIVIT-BOEM & $\infty$ & $\infty$ & $\infty$ & $\infty$ & $      8411.18$ & $      8411.18$ & $      8411.18$ & $      8411.18$ & $        71.16$ sec    & $       5.9445$  & $       0.6115$ \\ 
                 CGC & $      6213.58$ & $      6148.87$ & $      6141.14$ & $      6141.14$ & $      6141.14$ & $      6141.14$ & $      6141.14$ & $      6141.14$ & $         2.58$ sec    & $       1.9292$  & $       0.8413$ \\ 
                  HC & $      6581.68$ & $      6581.68$ & $      6581.68$ & $      6581.68$ & $      6581.68$ & $      6581.68$ & $      6581.68$ & $      6581.68$ & $         0.01$ sec    & $       2.6773$  & $       0.6886$ \\ 
              HC-CGC & $      6148.26$ & $      6144.46$ & $      6144.46$ & $      6144.46$ & $      6144.46$ & $      6144.46$ & $      6144.46$ & $      6144.46$ & $         0.66$ sec    & $       2.3084$  & $       0.7383$ \\ 
              ogm-KL & $      7306.79$ & $      6284.85$ & $      6267.68$ & $      6267.68$ & $      6267.68$ & $      6267.68$ & $      6267.68$ & $      6267.68$ & $         1.53$ sec    & $       1.7054$  & $       0.8602$ \\ 
    CC-Fusion-HC-CGC & $      6146.87$ & $      6145.64$ & $      6135.52$ & $      6135.52$ & $      6135.52$ & $      6135.52$ & $      6135.52$ & $      6135.52$ & $         2.09$ sec    & $       2.5261$  & $       0.7068$ \\ 
     CC-Fusion-HC-MC & $      6140.25$ & $      6134.87$ & $      6132.91$ & $      6132.91$ & $      6132.91$ & $      6132.91$ & $      6132.91$ & $      6132.91$ & $         3.72$ sec    & $       2.5320$  & $       0.7056$ \\ 
    CC-Fusion-WS-CGC & $      6150.94$ & $      6134.59$ & $      6132.47$ & $      6132.47$ & $      6132.47$ & $      6132.47$ & $      6132.47$ & $      6132.47$ & $         1.82$ sec    & $       1.9610$  & $       0.8349$ \\ 
     CC-Fusion-WS-MC & $      6179.58$ & $      6147.24$ & $      6130.44$ & $      6130.44$ & $      6130.44$ & $      6130.44$ & $      6130.44$ & $      6130.44$ & $         8.10$ sec    & $       2.5520$  & $       0.7041$ \\ 
\cmidrule{1-1} 
           MCR-CCFDB & $      6125.73$ & $      6125.73$ & $      6125.73$ & $      6125.73$ & $      6125.73$ & $      6125.73$ & $      6125.73$ & $      6125.73$ & $         0.42$ sec    & $       2.3811$  & $       0.7322$ \\ 
\cmidrule{1-1} 
           MCI-CCIFD & $      6144.84$ & $      6125.73$ & $      6125.73$ & $      6125.73$ & $      6125.73$ & $      6125.73$ & $      6125.73$ & $      6125.73$ & $         0.60$ sec    & $       2.3811$  & $       0.7322$ \\ 
\bottomrule
\end{tabular}
\end{table}

\begin{table}[H]
\scriptsize
\centering
\caption{image-seg (236037.bmp)}
\label{tab:anytimetable-image-seg-236037.bmp}
\begin{tabular}{lrrrrrrrrrrr}
\toprule
           algorithm &                                   \multicolumn{8}{c}{value} & \multicolumn{1}{c}{time}    & \multicolumn{1}{c}{VI}  & \multicolumn{1}{c}{RI} \\  
\cmidrule(lr){2-9}\cmidrule(lr){10-10} \cmidrule(lr){11-11} \cmidrule(lr){12-12}   
                     & \multicolumn{1}{c}{(0.5 sec)} & \multicolumn{1}{c}{(1 sec)} & \multicolumn{1}{c}{(10 sec)} & \multicolumn{1}{c}{(60 sec)} & \multicolumn{1}{c}{(300 sec)} & \multicolumn{1}{c}{(600 sec)} & \multicolumn{1}{c}{(1800 sec)} & \multicolumn{1}{c}{(end)} & \multicolumn{1}{c}{(end)}    & \multicolumn{1}{c}{(end)}   & \multicolumn{1}{c}{(end)}  \\ \midrule 
          PIVIT-BOEM & $\infty$ & $\infty$ & $\infty$ & $\infty$ & $     11411.55$ & $     11411.55$ & $     11411.55$ & $     11411.55$ & $       182.21$ sec    & $       7.5487$  & $       0.6430$ \\ 
                 CGC & $      9577.14$ & $      9478.78$ & $      9154.99$ & $      9140.68$ & $      9140.68$ & $      9140.68$ & $      9140.68$ & $      9140.68$ & $        20.12$ sec    & $       4.4172$  & $       0.5410$ \\ 
                  HC & $      9863.37$ & $      9863.37$ & $      9863.37$ & $      9863.37$ & $      9863.37$ & $      9863.37$ & $      9863.37$ & $      9863.37$ & $         0.01$ sec    & $       4.2743$  & $       0.6661$ \\ 
              HC-CGC & $      9151.42$ & $      9115.52$ & $      9113.71$ & $      9113.71$ & $      9113.71$ & $      9113.71$ & $      9113.71$ & $      9113.71$ & $         1.24$ sec    & $       4.3505$  & $       0.6669$ \\ 
              ogm-KL & $     12496.16$ & $     12496.16$ & $      9479.67$ & $      9479.67$ & $      9479.67$ & $      9479.67$ & $      9479.67$ & $      9479.67$ & $         4.92$ sec    & $       3.1258$  & $       0.4677$ \\ 
    CC-Fusion-HC-CGC & $      9185.41$ & $      9179.06$ & $      9176.54$ & $      9176.54$ & $      9176.54$ & $      9176.54$ & $      9176.54$ & $      9176.54$ & $         2.89$ sec    & $       4.6518$  & $       0.6546$ \\ 
     CC-Fusion-HC-MC & $      9259.59$ & $      9132.00$ & $      9062.61$ & $      9061.18$ & $      9061.18$ & $      9061.18$ & $      9061.18$ & $      9061.18$ & $        23.94$ sec    & $       4.7557$  & $       0.6534$ \\ 
    CC-Fusion-WS-CGC & $      9272.47$ & $      9249.77$ & $      9249.77$ & $      9249.77$ & $      9249.77$ & $      9249.77$ & $      9249.77$ & $      9249.77$ & $         2.29$ sec    & $       4.8479$  & $       0.6395$ \\ 
     CC-Fusion-WS-MC & $      9823.29$ & $      9346.86$ & $      9066.91$ & $      9060.84$ & $      9060.84$ & $      9060.84$ & $      9060.84$ & $      9060.84$ & $        50.69$ sec    & $       4.7584$  & $       0.6533$ \\ 
\cmidrule{1-1} 
           MCR-CCFDB & $     12499.39$ & $     10650.43$ & $      9072.01$ & $      9072.01$ & $      9072.01$ & $      9072.01$ & $      9072.01$ & $      9072.01$ & $         1.71$ sec    & $       4.8120$  & $       0.6529$ \\ 
\cmidrule{1-1} 
           MCI-CCIFD & $     10293.90$ & $      9232.77$ & $      9060.84$ & $      9060.84$ & $      9060.84$ & $      9060.84$ & $      9060.84$ & $      9060.84$ & $         3.61$ sec    & $       4.7584$  & $       0.6533$ \\ 
\bottomrule
\end{tabular}
\end{table}

\begin{table}[H]
\scriptsize
\centering
\caption{image-seg (24077.bmp)}
\label{tab:anytimetable-image-seg-24077.bmp}
\begin{tabular}{lrrrrrrrrrrr}
\toprule
           algorithm &                                   \multicolumn{8}{c}{value} & \multicolumn{1}{c}{time}    & \multicolumn{1}{c}{VI}  & \multicolumn{1}{c}{RI} \\  
\cmidrule(lr){2-9}\cmidrule(lr){10-10} \cmidrule(lr){11-11} \cmidrule(lr){12-12}   
                     & \multicolumn{1}{c}{(0.5 sec)} & \multicolumn{1}{c}{(1 sec)} & \multicolumn{1}{c}{(10 sec)} & \multicolumn{1}{c}{(60 sec)} & \multicolumn{1}{c}{(300 sec)} & \multicolumn{1}{c}{(600 sec)} & \multicolumn{1}{c}{(1800 sec)} & \multicolumn{1}{c}{(end)} & \multicolumn{1}{c}{(end)}    & \multicolumn{1}{c}{(end)}   & \multicolumn{1}{c}{(end)}  \\ \midrule 
          PIVIT-BOEM & $\infty$ & $\infty$ & $\infty$ & $      5787.16$ & $      5787.16$ & $      5787.16$ & $      5787.16$ & $      5787.16$ & $        31.09$ sec    & $       4.3431$  & $       0.9140$ \\ 
                 CGC & $      4773.14$ & $      4773.14$ & $      4773.14$ & $      4773.14$ & $      4773.14$ & $      4773.14$ & $      4773.14$ & $      4773.14$ & $         0.12$ sec    & $       3.0879$  & $       0.9175$ \\ 
                  HC & $      5151.15$ & $      5151.15$ & $      5151.15$ & $      5151.15$ & $      5151.15$ & $      5151.15$ & $      5151.15$ & $      5151.15$ & $         0.00$ sec    & $       2.8252$  & $       0.9219$ \\ 
              HC-CGC & $      4778.48$ & $      4778.48$ & $      4778.48$ & $      4778.48$ & $      4778.48$ & $      4778.48$ & $      4778.48$ & $      4778.48$ & $         0.16$ sec    & $       3.0422$  & $       0.9185$ \\ 
              ogm-KL & $      4872.82$ & $      4867.75$ & $      4867.75$ & $      4867.75$ & $      4867.75$ & $      4867.75$ & $      4867.75$ & $      4867.75$ & $         0.76$ sec    & $       3.3160$  & $       0.8637$ \\ 
    CC-Fusion-HC-CGC & $      4764.91$ & $      4764.12$ & $      4762.09$ & $      4762.09$ & $      4762.09$ & $      4762.09$ & $      4762.09$ & $      4762.09$ & $         1.79$ sec    & $       3.0741$  & $       0.9188$ \\ 
     CC-Fusion-HC-MC & $      4773.03$ & $      4763.78$ & $      4763.78$ & $      4763.78$ & $      4763.78$ & $      4763.78$ & $      4763.78$ & $      4763.78$ & $         2.84$ sec    & $       3.0819$  & $       0.9189$ \\ 
    CC-Fusion-WS-CGC & $      4771.27$ & $      4768.44$ & $      4768.44$ & $      4768.44$ & $      4768.44$ & $      4768.44$ & $      4768.44$ & $      4768.44$ & $         1.29$ sec    & $       3.0367$  & $       0.9196$ \\ 
     CC-Fusion-WS-MC & $      4814.81$ & $      4784.92$ & $      4763.78$ & $      4763.78$ & $      4763.78$ & $      4763.78$ & $      4763.78$ & $      4763.78$ & $         8.92$ sec    & $       3.0819$  & $       0.9189$ \\ 
\cmidrule{1-1} 
           MCR-CCFDB & $      4766.37$ & $      4766.37$ & $      4766.37$ & $      4766.37$ & $      4766.37$ & $      4766.37$ & $      4766.37$ & $      4766.37$ & $         0.13$ sec    & $       3.0835$  & $       0.9188$ \\ 
\cmidrule{1-1} 
           MCI-CCIFD & $      5145.78$ & $      4761.98$ & $      4761.98$ & $      4761.98$ & $      4761.98$ & $      4761.98$ & $      4761.98$ & $      4761.98$ & $         0.73$ sec    & $       3.0783$  & $       0.9188$ \\ 
\bottomrule
\end{tabular}
\end{table}

\begin{table}[H]
\scriptsize
\centering
\caption{image-seg (241004.bmp)}
\label{tab:anytimetable-image-seg-241004.bmp}
\begin{tabular}{lrrrrrrrrrrr}
\toprule
           algorithm &                                   \multicolumn{8}{c}{value} & \multicolumn{1}{c}{time}    & \multicolumn{1}{c}{VI}  & \multicolumn{1}{c}{RI} \\  
\cmidrule(lr){2-9}\cmidrule(lr){10-10} \cmidrule(lr){11-11} \cmidrule(lr){12-12}   
                     & \multicolumn{1}{c}{(0.5 sec)} & \multicolumn{1}{c}{(1 sec)} & \multicolumn{1}{c}{(10 sec)} & \multicolumn{1}{c}{(60 sec)} & \multicolumn{1}{c}{(300 sec)} & \multicolumn{1}{c}{(600 sec)} & \multicolumn{1}{c}{(1800 sec)} & \multicolumn{1}{c}{(end)} & \multicolumn{1}{c}{(end)}    & \multicolumn{1}{c}{(end)}   & \multicolumn{1}{c}{(end)}  \\ \midrule 
          PIVIT-BOEM & $      1567.60$ & $      1567.60$ & $      1567.60$ & $      1567.60$ & $      1567.60$ & $      1567.60$ & $      1567.60$ & $      1567.60$ & $         0.31$ sec    & $       1.7650$  & $       0.9361$ \\ 
                 CGC & $      1060.32$ & $      1060.32$ & $      1060.32$ & $      1060.32$ & $      1060.32$ & $      1060.32$ & $      1060.32$ & $      1060.32$ & $         0.01$ sec    & $       1.3261$  & $       0.9077$ \\ 
                  HC & $      1107.76$ & $      1107.76$ & $      1107.76$ & $      1107.76$ & $      1107.76$ & $      1107.76$ & $      1107.76$ & $      1107.76$ & $         0.00$ sec    & $       1.1400$  & $       0.9381$ \\ 
              HC-CGC & $      1060.82$ & $      1060.82$ & $      1060.82$ & $      1060.82$ & $      1060.82$ & $      1060.82$ & $      1060.82$ & $      1060.82$ & $         0.01$ sec    & $       1.3406$  & $       0.9076$ \\ 
              ogm-KL & $      1113.75$ & $      1113.75$ & $      1113.75$ & $      1113.75$ & $      1113.75$ & $      1113.75$ & $      1113.75$ & $      1113.75$ & $         0.01$ sec    & $       2.1385$  & $       0.7698$ \\ 
    CC-Fusion-HC-CGC & $      1057.42$ & $      1057.42$ & $      1057.42$ & $      1057.42$ & $      1057.42$ & $      1057.42$ & $      1057.42$ & $      1057.42$ & $         0.10$ sec    & $       1.3036$  & $       0.9090$ \\ 
     CC-Fusion-HC-MC & $      1057.42$ & $      1057.14$ & $      1057.14$ & $      1057.14$ & $      1057.14$ & $      1057.14$ & $      1057.14$ & $      1057.14$ & $         1.33$ sec    & $       1.3521$  & $       0.9086$ \\ 
    CC-Fusion-WS-CGC & $      1057.42$ & $      1057.42$ & $      1057.42$ & $      1057.42$ & $      1057.42$ & $      1057.42$ & $      1057.42$ & $      1057.42$ & $         0.07$ sec    & $       1.3036$  & $       0.9090$ \\ 
     CC-Fusion-WS-MC & $      1057.42$ & $      1057.42$ & $      1057.42$ & $      1057.42$ & $      1057.42$ & $      1057.42$ & $      1057.42$ & $      1057.42$ & $         0.80$ sec    & $       1.3036$  & $       0.9090$ \\ 
\cmidrule{1-1} 
           MCR-CCFDB & $      1057.14$ & $      1057.14$ & $      1057.14$ & $      1057.14$ & $      1057.14$ & $      1057.14$ & $      1057.14$ & $      1057.14$ & $         0.01$ sec    & $       1.3521$  & $       0.9086$ \\ 
\cmidrule{1-1} 
           MCI-CCIFD & $      1057.14$ & $      1057.14$ & $      1057.14$ & $      1057.14$ & $      1057.14$ & $      1057.14$ & $      1057.14$ & $      1057.14$ & $         0.02$ sec    & $       1.3521$  & $       0.9086$ \\ 
\bottomrule
\end{tabular}
\end{table}

\begin{table}[H]
\scriptsize
\centering
\caption{image-seg (241048.bmp)}
\label{tab:anytimetable-image-seg-241048.bmp}
\begin{tabular}{lrrrrrrrrrrr}
\toprule
           algorithm &                                   \multicolumn{8}{c}{value} & \multicolumn{1}{c}{time}    & \multicolumn{1}{c}{VI}  & \multicolumn{1}{c}{RI} \\  
\cmidrule(lr){2-9}\cmidrule(lr){10-10} \cmidrule(lr){11-11} \cmidrule(lr){12-12}   
                     & \multicolumn{1}{c}{(0.5 sec)} & \multicolumn{1}{c}{(1 sec)} & \multicolumn{1}{c}{(10 sec)} & \multicolumn{1}{c}{(60 sec)} & \multicolumn{1}{c}{(300 sec)} & \multicolumn{1}{c}{(600 sec)} & \multicolumn{1}{c}{(1800 sec)} & \multicolumn{1}{c}{(end)} & \multicolumn{1}{c}{(end)}    & \multicolumn{1}{c}{(end)}   & \multicolumn{1}{c}{(end)}  \\ \midrule 
          PIVIT-BOEM & $\infty$ & $\infty$ & $\infty$ & $      6765.74$ & $      6765.74$ & $      6765.74$ & $      6765.74$ & $      6765.74$ & $        28.05$ sec    & $       5.0087$  & $       0.8553$ \\ 
                 CGC & $      4772.18$ & $      4764.50$ & $      4764.50$ & $      4764.50$ & $      4764.50$ & $      4764.50$ & $      4764.50$ & $      4764.50$ & $         0.84$ sec    & $       2.6705$  & $       0.8300$ \\ 
                  HC & $      5266.78$ & $      5266.78$ & $      5266.78$ & $      5266.78$ & $      5266.78$ & $      5266.78$ & $      5266.78$ & $      5266.78$ & $         0.00$ sec    & $       2.5360$  & $       0.8592$ \\ 
              HC-CGC & $      4754.34$ & $      4753.32$ & $      4753.32$ & $      4753.32$ & $      4753.32$ & $      4753.32$ & $      4753.32$ & $      4753.32$ & $         0.56$ sec    & $       2.6361$  & $       0.8353$ \\ 
              ogm-KL & $      4902.61$ & $      4896.67$ & $      4896.67$ & $      4896.67$ & $      4896.67$ & $      4896.67$ & $      4896.67$ & $      4896.67$ & $         0.67$ sec    & $       3.3921$  & $       0.6255$ \\ 
    CC-Fusion-HC-CGC & $      4756.74$ & $      4753.36$ & $      4753.36$ & $      4753.36$ & $      4753.36$ & $      4753.36$ & $      4753.36$ & $      4753.36$ & $         1.30$ sec    & $       2.6228$  & $       0.8451$ \\ 
     CC-Fusion-HC-MC & $      4768.44$ & $      4746.60$ & $      4731.58$ & $      4730.95$ & $      4730.95$ & $      4730.95$ & $      4730.95$ & $      4730.95$ & $        16.83$ sec    & $       2.4583$  & $       0.8849$ \\ 
    CC-Fusion-WS-CGC & $      4778.74$ & $      4773.50$ & $      4773.50$ & $      4773.50$ & $      4773.50$ & $      4773.50$ & $      4773.50$ & $      4773.50$ & $         0.93$ sec    & $       2.6370$  & $       0.8355$ \\ 
     CC-Fusion-WS-MC & $      4828.70$ & $      4783.70$ & $      4735.80$ & $      4735.80$ & $      4735.80$ & $      4735.80$ & $      4735.80$ & $      4735.80$ & $        17.78$ sec    & $       2.4438$  & $       0.8873$ \\ 
\cmidrule{1-1} 
           MCR-CCFDB & $      4740.05$ & $      4740.05$ & $      4740.05$ & $      4740.05$ & $      4740.05$ & $      4740.05$ & $      4740.05$ & $      4740.05$ & $         0.27$ sec    & $       2.4620$  & $       0.8849$ \\ 
\cmidrule{1-1} 
           MCI-CCIFD & $      4970.71$ & $      4742.82$ & $      4730.95$ & $      4730.95$ & $      4730.95$ & $      4730.95$ & $      4730.95$ & $      4730.95$ & $         1.24$ sec    & $       2.4583$  & $       0.8849$ \\ 
\bottomrule
\end{tabular}
\end{table}

\begin{table}[H]
\scriptsize
\centering
\caption{image-seg (253027.bmp)}
\label{tab:anytimetable-image-seg-253027.bmp}
\begin{tabular}{lrrrrrrrrrrr}
\toprule
           algorithm &                                   \multicolumn{8}{c}{value} & \multicolumn{1}{c}{time}    & \multicolumn{1}{c}{VI}  & \multicolumn{1}{c}{RI} \\  
\cmidrule(lr){2-9}\cmidrule(lr){10-10} \cmidrule(lr){11-11} \cmidrule(lr){12-12}   
                     & \multicolumn{1}{c}{(0.5 sec)} & \multicolumn{1}{c}{(1 sec)} & \multicolumn{1}{c}{(10 sec)} & \multicolumn{1}{c}{(60 sec)} & \multicolumn{1}{c}{(300 sec)} & \multicolumn{1}{c}{(600 sec)} & \multicolumn{1}{c}{(1800 sec)} & \multicolumn{1}{c}{(end)} & \multicolumn{1}{c}{(end)}    & \multicolumn{1}{c}{(end)}   & \multicolumn{1}{c}{(end)}  \\ \midrule 
          PIVIT-BOEM & $\infty$ & $\infty$ & $\infty$ & $\infty$ & $      8496.07$ & $      8496.07$ & $      8496.07$ & $      8496.07$ & $        92.25$ sec    & $       5.8696$  & $       0.5088$ \\ 
                 CGC & $      6767.89$ & $      6670.66$ & $      6646.71$ & $      6646.71$ & $      6646.71$ & $      6646.71$ & $      6646.71$ & $      6646.71$ & $         1.57$ sec    & $       2.1409$  & $       0.8347$ \\ 
                  HC & $      7091.05$ & $      7091.05$ & $      7091.05$ & $      7091.05$ & $      7091.05$ & $      7091.05$ & $      7091.05$ & $      7091.05$ & $         0.01$ sec    & $       1.9390$  & $       0.8679$ \\ 
              HC-CGC & $      6627.71$ & $      6627.71$ & $      6627.71$ & $      6627.71$ & $      6627.71$ & $      6627.71$ & $      6627.71$ & $      6627.71$ & $         0.27$ sec    & $       1.7577$  & $       0.9160$ \\ 
              ogm-KL & $     12302.90$ & $      6954.63$ & $      6899.47$ & $      6899.47$ & $      6899.47$ & $      6899.47$ & $      6899.47$ & $      6899.47$ & $         2.11$ sec    & $       2.1247$  & $       0.7032$ \\ 
    CC-Fusion-HC-CGC & $      6652.41$ & $      6642.74$ & $      6642.74$ & $      6642.74$ & $      6642.74$ & $      6642.74$ & $      6642.74$ & $      6642.74$ & $         1.53$ sec    & $       1.7541$  & $       0.9007$ \\ 
     CC-Fusion-HC-MC & $      6774.52$ & $      6625.84$ & $      6606.62$ & $      6606.62$ & $      6606.62$ & $      6606.62$ & $      6606.62$ & $      6606.62$ & $        18.55$ sec    & $       1.7626$  & $       0.9155$ \\ 
    CC-Fusion-WS-CGC & $      6709.48$ & $      6674.79$ & $      6671.77$ & $      6671.77$ & $      6671.77$ & $      6671.77$ & $      6671.77$ & $      6671.77$ & $         2.00$ sec    & $       1.7581$  & $       0.9190$ \\ 
     CC-Fusion-WS-MC & $      6889.53$ & $      6683.12$ & $      6607.07$ & $      6606.62$ & $      6606.62$ & $      6606.62$ & $      6606.62$ & $      6606.62$ & $        20.40$ sec    & $       1.7626$  & $       0.9155$ \\ 
\cmidrule{1-1} 
           MCR-CCFDB & $     10535.57$ & $      7039.94$ & $      6609.76$ & $      6609.76$ & $      6609.76$ & $      6609.76$ & $      6609.76$ & $      6609.76$ & $         1.14$ sec    & $       1.7652$  & $       0.9156$ \\ 
\cmidrule{1-1} 
           MCI-CCIFD & $      7692.95$ & $      6699.15$ & $      6606.62$ & $      6606.62$ & $      6606.62$ & $      6606.62$ & $      6606.62$ & $      6606.62$ & $         1.14$ sec    & $       1.7626$  & $       0.9155$ \\ 
\bottomrule
\end{tabular}
\end{table}

\begin{table}[H]
\scriptsize
\centering
\caption{image-seg (253055.bmp)}
\label{tab:anytimetable-image-seg-253055.bmp}
\begin{tabular}{lrrrrrrrrrrr}
\toprule
           algorithm &                                   \multicolumn{8}{c}{value} & \multicolumn{1}{c}{time}    & \multicolumn{1}{c}{VI}  & \multicolumn{1}{c}{RI} \\  
\cmidrule(lr){2-9}\cmidrule(lr){10-10} \cmidrule(lr){11-11} \cmidrule(lr){12-12}   
                     & \multicolumn{1}{c}{(0.5 sec)} & \multicolumn{1}{c}{(1 sec)} & \multicolumn{1}{c}{(10 sec)} & \multicolumn{1}{c}{(60 sec)} & \multicolumn{1}{c}{(300 sec)} & \multicolumn{1}{c}{(600 sec)} & \multicolumn{1}{c}{(1800 sec)} & \multicolumn{1}{c}{(end)} & \multicolumn{1}{c}{(end)}    & \multicolumn{1}{c}{(end)}   & \multicolumn{1}{c}{(end)}  \\ \midrule 
          PIVIT-BOEM & $\infty$ & $      2130.97$ & $      2130.97$ & $      2130.97$ & $      2130.97$ & $      2130.97$ & $      2130.97$ & $      2130.97$ & $         0.79$ sec    & $       3.7111$  & $       0.6217$ \\ 
                 CGC & $      1527.79$ & $      1527.79$ & $      1527.79$ & $      1527.79$ & $      1527.79$ & $      1527.79$ & $      1527.79$ & $      1527.79$ & $         0.03$ sec    & $       1.2598$  & $       0.8672$ \\ 
                  HC & $      1696.76$ & $      1696.76$ & $      1696.76$ & $      1696.76$ & $      1696.76$ & $      1696.76$ & $      1696.76$ & $      1696.76$ & $         0.00$ sec    & $       1.0510$  & $       0.8921$ \\ 
              HC-CGC & $      1512.99$ & $      1512.99$ & $      1512.99$ & $      1512.99$ & $      1512.99$ & $      1512.99$ & $      1512.99$ & $      1512.99$ & $         0.05$ sec    & $       1.1069$  & $       0.8844$ \\ 
              ogm-KL & $      1553.88$ & $      1553.88$ & $      1553.88$ & $      1553.88$ & $      1553.88$ & $      1553.88$ & $      1553.88$ & $      1553.88$ & $         0.04$ sec    & $       1.1437$  & $       0.8766$ \\ 
    CC-Fusion-HC-CGC & $      1502.22$ & $      1502.22$ & $      1502.22$ & $      1502.22$ & $      1502.22$ & $      1502.22$ & $      1502.22$ & $      1502.22$ & $         0.12$ sec    & $       1.0373$  & $       0.8941$ \\ 
     CC-Fusion-HC-MC & $      1502.16$ & $      1502.16$ & $      1502.16$ & $      1502.16$ & $      1502.16$ & $      1502.16$ & $      1502.16$ & $      1502.16$ & $         0.88$ sec    & $       1.0360$  & $       0.8945$ \\ 
    CC-Fusion-WS-CGC & $      1502.16$ & $      1502.16$ & $      1502.16$ & $      1502.16$ & $      1502.16$ & $      1502.16$ & $      1502.16$ & $      1502.16$ & $         0.16$ sec    & $       1.0360$  & $       0.8945$ \\ 
     CC-Fusion-WS-MC & $      1503.06$ & $      1502.16$ & $      1502.16$ & $      1502.16$ & $      1502.16$ & $      1502.16$ & $      1502.16$ & $      1502.16$ & $         1.70$ sec    & $       1.0360$  & $       0.8945$ \\ 
\cmidrule{1-1} 
           MCR-CCFDB & $      1503.91$ & $      1503.91$ & $      1503.91$ & $      1503.91$ & $      1503.91$ & $      1503.91$ & $      1503.91$ & $      1503.91$ & $         0.03$ sec    & $       1.0364$  & $       0.8945$ \\ 
\cmidrule{1-1} 
           MCI-CCIFD & $      1502.16$ & $      1502.16$ & $      1502.16$ & $      1502.16$ & $      1502.16$ & $      1502.16$ & $      1502.16$ & $      1502.16$ & $         0.22$ sec    & $       1.0360$  & $       0.8945$ \\ 
\bottomrule
\end{tabular}
\end{table}

\begin{table}[H]
\scriptsize
\centering
\caption{image-seg (260058.bmp)}
\label{tab:anytimetable-image-seg-260058.bmp}
\begin{tabular}{lrrrrrrrrrrr}
\toprule
           algorithm &                                   \multicolumn{8}{c}{value} & \multicolumn{1}{c}{time}    & \multicolumn{1}{c}{VI}  & \multicolumn{1}{c}{RI} \\  
\cmidrule(lr){2-9}\cmidrule(lr){10-10} \cmidrule(lr){11-11} \cmidrule(lr){12-12}   
                     & \multicolumn{1}{c}{(0.5 sec)} & \multicolumn{1}{c}{(1 sec)} & \multicolumn{1}{c}{(10 sec)} & \multicolumn{1}{c}{(60 sec)} & \multicolumn{1}{c}{(300 sec)} & \multicolumn{1}{c}{(600 sec)} & \multicolumn{1}{c}{(1800 sec)} & \multicolumn{1}{c}{(end)} & \multicolumn{1}{c}{(end)}    & \multicolumn{1}{c}{(end)}   & \multicolumn{1}{c}{(end)}  \\ \midrule 
          PIVIT-BOEM & $      1937.28$ & $      1937.28$ & $      1937.28$ & $      1937.28$ & $      1937.28$ & $      1937.28$ & $      1937.28$ & $      1937.28$ & $         0.40$ sec    & $       2.9608$  & $       0.7191$ \\ 
                 CGC & $      1087.76$ & $      1087.76$ & $      1087.76$ & $      1087.76$ & $      1087.76$ & $      1087.76$ & $      1087.76$ & $      1087.76$ & $         0.02$ sec    & $       0.7246$  & $       0.9240$ \\ 
                  HC & $      1194.32$ & $      1194.32$ & $      1194.32$ & $      1194.32$ & $      1194.32$ & $      1194.32$ & $      1194.32$ & $      1194.32$ & $         0.00$ sec    & $       0.8468$  & $       0.9008$ \\ 
              HC-CGC & $      1084.68$ & $      1084.68$ & $      1084.68$ & $      1084.68$ & $      1084.68$ & $      1084.68$ & $      1084.68$ & $      1084.68$ & $         0.02$ sec    & $       0.7255$  & $       0.9244$ \\ 
              ogm-KL & $      1110.94$ & $      1110.94$ & $      1110.94$ & $      1110.94$ & $      1110.94$ & $      1110.94$ & $      1110.94$ & $      1110.94$ & $         0.01$ sec    & $       0.8036$  & $       0.9044$ \\ 
    CC-Fusion-HC-CGC & $      1084.26$ & $      1084.26$ & $      1084.26$ & $      1084.26$ & $      1084.26$ & $      1084.26$ & $      1084.26$ & $      1084.26$ & $         0.07$ sec    & $       0.7233$  & $       0.9244$ \\ 
     CC-Fusion-HC-MC & $      1084.26$ & $      1084.26$ & $      1084.26$ & $      1084.26$ & $      1084.26$ & $      1084.26$ & $      1084.26$ & $      1084.26$ & $         0.63$ sec    & $       0.7233$  & $       0.9244$ \\ 
    CC-Fusion-WS-CGC & $      1084.26$ & $      1084.26$ & $      1084.26$ & $      1084.26$ & $      1084.26$ & $      1084.26$ & $      1084.26$ & $      1084.26$ & $         0.06$ sec    & $       0.7233$  & $       0.9244$ \\ 
     CC-Fusion-WS-MC & $      1084.26$ & $      1084.26$ & $      1084.26$ & $      1084.26$ & $      1084.26$ & $      1084.26$ & $      1084.26$ & $      1084.26$ & $         0.72$ sec    & $       0.7233$  & $       0.9244$ \\ 
\cmidrule{1-1} 
           MCR-CCFDB & $      1084.26$ & $      1084.26$ & $      1084.26$ & $      1084.26$ & $      1084.26$ & $      1084.26$ & $      1084.26$ & $      1084.26$ & $         0.01$ sec    & $       0.7233$  & $       0.9244$ \\ 
\cmidrule{1-1} 
           MCI-CCIFD & $      1084.26$ & $      1084.26$ & $      1084.26$ & $      1084.26$ & $      1084.26$ & $      1084.26$ & $      1084.26$ & $      1084.26$ & $         0.05$ sec    & $       0.7233$  & $       0.9244$ \\ 
\bottomrule
\end{tabular}
\end{table}

\begin{table}[H]
\scriptsize
\centering
\caption{image-seg (271035.bmp)}
\label{tab:anytimetable-image-seg-271035.bmp}
\begin{tabular}{lrrrrrrrrrrr}
\toprule
           algorithm &                                   \multicolumn{8}{c}{value} & \multicolumn{1}{c}{time}    & \multicolumn{1}{c}{VI}  & \multicolumn{1}{c}{RI} \\  
\cmidrule(lr){2-9}\cmidrule(lr){10-10} \cmidrule(lr){11-11} \cmidrule(lr){12-12}   
                     & \multicolumn{1}{c}{(0.5 sec)} & \multicolumn{1}{c}{(1 sec)} & \multicolumn{1}{c}{(10 sec)} & \multicolumn{1}{c}{(60 sec)} & \multicolumn{1}{c}{(300 sec)} & \multicolumn{1}{c}{(600 sec)} & \multicolumn{1}{c}{(1800 sec)} & \multicolumn{1}{c}{(end)} & \multicolumn{1}{c}{(end)}    & \multicolumn{1}{c}{(end)}   & \multicolumn{1}{c}{(end)}  \\ \midrule 
          PIVIT-BOEM & $\infty$ & $\infty$ & $\infty$ & $      4674.85$ & $      4674.85$ & $      4674.85$ & $      4674.85$ & $      4674.85$ & $        11.71$ sec    & $       4.5644$  & $       0.8378$ \\ 
                 CGC & $      3640.12$ & $      3640.12$ & $      3640.12$ & $      3640.12$ & $      3640.12$ & $      3640.12$ & $      3640.12$ & $      3640.12$ & $         0.13$ sec    & $       2.7983$  & $       0.8598$ \\ 
                  HC & $      4002.40$ & $      4002.40$ & $      4002.40$ & $      4002.40$ & $      4002.40$ & $      4002.40$ & $      4002.40$ & $      4002.40$ & $         0.00$ sec    & $       2.8257$  & $       0.8578$ \\ 
              HC-CGC & $      3631.79$ & $      3631.79$ & $      3631.79$ & $      3631.79$ & $      3631.79$ & $      3631.79$ & $      3631.79$ & $      3631.79$ & $         0.06$ sec    & $       2.9024$  & $       0.8555$ \\ 
              ogm-KL & $      3834.56$ & $      3812.22$ & $      3812.22$ & $      3812.22$ & $      3812.22$ & $      3812.22$ & $      3812.22$ & $      3812.22$ & $         0.68$ sec    & $       4.0265$  & $       0.6491$ \\ 
    CC-Fusion-HC-CGC & $      3630.63$ & $      3630.63$ & $      3630.63$ & $      3630.63$ & $      3630.63$ & $      3630.63$ & $      3630.63$ & $      3630.63$ & $         0.48$ sec    & $       2.7748$  & $       0.8609$ \\ 
     CC-Fusion-HC-MC & $      3625.92$ & $      3621.48$ & $      3621.48$ & $      3621.48$ & $      3621.48$ & $      3621.48$ & $      3621.48$ & $      3621.48$ & $         2.54$ sec    & $       2.8771$  & $       0.8571$ \\ 
    CC-Fusion-WS-CGC & $      3637.81$ & $      3635.08$ & $      3635.08$ & $      3635.08$ & $      3635.08$ & $      3635.08$ & $      3635.08$ & $      3635.08$ & $         0.83$ sec    & $       2.8803$  & $       0.8556$ \\ 
     CC-Fusion-WS-MC & $      3631.88$ & $      3623.24$ & $      3621.00$ & $      3621.00$ & $      3621.00$ & $      3621.00$ & $      3621.00$ & $      3621.00$ & $         7.90$ sec    & $       2.8324$  & $       0.8586$ \\ 
\cmidrule{1-1} 
           MCR-CCFDB & $      3625.90$ & $      3625.90$ & $      3625.90$ & $      3625.90$ & $      3625.90$ & $      3625.90$ & $      3625.90$ & $      3625.90$ & $         0.15$ sec    & $       2.8115$  & $       0.8592$ \\ 
\cmidrule{1-1} 
           MCI-CCIFD & $      3691.49$ & $      3621.00$ & $      3621.00$ & $      3621.00$ & $      3621.00$ & $      3621.00$ & $      3621.00$ & $      3621.00$ & $         0.97$ sec    & $       2.8324$  & $       0.8586$ \\ 
\bottomrule
\end{tabular}
\end{table}

\begin{table}[H]
\scriptsize
\centering
\caption{image-seg (285079.bmp)}
\label{tab:anytimetable-image-seg-285079.bmp}
\begin{tabular}{lrrrrrrrrrrr}
\toprule
           algorithm &                                   \multicolumn{8}{c}{value} & \multicolumn{1}{c}{time}    & \multicolumn{1}{c}{VI}  & \multicolumn{1}{c}{RI} \\  
\cmidrule(lr){2-9}\cmidrule(lr){10-10} \cmidrule(lr){11-11} \cmidrule(lr){12-12}   
                     & \multicolumn{1}{c}{(0.5 sec)} & \multicolumn{1}{c}{(1 sec)} & \multicolumn{1}{c}{(10 sec)} & \multicolumn{1}{c}{(60 sec)} & \multicolumn{1}{c}{(300 sec)} & \multicolumn{1}{c}{(600 sec)} & \multicolumn{1}{c}{(1800 sec)} & \multicolumn{1}{c}{(end)} & \multicolumn{1}{c}{(end)}    & \multicolumn{1}{c}{(end)}   & \multicolumn{1}{c}{(end)}  \\ \midrule 
          PIVIT-BOEM & $\infty$ & $\infty$ & $\infty$ & $      7855.54$ & $      7855.54$ & $      7855.54$ & $      7855.54$ & $      7855.54$ & $        52.10$ sec    & $       5.7799$  & $       0.7437$ \\ 
                 CGC & $      5648.81$ & $      5625.98$ & $      5625.08$ & $      5625.08$ & $      5625.08$ & $      5625.08$ & $      5625.08$ & $      5625.08$ & $         1.50$ sec    & $       2.9455$  & $       0.7653$ \\ 
                  HC & $      6099.78$ & $      6099.78$ & $      6099.78$ & $      6099.78$ & $      6099.78$ & $      6099.78$ & $      6099.78$ & $      6099.78$ & $         0.01$ sec    & $       2.9178$  & $       0.7673$ \\ 
              HC-CGC & $      5659.28$ & $      5635.10$ & $      5631.01$ & $      5631.01$ & $      5631.01$ & $      5631.01$ & $      5631.01$ & $      5631.01$ & $         2.24$ sec    & $       2.8327$  & $       0.7822$ \\ 
              ogm-KL & $      7465.28$ & $      5749.35$ & $      5742.46$ & $      5742.46$ & $      5742.46$ & $      5742.46$ & $      5742.46$ & $      5742.46$ & $         2.31$ sec    & $       3.4650$  & $       0.5453$ \\ 
    CC-Fusion-HC-CGC & $      5619.11$ & $      5617.00$ & $      5617.00$ & $      5617.00$ & $      5617.00$ & $      5617.00$ & $      5617.00$ & $      5617.00$ & $         1.18$ sec    & $       2.8579$  & $       0.7976$ \\ 
     CC-Fusion-HC-MC & $      5625.88$ & $      5618.23$ & $      5610.71$ & $      5610.71$ & $      5610.71$ & $      5610.71$ & $      5610.71$ & $      5610.71$ & $         4.95$ sec    & $       2.8881$  & $       0.7961$ \\ 
    CC-Fusion-WS-CGC & $      5630.62$ & $      5627.83$ & $      5627.83$ & $      5627.83$ & $      5627.83$ & $      5627.83$ & $      5627.83$ & $      5627.83$ & $         1.04$ sec    & $       2.8338$  & $       0.7944$ \\ 
     CC-Fusion-WS-MC & $      5797.42$ & $      5681.36$ & $      5610.12$ & $      5610.12$ & $      5610.12$ & $      5610.12$ & $      5610.12$ & $      5610.12$ & $        10.82$ sec    & $       2.8816$  & $       0.7963$ \\ 
\cmidrule{1-1} 
           MCR-CCFDB & $      5614.90$ & $      5614.90$ & $      5614.90$ & $      5614.90$ & $      5614.90$ & $      5614.90$ & $      5614.90$ & $      5614.90$ & $         0.31$ sec    & $       2.8904$  & $       0.7961$ \\ 
\cmidrule{1-1} 
           MCI-CCIFD & $      5648.18$ & $      5648.18$ & $      5610.12$ & $      5610.12$ & $      5610.12$ & $      5610.12$ & $      5610.12$ & $      5610.12$ & $         2.80$ sec    & $       2.8816$  & $       0.7963$ \\ 
\bottomrule
\end{tabular}
\end{table}

\begin{table}[H]
\scriptsize
\centering
\caption{image-seg (291000.bmp)}
\label{tab:anytimetable-image-seg-291000.bmp}
\begin{tabular}{lrrrrrrrrrrr}
\toprule
           algorithm &                                   \multicolumn{8}{c}{value} & \multicolumn{1}{c}{time}    & \multicolumn{1}{c}{VI}  & \multicolumn{1}{c}{RI} \\  
\cmidrule(lr){2-9}\cmidrule(lr){10-10} \cmidrule(lr){11-11} \cmidrule(lr){12-12}   
                     & \multicolumn{1}{c}{(0.5 sec)} & \multicolumn{1}{c}{(1 sec)} & \multicolumn{1}{c}{(10 sec)} & \multicolumn{1}{c}{(60 sec)} & \multicolumn{1}{c}{(300 sec)} & \multicolumn{1}{c}{(600 sec)} & \multicolumn{1}{c}{(1800 sec)} & \multicolumn{1}{c}{(end)} & \multicolumn{1}{c}{(end)}    & \multicolumn{1}{c}{(end)}   & \multicolumn{1}{c}{(end)}  \\ \midrule 
          PIVIT-BOEM & $\infty$ & $\infty$ & $\infty$ & $\infty$ & $     15013.35$ & $     15013.35$ & $     15013.35$ & $     15013.35$ & $       287.04$ sec    & $       8.0721$  & $       0.6083$ \\ 
                 CGC & $     10401.10$ & $     10384.56$ & $     10276.04$ & $     10230.14$ & $     10230.14$ & $     10230.14$ & $     10230.14$ & $     10230.14$ & $        25.28$ sec    & $       2.4443$  & $       0.7835$ \\ 
                  HC & $     10833.66$ & $     10833.66$ & $     10833.66$ & $     10833.66$ & $     10833.66$ & $     10833.66$ & $     10833.66$ & $     10833.66$ & $         0.01$ sec    & $       2.7142$  & $       0.7603$ \\ 
              HC-CGC & $     10478.68$ & $     10383.13$ & $     10225.51$ & $     10225.12$ & $     10225.12$ & $     10225.12$ & $     10225.12$ & $     10225.12$ & $        14.56$ sec    & $       2.3635$  & $       0.7926$ \\ 
              ogm-KL & $     12318.39$ & $     12318.39$ & $     10442.94$ & $     10442.94$ & $     10442.94$ & $     10442.94$ & $     10442.94$ & $     10442.94$ & $         4.06$ sec    & $       2.3676$  & $       0.5473$ \\ 
    CC-Fusion-HC-CGC & $     10242.90$ & $     10236.64$ & $     10233.77$ & $     10233.77$ & $     10233.77$ & $     10233.77$ & $     10233.77$ & $     10233.77$ & $         2.11$ sec    & $       2.3766$  & $       0.7933$ \\ 
     CC-Fusion-HC-MC & $     10226.70$ & $     10219.34$ & $     10208.87$ & $     10208.87$ & $     10208.87$ & $     10208.87$ & $     10208.87$ & $     10208.87$ & $         5.14$ sec    & $       2.3534$  & $       0.7966$ \\ 
    CC-Fusion-WS-CGC & $     10272.06$ & $     10251.93$ & $     10249.53$ & $     10249.53$ & $     10249.53$ & $     10249.53$ & $     10249.53$ & $     10249.53$ & $         2.12$ sec    & $       2.4124$  & $       0.7863$ \\ 
     CC-Fusion-WS-MC & $     10252.46$ & $     10237.71$ & $     10208.87$ & $     10208.87$ & $     10208.87$ & $     10208.87$ & $     10208.87$ & $     10208.87$ & $        13.90$ sec    & $       2.3534$  & $       0.7966$ \\ 
\cmidrule{1-1} 
           MCR-CCFDB & $     11287.95$ & $     10388.61$ & $     10209.16$ & $     10209.16$ & $     10209.16$ & $     10209.16$ & $     10209.16$ & $     10209.16$ & $         1.69$ sec    & $       2.3579$  & $       0.7964$ \\ 
\cmidrule{1-1} 
           MCI-CCIFD & $     10460.61$ & $     10298.44$ & $     10208.87$ & $     10208.87$ & $     10208.87$ & $     10208.87$ & $     10208.87$ & $     10208.87$ & $         3.45$ sec    & $       2.3534$  & $       0.7966$ \\ 
\bottomrule
\end{tabular}
\end{table}

\begin{table}[H]
\scriptsize
\centering
\caption{image-seg (295087.bmp)}
\label{tab:anytimetable-image-seg-295087.bmp}
\begin{tabular}{lrrrrrrrrrrr}
\toprule
           algorithm &                                   \multicolumn{8}{c}{value} & \multicolumn{1}{c}{time}    & \multicolumn{1}{c}{VI}  & \multicolumn{1}{c}{RI} \\  
\cmidrule(lr){2-9}\cmidrule(lr){10-10} \cmidrule(lr){11-11} \cmidrule(lr){12-12}   
                     & \multicolumn{1}{c}{(0.5 sec)} & \multicolumn{1}{c}{(1 sec)} & \multicolumn{1}{c}{(10 sec)} & \multicolumn{1}{c}{(60 sec)} & \multicolumn{1}{c}{(300 sec)} & \multicolumn{1}{c}{(600 sec)} & \multicolumn{1}{c}{(1800 sec)} & \multicolumn{1}{c}{(end)} & \multicolumn{1}{c}{(end)}    & \multicolumn{1}{c}{(end)}   & \multicolumn{1}{c}{(end)}  \\ \midrule 
          PIVIT-BOEM & $\infty$ & $\infty$ & $\infty$ & $      5960.92$ & $      5960.92$ & $      5960.92$ & $      5960.92$ & $      5960.92$ & $        22.43$ sec    & $       4.0987$  & $       0.8790$ \\ 
                 CGC & $      4319.34$ & $      4315.46$ & $      4315.46$ & $      4315.46$ & $      4315.46$ & $      4315.46$ & $      4315.46$ & $      4315.46$ & $         0.66$ sec    & $       2.2703$  & $       0.8837$ \\ 
                  HC & $      4671.04$ & $      4671.04$ & $      4671.04$ & $      4671.04$ & $      4671.04$ & $      4671.04$ & $      4671.04$ & $      4671.04$ & $         0.00$ sec    & $       2.2425$  & $       0.8856$ \\ 
              HC-CGC & $      4310.37$ & $      4310.37$ & $      4310.37$ & $      4310.37$ & $      4310.37$ & $      4310.37$ & $      4310.37$ & $      4310.37$ & $         0.23$ sec    & $       2.2692$  & $       0.8876$ \\ 
              ogm-KL & $      4437.74$ & $      4434.65$ & $      4434.65$ & $      4434.65$ & $      4434.65$ & $      4434.65$ & $      4434.65$ & $      4434.65$ & $         1.19$ sec    & $       2.3252$  & $       0.7697$ \\ 
    CC-Fusion-HC-CGC & $      4293.28$ & $      4293.21$ & $      4293.21$ & $      4293.21$ & $      4293.21$ & $      4293.21$ & $      4293.21$ & $      4293.21$ & $         0.92$ sec    & $       2.2811$  & $       0.8855$ \\ 
     CC-Fusion-HC-MC & $      4294.03$ & $      4292.52$ & $      4291.40$ & $      4291.40$ & $      4291.40$ & $      4291.40$ & $      4291.40$ & $      4291.40$ & $         2.50$ sec    & $       2.3224$  & $       0.8862$ \\ 
    CC-Fusion-WS-CGC & $      4305.88$ & $      4305.88$ & $      4305.88$ & $      4305.88$ & $      4305.88$ & $      4305.88$ & $      4305.88$ & $      4305.88$ & $         0.58$ sec    & $       2.2452$  & $       0.8860$ \\ 
     CC-Fusion-WS-MC & $      4328.58$ & $      4304.41$ & $      4290.54$ & $      4290.54$ & $      4290.54$ & $      4290.54$ & $      4290.54$ & $      4290.54$ & $         3.98$ sec    & $       2.1874$  & $       0.9035$ \\ 
\cmidrule{1-1} 
           MCR-CCFDB & $      4290.54$ & $      4290.54$ & $      4290.54$ & $      4290.54$ & $      4290.54$ & $      4290.54$ & $      4290.54$ & $      4290.54$ & $         0.19$ sec    & $       2.1874$  & $       0.9035$ \\ 
\cmidrule{1-1} 
           MCI-CCIFD & $      4295.87$ & $      4295.87$ & $      4290.54$ & $      4290.54$ & $      4290.54$ & $      4290.54$ & $      4290.54$ & $      4290.54$ & $         1.45$ sec    & $       2.1874$  & $       0.9035$ \\ 
\bottomrule
\end{tabular}
\end{table}

\begin{table}[H]
\scriptsize
\centering
\caption{image-seg (296007.bmp)}
\label{tab:anytimetable-image-seg-296007.bmp}
\begin{tabular}{lrrrrrrrrrrr}
\toprule
           algorithm &                                   \multicolumn{8}{c}{value} & \multicolumn{1}{c}{time}    & \multicolumn{1}{c}{VI}  & \multicolumn{1}{c}{RI} \\  
\cmidrule(lr){2-9}\cmidrule(lr){10-10} \cmidrule(lr){11-11} \cmidrule(lr){12-12}   
                     & \multicolumn{1}{c}{(0.5 sec)} & \multicolumn{1}{c}{(1 sec)} & \multicolumn{1}{c}{(10 sec)} & \multicolumn{1}{c}{(60 sec)} & \multicolumn{1}{c}{(300 sec)} & \multicolumn{1}{c}{(600 sec)} & \multicolumn{1}{c}{(1800 sec)} & \multicolumn{1}{c}{(end)} & \multicolumn{1}{c}{(end)}    & \multicolumn{1}{c}{(end)}   & \multicolumn{1}{c}{(end)}  \\ \midrule 
          PIVIT-BOEM & $\infty$ & $\infty$ & $      3437.10$ & $      3437.10$ & $      3437.10$ & $      3437.10$ & $      3437.10$ & $      3437.10$ & $         2.85$ sec    & $       2.7535$  & $       0.8987$ \\ 
                 CGC & $      2308.94$ & $      2308.94$ & $      2308.94$ & $      2308.94$ & $      2308.94$ & $      2308.94$ & $      2308.94$ & $      2308.94$ & $         0.25$ sec    & $       1.7274$  & $       0.8217$ \\ 
                  HC & $      2584.21$ & $      2584.21$ & $      2584.21$ & $      2584.21$ & $      2584.21$ & $      2584.21$ & $      2584.21$ & $      2584.21$ & $         0.00$ sec    & $       1.5971$  & $       0.8454$ \\ 
              HC-CGC & $      2308.90$ & $      2308.90$ & $      2308.90$ & $      2308.90$ & $      2308.90$ & $      2308.90$ & $      2308.90$ & $      2308.90$ & $         0.10$ sec    & $       1.5527$  & $       0.8505$ \\ 
              ogm-KL & $      2401.42$ & $      2401.42$ & $      2401.42$ & $      2401.42$ & $      2401.42$ & $      2401.42$ & $      2401.42$ & $      2401.42$ & $         0.09$ sec    & $       1.7012$  & $       0.8463$ \\ 
    CC-Fusion-HC-CGC & $      2301.79$ & $      2301.79$ & $      2301.79$ & $      2301.79$ & $      2301.79$ & $      2301.79$ & $      2301.79$ & $      2301.79$ & $         0.38$ sec    & $       1.6933$  & $       0.8231$ \\ 
     CC-Fusion-HC-MC & $      2293.13$ & $      2293.13$ & $      2293.13$ & $      2293.13$ & $      2293.13$ & $      2293.13$ & $      2293.13$ & $      2293.13$ & $         1.25$ sec    & $       1.5506$  & $       0.8518$ \\ 
    CC-Fusion-WS-CGC & $      2294.30$ & $      2294.30$ & $      2294.30$ & $      2294.30$ & $      2294.30$ & $      2294.30$ & $      2294.30$ & $      2294.30$ & $         0.45$ sec    & $       1.5479$  & $       0.8516$ \\ 
     CC-Fusion-WS-MC & $      2293.17$ & $      2293.13$ & $      2293.13$ & $      2293.13$ & $      2293.13$ & $      2293.13$ & $      2293.13$ & $      2293.13$ & $         1.54$ sec    & $       1.5506$  & $       0.8518$ \\ 
\cmidrule{1-1} 
           MCR-CCFDB & $      2293.13$ & $      2293.13$ & $      2293.13$ & $      2293.13$ & $      2293.13$ & $      2293.13$ & $      2293.13$ & $      2293.13$ & $         0.07$ sec    & $       1.5506$  & $       0.8518$ \\ 
\cmidrule{1-1} 
           MCI-CCIFD & $      2293.13$ & $      2293.13$ & $      2293.13$ & $      2293.13$ & $      2293.13$ & $      2293.13$ & $      2293.13$ & $      2293.13$ & $         0.13$ sec    & $       1.5506$  & $       0.8518$ \\ 
\bottomrule
\end{tabular}
\end{table}

\begin{table}[H]
\scriptsize
\centering
\caption{image-seg (296059.bmp)}
\label{tab:anytimetable-image-seg-296059.bmp}
\begin{tabular}{lrrrrrrrrrrr}
\toprule
           algorithm &                                   \multicolumn{8}{c}{value} & \multicolumn{1}{c}{time}    & \multicolumn{1}{c}{VI}  & \multicolumn{1}{c}{RI} \\  
\cmidrule(lr){2-9}\cmidrule(lr){10-10} \cmidrule(lr){11-11} \cmidrule(lr){12-12}   
                     & \multicolumn{1}{c}{(0.5 sec)} & \multicolumn{1}{c}{(1 sec)} & \multicolumn{1}{c}{(10 sec)} & \multicolumn{1}{c}{(60 sec)} & \multicolumn{1}{c}{(300 sec)} & \multicolumn{1}{c}{(600 sec)} & \multicolumn{1}{c}{(1800 sec)} & \multicolumn{1}{c}{(end)} & \multicolumn{1}{c}{(end)}    & \multicolumn{1}{c}{(end)}   & \multicolumn{1}{c}{(end)}  \\ \midrule 
          PIVIT-BOEM & $\infty$ & $\infty$ & $      3092.48$ & $      3092.48$ & $      3092.48$ & $      3092.48$ & $      3092.48$ & $      3092.48$ & $         1.94$ sec    & $       3.0536$  & $       0.8713$ \\ 
                 CGC & $      2050.88$ & $      2050.88$ & $      2050.88$ & $      2050.88$ & $      2050.88$ & $      2050.88$ & $      2050.88$ & $      2050.88$ & $         0.18$ sec    & $       1.9190$  & $       0.7849$ \\ 
                  HC & $      2317.19$ & $      2317.19$ & $      2317.19$ & $      2317.19$ & $      2317.19$ & $      2317.19$ & $      2317.19$ & $      2317.19$ & $         0.00$ sec    & $       1.9539$  & $       0.7603$ \\ 
              HC-CGC & $      2051.09$ & $      2051.09$ & $      2051.09$ & $      2051.09$ & $      2051.09$ & $      2051.09$ & $      2051.09$ & $      2051.09$ & $         0.09$ sec    & $       1.9365$  & $       0.7932$ \\ 
              ogm-KL & $      2162.03$ & $      2162.03$ & $      2162.03$ & $      2162.03$ & $      2162.03$ & $      2162.03$ & $      2162.03$ & $      2162.03$ & $         0.05$ sec    & $       2.4812$  & $       0.6696$ \\ 
    CC-Fusion-HC-CGC & $      2044.73$ & $      2044.73$ & $      2044.73$ & $      2044.73$ & $      2044.73$ & $      2044.73$ & $      2044.73$ & $      2044.73$ & $         0.38$ sec    & $       1.9279$  & $       0.7938$ \\ 
     CC-Fusion-HC-MC & $      2044.71$ & $      2044.71$ & $      2044.71$ & $      2044.71$ & $      2044.71$ & $      2044.71$ & $      2044.71$ & $      2044.71$ & $         1.02$ sec    & $       1.9199$  & $       0.7991$ \\ 
    CC-Fusion-WS-CGC & $      2045.37$ & $      2045.37$ & $      2045.37$ & $      2045.37$ & $      2045.37$ & $      2045.37$ & $      2045.37$ & $      2045.37$ & $         0.26$ sec    & $       1.9287$  & $       0.7947$ \\ 
     CC-Fusion-WS-MC & $      2044.71$ & $      2044.71$ & $      2044.71$ & $      2044.71$ & $      2044.71$ & $      2044.71$ & $      2044.71$ & $      2044.71$ & $         1.45$ sec    & $       1.9199$  & $       0.7991$ \\ 
\cmidrule{1-1} 
           MCR-CCFDB & $      2045.16$ & $      2045.16$ & $      2045.16$ & $      2045.16$ & $      2045.16$ & $      2045.16$ & $      2045.16$ & $      2045.16$ & $         0.09$ sec    & $       1.9205$  & $       0.7992$ \\ 
\cmidrule{1-1} 
           MCI-CCIFD & $      2044.71$ & $      2044.71$ & $      2044.71$ & $      2044.71$ & $      2044.71$ & $      2044.71$ & $      2044.71$ & $      2044.71$ & $         0.45$ sec    & $       1.9199$  & $       0.7991$ \\ 
\bottomrule
\end{tabular}
\end{table}

\begin{table}[H]
\scriptsize
\centering
\caption{image-seg (299086.bmp)}
\label{tab:anytimetable-image-seg-299086.bmp}
\begin{tabular}{lrrrrrrrrrrr}
\toprule
           algorithm &                                   \multicolumn{8}{c}{value} & \multicolumn{1}{c}{time}    & \multicolumn{1}{c}{VI}  & \multicolumn{1}{c}{RI} \\  
\cmidrule(lr){2-9}\cmidrule(lr){10-10} \cmidrule(lr){11-11} \cmidrule(lr){12-12}   
                     & \multicolumn{1}{c}{(0.5 sec)} & \multicolumn{1}{c}{(1 sec)} & \multicolumn{1}{c}{(10 sec)} & \multicolumn{1}{c}{(60 sec)} & \multicolumn{1}{c}{(300 sec)} & \multicolumn{1}{c}{(600 sec)} & \multicolumn{1}{c}{(1800 sec)} & \multicolumn{1}{c}{(end)} & \multicolumn{1}{c}{(end)}    & \multicolumn{1}{c}{(end)}   & \multicolumn{1}{c}{(end)}  \\ \midrule 
          PIVIT-BOEM & $\infty$ & $      2049.19$ & $      2049.19$ & $      2049.19$ & $      2049.19$ & $      2049.19$ & $      2049.19$ & $      2049.19$ & $         0.85$ sec    & $       2.1254$  & $       0.8727$ \\ 
                 CGC & $      1559.66$ & $      1559.66$ & $      1559.66$ & $      1559.66$ & $      1559.66$ & $      1559.66$ & $      1559.66$ & $      1559.66$ & $         0.03$ sec    & $       1.5639$  & $       0.8551$ \\ 
                  HC & $      1683.59$ & $      1683.59$ & $      1683.59$ & $      1683.59$ & $      1683.59$ & $      1683.59$ & $      1683.59$ & $      1683.59$ & $         0.00$ sec    & $       1.3613$  & $       0.8969$ \\ 
              HC-CGC & $      1566.34$ & $      1566.34$ & $      1566.34$ & $      1566.34$ & $      1566.34$ & $      1566.34$ & $      1566.34$ & $      1566.34$ & $         0.01$ sec    & $       1.5246$  & $       0.8558$ \\ 
              ogm-KL & $      1622.71$ & $      1622.71$ & $      1622.71$ & $      1622.71$ & $      1622.71$ & $      1622.71$ & $      1622.71$ & $      1622.71$ & $         0.05$ sec    & $       1.9319$  & $       0.7621$ \\ 
    CC-Fusion-HC-CGC & $      1559.13$ & $      1559.13$ & $      1559.13$ & $      1559.13$ & $      1559.13$ & $      1559.13$ & $      1559.13$ & $      1559.13$ & $         0.21$ sec    & $       1.4884$  & $       0.8599$ \\ 
     CC-Fusion-HC-MC & $      1557.24$ & $      1557.24$ & $      1557.24$ & $      1557.24$ & $      1557.24$ & $      1557.24$ & $      1557.24$ & $      1557.24$ & $         0.89$ sec    & $       1.5484$  & $       0.8557$ \\ 
    CC-Fusion-WS-CGC & $      1559.28$ & $      1559.28$ & $      1559.28$ & $      1559.28$ & $      1559.28$ & $      1559.28$ & $      1559.28$ & $      1559.28$ & $         0.30$ sec    & $       1.4873$  & $       0.8599$ \\ 
     CC-Fusion-WS-MC & $      1557.24$ & $      1557.24$ & $      1557.24$ & $      1557.24$ & $      1557.24$ & $      1557.24$ & $      1557.24$ & $      1557.24$ & $         1.21$ sec    & $       1.5484$  & $       0.8557$ \\ 
\cmidrule{1-1} 
           MCR-CCFDB & $      1557.24$ & $      1557.24$ & $      1557.24$ & $      1557.24$ & $      1557.24$ & $      1557.24$ & $      1557.24$ & $      1557.24$ & $         0.03$ sec    & $       1.5484$  & $       0.8557$ \\ 
\cmidrule{1-1} 
           MCI-CCIFD & $      1557.24$ & $      1557.24$ & $      1557.24$ & $      1557.24$ & $      1557.24$ & $      1557.24$ & $      1557.24$ & $      1557.24$ & $         0.03$ sec    & $       1.5484$  & $       0.8557$ \\ 
\bottomrule
\end{tabular}
\end{table}

\begin{table}[H]
\scriptsize
\centering
\caption{image-seg (300091.bmp)}
\label{tab:anytimetable-image-seg-300091.bmp}
\begin{tabular}{lrrrrrrrrrrr}
\toprule
           algorithm &                                   \multicolumn{8}{c}{value} & \multicolumn{1}{c}{time}    & \multicolumn{1}{c}{VI}  & \multicolumn{1}{c}{RI} \\  
\cmidrule(lr){2-9}\cmidrule(lr){10-10} \cmidrule(lr){11-11} \cmidrule(lr){12-12}   
                     & \multicolumn{1}{c}{(0.5 sec)} & \multicolumn{1}{c}{(1 sec)} & \multicolumn{1}{c}{(10 sec)} & \multicolumn{1}{c}{(60 sec)} & \multicolumn{1}{c}{(300 sec)} & \multicolumn{1}{c}{(600 sec)} & \multicolumn{1}{c}{(1800 sec)} & \multicolumn{1}{c}{(end)} & \multicolumn{1}{c}{(end)}    & \multicolumn{1}{c}{(end)}   & \multicolumn{1}{c}{(end)}  \\ \midrule 
          PIVIT-BOEM & $\infty$ & $\infty$ & $      2285.23$ & $      2285.23$ & $      2285.23$ & $      2285.23$ & $      2285.23$ & $      2285.23$ & $         1.03$ sec    & $       2.8335$  & $       0.6035$ \\ 
                 CGC & $      1495.59$ & $      1495.59$ & $      1495.59$ & $      1495.59$ & $      1495.59$ & $      1495.59$ & $      1495.59$ & $      1495.59$ & $         0.04$ sec    & $       0.7039$  & $       0.8965$ \\ 
                  HC & $      1595.38$ & $      1595.38$ & $      1595.38$ & $      1595.38$ & $      1595.38$ & $      1595.38$ & $      1595.38$ & $      1595.38$ & $         0.00$ sec    & $       1.1432$  & $       0.6750$ \\ 
              HC-CGC & $      1495.10$ & $      1495.10$ & $      1495.10$ & $      1495.10$ & $      1495.10$ & $      1495.10$ & $      1495.10$ & $      1495.10$ & $         0.11$ sec    & $       0.7081$  & $       0.8964$ \\ 
              ogm-KL & $      1524.61$ & $      1524.61$ & $      1524.61$ & $      1524.61$ & $      1524.61$ & $      1524.61$ & $      1524.61$ & $      1524.61$ & $         0.07$ sec    & $       1.2632$  & $       0.6660$ \\ 
    CC-Fusion-HC-CGC & $      1496.74$ & $      1496.74$ & $      1496.74$ & $      1496.74$ & $      1496.74$ & $      1496.74$ & $      1496.74$ & $      1496.74$ & $         0.14$ sec    & $       0.8338$  & $       0.8662$ \\ 
     CC-Fusion-HC-MC & $      1495.10$ & $      1495.10$ & $      1495.10$ & $      1495.10$ & $      1495.10$ & $      1495.10$ & $      1495.10$ & $      1495.10$ & $         0.94$ sec    & $       0.7081$  & $       0.8964$ \\ 
    CC-Fusion-WS-CGC & $      1495.59$ & $      1495.59$ & $      1495.59$ & $      1495.59$ & $      1495.59$ & $      1495.59$ & $      1495.59$ & $      1495.59$ & $         0.14$ sec    & $       0.7039$  & $       0.8965$ \\ 
     CC-Fusion-WS-MC & $      1495.10$ & $      1495.10$ & $      1495.10$ & $      1495.10$ & $      1495.10$ & $      1495.10$ & $      1495.10$ & $      1495.10$ & $         1.27$ sec    & $       0.7081$  & $       0.8964$ \\ 
\cmidrule{1-1} 
           MCR-CCFDB & $      1501.09$ & $      1501.09$ & $      1501.09$ & $      1501.09$ & $      1501.09$ & $      1501.09$ & $      1501.09$ & $      1501.09$ & $         0.04$ sec    & $       0.7209$  & $       0.8953$ \\ 
\cmidrule{1-1} 
           MCI-CCIFD & $      1495.10$ & $      1495.10$ & $      1495.10$ & $      1495.10$ & $      1495.10$ & $      1495.10$ & $      1495.10$ & $      1495.10$ & $         0.45$ sec    & $       0.7081$  & $       0.8964$ \\ 
\bottomrule
\end{tabular}
\end{table}

\begin{table}[H]
\scriptsize
\centering
\caption{image-seg (302008.bmp)}
\label{tab:anytimetable-image-seg-302008.bmp}
\begin{tabular}{lrrrrrrrrrrr}
\toprule
           algorithm &                                   \multicolumn{8}{c}{value} & \multicolumn{1}{c}{time}    & \multicolumn{1}{c}{VI}  & \multicolumn{1}{c}{RI} \\  
\cmidrule(lr){2-9}\cmidrule(lr){10-10} \cmidrule(lr){11-11} \cmidrule(lr){12-12}   
                     & \multicolumn{1}{c}{(0.5 sec)} & \multicolumn{1}{c}{(1 sec)} & \multicolumn{1}{c}{(10 sec)} & \multicolumn{1}{c}{(60 sec)} & \multicolumn{1}{c}{(300 sec)} & \multicolumn{1}{c}{(600 sec)} & \multicolumn{1}{c}{(1800 sec)} & \multicolumn{1}{c}{(end)} & \multicolumn{1}{c}{(end)}    & \multicolumn{1}{c}{(end)}   & \multicolumn{1}{c}{(end)}  \\ \midrule 
          PIVIT-BOEM & $\infty$ & $\infty$ & $      3300.15$ & $      3300.15$ & $      3300.15$ & $      3300.15$ & $      3300.15$ & $      3300.15$ & $         5.29$ sec    & $       3.1568$  & $       0.8242$ \\ 
                 CGC & $      2544.50$ & $      2544.50$ & $      2544.50$ & $      2544.50$ & $      2544.50$ & $      2544.50$ & $      2544.50$ & $      2544.50$ & $         0.07$ sec    & $       2.3072$  & $       0.7304$ \\ 
                  HC & $      2739.19$ & $      2739.19$ & $      2739.19$ & $      2739.19$ & $      2739.19$ & $      2739.19$ & $      2739.19$ & $      2739.19$ & $         0.00$ sec    & $       2.8436$  & $       0.5283$ \\ 
              HC-CGC & $      2544.52$ & $      2544.52$ & $      2544.52$ & $      2544.52$ & $      2544.52$ & $      2544.52$ & $      2544.52$ & $      2544.52$ & $         0.04$ sec    & $       2.2610$  & $       0.7755$ \\ 
              ogm-KL & $      2582.25$ & $      2582.25$ & $      2582.25$ & $      2582.25$ & $      2582.25$ & $      2582.25$ & $      2582.25$ & $      2582.25$ & $         0.12$ sec    & $       2.9797$  & $       0.5291$ \\ 
    CC-Fusion-HC-CGC & $      2543.23$ & $      2543.23$ & $      2543.23$ & $      2543.23$ & $      2543.23$ & $      2543.23$ & $      2543.23$ & $      2543.23$ & $         0.40$ sec    & $       2.3054$  & $       0.7320$ \\ 
     CC-Fusion-HC-MC & $      2543.23$ & $      2543.23$ & $      2543.23$ & $      2543.23$ & $      2543.23$ & $      2543.23$ & $      2543.23$ & $      2543.23$ & $         1.16$ sec    & $       2.3054$  & $       0.7320$ \\ 
    CC-Fusion-WS-CGC & $      2543.25$ & $      2543.25$ & $      2543.25$ & $      2543.25$ & $      2543.25$ & $      2543.25$ & $      2543.25$ & $      2543.25$ & $         0.27$ sec    & $       2.3051$  & $       0.7320$ \\ 
     CC-Fusion-WS-MC & $      2544.38$ & $      2543.23$ & $      2543.23$ & $      2543.23$ & $      2543.23$ & $      2543.23$ & $      2543.23$ & $      2543.23$ & $         1.70$ sec    & $       2.3054$  & $       0.7320$ \\ 
\cmidrule{1-1} 
           MCR-CCFDB & $      2543.23$ & $      2543.23$ & $      2543.23$ & $      2543.23$ & $      2543.23$ & $      2543.23$ & $      2543.23$ & $      2543.23$ & $         0.03$ sec    & $       2.3054$  & $       0.7320$ \\ 
\cmidrule{1-1} 
           MCI-CCIFD & $      2543.23$ & $      2543.23$ & $      2543.23$ & $      2543.23$ & $      2543.23$ & $      2543.23$ & $      2543.23$ & $      2543.23$ & $         0.08$ sec    & $       2.3054$  & $       0.7320$ \\ 
\bottomrule
\end{tabular}
\end{table}

\begin{table}[H]
\scriptsize
\centering
\caption{image-seg (304034.bmp)}
\label{tab:anytimetable-image-seg-304034.bmp}
\begin{tabular}{lrrrrrrrrrrr}
\toprule
           algorithm &                                   \multicolumn{8}{c}{value} & \multicolumn{1}{c}{time}    & \multicolumn{1}{c}{VI}  & \multicolumn{1}{c}{RI} \\  
\cmidrule(lr){2-9}\cmidrule(lr){10-10} \cmidrule(lr){11-11} \cmidrule(lr){12-12}   
                     & \multicolumn{1}{c}{(0.5 sec)} & \multicolumn{1}{c}{(1 sec)} & \multicolumn{1}{c}{(10 sec)} & \multicolumn{1}{c}{(60 sec)} & \multicolumn{1}{c}{(300 sec)} & \multicolumn{1}{c}{(600 sec)} & \multicolumn{1}{c}{(1800 sec)} & \multicolumn{1}{c}{(end)} & \multicolumn{1}{c}{(end)}    & \multicolumn{1}{c}{(end)}   & \multicolumn{1}{c}{(end)}  \\ \midrule 
          PIVIT-BOEM & $\infty$ & $\infty$ & $\infty$ & $\infty$ & $     10695.62$ & $     10695.62$ & $     10695.62$ & $     10695.62$ & $       129.09$ sec    & $       8.2526$  & $       0.3734$ \\ 
                 CGC & $      8126.66$ & $      8100.40$ & $      7867.15$ & $      7867.15$ & $      7867.15$ & $      7867.15$ & $      7867.15$ & $      7867.15$ & $         5.62$ sec    & $       3.9446$  & $       0.4840$ \\ 
                  HC & $      8581.17$ & $      8581.17$ & $      8581.17$ & $      8581.17$ & $      8581.17$ & $      8581.17$ & $      8581.17$ & $      8581.17$ & $         0.01$ sec    & $       4.1439$  & $       0.4515$ \\ 
              HC-CGC & $      7957.98$ & $      7878.87$ & $      7850.47$ & $      7850.47$ & $      7850.47$ & $      7850.47$ & $      7850.47$ & $      7850.47$ & $         2.44$ sec    & $       3.7730$  & $       0.5135$ \\ 
              ogm-KL & $     10653.69$ & $     10653.69$ & $      8191.05$ & $      8191.05$ & $      8191.05$ & $      8191.05$ & $      8191.05$ & $      8191.05$ & $         4.18$ sec    & $       2.2853$  & $       0.5362$ \\ 
    CC-Fusion-HC-CGC & $      7914.44$ & $      7914.44$ & $      7914.44$ & $      7914.44$ & $      7914.44$ & $      7914.44$ & $      7914.44$ & $      7914.44$ & $         1.17$ sec    & $       4.0884$  & $       0.4523$ \\ 
     CC-Fusion-HC-MC & $      7896.66$ & $      7843.87$ & $      7835.47$ & $      7835.47$ & $      7835.47$ & $      7835.47$ & $      7835.47$ & $      7835.47$ & $         7.08$ sec    & $       4.2012$  & $       0.4451$ \\ 
    CC-Fusion-WS-CGC & $      7976.09$ & $      7937.90$ & $      7924.40$ & $      7924.40$ & $      7924.40$ & $      7924.40$ & $      7924.40$ & $      7924.40$ & $         2.19$ sec    & $       4.2174$  & $       0.4451$ \\ 
     CC-Fusion-WS-MC & $      8075.21$ & $      7936.09$ & $      7836.00$ & $      7836.00$ & $      7836.00$ & $      7836.00$ & $      7836.00$ & $      7836.00$ & $        14.62$ sec    & $       4.2100$  & $       0.4447$ \\ 
\cmidrule{1-1} 
           MCR-CCFDB & $      9888.74$ & $      7860.77$ & $      7849.10$ & $      7849.10$ & $      7849.10$ & $      7849.10$ & $      7849.10$ & $      7849.10$ & $         1.06$ sec    & $       4.2626$  & $       0.4439$ \\ 
\cmidrule{1-1} 
           MCI-CCIFD & $      8155.93$ & $      7914.90$ & $      7835.47$ & $      7835.47$ & $      7835.47$ & $      7835.47$ & $      7835.47$ & $      7835.47$ & $         3.10$ sec    & $       4.2017$  & $       0.4451$ \\ 
\bottomrule
\end{tabular}
\end{table}

\begin{table}[H]
\scriptsize
\centering
\caption{image-seg (304074.bmp)}
\label{tab:anytimetable-image-seg-304074.bmp}
\begin{tabular}{lrrrrrrrrrrr}
\toprule
           algorithm &                                   \multicolumn{8}{c}{value} & \multicolumn{1}{c}{time}    & \multicolumn{1}{c}{VI}  & \multicolumn{1}{c}{RI} \\  
\cmidrule(lr){2-9}\cmidrule(lr){10-10} \cmidrule(lr){11-11} \cmidrule(lr){12-12}   
                     & \multicolumn{1}{c}{(0.5 sec)} & \multicolumn{1}{c}{(1 sec)} & \multicolumn{1}{c}{(10 sec)} & \multicolumn{1}{c}{(60 sec)} & \multicolumn{1}{c}{(300 sec)} & \multicolumn{1}{c}{(600 sec)} & \multicolumn{1}{c}{(1800 sec)} & \multicolumn{1}{c}{(end)} & \multicolumn{1}{c}{(end)}    & \multicolumn{1}{c}{(end)}   & \multicolumn{1}{c}{(end)}  \\ \midrule 
          PIVIT-BOEM & $\infty$ & $\infty$ & $\infty$ & $      6235.71$ & $      6235.71$ & $      6235.71$ & $      6235.71$ & $      6235.71$ & $        17.96$ sec    & $       5.1510$  & $       0.8089$ \\ 
                 CGC & $      3898.47$ & $      3898.47$ & $      3898.47$ & $      3898.47$ & $      3898.47$ & $      3898.47$ & $      3898.47$ & $      3898.47$ & $         0.25$ sec    & $       1.7082$  & $       0.9206$ \\ 
                  HC & $      4331.65$ & $      4331.65$ & $      4331.65$ & $      4331.65$ & $      4331.65$ & $      4331.65$ & $      4331.65$ & $      4331.65$ & $         0.00$ sec    & $       1.6250$  & $       0.9001$ \\ 
              HC-CGC & $      3901.26$ & $      3901.26$ & $      3901.26$ & $      3901.26$ & $      3901.26$ & $      3901.26$ & $      3901.26$ & $      3901.26$ & $         0.19$ sec    & $       1.7033$  & $       0.9210$ \\ 
              ogm-KL & $      4129.57$ & $      4128.25$ & $      4128.25$ & $      4128.25$ & $      4128.25$ & $      4128.25$ & $      4128.25$ & $      4128.25$ & $         0.69$ sec    & $       3.1503$  & $       0.4574$ \\ 
    CC-Fusion-HC-CGC & $      3893.13$ & $      3893.13$ & $      3893.13$ & $      3893.13$ & $      3893.13$ & $      3893.13$ & $      3893.13$ & $      3893.13$ & $         0.74$ sec    & $       1.6286$  & $       0.9304$ \\ 
     CC-Fusion-HC-MC & $      3891.88$ & $      3891.88$ & $      3891.88$ & $      3891.88$ & $      3891.88$ & $      3891.88$ & $      3891.88$ & $      3891.88$ & $         1.33$ sec    & $       1.6939$  & $       0.9259$ \\ 
    CC-Fusion-WS-CGC & $      3915.84$ & $      3904.80$ & $      3903.82$ & $      3903.82$ & $      3903.82$ & $      3903.82$ & $      3903.82$ & $      3903.82$ & $         1.24$ sec    & $       1.6451$  & $       0.9281$ \\ 
     CC-Fusion-WS-MC & $      3892.96$ & $      3892.06$ & $      3892.06$ & $      3892.06$ & $      3892.06$ & $      3892.06$ & $      3892.06$ & $      3892.06$ & $         1.88$ sec    & $       1.6788$  & $       0.9267$ \\ 
\cmidrule{1-1} 
           MCR-CCFDB & $      3891.88$ & $      3891.88$ & $      3891.88$ & $      3891.88$ & $      3891.88$ & $      3891.88$ & $      3891.88$ & $      3891.88$ & $         0.18$ sec    & $       1.6939$  & $       0.9259$ \\ 
\cmidrule{1-1} 
           MCI-CCIFD & $      3891.88$ & $      3891.88$ & $      3891.88$ & $      3891.88$ & $      3891.88$ & $      3891.88$ & $      3891.88$ & $      3891.88$ & $         0.16$ sec    & $       1.6939$  & $       0.9259$ \\ 
\bottomrule
\end{tabular}
\end{table}

\begin{table}[H]
\scriptsize
\centering
\caption{image-seg (306005.bmp)}
\label{tab:anytimetable-image-seg-306005.bmp}
\begin{tabular}{lrrrrrrrrrrr}
\toprule
           algorithm &                                   \multicolumn{8}{c}{value} & \multicolumn{1}{c}{time}    & \multicolumn{1}{c}{VI}  & \multicolumn{1}{c}{RI} \\  
\cmidrule(lr){2-9}\cmidrule(lr){10-10} \cmidrule(lr){11-11} \cmidrule(lr){12-12}   
                     & \multicolumn{1}{c}{(0.5 sec)} & \multicolumn{1}{c}{(1 sec)} & \multicolumn{1}{c}{(10 sec)} & \multicolumn{1}{c}{(60 sec)} & \multicolumn{1}{c}{(300 sec)} & \multicolumn{1}{c}{(600 sec)} & \multicolumn{1}{c}{(1800 sec)} & \multicolumn{1}{c}{(end)} & \multicolumn{1}{c}{(end)}    & \multicolumn{1}{c}{(end)}   & \multicolumn{1}{c}{(end)}  \\ \midrule 
          PIVIT-BOEM & $\infty$ & $\infty$ & $\infty$ & $      6098.68$ & $      6098.68$ & $      6098.68$ & $      6098.68$ & $      6098.68$ & $        21.27$ sec    & $       5.7029$  & $       0.7076$ \\ 
                 CGC & $      4364.23$ & $      4340.79$ & $      4340.79$ & $      4340.79$ & $      4340.79$ & $      4340.79$ & $      4340.79$ & $      4340.79$ & $         1.20$ sec    & $       2.7518$  & $       0.5707$ \\ 
                  HC & $      4687.29$ & $      4687.29$ & $      4687.29$ & $      4687.29$ & $      4687.29$ & $      4687.29$ & $      4687.29$ & $      4687.29$ & $         0.00$ sec    & $       2.3970$  & $       0.7926$ \\ 
              HC-CGC & $      4309.50$ & $      4309.50$ & $      4309.50$ & $      4309.50$ & $      4309.50$ & $      4309.50$ & $      4309.50$ & $      4309.50$ & $         0.33$ sec    & $       2.3532$  & $       0.7978$ \\ 
              ogm-KL & $      4517.30$ & $      4516.90$ & $      4516.90$ & $      4516.90$ & $      4516.90$ & $      4516.90$ & $      4516.90$ & $      4516.90$ & $         0.92$ sec    & $       2.7038$  & $       0.5053$ \\ 
    CC-Fusion-HC-CGC & $      4307.25$ & $      4307.25$ & $      4307.25$ & $      4307.25$ & $      4307.25$ & $      4307.25$ & $      4307.25$ & $      4307.25$ & $         0.83$ sec    & $       2.3414$  & $       0.8017$ \\ 
     CC-Fusion-HC-MC & $      4313.10$ & $      4302.73$ & $      4290.66$ & $      4290.66$ & $      4290.66$ & $      4290.66$ & $      4290.66$ & $      4290.66$ & $         5.51$ sec    & $       2.3725$  & $       0.8041$ \\ 
    CC-Fusion-WS-CGC & $      4323.27$ & $      4323.27$ & $      4323.27$ & $      4323.27$ & $      4323.27$ & $      4323.27$ & $      4323.27$ & $      4323.27$ & $         0.48$ sec    & $       2.3361$  & $       0.7977$ \\ 
     CC-Fusion-WS-MC & $      4354.16$ & $      4316.02$ & $      4290.66$ & $      4290.66$ & $      4290.66$ & $      4290.66$ & $      4290.66$ & $      4290.66$ & $         9.40$ sec    & $       2.3699$  & $       0.8042$ \\ 
\cmidrule{1-1} 
           MCR-CCFDB & $      4292.90$ & $      4292.90$ & $      4292.90$ & $      4292.90$ & $      4292.90$ & $      4292.90$ & $      4292.90$ & $      4292.90$ & $         0.34$ sec    & $       2.3764$  & $       0.8087$ \\ 
\cmidrule{1-1} 
           MCI-CCIFD & $      4301.93$ & $      4301.93$ & $      4290.25$ & $      4290.25$ & $      4290.25$ & $      4290.25$ & $      4290.25$ & $      4290.25$ & $         1.34$ sec    & $       2.3752$  & $       0.8087$ \\ 
\bottomrule
\end{tabular}
\end{table}

\begin{table}[H]
\scriptsize
\centering
\caption{image-seg (3096.bmp)}
\label{tab:anytimetable-image-seg-3096.bmp}
\begin{tabular}{lrrrrrrrrrrr}
\toprule
           algorithm &                                   \multicolumn{8}{c}{value} & \multicolumn{1}{c}{time}    & \multicolumn{1}{c}{VI}  & \multicolumn{1}{c}{RI} \\  
\cmidrule(lr){2-9}\cmidrule(lr){10-10} \cmidrule(lr){11-11} \cmidrule(lr){12-12}   
                     & \multicolumn{1}{c}{(0.5 sec)} & \multicolumn{1}{c}{(1 sec)} & \multicolumn{1}{c}{(10 sec)} & \multicolumn{1}{c}{(60 sec)} & \multicolumn{1}{c}{(300 sec)} & \multicolumn{1}{c}{(600 sec)} & \multicolumn{1}{c}{(1800 sec)} & \multicolumn{1}{c}{(end)} & \multicolumn{1}{c}{(end)}    & \multicolumn{1}{c}{(end)}   & \multicolumn{1}{c}{(end)}  \\ \midrule 
          PIVIT-BOEM & $       493.39$ & $       493.39$ & $       493.39$ & $       493.39$ & $       493.39$ & $       493.39$ & $       493.39$ & $       493.39$ & $         0.05$ sec    & $       1.0129$  & $       0.8193$ \\ 
                 CGC & $       396.90$ & $       396.90$ & $       396.90$ & $       396.90$ & $       396.90$ & $       396.90$ & $       396.90$ & $       396.90$ & $         0.00$ sec    & $       0.5448$  & $       0.8728$ \\ 
                  HC & $       411.27$ & $       411.27$ & $       411.27$ & $       411.27$ & $       411.27$ & $       411.27$ & $       411.27$ & $       411.27$ & $         0.00$ sec    & $       0.5381$  & $       0.8730$ \\ 
              HC-CGC & $       396.90$ & $       396.90$ & $       396.90$ & $       396.90$ & $       396.90$ & $       396.90$ & $       396.90$ & $       396.90$ & $         0.00$ sec    & $       0.5448$  & $       0.8728$ \\ 
              ogm-KL & $       400.75$ & $       400.75$ & $       400.75$ & $       400.75$ & $       400.75$ & $       400.75$ & $       400.75$ & $       400.75$ & $         0.00$ sec    & $       0.5148$  & $       0.8723$ \\ 
    CC-Fusion-HC-CGC & $       396.90$ & $       396.90$ & $       396.90$ & $       396.90$ & $       396.90$ & $       396.90$ & $       396.90$ & $       396.90$ & $         0.04$ sec    & $       0.5448$  & $       0.8728$ \\ 
     CC-Fusion-HC-MC & $       396.90$ & $       396.90$ & $       396.90$ & $       396.90$ & $       396.90$ & $       396.90$ & $       396.90$ & $       396.90$ & $         0.52$ sec    & $       0.5448$  & $       0.8728$ \\ 
    CC-Fusion-WS-CGC & $       396.90$ & $       396.90$ & $       396.90$ & $       396.90$ & $       396.90$ & $       396.90$ & $       396.90$ & $       396.90$ & $         0.03$ sec    & $       0.5448$  & $       0.8728$ \\ 
     CC-Fusion-WS-MC & $       396.90$ & $       396.90$ & $       396.90$ & $       396.90$ & $       396.90$ & $       396.90$ & $       396.90$ & $       396.90$ & $         0.53$ sec    & $       0.5448$  & $       0.8728$ \\ 
\cmidrule{1-1} 
           MCR-CCFDB & $       396.90$ & $       396.90$ & $       396.90$ & $       396.90$ & $       396.90$ & $       396.90$ & $       396.90$ & $       396.90$ & $         0.00$ sec    & $       0.5448$  & $       0.8728$ \\ 
\cmidrule{1-1} 
           MCI-CCIFD & $       396.90$ & $       396.90$ & $       396.90$ & $       396.90$ & $       396.90$ & $       396.90$ & $       396.90$ & $       396.90$ & $         0.01$ sec    & $       0.5448$  & $       0.8728$ \\ 
\bottomrule
\end{tabular}
\end{table}

\begin{table}[H]
\scriptsize
\centering
\caption{image-seg (33039.bmp)}
\label{tab:anytimetable-image-seg-33039.bmp}
\begin{tabular}{lrrrrrrrrrrr}
\toprule
           algorithm &                                   \multicolumn{8}{c}{value} & \multicolumn{1}{c}{time}    & \multicolumn{1}{c}{VI}  & \multicolumn{1}{c}{RI} \\  
\cmidrule(lr){2-9}\cmidrule(lr){10-10} \cmidrule(lr){11-11} \cmidrule(lr){12-12}   
                     & \multicolumn{1}{c}{(0.5 sec)} & \multicolumn{1}{c}{(1 sec)} & \multicolumn{1}{c}{(10 sec)} & \multicolumn{1}{c}{(60 sec)} & \multicolumn{1}{c}{(300 sec)} & \multicolumn{1}{c}{(600 sec)} & \multicolumn{1}{c}{(1800 sec)} & \multicolumn{1}{c}{(end)} & \multicolumn{1}{c}{(end)}    & \multicolumn{1}{c}{(end)}   & \multicolumn{1}{c}{(end)}  \\ \midrule 
          PIVIT-BOEM & $\infty$ & $\infty$ & $\infty$ & $\infty$ & $     10265.62$ & $     10265.62$ & $     10265.62$ & $     10265.62$ & $       137.48$ sec    & $       6.8874$  & $       0.7556$ \\ 
                 CGC & $      8517.48$ & $      8162.64$ & $      8112.64$ & $      8112.64$ & $      8112.64$ & $      8112.64$ & $      8112.64$ & $      8112.64$ & $         1.81$ sec    & $       4.3825$  & $       0.7517$ \\ 
                  HC & $      9246.92$ & $      9246.92$ & $      9246.92$ & $      9246.92$ & $      9246.92$ & $      9246.92$ & $      9246.92$ & $      9246.92$ & $         0.01$ sec    & $       4.1089$  & $       0.7638$ \\ 
              HC-CGC & $      8096.34$ & $      8095.85$ & $      8095.85$ & $      8095.85$ & $      8095.85$ & $      8095.85$ & $      8095.85$ & $      8095.85$ & $         0.62$ sec    & $       4.4446$  & $       0.7561$ \\ 
              ogm-KL & $     12899.70$ & $      8631.80$ & $      8582.09$ & $      8582.09$ & $      8582.09$ & $      8582.09$ & $      8582.09$ & $      8582.09$ & $         3.15$ sec    & $       3.9329$  & $       0.5515$ \\ 
    CC-Fusion-HC-CGC & $      8166.00$ & $      8163.93$ & $      8145.97$ & $      8145.97$ & $      8145.97$ & $      8145.97$ & $      8145.97$ & $      8145.97$ & $         2.78$ sec    & $       4.2699$  & $       0.7715$ \\ 
     CC-Fusion-HC-MC & $      8158.60$ & $      8109.44$ & $      8069.67$ & $      8069.67$ & $      8069.67$ & $      8069.67$ & $      8069.67$ & $      8069.67$ & $        12.27$ sec    & $       4.4347$  & $       0.7637$ \\ 
    CC-Fusion-WS-CGC & $      8264.14$ & $      8243.85$ & $      8221.18$ & $      8221.18$ & $      8221.18$ & $      8221.18$ & $      8221.18$ & $      8221.18$ & $         2.80$ sec    & $       4.4419$  & $       0.7470$ \\ 
     CC-Fusion-WS-MC & $      8853.16$ & $      8272.73$ & $      8078.66$ & $      8069.67$ & $      8069.67$ & $      8069.67$ & $      8069.67$ & $      8069.67$ & $        32.00$ sec    & $       4.4347$  & $       0.7637$ \\ 
\cmidrule{1-1} 
           MCR-CCFDB & $     11568.47$ & $      8827.80$ & $      8102.26$ & $      8102.26$ & $      8102.26$ & $      8102.26$ & $      8102.26$ & $      8102.26$ & $         1.21$ sec    & $       4.4959$  & $       0.7621$ \\ 
\cmidrule{1-1} 
           MCI-CCIFD & $      8788.75$ & $      8439.03$ & $      8069.67$ & $      8069.67$ & $      8069.67$ & $      8069.67$ & $      8069.67$ & $      8069.67$ & $         2.90$ sec    & $       4.4347$  & $       0.7637$ \\ 
\bottomrule
\end{tabular}
\end{table}

\begin{table}[H]
\scriptsize
\centering
\caption{image-seg (351093.bmp)}
\label{tab:anytimetable-image-seg-351093.bmp}
\begin{tabular}{lrrrrrrrrrrr}
\toprule
           algorithm &                                   \multicolumn{8}{c}{value} & \multicolumn{1}{c}{time}    & \multicolumn{1}{c}{VI}  & \multicolumn{1}{c}{RI} \\  
\cmidrule(lr){2-9}\cmidrule(lr){10-10} \cmidrule(lr){11-11} \cmidrule(lr){12-12}   
                     & \multicolumn{1}{c}{(0.5 sec)} & \multicolumn{1}{c}{(1 sec)} & \multicolumn{1}{c}{(10 sec)} & \multicolumn{1}{c}{(60 sec)} & \multicolumn{1}{c}{(300 sec)} & \multicolumn{1}{c}{(600 sec)} & \multicolumn{1}{c}{(1800 sec)} & \multicolumn{1}{c}{(end)} & \multicolumn{1}{c}{(end)}    & \multicolumn{1}{c}{(end)}   & \multicolumn{1}{c}{(end)}  \\ \midrule 
          PIVIT-BOEM & $\infty$ & $\infty$ & $\infty$ & $\infty$ & $      8298.39$ & $      8298.39$ & $      8298.39$ & $      8298.39$ & $        60.33$ sec    & $       4.8819$  & $       0.8406$ \\ 
                 CGC & $      6289.91$ & $      6157.80$ & $      6156.33$ & $      6156.33$ & $      6156.33$ & $      6156.33$ & $      6156.33$ & $      6156.33$ & $         1.29$ sec    & $       2.5166$  & $       0.8755$ \\ 
                  HC & $      6602.99$ & $      6602.99$ & $      6602.99$ & $      6602.99$ & $      6602.99$ & $      6602.99$ & $      6602.99$ & $      6602.99$ & $         0.01$ sec    & $       2.4726$  & $       0.8769$ \\ 
              HC-CGC & $      6133.67$ & $      6129.15$ & $      6129.15$ & $      6129.15$ & $      6129.15$ & $      6129.15$ & $      6129.15$ & $      6129.15$ & $         0.99$ sec    & $       2.3687$  & $       0.8850$ \\ 
              ogm-KL & $      8371.23$ & $      6361.43$ & $      6337.40$ & $      6337.40$ & $      6337.40$ & $      6337.40$ & $      6337.40$ & $      6337.40$ & $         1.87$ sec    & $       2.7285$  & $       0.7185$ \\ 
    CC-Fusion-HC-CGC & $      6142.75$ & $      6129.27$ & $      6124.74$ & $      6124.74$ & $      6124.74$ & $      6124.74$ & $      6124.74$ & $      6124.74$ & $         2.56$ sec    & $       2.6381$  & $       0.8455$ \\ 
     CC-Fusion-HC-MC & $      6131.78$ & $      6111.90$ & $      6108.57$ & $      6108.57$ & $      6108.57$ & $      6108.57$ & $      6108.57$ & $      6108.57$ & $         6.81$ sec    & $       2.5059$  & $       0.8791$ \\ 
    CC-Fusion-WS-CGC & $      6207.70$ & $      6188.15$ & $      6162.41$ & $      6162.41$ & $      6162.41$ & $      6162.41$ & $      6162.41$ & $      6162.41$ & $         1.85$ sec    & $       2.6726$  & $       0.8488$ \\ 
     CC-Fusion-WS-MC & $      6535.99$ & $      6293.85$ & $      6106.71$ & $      6105.28$ & $      6105.28$ & $      6105.28$ & $      6105.28$ & $      6105.28$ & $        14.75$ sec    & $       2.5626$  & $       0.8776$ \\ 
\cmidrule{1-1} 
           MCR-CCFDB & $      6205.45$ & $      6111.09$ & $      6111.09$ & $      6111.09$ & $      6111.09$ & $      6111.09$ & $      6111.09$ & $      6111.09$ & $         0.61$ sec    & $       2.5818$  & $       0.8773$ \\ 
\cmidrule{1-1} 
           MCI-CCIFD & $      6176.71$ & $      6162.71$ & $      6105.28$ & $      6105.28$ & $      6105.28$ & $      6105.28$ & $      6105.28$ & $      6105.28$ & $         2.26$ sec    & $       2.5626$  & $       0.8776$ \\ 
\bottomrule
\end{tabular}
\end{table}

\begin{table}[H]
\scriptsize
\centering
\caption{image-seg (361010.bmp)}
\label{tab:anytimetable-image-seg-361010.bmp}
\begin{tabular}{lrrrrrrrrrrr}
\toprule
           algorithm &                                   \multicolumn{8}{c}{value} & \multicolumn{1}{c}{time}    & \multicolumn{1}{c}{VI}  & \multicolumn{1}{c}{RI} \\  
\cmidrule(lr){2-9}\cmidrule(lr){10-10} \cmidrule(lr){11-11} \cmidrule(lr){12-12}   
                     & \multicolumn{1}{c}{(0.5 sec)} & \multicolumn{1}{c}{(1 sec)} & \multicolumn{1}{c}{(10 sec)} & \multicolumn{1}{c}{(60 sec)} & \multicolumn{1}{c}{(300 sec)} & \multicolumn{1}{c}{(600 sec)} & \multicolumn{1}{c}{(1800 sec)} & \multicolumn{1}{c}{(end)} & \multicolumn{1}{c}{(end)}    & \multicolumn{1}{c}{(end)}   & \multicolumn{1}{c}{(end)}  \\ \midrule 
          PIVIT-BOEM & $\infty$ & $\infty$ & $\infty$ & $      4475.74$ & $      4475.74$ & $      4475.74$ & $      4475.74$ & $      4475.74$ & $        10.96$ sec    & $       4.8133$  & $       0.7521$ \\ 
                 CGC & $      3365.90$ & $      3365.90$ & $      3365.90$ & $      3365.90$ & $      3365.90$ & $      3365.90$ & $      3365.90$ & $      3365.90$ & $         0.22$ sec    & $       2.1146$  & $       0.8095$ \\ 
                  HC & $      3676.00$ & $      3676.00$ & $      3676.00$ & $      3676.00$ & $      3676.00$ & $      3676.00$ & $      3676.00$ & $      3676.00$ & $         0.00$ sec    & $       2.1945$  & $       0.8549$ \\ 
              HC-CGC & $      3363.51$ & $      3363.51$ & $      3363.51$ & $      3363.51$ & $      3363.51$ & $      3363.51$ & $      3363.51$ & $      3363.51$ & $         0.13$ sec    & $       1.8938$  & $       0.8918$ \\ 
              ogm-KL & $      3440.32$ & $      3440.32$ & $      3440.32$ & $      3440.32$ & $      3440.32$ & $      3440.32$ & $      3440.32$ & $      3440.32$ & $         0.52$ sec    & $       2.4798$  & $       0.7033$ \\ 
    CC-Fusion-HC-CGC & $      3362.08$ & $      3362.08$ & $      3362.08$ & $      3362.08$ & $      3362.08$ & $      3362.08$ & $      3362.08$ & $      3362.08$ & $         0.73$ sec    & $       1.6718$  & $       0.9425$ \\ 
     CC-Fusion-HC-MC & $      3361.39$ & $      3361.02$ & $      3361.02$ & $      3361.02$ & $      3361.02$ & $      3361.02$ & $      3361.02$ & $      3361.02$ & $         1.62$ sec    & $       1.6743$  & $       0.9426$ \\ 
    CC-Fusion-WS-CGC & $      3364.87$ & $      3364.87$ & $      3364.87$ & $      3364.87$ & $      3364.87$ & $      3364.87$ & $      3364.87$ & $      3364.87$ & $         0.59$ sec    & $       2.1212$  & $       0.8095$ \\ 
     CC-Fusion-WS-MC & $      3367.25$ & $      3361.32$ & $      3361.02$ & $      3361.02$ & $      3361.02$ & $      3361.02$ & $      3361.02$ & $      3361.02$ & $         2.42$ sec    & $       1.6743$  & $       0.9426$ \\ 
\cmidrule{1-1} 
           MCR-CCFDB & $      3361.02$ & $      3361.02$ & $      3361.02$ & $      3361.02$ & $      3361.02$ & $      3361.02$ & $      3361.02$ & $      3361.02$ & $         0.05$ sec    & $       1.6743$  & $       0.9426$ \\ 
\cmidrule{1-1} 
           MCI-CCIFD & $      3361.02$ & $      3361.02$ & $      3361.02$ & $      3361.02$ & $      3361.02$ & $      3361.02$ & $      3361.02$ & $      3361.02$ & $         0.12$ sec    & $       1.6743$  & $       0.9426$ \\ 
\bottomrule
\end{tabular}
\end{table}

\begin{table}[H]
\scriptsize
\centering
\caption{image-seg (37073.bmp)}
\label{tab:anytimetable-image-seg-37073.bmp}
\begin{tabular}{lrrrrrrrrrrr}
\toprule
           algorithm &                                   \multicolumn{8}{c}{value} & \multicolumn{1}{c}{time}    & \multicolumn{1}{c}{VI}  & \multicolumn{1}{c}{RI} \\  
\cmidrule(lr){2-9}\cmidrule(lr){10-10} \cmidrule(lr){11-11} \cmidrule(lr){12-12}   
                     & \multicolumn{1}{c}{(0.5 sec)} & \multicolumn{1}{c}{(1 sec)} & \multicolumn{1}{c}{(10 sec)} & \multicolumn{1}{c}{(60 sec)} & \multicolumn{1}{c}{(300 sec)} & \multicolumn{1}{c}{(600 sec)} & \multicolumn{1}{c}{(1800 sec)} & \multicolumn{1}{c}{(end)} & \multicolumn{1}{c}{(end)}    & \multicolumn{1}{c}{(end)}   & \multicolumn{1}{c}{(end)}  \\ \midrule 
          PIVIT-BOEM & $\infty$ & $\infty$ & $      2618.31$ & $      2618.31$ & $      2618.31$ & $      2618.31$ & $      2618.31$ & $      2618.31$ & $         1.67$ sec    & $       2.9342$  & $       0.7531$ \\ 
                 CGC & $      1976.05$ & $      1976.05$ & $      1976.05$ & $      1976.05$ & $      1976.05$ & $      1976.05$ & $      1976.05$ & $      1976.05$ & $         0.03$ sec    & $       2.6728$  & $       0.6803$ \\ 
                  HC & $      2077.92$ & $      2077.92$ & $      2077.92$ & $      2077.92$ & $      2077.92$ & $      2077.92$ & $      2077.92$ & $      2077.92$ & $         0.00$ sec    & $       2.6292$  & $       0.6904$ \\ 
              HC-CGC & $      1976.38$ & $      1976.38$ & $      1976.38$ & $      1976.38$ & $      1976.38$ & $      1976.38$ & $      1976.38$ & $      1976.38$ & $         0.02$ sec    & $       2.6558$  & $       0.6803$ \\ 
              ogm-KL & $      2050.47$ & $      2050.47$ & $      2050.47$ & $      2050.47$ & $      2050.47$ & $      2050.47$ & $      2050.47$ & $      2050.47$ & $         0.10$ sec    & $       3.1116$  & $       0.5823$ \\ 
    CC-Fusion-HC-CGC & $      1975.00$ & $      1975.00$ & $      1975.00$ & $      1975.00$ & $      1975.00$ & $      1975.00$ & $      1975.00$ & $      1975.00$ & $         0.38$ sec    & $       2.6679$  & $       0.6803$ \\ 
     CC-Fusion-HC-MC & $      1975.00$ & $      1975.00$ & $      1975.00$ & $      1975.00$ & $      1975.00$ & $      1975.00$ & $      1975.00$ & $      1975.00$ & $         0.93$ sec    & $       2.6679$  & $       0.6803$ \\ 
    CC-Fusion-WS-CGC & $      1975.68$ & $      1975.68$ & $      1975.68$ & $      1975.68$ & $      1975.68$ & $      1975.68$ & $      1975.68$ & $      1975.68$ & $         0.23$ sec    & $       2.6543$  & $       0.6816$ \\ 
     CC-Fusion-WS-MC & $      1975.44$ & $      1975.00$ & $      1975.00$ & $      1975.00$ & $      1975.00$ & $      1975.00$ & $      1975.00$ & $      1975.00$ & $         1.57$ sec    & $       2.6679$  & $       0.6803$ \\ 
\cmidrule{1-1} 
           MCR-CCFDB & $      1975.00$ & $      1975.00$ & $      1975.00$ & $      1975.00$ & $      1975.00$ & $      1975.00$ & $      1975.00$ & $      1975.00$ & $         0.02$ sec    & $       2.6679$  & $       0.6803$ \\ 
\cmidrule{1-1} 
           MCI-CCIFD & $      1975.00$ & $      1975.00$ & $      1975.00$ & $      1975.00$ & $      1975.00$ & $      1975.00$ & $      1975.00$ & $      1975.00$ & $         0.05$ sec    & $       2.6679$  & $       0.6803$ \\ 
\bottomrule
\end{tabular}
\end{table}

\begin{table}[H]
\scriptsize
\centering
\caption{image-seg (376043.bmp)}
\label{tab:anytimetable-image-seg-376043.bmp}
\begin{tabular}{lrrrrrrrrrrr}
\toprule
           algorithm &                                   \multicolumn{8}{c}{value} & \multicolumn{1}{c}{time}    & \multicolumn{1}{c}{VI}  & \multicolumn{1}{c}{RI} \\  
\cmidrule(lr){2-9}\cmidrule(lr){10-10} \cmidrule(lr){11-11} \cmidrule(lr){12-12}   
                     & \multicolumn{1}{c}{(0.5 sec)} & \multicolumn{1}{c}{(1 sec)} & \multicolumn{1}{c}{(10 sec)} & \multicolumn{1}{c}{(60 sec)} & \multicolumn{1}{c}{(300 sec)} & \multicolumn{1}{c}{(600 sec)} & \multicolumn{1}{c}{(1800 sec)} & \multicolumn{1}{c}{(end)} & \multicolumn{1}{c}{(end)}    & \multicolumn{1}{c}{(end)}   & \multicolumn{1}{c}{(end)}  \\ \midrule 
          PIVIT-BOEM & $\infty$ & $\infty$ & $\infty$ & $      9278.95$ & $      9278.95$ & $      9278.95$ & $      9278.95$ & $      9278.95$ & $        59.55$ sec    & $       6.3516$  & $       0.7298$ \\ 
                 CGC & $      6035.99$ & $      5923.43$ & $      5897.16$ & $      5897.16$ & $      5897.16$ & $      5897.16$ & $      5897.16$ & $      5897.16$ & $         1.42$ sec    & $       2.0213$  & $       0.8573$ \\ 
                  HC & $      6460.85$ & $      6460.85$ & $      6460.85$ & $      6460.85$ & $      6460.85$ & $      6460.85$ & $      6460.85$ & $      6460.85$ & $         0.01$ sec    & $       2.2194$  & $       0.8380$ \\ 
              HC-CGC & $      5915.44$ & $      5901.44$ & $      5901.44$ & $      5901.44$ & $      5901.44$ & $      5901.44$ & $      5901.44$ & $      5901.44$ & $         1.26$ sec    & $       1.7335$  & $       0.9014$ \\ 
              ogm-KL & $      6108.39$ & $      6099.05$ & $      6099.05$ & $      6099.05$ & $      6099.05$ & $      6099.05$ & $      6099.05$ & $      6099.05$ & $         1.03$ sec    & $       2.7470$  & $       0.4549$ \\ 
    CC-Fusion-HC-CGC & $      5905.66$ & $      5905.66$ & $      5905.66$ & $      5905.66$ & $      5905.66$ & $      5905.66$ & $      5905.66$ & $      5905.66$ & $         0.83$ sec    & $       1.8574$  & $       0.8528$ \\ 
     CC-Fusion-HC-MC & $      5863.83$ & $      5863.83$ & $      5863.83$ & $      5863.83$ & $      5863.83$ & $      5863.83$ & $      5863.83$ & $      5863.83$ & $         2.44$ sec    & $       2.0117$  & $       0.8575$ \\ 
    CC-Fusion-WS-CGC & $      5932.06$ & $      5926.86$ & $      5926.86$ & $      5926.86$ & $      5926.86$ & $      5926.86$ & $      5926.86$ & $      5926.86$ & $         1.43$ sec    & $       1.8695$  & $       0.8933$ \\ 
     CC-Fusion-WS-MC & $      5901.92$ & $      5873.73$ & $      5863.83$ & $      5863.83$ & $      5863.83$ & $      5863.83$ & $      5863.83$ & $      5863.83$ & $         5.40$ sec    & $       2.0117$  & $       0.8575$ \\ 
\cmidrule{1-1} 
           MCR-CCFDB & $      5908.00$ & $      5863.83$ & $      5863.83$ & $      5863.83$ & $      5863.83$ & $      5863.83$ & $      5863.83$ & $      5863.83$ & $         0.54$ sec    & $       2.0117$  & $       0.8575$ \\ 
\cmidrule{1-1} 
           MCI-CCIFD & $      6022.11$ & $      5960.30$ & $      5863.83$ & $      5863.83$ & $      5863.83$ & $      5863.83$ & $      5863.83$ & $      5863.83$ & $         1.18$ sec    & $       2.0117$  & $       0.8575$ \\ 
\bottomrule
\end{tabular}
\end{table}

\begin{table}[H]
\scriptsize
\centering
\caption{image-seg (38082.bmp)}
\label{tab:anytimetable-image-seg-38082.bmp}
\begin{tabular}{lrrrrrrrrrrr}
\toprule
           algorithm &                                   \multicolumn{8}{c}{value} & \multicolumn{1}{c}{time}    & \multicolumn{1}{c}{VI}  & \multicolumn{1}{c}{RI} \\  
\cmidrule(lr){2-9}\cmidrule(lr){10-10} \cmidrule(lr){11-11} \cmidrule(lr){12-12}   
                     & \multicolumn{1}{c}{(0.5 sec)} & \multicolumn{1}{c}{(1 sec)} & \multicolumn{1}{c}{(10 sec)} & \multicolumn{1}{c}{(60 sec)} & \multicolumn{1}{c}{(300 sec)} & \multicolumn{1}{c}{(600 sec)} & \multicolumn{1}{c}{(1800 sec)} & \multicolumn{1}{c}{(end)} & \multicolumn{1}{c}{(end)}    & \multicolumn{1}{c}{(end)}   & \multicolumn{1}{c}{(end)}  \\ \midrule 
          PIVIT-BOEM & $\infty$ & $\infty$ & $\infty$ & $\infty$ & $     10920.80$ & $     10920.80$ & $     10920.80$ & $     10920.80$ & $       137.63$ sec    & $       7.1179$  & $       0.6849$ \\ 
                 CGC & $      8466.26$ & $      8369.76$ & $      8130.41$ & $      8130.41$ & $      8130.41$ & $      8130.41$ & $      8130.41$ & $      8130.41$ & $         5.78$ sec    & $       3.7813$  & $       0.6816$ \\ 
                  HC & $      8882.31$ & $      8882.31$ & $      8882.31$ & $      8882.31$ & $      8882.31$ & $      8882.31$ & $      8882.31$ & $      8882.31$ & $         0.01$ sec    & $       3.7426$  & $       0.7027$ \\ 
              HC-CGC & $      8125.70$ & $      8104.12$ & $      8104.12$ & $      8104.12$ & $      8104.12$ & $      8104.12$ & $      8104.12$ & $      8104.12$ & $         0.90$ sec    & $       3.6727$  & $       0.7011$ \\ 
              ogm-KL & $     10547.29$ & $      8507.28$ & $      8480.50$ & $      8480.50$ & $      8480.50$ & $      8480.50$ & $      8480.50$ & $      8480.50$ & $         2.40$ sec    & $       3.2592$  & $       0.4623$ \\ 
    CC-Fusion-HC-CGC & $      8213.42$ & $      8210.70$ & $      8206.21$ & $      8206.21$ & $      8206.21$ & $      8206.21$ & $      8206.21$ & $      8206.21$ & $         3.36$ sec    & $       3.7325$  & $       0.7067$ \\ 
     CC-Fusion-HC-MC & $      8180.42$ & $      8105.80$ & $      8061.66$ & $      8060.44$ & $      8060.44$ & $      8060.44$ & $      8060.44$ & $      8060.44$ & $        26.56$ sec    & $       3.8167$  & $       0.7112$ \\ 
    CC-Fusion-WS-CGC & $      8305.69$ & $      8272.16$ & $      8243.25$ & $      8243.25$ & $      8243.25$ & $      8243.25$ & $      8243.25$ & $      8243.25$ & $         4.12$ sec    & $       3.9664$  & $       0.6931$ \\ 
     CC-Fusion-WS-MC & $      8939.73$ & $      8602.95$ & $      8062.32$ & $      8060.34$ & $      8060.34$ & $      8060.34$ & $      8060.34$ & $      8060.34$ & $        25.67$ sec    & $       3.8232$  & $       0.7083$ \\ 
\cmidrule{1-1} 
           MCR-CCFDB & $      9969.00$ & $      9669.92$ & $      8080.61$ & $      8080.61$ & $      8080.61$ & $      8080.61$ & $      8080.61$ & $      8080.61$ & $         1.78$ sec    & $       3.9109$  & $       0.7037$ \\ 
\cmidrule{1-1} 
           MCI-CCIFD & $      8746.25$ & $      8334.10$ & $      8060.34$ & $      8060.34$ & $      8060.34$ & $      8060.34$ & $      8060.34$ & $      8060.34$ & $         3.84$ sec    & $       3.8232$  & $       0.7083$ \\ 
\bottomrule
\end{tabular}
\end{table}

\begin{table}[H]
\scriptsize
\centering
\caption{image-seg (38092.bmp)}
\label{tab:anytimetable-image-seg-38092.bmp}
\begin{tabular}{lrrrrrrrrrrr}
\toprule
           algorithm &                                   \multicolumn{8}{c}{value} & \multicolumn{1}{c}{time}    & \multicolumn{1}{c}{VI}  & \multicolumn{1}{c}{RI} \\  
\cmidrule(lr){2-9}\cmidrule(lr){10-10} \cmidrule(lr){11-11} \cmidrule(lr){12-12}   
                     & \multicolumn{1}{c}{(0.5 sec)} & \multicolumn{1}{c}{(1 sec)} & \multicolumn{1}{c}{(10 sec)} & \multicolumn{1}{c}{(60 sec)} & \multicolumn{1}{c}{(300 sec)} & \multicolumn{1}{c}{(600 sec)} & \multicolumn{1}{c}{(1800 sec)} & \multicolumn{1}{c}{(end)} & \multicolumn{1}{c}{(end)}    & \multicolumn{1}{c}{(end)}   & \multicolumn{1}{c}{(end)}  \\ \midrule 
          PIVIT-BOEM & $\infty$ & $\infty$ & $\infty$ & $      6142.02$ & $      6142.02$ & $      6142.02$ & $      6142.02$ & $      6142.02$ & $        19.63$ sec    & $       4.5036$  & $       0.8366$ \\ 
                 CGC & $      4080.77$ & $      4080.77$ & $      4080.77$ & $      4080.77$ & $      4080.77$ & $      4080.77$ & $      4080.77$ & $      4080.77$ & $         0.45$ sec    & $       1.6383$  & $       0.9200$ \\ 
                  HC & $      4315.97$ & $      4315.97$ & $      4315.97$ & $      4315.97$ & $      4315.97$ & $      4315.97$ & $      4315.97$ & $      4315.97$ & $         0.00$ sec    & $       1.6717$  & $       0.9156$ \\ 
              HC-CGC & $      4081.02$ & $      4081.02$ & $      4081.02$ & $      4081.02$ & $      4081.02$ & $      4081.02$ & $      4081.02$ & $      4081.02$ & $         0.45$ sec    & $       1.6712$  & $       0.9201$ \\ 
              ogm-KL & $      4325.15$ & $      4310.52$ & $      4310.52$ & $      4310.52$ & $      4310.52$ & $      4310.52$ & $      4310.52$ & $      4310.52$ & $         0.81$ sec    & $       2.6418$  & $       0.7319$ \\ 
    CC-Fusion-HC-CGC & $      4075.17$ & $      4074.42$ & $      4074.42$ & $      4074.42$ & $      4074.42$ & $      4074.42$ & $      4074.42$ & $      4074.42$ & $         1.10$ sec    & $       1.6178$  & $       0.9207$ \\ 
     CC-Fusion-HC-MC & $      4071.86$ & $      4071.86$ & $      4071.86$ & $      4071.86$ & $      4071.86$ & $      4071.86$ & $      4071.86$ & $      4071.86$ & $         1.50$ sec    & $       1.6181$  & $       0.9228$ \\ 
    CC-Fusion-WS-CGC & $      4093.97$ & $      4093.97$ & $      4093.97$ & $      4093.97$ & $      4093.97$ & $      4093.97$ & $      4093.97$ & $      4093.97$ & $         0.35$ sec    & $       1.6584$  & $       0.9098$ \\ 
     CC-Fusion-WS-MC & $      4073.35$ & $      4071.86$ & $      4071.86$ & $      4071.86$ & $      4071.86$ & $      4071.86$ & $      4071.86$ & $      4071.86$ & $         2.29$ sec    & $       1.6181$  & $       0.9228$ \\ 
\cmidrule{1-1} 
           MCR-CCFDB & $      4071.86$ & $      4071.86$ & $      4071.86$ & $      4071.86$ & $      4071.86$ & $      4071.86$ & $      4071.86$ & $      4071.86$ & $         0.16$ sec    & $       1.6181$  & $       0.9228$ \\ 
\cmidrule{1-1} 
           MCI-CCIFD & $      4071.86$ & $      4071.86$ & $      4071.86$ & $      4071.86$ & $      4071.86$ & $      4071.86$ & $      4071.86$ & $      4071.86$ & $         0.13$ sec    & $       1.6181$  & $       0.9228$ \\ 
\bottomrule
\end{tabular}
\end{table}

\begin{table}[H]
\scriptsize
\centering
\caption{image-seg (385039.bmp)}
\label{tab:anytimetable-image-seg-385039.bmp}
\begin{tabular}{lrrrrrrrrrrr}
\toprule
           algorithm &                                   \multicolumn{8}{c}{value} & \multicolumn{1}{c}{time}    & \multicolumn{1}{c}{VI}  & \multicolumn{1}{c}{RI} \\  
\cmidrule(lr){2-9}\cmidrule(lr){10-10} \cmidrule(lr){11-11} \cmidrule(lr){12-12}   
                     & \multicolumn{1}{c}{(0.5 sec)} & \multicolumn{1}{c}{(1 sec)} & \multicolumn{1}{c}{(10 sec)} & \multicolumn{1}{c}{(60 sec)} & \multicolumn{1}{c}{(300 sec)} & \multicolumn{1}{c}{(600 sec)} & \multicolumn{1}{c}{(1800 sec)} & \multicolumn{1}{c}{(end)} & \multicolumn{1}{c}{(end)}    & \multicolumn{1}{c}{(end)}   & \multicolumn{1}{c}{(end)}  \\ \midrule 
          PIVIT-BOEM & $\infty$ & $\infty$ & $\infty$ & $      5299.37$ & $      5299.37$ & $      5299.37$ & $      5299.37$ & $      5299.37$ & $        16.35$ sec    & $       3.7730$  & $       0.8590$ \\ 
                 CGC & $      3761.13$ & $      3761.13$ & $      3761.13$ & $      3761.13$ & $      3761.13$ & $      3761.13$ & $      3761.13$ & $      3761.13$ & $         0.14$ sec    & $       2.5323$  & $       0.8615$ \\ 
                  HC & $      3964.74$ & $      3964.74$ & $      3964.74$ & $      3964.74$ & $      3964.74$ & $      3964.74$ & $      3964.74$ & $      3964.74$ & $         0.00$ sec    & $       2.3986$  & $       0.8731$ \\ 
              HC-CGC & $      3758.48$ & $      3758.48$ & $      3758.48$ & $      3758.48$ & $      3758.48$ & $      3758.48$ & $      3758.48$ & $      3758.48$ & $         0.05$ sec    & $       2.5622$  & $       0.8650$ \\ 
              ogm-KL & $      3871.27$ & $      3870.70$ & $      3870.70$ & $      3870.70$ & $      3870.70$ & $      3870.70$ & $      3870.70$ & $      3870.70$ & $         0.64$ sec    & $       3.3107$  & $       0.6584$ \\ 
    CC-Fusion-HC-CGC & $      3746.35$ & $      3746.35$ & $      3746.35$ & $      3746.35$ & $      3746.35$ & $      3746.35$ & $      3746.35$ & $      3746.35$ & $         0.66$ sec    & $       2.4966$  & $       0.8704$ \\ 
     CC-Fusion-HC-MC & $      3745.97$ & $      3745.97$ & $      3745.53$ & $      3745.53$ & $      3745.53$ & $      3745.53$ & $      3745.53$ & $      3745.53$ & $         2.57$ sec    & $       2.4426$  & $       0.8730$ \\ 
    CC-Fusion-WS-CGC & $      3754.99$ & $      3754.99$ & $      3754.99$ & $      3754.99$ & $      3754.99$ & $      3754.99$ & $      3754.99$ & $      3754.99$ & $         0.54$ sec    & $       2.4438$  & $       0.8727$ \\ 
     CC-Fusion-WS-MC & $      3748.12$ & $      3745.53$ & $      3745.53$ & $      3745.53$ & $      3745.53$ & $      3745.53$ & $      3745.53$ & $      3745.53$ & $         2.93$ sec    & $       2.4426$  & $       0.8730$ \\ 
\cmidrule{1-1} 
           MCR-CCFDB & $      3747.90$ & $      3747.90$ & $      3747.90$ & $      3747.90$ & $      3747.90$ & $      3747.90$ & $      3747.90$ & $      3747.90$ & $         0.08$ sec    & $       2.4300$  & $       0.8733$ \\ 
\cmidrule{1-1} 
           MCI-CCIFD & $      3745.53$ & $      3745.53$ & $      3745.53$ & $      3745.53$ & $      3745.53$ & $      3745.53$ & $      3745.53$ & $      3745.53$ & $         0.36$ sec    & $       2.4426$  & $       0.8730$ \\ 
\bottomrule
\end{tabular}
\end{table}

\begin{table}[H]
\scriptsize
\centering
\caption{image-seg (41033.bmp)}
\label{tab:anytimetable-image-seg-41033.bmp}
\begin{tabular}{lrrrrrrrrrrr}
\toprule
           algorithm &                                   \multicolumn{8}{c}{value} & \multicolumn{1}{c}{time}    & \multicolumn{1}{c}{VI}  & \multicolumn{1}{c}{RI} \\  
\cmidrule(lr){2-9}\cmidrule(lr){10-10} \cmidrule(lr){11-11} \cmidrule(lr){12-12}   
                     & \multicolumn{1}{c}{(0.5 sec)} & \multicolumn{1}{c}{(1 sec)} & \multicolumn{1}{c}{(10 sec)} & \multicolumn{1}{c}{(60 sec)} & \multicolumn{1}{c}{(300 sec)} & \multicolumn{1}{c}{(600 sec)} & \multicolumn{1}{c}{(1800 sec)} & \multicolumn{1}{c}{(end)} & \multicolumn{1}{c}{(end)}    & \multicolumn{1}{c}{(end)}   & \multicolumn{1}{c}{(end)}  \\ \midrule 
          PIVIT-BOEM & $\infty$ & $\infty$ & $      3348.87$ & $      3348.87$ & $      3348.87$ & $      3348.87$ & $      3348.87$ & $      3348.87$ & $         2.02$ sec    & $       3.8520$  & $       0.8114$ \\ 
                 CGC & $      2001.65$ & $      2001.65$ & $      2001.65$ & $      2001.65$ & $      2001.65$ & $      2001.65$ & $      2001.65$ & $      2001.65$ & $         0.11$ sec    & $       1.7862$  & $       0.8220$ \\ 
                  HC & $      2162.63$ & $      2162.63$ & $      2162.63$ & $      2162.63$ & $      2162.63$ & $      2162.63$ & $      2162.63$ & $      2162.63$ & $         0.00$ sec    & $       1.9961$  & $       0.8102$ \\ 
              HC-CGC & $      2003.88$ & $      2003.88$ & $      2003.88$ & $      2003.88$ & $      2003.88$ & $      2003.88$ & $      2003.88$ & $      2003.88$ & $         0.06$ sec    & $       1.8169$  & $       0.8303$ \\ 
              ogm-KL & $      2086.82$ & $      2086.82$ & $      2086.82$ & $      2086.82$ & $      2086.82$ & $      2086.82$ & $      2086.82$ & $      2086.82$ & $         0.05$ sec    & $       2.3014$  & $       0.6176$ \\ 
    CC-Fusion-HC-CGC & $      1994.82$ & $      1994.82$ & $      1994.82$ & $      1994.82$ & $      1994.82$ & $      1994.82$ & $      1994.82$ & $      1994.82$ & $         0.43$ sec    & $       1.5248$  & $       0.8871$ \\ 
     CC-Fusion-HC-MC & $      1998.14$ & $      1994.24$ & $      1994.24$ & $      1994.24$ & $      1994.24$ & $      1994.24$ & $      1994.24$ & $      1994.24$ & $         1.72$ sec    & $       1.9850$  & $       0.8289$ \\ 
    CC-Fusion-WS-CGC & $      1999.38$ & $      1999.38$ & $      1999.38$ & $      1999.38$ & $      1999.38$ & $      1999.38$ & $      1999.38$ & $      1999.38$ & $         0.50$ sec    & $       1.8603$  & $       0.8381$ \\ 
     CC-Fusion-WS-MC & $      1994.82$ & $      1994.82$ & $      1994.82$ & $      1994.82$ & $      1994.82$ & $      1994.82$ & $      1994.82$ & $      1994.82$ & $         1.54$ sec    & $       1.5248$  & $       0.8871$ \\ 
\cmidrule{1-1} 
           MCR-CCFDB & $      1994.24$ & $      1994.24$ & $      1994.24$ & $      1994.24$ & $      1994.24$ & $      1994.24$ & $      1994.24$ & $      1994.24$ & $         0.06$ sec    & $       1.9850$  & $       0.8289$ \\ 
\cmidrule{1-1} 
           MCI-CCIFD & $      1994.24$ & $      1994.24$ & $      1994.24$ & $      1994.24$ & $      1994.24$ & $      1994.24$ & $      1994.24$ & $      1994.24$ & $         0.14$ sec    & $       1.9850$  & $       0.8289$ \\ 
\bottomrule
\end{tabular}
\end{table}

\begin{table}[H]
\scriptsize
\centering
\caption{image-seg (41069.bmp)}
\label{tab:anytimetable-image-seg-41069.bmp}
\begin{tabular}{lrrrrrrrrrrr}
\toprule
           algorithm &                                   \multicolumn{8}{c}{value} & \multicolumn{1}{c}{time}    & \multicolumn{1}{c}{VI}  & \multicolumn{1}{c}{RI} \\  
\cmidrule(lr){2-9}\cmidrule(lr){10-10} \cmidrule(lr){11-11} \cmidrule(lr){12-12}   
                     & \multicolumn{1}{c}{(0.5 sec)} & \multicolumn{1}{c}{(1 sec)} & \multicolumn{1}{c}{(10 sec)} & \multicolumn{1}{c}{(60 sec)} & \multicolumn{1}{c}{(300 sec)} & \multicolumn{1}{c}{(600 sec)} & \multicolumn{1}{c}{(1800 sec)} & \multicolumn{1}{c}{(end)} & \multicolumn{1}{c}{(end)}    & \multicolumn{1}{c}{(end)}   & \multicolumn{1}{c}{(end)}  \\ \midrule 
          PIVIT-BOEM & $\infty$ & $\infty$ & $\infty$ & $     10802.67$ & $     10802.67$ & $     10802.67$ & $     10802.67$ & $     10802.67$ & $        54.00$ sec    & $       6.1801$  & $       0.6626$ \\ 
                 CGC & $      5141.05$ & $      5126.73$ & $      5126.73$ & $      5126.73$ & $      5126.73$ & $      5126.73$ & $      5126.73$ & $      5126.73$ & $         0.93$ sec    & $       2.1375$  & $       0.5709$ \\ 
                  HC & $      5603.76$ & $      5603.76$ & $      5603.76$ & $      5603.76$ & $      5603.76$ & $      5603.76$ & $      5603.76$ & $      5603.76$ & $         0.01$ sec    & $       2.1274$  & $       0.4272$ \\ 
              HC-CGC & $      5130.38$ & $      5128.55$ & $      5128.55$ & $      5128.55$ & $      5128.55$ & $      5128.55$ & $      5128.55$ & $      5128.55$ & $         0.67$ sec    & $       2.1342$  & $       0.5720$ \\ 
              ogm-KL & $      5269.73$ & $      5269.73$ & $      5269.73$ & $      5269.73$ & $      5269.73$ & $      5269.73$ & $      5269.73$ & $      5269.73$ & $         0.27$ sec    & $       2.2871$  & $       0.4861$ \\ 
    CC-Fusion-HC-CGC & $      5124.32$ & $      5124.32$ & $      5124.32$ & $      5124.32$ & $      5124.32$ & $      5124.32$ & $      5124.32$ & $      5124.32$ & $         0.69$ sec    & $       1.9637$  & $       0.6332$ \\ 
     CC-Fusion-HC-MC & $      5122.43$ & $      5120.95$ & $      5120.95$ & $      5120.95$ & $      5120.95$ & $      5120.95$ & $      5120.95$ & $      5120.95$ & $         2.90$ sec    & $       2.0463$  & $       0.5931$ \\ 
    CC-Fusion-WS-CGC & $      5119.45$ & $      5119.45$ & $      5119.45$ & $      5119.45$ & $      5119.45$ & $      5119.45$ & $      5119.45$ & $      5119.45$ & $         0.47$ sec    & $       2.0218$  & $       0.6240$ \\ 
     CC-Fusion-WS-MC & $      5112.91$ & $      5110.96$ & $      5110.96$ & $      5110.96$ & $      5110.96$ & $      5110.96$ & $      5110.96$ & $      5110.96$ & $         3.84$ sec    & $       2.1405$  & $       0.5953$ \\ 
\cmidrule{1-1} 
           MCR-CCFDB & $      5429.51$ & $      5200.11$ & $      5114.98$ & $      5114.98$ & $      5114.98$ & $      5114.98$ & $      5114.98$ & $      5114.98$ & $         1.59$ sec    & $       2.1409$  & $       0.5953$ \\ 
\cmidrule{1-1} 
           MCI-CCIFD & $      5411.57$ & $      5410.52$ & $      5110.96$ & $      5110.96$ & $      5110.96$ & $      5110.96$ & $      5110.96$ & $      5110.96$ & $         5.78$ sec    & $       2.1405$  & $       0.5953$ \\ 
\bottomrule
\end{tabular}
\end{table}

\begin{table}[H]
\scriptsize
\centering
\caption{image-seg (42012.bmp)}
\label{tab:anytimetable-image-seg-42012.bmp}
\begin{tabular}{lrrrrrrrrrrr}
\toprule
           algorithm &                                   \multicolumn{8}{c}{value} & \multicolumn{1}{c}{time}    & \multicolumn{1}{c}{VI}  & \multicolumn{1}{c}{RI} \\  
\cmidrule(lr){2-9}\cmidrule(lr){10-10} \cmidrule(lr){11-11} \cmidrule(lr){12-12}   
                     & \multicolumn{1}{c}{(0.5 sec)} & \multicolumn{1}{c}{(1 sec)} & \multicolumn{1}{c}{(10 sec)} & \multicolumn{1}{c}{(60 sec)} & \multicolumn{1}{c}{(300 sec)} & \multicolumn{1}{c}{(600 sec)} & \multicolumn{1}{c}{(1800 sec)} & \multicolumn{1}{c}{(end)} & \multicolumn{1}{c}{(end)}    & \multicolumn{1}{c}{(end)}   & \multicolumn{1}{c}{(end)}  \\ \midrule 
          PIVIT-BOEM & $\infty$ & $\infty$ & $      4760.93$ & $      4760.93$ & $      4760.93$ & $      4760.93$ & $      4760.93$ & $      4760.93$ & $         8.92$ sec    & $       4.4133$  & $       0.8811$ \\ 
                 CGC & $      3306.02$ & $      3306.02$ & $      3306.02$ & $      3306.02$ & $      3306.02$ & $      3306.02$ & $      3306.02$ & $      3306.02$ & $         0.45$ sec    & $       3.6697$  & $       0.5933$ \\ 
                  HC & $      3654.22$ & $      3654.22$ & $      3654.22$ & $      3654.22$ & $      3654.22$ & $      3654.22$ & $      3654.22$ & $      3654.22$ & $         0.00$ sec    & $       3.2283$  & $       0.7040$ \\ 
              HC-CGC & $      3294.22$ & $      3294.22$ & $      3294.22$ & $      3294.22$ & $      3294.22$ & $      3294.22$ & $      3294.22$ & $      3294.22$ & $         0.19$ sec    & $       3.1899$  & $       0.7246$ \\ 
              ogm-KL & $      3485.03$ & $      3485.03$ & $      3485.03$ & $      3485.03$ & $      3485.03$ & $      3485.03$ & $      3485.03$ & $      3485.03$ & $         0.16$ sec    & $       4.0526$  & $       0.4037$ \\ 
    CC-Fusion-HC-CGC & $      3251.87$ & $      3249.10$ & $      3249.10$ & $      3249.10$ & $      3249.10$ & $      3249.10$ & $      3249.10$ & $      3249.10$ & $         1.22$ sec    & $       2.9865$  & $       0.7935$ \\ 
     CC-Fusion-HC-MC & $      3250.41$ & $      3249.05$ & $      3249.05$ & $      3249.05$ & $      3249.05$ & $      3249.05$ & $      3249.05$ & $      3249.05$ & $         2.22$ sec    & $       3.0847$  & $       0.7506$ \\ 
    CC-Fusion-WS-CGC & $      3267.25$ & $      3267.25$ & $      3267.25$ & $      3267.25$ & $      3267.25$ & $      3267.25$ & $      3267.25$ & $      3267.25$ & $         0.64$ sec    & $       3.1774$  & $       0.7278$ \\ 
     CC-Fusion-WS-MC & $      3250.28$ & $      3250.28$ & $      3248.70$ & $      3248.70$ & $      3248.70$ & $      3248.70$ & $      3248.70$ & $      3248.70$ & $         6.74$ sec    & $       3.0055$  & $       0.7922$ \\ 
\cmidrule{1-1} 
           MCR-CCFDB & $      3258.45$ & $      3258.45$ & $      3258.45$ & $      3258.45$ & $      3258.45$ & $      3258.45$ & $      3258.45$ & $      3258.45$ & $         0.30$ sec    & $       3.0155$  & $       0.7978$ \\ 
\cmidrule{1-1} 
           MCI-CCIFD & $      3550.49$ & $      3272.63$ & $      3248.70$ & $      3248.70$ & $      3248.70$ & $      3248.70$ & $      3248.70$ & $      3248.70$ & $         1.62$ sec    & $       3.0055$  & $       0.7922$ \\ 
\bottomrule
\end{tabular}
\end{table}

\begin{table}[H]
\scriptsize
\centering
\caption{image-seg (42049.bmp)}
\label{tab:anytimetable-image-seg-42049.bmp}
\begin{tabular}{lrrrrrrrrrrr}
\toprule
           algorithm &                                   \multicolumn{8}{c}{value} & \multicolumn{1}{c}{time}    & \multicolumn{1}{c}{VI}  & \multicolumn{1}{c}{RI} \\  
\cmidrule(lr){2-9}\cmidrule(lr){10-10} \cmidrule(lr){11-11} \cmidrule(lr){12-12}   
                     & \multicolumn{1}{c}{(0.5 sec)} & \multicolumn{1}{c}{(1 sec)} & \multicolumn{1}{c}{(10 sec)} & \multicolumn{1}{c}{(60 sec)} & \multicolumn{1}{c}{(300 sec)} & \multicolumn{1}{c}{(600 sec)} & \multicolumn{1}{c}{(1800 sec)} & \multicolumn{1}{c}{(end)} & \multicolumn{1}{c}{(end)}    & \multicolumn{1}{c}{(end)}   & \multicolumn{1}{c}{(end)}  \\ \midrule 
          PIVIT-BOEM & $      1278.22$ & $      1278.22$ & $      1278.22$ & $      1278.22$ & $      1278.22$ & $      1278.22$ & $      1278.22$ & $      1278.22$ & $         0.36$ sec    & $       1.2817$  & $       0.9746$ \\ 
                 CGC & $      1072.33$ & $      1072.33$ & $      1072.33$ & $      1072.33$ & $      1072.33$ & $      1072.33$ & $      1072.33$ & $      1072.33$ & $         0.01$ sec    & $       0.8764$  & $       0.9772$ \\ 
                  HC & $      1123.37$ & $      1123.37$ & $      1123.37$ & $      1123.37$ & $      1123.37$ & $      1123.37$ & $      1123.37$ & $      1123.37$ & $         0.00$ sec    & $       0.8728$  & $       0.9773$ \\ 
              HC-CGC & $      1072.99$ & $      1072.99$ & $      1072.99$ & $      1072.99$ & $      1072.99$ & $      1072.99$ & $      1072.99$ & $      1072.99$ & $         0.01$ sec    & $       0.8788$  & $       0.9772$ \\ 
              ogm-KL & $      1090.54$ & $      1090.54$ & $      1090.54$ & $      1090.54$ & $      1090.54$ & $      1090.54$ & $      1090.54$ & $      1090.54$ & $         0.02$ sec    & $       0.8826$  & $       0.9686$ \\ 
    CC-Fusion-HC-CGC & $      1069.67$ & $      1069.67$ & $      1069.67$ & $      1069.67$ & $      1069.67$ & $      1069.67$ & $      1069.67$ & $      1069.67$ & $         0.13$ sec    & $       0.8922$  & $       0.9769$ \\ 
     CC-Fusion-HC-MC & $      1069.67$ & $      1069.67$ & $      1069.67$ & $      1069.67$ & $      1069.67$ & $      1069.67$ & $      1069.67$ & $      1069.67$ & $         0.93$ sec    & $       0.8922$  & $       0.9769$ \\ 
    CC-Fusion-WS-CGC & $      1070.15$ & $      1070.15$ & $      1070.15$ & $      1070.15$ & $      1070.15$ & $      1070.15$ & $      1070.15$ & $      1070.15$ & $         0.10$ sec    & $       0.8894$  & $       0.9769$ \\ 
     CC-Fusion-WS-MC & $      1069.76$ & $      1069.22$ & $      1069.22$ & $      1069.22$ & $      1069.22$ & $      1069.22$ & $      1069.22$ & $      1069.22$ & $         1.25$ sec    & $       0.8971$  & $       0.9768$ \\ 
\cmidrule{1-1} 
           MCR-CCFDB & $      1069.22$ & $      1069.22$ & $      1069.22$ & $      1069.22$ & $      1069.22$ & $      1069.22$ & $      1069.22$ & $      1069.22$ & $         0.01$ sec    & $       0.8971$  & $       0.9768$ \\ 
\cmidrule{1-1} 
           MCI-CCIFD & $      1069.22$ & $      1069.22$ & $      1069.22$ & $      1069.22$ & $      1069.22$ & $      1069.22$ & $      1069.22$ & $      1069.22$ & $         0.05$ sec    & $       0.8971$  & $       0.9768$ \\ 
\bottomrule
\end{tabular}
\end{table}

\begin{table}[H]
\scriptsize
\centering
\caption{image-seg (43074.bmp)}
\label{tab:anytimetable-image-seg-43074.bmp}
\begin{tabular}{lrrrrrrrrrrr}
\toprule
           algorithm &                                   \multicolumn{8}{c}{value} & \multicolumn{1}{c}{time}    & \multicolumn{1}{c}{VI}  & \multicolumn{1}{c}{RI} \\  
\cmidrule(lr){2-9}\cmidrule(lr){10-10} \cmidrule(lr){11-11} \cmidrule(lr){12-12}   
                     & \multicolumn{1}{c}{(0.5 sec)} & \multicolumn{1}{c}{(1 sec)} & \multicolumn{1}{c}{(10 sec)} & \multicolumn{1}{c}{(60 sec)} & \multicolumn{1}{c}{(300 sec)} & \multicolumn{1}{c}{(600 sec)} & \multicolumn{1}{c}{(1800 sec)} & \multicolumn{1}{c}{(end)} & \multicolumn{1}{c}{(end)}    & \multicolumn{1}{c}{(end)}   & \multicolumn{1}{c}{(end)}  \\ \midrule 
          PIVIT-BOEM & $\infty$ & $\infty$ & $      4180.23$ & $      4180.23$ & $      4180.23$ & $      4180.23$ & $      4180.23$ & $      4180.23$ & $         3.51$ sec    & $       4.7877$  & $       0.6292$ \\ 
                 CGC & $      2343.40$ & $      2343.40$ & $      2343.40$ & $      2343.40$ & $      2343.40$ & $      2343.40$ & $      2343.40$ & $      2343.40$ & $         0.42$ sec    & $       1.5187$  & $       0.4951$ \\ 
                  HC & $      2466.99$ & $      2466.99$ & $      2466.99$ & $      2466.99$ & $      2466.99$ & $      2466.99$ & $      2466.99$ & $      2466.99$ & $         0.00$ sec    & $       1.4852$  & $       0.5269$ \\ 
              HC-CGC & $      2355.67$ & $      2355.67$ & $      2355.67$ & $      2355.67$ & $      2355.67$ & $      2355.67$ & $      2355.67$ & $      2355.67$ & $         0.24$ sec    & $       1.4402$  & $       0.5303$ \\ 
              ogm-KL & $      2390.74$ & $      2390.74$ & $      2390.74$ & $      2390.74$ & $      2390.74$ & $      2390.74$ & $      2390.74$ & $      2390.74$ & $         0.09$ sec    & $       1.6198$  & $       0.4540$ \\ 
    CC-Fusion-HC-CGC & $      2334.97$ & $      2334.97$ & $      2334.97$ & $      2334.97$ & $      2334.97$ & $      2334.97$ & $      2334.97$ & $      2334.97$ & $         0.24$ sec    & $       1.4456$  & $       0.5159$ \\ 
     CC-Fusion-HC-MC & $      2332.83$ & $      2332.83$ & $      2332.83$ & $      2332.83$ & $      2332.83$ & $      2332.83$ & $      2332.83$ & $      2332.83$ & $         1.09$ sec    & $       1.4105$  & $       0.5312$ \\ 
    CC-Fusion-WS-CGC & $      2336.13$ & $      2336.13$ & $      2336.13$ & $      2336.13$ & $      2336.13$ & $      2336.13$ & $      2336.13$ & $      2336.13$ & $         0.17$ sec    & $       1.4060$  & $       0.5293$ \\ 
     CC-Fusion-WS-MC & $      2333.36$ & $      2333.36$ & $      2332.83$ & $      2332.83$ & $      2332.83$ & $      2332.83$ & $      2332.83$ & $      2332.83$ & $         2.56$ sec    & $       1.4105$  & $       0.5312$ \\ 
\cmidrule{1-1} 
           MCR-CCFDB & $      2343.74$ & $      2343.74$ & $      2343.74$ & $      2343.74$ & $      2343.74$ & $      2343.74$ & $      2343.74$ & $      2343.74$ & $         0.25$ sec    & $       1.4250$  & $       0.5319$ \\ 
\cmidrule{1-1} 
           MCI-CCIFD & $      2334.33$ & $      2332.83$ & $      2332.83$ & $      2332.83$ & $      2332.83$ & $      2332.83$ & $      2332.83$ & $      2332.83$ & $         0.78$ sec    & $       1.4105$  & $       0.5312$ \\ 
\bottomrule
\end{tabular}
\end{table}

\begin{table}[H]
\scriptsize
\centering
\caption{image-seg (45096.bmp)}
\label{tab:anytimetable-image-seg-45096.bmp}
\begin{tabular}{lrrrrrrrrrrr}
\toprule
           algorithm &                                   \multicolumn{8}{c}{value} & \multicolumn{1}{c}{time}    & \multicolumn{1}{c}{VI}  & \multicolumn{1}{c}{RI} \\  
\cmidrule(lr){2-9}\cmidrule(lr){10-10} \cmidrule(lr){11-11} \cmidrule(lr){12-12}   
                     & \multicolumn{1}{c}{(0.5 sec)} & \multicolumn{1}{c}{(1 sec)} & \multicolumn{1}{c}{(10 sec)} & \multicolumn{1}{c}{(60 sec)} & \multicolumn{1}{c}{(300 sec)} & \multicolumn{1}{c}{(600 sec)} & \multicolumn{1}{c}{(1800 sec)} & \multicolumn{1}{c}{(end)} & \multicolumn{1}{c}{(end)}    & \multicolumn{1}{c}{(end)}   & \multicolumn{1}{c}{(end)}  \\ \midrule 
          PIVIT-BOEM & $      1274.33$ & $      1274.33$ & $      1274.33$ & $      1274.33$ & $      1274.33$ & $      1274.33$ & $      1274.33$ & $      1274.33$ & $         0.29$ sec    & $       1.9038$  & $       0.8445$ \\ 
                 CGC & $       986.93$ & $       986.93$ & $       986.93$ & $       986.93$ & $       986.93$ & $       986.93$ & $       986.93$ & $       986.93$ & $         0.02$ sec    & $       1.2652$  & $       0.8338$ \\ 
                  HC & $      1039.42$ & $      1039.42$ & $      1039.42$ & $      1039.42$ & $      1039.42$ & $      1039.42$ & $      1039.42$ & $      1039.42$ & $         0.00$ sec    & $       1.2131$  & $       0.8563$ \\ 
              HC-CGC & $       981.04$ & $       981.04$ & $       981.04$ & $       981.04$ & $       981.04$ & $       981.04$ & $       981.04$ & $       981.04$ & $         0.01$ sec    & $       1.1225$  & $       0.8694$ \\ 
              ogm-KL & $      1008.08$ & $      1008.08$ & $      1008.08$ & $      1008.08$ & $      1008.08$ & $      1008.08$ & $      1008.08$ & $      1008.08$ & $         0.02$ sec    & $       1.4049$  & $       0.8509$ \\ 
    CC-Fusion-HC-CGC & $       977.78$ & $       977.78$ & $       977.78$ & $       977.78$ & $       977.78$ & $       977.78$ & $       977.78$ & $       977.78$ & $         0.11$ sec    & $       1.1241$  & $       0.8681$ \\ 
     CC-Fusion-HC-MC & $       977.78$ & $       977.78$ & $       977.78$ & $       977.78$ & $       977.78$ & $       977.78$ & $       977.78$ & $       977.78$ & $         0.93$ sec    & $       1.1241$  & $       0.8681$ \\ 
    CC-Fusion-WS-CGC & $       977.78$ & $       977.78$ & $       977.78$ & $       977.78$ & $       977.78$ & $       977.78$ & $       977.78$ & $       977.78$ & $         0.10$ sec    & $       1.1241$  & $       0.8681$ \\ 
     CC-Fusion-WS-MC & $       978.60$ & $       977.78$ & $       977.78$ & $       977.78$ & $       977.78$ & $       977.78$ & $       977.78$ & $       977.78$ & $         1.37$ sec    & $       1.1241$  & $       0.8681$ \\ 
\cmidrule{1-1} 
           MCR-CCFDB & $       977.78$ & $       977.78$ & $       977.78$ & $       977.78$ & $       977.78$ & $       977.78$ & $       977.78$ & $       977.78$ & $         0.01$ sec    & $       1.1241$  & $       0.8681$ \\ 
\cmidrule{1-1} 
           MCI-CCIFD & $       977.78$ & $       977.78$ & $       977.78$ & $       977.78$ & $       977.78$ & $       977.78$ & $       977.78$ & $       977.78$ & $         0.02$ sec    & $       1.1241$  & $       0.8681$ \\ 
\bottomrule
\end{tabular}
\end{table}

\begin{table}[H]
\scriptsize
\centering
\caption{image-seg (54082.bmp)}
\label{tab:anytimetable-image-seg-54082.bmp}
\begin{tabular}{lrrrrrrrrrrr}
\toprule
           algorithm &                                   \multicolumn{8}{c}{value} & \multicolumn{1}{c}{time}    & \multicolumn{1}{c}{VI}  & \multicolumn{1}{c}{RI} \\  
\cmidrule(lr){2-9}\cmidrule(lr){10-10} \cmidrule(lr){11-11} \cmidrule(lr){12-12}   
                     & \multicolumn{1}{c}{(0.5 sec)} & \multicolumn{1}{c}{(1 sec)} & \multicolumn{1}{c}{(10 sec)} & \multicolumn{1}{c}{(60 sec)} & \multicolumn{1}{c}{(300 sec)} & \multicolumn{1}{c}{(600 sec)} & \multicolumn{1}{c}{(1800 sec)} & \multicolumn{1}{c}{(end)} & \multicolumn{1}{c}{(end)}    & \multicolumn{1}{c}{(end)}   & \multicolumn{1}{c}{(end)}  \\ \midrule 
          PIVIT-BOEM & $\infty$ & $\infty$ & $\infty$ & $      5411.16$ & $      5411.16$ & $      5411.16$ & $      5411.16$ & $      5411.16$ & $        12.14$ sec    & $       4.0959$  & $       0.8399$ \\ 
                 CGC & $      3895.76$ & $      3859.80$ & $      3830.13$ & $      3830.13$ & $      3830.13$ & $      3830.13$ & $      3830.13$ & $      3830.13$ & $         2.13$ sec    & $       2.5040$  & $       0.7027$ \\ 
                  HC & $      4423.13$ & $      4423.13$ & $      4423.13$ & $      4423.13$ & $      4423.13$ & $      4423.13$ & $      4423.13$ & $      4423.13$ & $         0.00$ sec    & $       2.4943$  & $       0.7394$ \\ 
              HC-CGC & $      3818.81$ & $      3818.81$ & $      3818.81$ & $      3818.81$ & $      3818.81$ & $      3818.81$ & $      3818.81$ & $      3818.81$ & $         0.54$ sec    & $       2.5136$  & $       0.7242$ \\ 
              ogm-KL & $      3923.38$ & $      3923.38$ & $      3923.38$ & $      3923.38$ & $      3923.38$ & $      3923.38$ & $      3923.38$ & $      3923.38$ & $         0.54$ sec    & $       3.1468$  & $       0.4740$ \\ 
    CC-Fusion-HC-CGC & $      3809.30$ & $      3809.30$ & $      3809.30$ & $      3809.30$ & $      3809.30$ & $      3809.30$ & $      3809.30$ & $      3809.30$ & $         0.52$ sec    & $       2.1327$  & $       0.7767$ \\ 
     CC-Fusion-HC-MC & $      3800.97$ & $      3799.90$ & $      3796.36$ & $      3796.36$ & $      3796.36$ & $      3796.36$ & $      3796.36$ & $      3796.36$ & $         4.80$ sec    & $       2.2149$  & $       0.7878$ \\ 
    CC-Fusion-WS-CGC & $      3817.98$ & $      3817.98$ & $      3817.98$ & $      3817.98$ & $      3817.98$ & $      3817.98$ & $      3817.98$ & $      3817.98$ & $         0.52$ sec    & $       2.1577$  & $       0.7741$ \\ 
     CC-Fusion-WS-MC & $      3804.23$ & $      3801.70$ & $      3796.36$ & $      3796.36$ & $      3796.36$ & $      3796.36$ & $      3796.36$ & $      3796.36$ & $         7.33$ sec    & $       2.2149$  & $       0.7878$ \\ 
\cmidrule{1-1} 
           MCR-CCFDB & $      3966.75$ & $      3798.55$ & $      3798.55$ & $      3798.55$ & $      3798.55$ & $      3798.55$ & $      3798.55$ & $      3798.55$ & $         0.55$ sec    & $       2.2159$  & $       0.7879$ \\ 
\cmidrule{1-1} 
           MCI-CCIFD & $      4111.43$ & $      3796.36$ & $      3796.36$ & $      3796.36$ & $      3796.36$ & $      3796.36$ & $      3796.36$ & $      3796.36$ & $         0.91$ sec    & $       2.2149$  & $       0.7878$ \\ 
\bottomrule
\end{tabular}
\end{table}

\begin{table}[H]
\scriptsize
\centering
\caption{image-seg (55073.bmp)}
\label{tab:anytimetable-image-seg-55073.bmp}
\begin{tabular}{lrrrrrrrrrrr}
\toprule
           algorithm &                                   \multicolumn{8}{c}{value} & \multicolumn{1}{c}{time}    & \multicolumn{1}{c}{VI}  & \multicolumn{1}{c}{RI} \\  
\cmidrule(lr){2-9}\cmidrule(lr){10-10} \cmidrule(lr){11-11} \cmidrule(lr){12-12}   
                     & \multicolumn{1}{c}{(0.5 sec)} & \multicolumn{1}{c}{(1 sec)} & \multicolumn{1}{c}{(10 sec)} & \multicolumn{1}{c}{(60 sec)} & \multicolumn{1}{c}{(300 sec)} & \multicolumn{1}{c}{(600 sec)} & \multicolumn{1}{c}{(1800 sec)} & \multicolumn{1}{c}{(end)} & \multicolumn{1}{c}{(end)}    & \multicolumn{1}{c}{(end)}   & \multicolumn{1}{c}{(end)}  \\ \midrule 
          PIVIT-BOEM & $\infty$ & $\infty$ & $\infty$ & $\infty$ & $     11227.29$ & $     11227.29$ & $     11227.29$ & $     11227.29$ & $       129.95$ sec    & $       6.2945$  & $       0.8065$ \\ 
                 CGC & $      8175.28$ & $      8151.00$ & $      7877.64$ & $      7877.64$ & $      7877.64$ & $      7877.64$ & $      7877.64$ & $      7877.64$ & $         7.80$ sec    & $       3.2643$  & $       0.7393$ \\ 
                  HC & $      8788.90$ & $      8788.90$ & $      8788.90$ & $      8788.90$ & $      8788.90$ & $      8788.90$ & $      8788.90$ & $      8788.90$ & $         0.01$ sec    & $       3.7634$  & $       0.5934$ \\ 
              HC-CGC & $      8226.93$ & $      8042.20$ & $      7869.61$ & $      7869.61$ & $      7869.61$ & $      7869.61$ & $      7869.61$ & $      7869.61$ & $         3.87$ sec    & $       3.2468$  & $       0.7303$ \\ 
              ogm-KL & $     10194.77$ & $      8208.48$ & $      8200.84$ & $      8200.84$ & $      8200.84$ & $      8200.84$ & $      8200.84$ & $      8200.84$ & $         2.13$ sec    & $       3.7028$  & $       0.3849$ \\ 
    CC-Fusion-HC-CGC & $      7911.74$ & $      7908.80$ & $      7888.85$ & $      7888.85$ & $      7888.85$ & $      7888.85$ & $      7888.85$ & $      7888.85$ & $         3.02$ sec    & $       3.1893$  & $       0.7651$ \\ 
     CC-Fusion-HC-MC & $      7876.99$ & $      7867.43$ & $      7840.43$ & $      7838.19$ & $      7838.19$ & $      7838.19$ & $      7838.19$ & $      7838.19$ & $        21.37$ sec    & $       3.2175$  & $       0.7715$ \\ 
    CC-Fusion-WS-CGC & $      7924.96$ & $      7916.52$ & $      7916.52$ & $      7916.52$ & $      7916.52$ & $      7916.52$ & $      7916.52$ & $      7916.52$ & $         1.49$ sec    & $       3.1858$  & $       0.7784$ \\ 
     CC-Fusion-WS-MC & $      8258.20$ & $      7856.63$ & $      7835.96$ & $      7835.96$ & $      7835.96$ & $      7835.96$ & $      7835.96$ & $      7835.96$ & $        11.98$ sec    & $       3.1751$  & $       0.7955$ \\ 
\cmidrule{1-1} 
           MCR-CCFDB & $      9646.06$ & $      8785.58$ & $      7840.92$ & $      7840.92$ & $      7840.92$ & $      7840.92$ & $      7840.92$ & $      7840.92$ & $         1.66$ sec    & $       3.1883$  & $       0.7951$ \\ 
\cmidrule{1-1} 
           MCI-CCIFD & $      8325.52$ & $      8038.53$ & $      7835.96$ & $      7835.96$ & $      7835.96$ & $      7835.96$ & $      7835.96$ & $      7835.96$ & $         2.39$ sec    & $       3.1751$  & $       0.7955$ \\ 
\bottomrule
\end{tabular}
\end{table}

\begin{table}[H]
\scriptsize
\centering
\caption{image-seg (58060.bmp)}
\label{tab:anytimetable-image-seg-58060.bmp}
\begin{tabular}{lrrrrrrrrrrr}
\toprule
           algorithm &                                   \multicolumn{8}{c}{value} & \multicolumn{1}{c}{time}    & \multicolumn{1}{c}{VI}  & \multicolumn{1}{c}{RI} \\  
\cmidrule(lr){2-9}\cmidrule(lr){10-10} \cmidrule(lr){11-11} \cmidrule(lr){12-12}   
                     & \multicolumn{1}{c}{(0.5 sec)} & \multicolumn{1}{c}{(1 sec)} & \multicolumn{1}{c}{(10 sec)} & \multicolumn{1}{c}{(60 sec)} & \multicolumn{1}{c}{(300 sec)} & \multicolumn{1}{c}{(600 sec)} & \multicolumn{1}{c}{(1800 sec)} & \multicolumn{1}{c}{(end)} & \multicolumn{1}{c}{(end)}    & \multicolumn{1}{c}{(end)}   & \multicolumn{1}{c}{(end)}  \\ \midrule 
          PIVIT-BOEM & $\infty$ & $\infty$ & $\infty$ & $\infty$ & $     12405.93$ & $     12405.93$ & $     12405.93$ & $     12405.93$ & $       238.48$ sec    & $       6.5231$  & $       0.7081$ \\ 
                 CGC & $     10100.41$ & $     10049.51$ & $      9937.79$ & $      9937.79$ & $      9937.79$ & $      9937.79$ & $      9937.79$ & $      9937.79$ & $         4.94$ sec    & $       3.9160$  & $       0.6869$ \\ 
                  HC & $     10654.86$ & $     10654.86$ & $     10654.86$ & $     10654.86$ & $     10654.86$ & $     10654.86$ & $     10654.86$ & $     10654.86$ & $         0.01$ sec    & $       3.6932$  & $       0.7093$ \\ 
              HC-CGC & $      9996.53$ & $      9957.49$ & $      9928.04$ & $      9928.04$ & $      9928.04$ & $      9928.04$ & $      9928.04$ & $      9928.04$ & $         5.34$ sec    & $       3.8462$  & $       0.6983$ \\ 
              ogm-KL & $     12237.48$ & $     12237.48$ & $     10121.99$ & $     10121.99$ & $     10121.99$ & $     10121.99$ & $     10121.99$ & $     10121.99$ & $         8.48$ sec    & $       3.1929$  & $       0.5829$ \\ 
    CC-Fusion-HC-CGC & $      9960.95$ & $      9949.19$ & $      9940.45$ & $      9940.45$ & $      9940.45$ & $      9940.45$ & $      9940.45$ & $      9940.45$ & $         2.37$ sec    & $       3.7718$  & $       0.7120$ \\ 
     CC-Fusion-HC-MC & $     10319.83$ & $     10057.59$ & $      9887.79$ & $      9884.16$ & $      9884.16$ & $      9884.16$ & $      9884.16$ & $      9884.16$ & $        26.05$ sec    & $       3.9821$  & $       0.6991$ \\ 
    CC-Fusion-WS-CGC & $     10015.84$ & $     10009.19$ & $      9975.16$ & $      9975.16$ & $      9975.16$ & $      9975.16$ & $      9975.16$ & $      9975.16$ & $         2.81$ sec    & $       4.0326$  & $       0.7126$ \\ 
     CC-Fusion-WS-MC & $     10841.41$ & $     10319.52$ & $      9883.83$ & $      9882.52$ & $      9882.52$ & $      9882.52$ & $      9882.52$ & $      9882.52$ & $        20.72$ sec    & $       4.0159$  & $       0.6983$ \\ 
\cmidrule{1-1} 
           MCR-CCFDB & $     10976.84$ & $     10033.19$ & $      9890.34$ & $      9890.34$ & $      9890.34$ & $      9890.34$ & $      9890.34$ & $      9890.34$ & $         1.07$ sec    & $       4.0480$  & $       0.7000$ \\ 
\cmidrule{1-1} 
           MCI-CCIFD & $     10759.51$ & $     10256.02$ & $      9881.86$ & $      9881.86$ & $      9881.86$ & $      9881.86$ & $      9881.86$ & $      9881.86$ & $         5.31$ sec    & $       4.0160$  & $       0.6983$ \\ 
\bottomrule
\end{tabular}
\end{table}

\begin{table}[H]
\scriptsize
\centering
\caption{image-seg (62096.bmp)}
\label{tab:anytimetable-image-seg-62096.bmp}
\begin{tabular}{lrrrrrrrrrrr}
\toprule
           algorithm &                                   \multicolumn{8}{c}{value} & \multicolumn{1}{c}{time}    & \multicolumn{1}{c}{VI}  & \multicolumn{1}{c}{RI} \\  
\cmidrule(lr){2-9}\cmidrule(lr){10-10} \cmidrule(lr){11-11} \cmidrule(lr){12-12}   
                     & \multicolumn{1}{c}{(0.5 sec)} & \multicolumn{1}{c}{(1 sec)} & \multicolumn{1}{c}{(10 sec)} & \multicolumn{1}{c}{(60 sec)} & \multicolumn{1}{c}{(300 sec)} & \multicolumn{1}{c}{(600 sec)} & \multicolumn{1}{c}{(1800 sec)} & \multicolumn{1}{c}{(end)} & \multicolumn{1}{c}{(end)}    & \multicolumn{1}{c}{(end)}   & \multicolumn{1}{c}{(end)}  \\ \midrule 
          PIVIT-BOEM & $\infty$ & $\infty$ & $\infty$ & $      5405.43$ & $      5405.43$ & $      5405.43$ & $      5405.43$ & $      5405.43$ & $        11.32$ sec    & $       4.7248$  & $       0.6928$ \\ 
                 CGC & $      3430.05$ & $      3430.05$ & $      3430.05$ & $      3430.05$ & $      3430.05$ & $      3430.05$ & $      3430.05$ & $      3430.05$ & $         0.54$ sec    & $       0.9297$  & $       0.9309$ \\ 
                  HC & $      3724.76$ & $      3724.76$ & $      3724.76$ & $      3724.76$ & $      3724.76$ & $      3724.76$ & $      3724.76$ & $      3724.76$ & $         0.00$ sec    & $       1.5801$  & $       0.8092$ \\ 
              HC-CGC & $      3424.07$ & $      3424.07$ & $      3424.07$ & $      3424.07$ & $      3424.07$ & $      3424.07$ & $      3424.07$ & $      3424.07$ & $         0.69$ sec    & $       0.9593$  & $       0.9279$ \\ 
              ogm-KL & $      3505.74$ & $      3473.37$ & $      3473.37$ & $      3473.37$ & $      3473.37$ & $      3473.37$ & $      3473.37$ & $      3473.37$ & $         0.77$ sec    & $       1.3491$  & $       0.8301$ \\ 
    CC-Fusion-HC-CGC & $      3420.23$ & $      3420.23$ & $      3420.23$ & $      3420.23$ & $      3420.23$ & $      3420.23$ & $      3420.23$ & $      3420.23$ & $         0.28$ sec    & $       0.9189$  & $       0.9320$ \\ 
     CC-Fusion-HC-MC & $      3420.19$ & $      3419.40$ & $      3419.40$ & $      3419.40$ & $      3419.40$ & $      3419.40$ & $      3419.40$ & $      3419.40$ & $         1.92$ sec    & $       0.9513$  & $       0.9286$ \\ 
    CC-Fusion-WS-CGC & $      3420.43$ & $      3420.43$ & $      3420.43$ & $      3420.43$ & $      3420.43$ & $      3420.43$ & $      3420.43$ & $      3420.43$ & $         0.59$ sec    & $       1.3364$  & $       0.8389$ \\ 
     CC-Fusion-WS-MC & $      3420.34$ & $      3420.30$ & $      3420.30$ & $      3420.30$ & $      3420.30$ & $      3420.30$ & $      3420.30$ & $      3420.30$ & $         1.65$ sec    & $       1.3403$  & $       0.8387$ \\ 
\cmidrule{1-1} 
           MCR-CCFDB & $      3419.40$ & $      3419.40$ & $      3419.40$ & $      3419.40$ & $      3419.40$ & $      3419.40$ & $      3419.40$ & $      3419.40$ & $         0.29$ sec    & $       0.9513$  & $       0.9286$ \\ 
\cmidrule{1-1} 
           MCI-CCIFD & $      3451.59$ & $      3419.40$ & $      3419.40$ & $      3419.40$ & $      3419.40$ & $      3419.40$ & $      3419.40$ & $      3419.40$ & $         0.64$ sec    & $       0.9513$  & $       0.9286$ \\ 
\bottomrule
\end{tabular}
\end{table}

\begin{table}[H]
\scriptsize
\centering
\caption{image-seg (65033.bmp)}
\label{tab:anytimetable-image-seg-65033.bmp}
\begin{tabular}{lrrrrrrrrrrr}
\toprule
           algorithm &                                   \multicolumn{8}{c}{value} & \multicolumn{1}{c}{time}    & \multicolumn{1}{c}{VI}  & \multicolumn{1}{c}{RI} \\  
\cmidrule(lr){2-9}\cmidrule(lr){10-10} \cmidrule(lr){11-11} \cmidrule(lr){12-12}   
                     & \multicolumn{1}{c}{(0.5 sec)} & \multicolumn{1}{c}{(1 sec)} & \multicolumn{1}{c}{(10 sec)} & \multicolumn{1}{c}{(60 sec)} & \multicolumn{1}{c}{(300 sec)} & \multicolumn{1}{c}{(600 sec)} & \multicolumn{1}{c}{(1800 sec)} & \multicolumn{1}{c}{(end)} & \multicolumn{1}{c}{(end)}    & \multicolumn{1}{c}{(end)}   & \multicolumn{1}{c}{(end)}  \\ \midrule 
          PIVIT-BOEM & $\infty$ & $\infty$ & $\infty$ & $\infty$ & $     10141.12$ & $     10141.12$ & $     10141.12$ & $     10141.12$ & $       109.00$ sec    & $       6.2258$  & $       0.7925$ \\ 
                 CGC & $      7407.36$ & $      7405.17$ & $      7405.17$ & $      7405.17$ & $      7405.17$ & $      7405.17$ & $      7405.17$ & $      7405.17$ & $         0.51$ sec    & $       2.6930$  & $       0.8659$ \\ 
                  HC & $      7954.15$ & $      7954.15$ & $      7954.15$ & $      7954.15$ & $      7954.15$ & $      7954.15$ & $      7954.15$ & $      7954.15$ & $         0.01$ sec    & $       3.0427$  & $       0.8429$ \\ 
              HC-CGC & $      7389.37$ & $      7389.09$ & $      7389.09$ & $      7389.09$ & $      7389.09$ & $      7389.09$ & $      7389.09$ & $      7389.09$ & $         0.60$ sec    & $       2.6742$  & $       0.8713$ \\ 
              ogm-KL & $     10114.93$ & $      7617.46$ & $      7580.90$ & $      7580.90$ & $      7580.90$ & $      7580.90$ & $      7580.90$ & $      7580.90$ & $         2.76$ sec    & $       2.8783$  & $       0.7639$ \\ 
    CC-Fusion-HC-CGC & $      7391.59$ & $      7385.65$ & $      7381.94$ & $      7381.94$ & $      7381.94$ & $      7381.94$ & $      7381.94$ & $      7381.94$ & $         1.71$ sec    & $       2.8700$  & $       0.8584$ \\ 
     CC-Fusion-HC-MC & $      7374.58$ & $      7368.35$ & $      7366.03$ & $      7366.03$ & $      7366.03$ & $      7366.03$ & $      7366.03$ & $      7366.03$ & $         7.46$ sec    & $       2.9032$  & $       0.8580$ \\ 
    CC-Fusion-WS-CGC & $      7429.52$ & $      7421.64$ & $      7409.61$ & $      7409.61$ & $      7409.61$ & $      7409.61$ & $      7409.61$ & $      7409.61$ & $         2.29$ sec    & $       2.6585$  & $       0.8717$ \\ 
     CC-Fusion-WS-MC & $      7800.51$ & $      7414.54$ & $      7365.22$ & $      7365.22$ & $      7365.22$ & $      7365.22$ & $      7365.22$ & $      7365.22$ & $         8.89$ sec    & $       2.9382$  & $       0.8569$ \\ 
\cmidrule{1-1} 
           MCR-CCFDB & $      7365.34$ & $      7365.34$ & $      7365.34$ & $      7365.34$ & $      7365.34$ & $      7365.34$ & $      7365.34$ & $      7365.34$ & $         0.42$ sec    & $       2.9437$  & $       0.8568$ \\ 
\cmidrule{1-1} 
           MCI-CCIFD & $      7541.46$ & $      7404.64$ & $      7364.57$ & $      7364.57$ & $      7364.57$ & $      7364.57$ & $      7364.57$ & $      7364.57$ & $         1.63$ sec    & $       2.9440$  & $       0.8568$ \\ 
\bottomrule
\end{tabular}
\end{table}

\begin{table}[H]
\scriptsize
\centering
\caption{image-seg (66053.bmp)}
\label{tab:anytimetable-image-seg-66053.bmp}
\begin{tabular}{lrrrrrrrrrrr}
\toprule
           algorithm &                                   \multicolumn{8}{c}{value} & \multicolumn{1}{c}{time}    & \multicolumn{1}{c}{VI}  & \multicolumn{1}{c}{RI} \\  
\cmidrule(lr){2-9}\cmidrule(lr){10-10} \cmidrule(lr){11-11} \cmidrule(lr){12-12}   
                     & \multicolumn{1}{c}{(0.5 sec)} & \multicolumn{1}{c}{(1 sec)} & \multicolumn{1}{c}{(10 sec)} & \multicolumn{1}{c}{(60 sec)} & \multicolumn{1}{c}{(300 sec)} & \multicolumn{1}{c}{(600 sec)} & \multicolumn{1}{c}{(1800 sec)} & \multicolumn{1}{c}{(end)} & \multicolumn{1}{c}{(end)}    & \multicolumn{1}{c}{(end)}   & \multicolumn{1}{c}{(end)}  \\ \midrule 
          PIVIT-BOEM & $\infty$ & $\infty$ & $\infty$ & $      7108.55$ & $      7108.55$ & $      7108.55$ & $      7108.55$ & $      7108.55$ & $        28.04$ sec    & $       5.5736$  & $       0.7748$ \\ 
                 CGC & $      4442.78$ & $      4441.72$ & $      4441.72$ & $      4441.72$ & $      4441.72$ & $      4441.72$ & $      4441.72$ & $      4441.72$ & $         0.80$ sec    & $       2.5962$  & $       0.7745$ \\ 
                  HC & $      5048.16$ & $      5048.16$ & $      5048.16$ & $      5048.16$ & $      5048.16$ & $      5048.16$ & $      5048.16$ & $      5048.16$ & $         0.00$ sec    & $       2.7227$  & $       0.7642$ \\ 
              HC-CGC & $      4445.52$ & $      4439.17$ & $      4439.17$ & $      4439.17$ & $      4439.17$ & $      4439.17$ & $      4439.17$ & $      4439.17$ & $         1.02$ sec    & $       2.6002$  & $       0.7752$ \\ 
              ogm-KL & $      4524.61$ & $      4524.61$ & $      4524.61$ & $      4524.61$ & $      4524.61$ & $      4524.61$ & $      4524.61$ & $      4524.61$ & $         0.54$ sec    & $       2.7513$  & $       0.7182$ \\ 
    CC-Fusion-HC-CGC & $      4433.21$ & $      4433.18$ & $      4433.18$ & $      4433.18$ & $      4433.18$ & $      4433.18$ & $      4433.18$ & $      4433.18$ & $         1.00$ sec    & $       2.2642$  & $       0.8553$ \\ 
     CC-Fusion-HC-MC & $      4431.06$ & $      4427.25$ & $      4427.25$ & $      4427.25$ & $      4427.25$ & $      4427.25$ & $      4427.25$ & $      4427.25$ & $         2.65$ sec    & $       2.2623$  & $       0.8561$ \\ 
    CC-Fusion-WS-CGC & $      4451.69$ & $      4451.69$ & $      4451.69$ & $      4451.69$ & $      4451.69$ & $      4451.69$ & $      4451.69$ & $      4451.69$ & $         0.73$ sec    & $       2.2551$  & $       0.8568$ \\ 
     CC-Fusion-WS-MC & $      4433.82$ & $      4427.93$ & $      4427.93$ & $      4427.93$ & $      4427.93$ & $      4427.93$ & $      4427.93$ & $      4427.93$ & $         3.52$ sec    & $       2.2559$  & $       0.8565$ \\ 
\cmidrule{1-1} 
           MCR-CCFDB & $      4429.11$ & $      4429.11$ & $      4429.11$ & $      4429.11$ & $      4429.11$ & $      4429.11$ & $      4429.11$ & $      4429.11$ & $         0.27$ sec    & $       2.2649$  & $       0.8561$ \\ 
\cmidrule{1-1} 
           MCI-CCIFD & $      4493.26$ & $      4427.25$ & $      4427.25$ & $      4427.25$ & $      4427.25$ & $      4427.25$ & $      4427.25$ & $      4427.25$ & $         0.73$ sec    & $       2.2623$  & $       0.8561$ \\ 
\bottomrule
\end{tabular}
\end{table}

\begin{table}[H]
\scriptsize
\centering
\caption{image-seg (69015.bmp)}
\label{tab:anytimetable-image-seg-69015.bmp}
\begin{tabular}{lrrrrrrrrrrr}
\toprule
           algorithm &                                   \multicolumn{8}{c}{value} & \multicolumn{1}{c}{time}    & \multicolumn{1}{c}{VI}  & \multicolumn{1}{c}{RI} \\  
\cmidrule(lr){2-9}\cmidrule(lr){10-10} \cmidrule(lr){11-11} \cmidrule(lr){12-12}   
                     & \multicolumn{1}{c}{(0.5 sec)} & \multicolumn{1}{c}{(1 sec)} & \multicolumn{1}{c}{(10 sec)} & \multicolumn{1}{c}{(60 sec)} & \multicolumn{1}{c}{(300 sec)} & \multicolumn{1}{c}{(600 sec)} & \multicolumn{1}{c}{(1800 sec)} & \multicolumn{1}{c}{(end)} & \multicolumn{1}{c}{(end)}    & \multicolumn{1}{c}{(end)}   & \multicolumn{1}{c}{(end)}  \\ \midrule 
          PIVIT-BOEM & $\infty$ & $\infty$ & $\infty$ & $      5878.27$ & $      5878.27$ & $      5878.27$ & $      5878.27$ & $      5878.27$ & $        17.77$ sec    & $       5.2817$  & $       0.8164$ \\ 
                 CGC & $      4090.96$ & $      4088.12$ & $      4088.12$ & $      4088.12$ & $      4088.12$ & $      4088.12$ & $      4088.12$ & $      4088.12$ & $         0.78$ sec    & $       2.8705$  & $       0.7595$ \\ 
                  HC & $      4308.17$ & $      4308.17$ & $      4308.17$ & $      4308.17$ & $      4308.17$ & $      4308.17$ & $      4308.17$ & $      4308.17$ & $         0.00$ sec    & $       2.9258$  & $       0.8392$ \\ 
              HC-CGC & $      4081.07$ & $      4081.07$ & $      4081.07$ & $      4081.07$ & $      4081.07$ & $      4081.07$ & $      4081.07$ & $      4081.07$ & $         0.13$ sec    & $       2.9965$  & $       0.8001$ \\ 
              ogm-KL & $      4161.61$ & $      4161.61$ & $      4161.61$ & $      4161.61$ & $      4161.61$ & $      4161.61$ & $      4161.61$ & $      4161.61$ & $         0.61$ sec    & $       3.3795$  & $       0.5935$ \\ 
    CC-Fusion-HC-CGC & $      4026.15$ & $      4025.79$ & $      4025.79$ & $      4025.79$ & $      4025.79$ & $      4025.79$ & $      4025.79$ & $      4025.79$ & $         0.91$ sec    & $       2.5500$  & $       0.8615$ \\ 
     CC-Fusion-HC-MC & $      4028.50$ & $      4025.34$ & $      4024.45$ & $      4024.45$ & $      4024.45$ & $      4024.45$ & $      4024.45$ & $      4024.45$ & $         5.37$ sec    & $       2.5936$  & $       0.8600$ \\ 
    CC-Fusion-WS-CGC & $      4028.23$ & $      4028.23$ & $      4028.23$ & $      4028.23$ & $      4028.23$ & $      4028.23$ & $      4028.23$ & $      4028.23$ & $         0.62$ sec    & $       2.5781$  & $       0.8604$ \\ 
     CC-Fusion-WS-MC & $      4049.88$ & $      4032.07$ & $      4024.45$ & $      4024.45$ & $      4024.45$ & $      4024.45$ & $      4024.45$ & $      4024.45$ & $         9.87$ sec    & $       2.5936$  & $       0.8600$ \\ 
\cmidrule{1-1} 
           MCR-CCFDB & $      4026.56$ & $      4026.56$ & $      4026.56$ & $      4026.56$ & $      4026.56$ & $      4026.56$ & $      4026.56$ & $      4026.56$ & $         0.24$ sec    & $       2.5948$  & $       0.8600$ \\ 
\cmidrule{1-1} 
           MCI-CCIFD & $      4252.47$ & $      4028.52$ & $      4024.45$ & $      4024.45$ & $      4024.45$ & $      4024.45$ & $      4024.45$ & $      4024.45$ & $         1.14$ sec    & $       2.5936$  & $       0.8600$ \\ 
\bottomrule
\end{tabular}
\end{table}

\begin{table}[H]
\scriptsize
\centering
\caption{image-seg (69020.bmp)}
\label{tab:anytimetable-image-seg-69020.bmp}
\begin{tabular}{lrrrrrrrrrrr}
\toprule
           algorithm &                                   \multicolumn{8}{c}{value} & \multicolumn{1}{c}{time}    & \multicolumn{1}{c}{VI}  & \multicolumn{1}{c}{RI} \\  
\cmidrule(lr){2-9}\cmidrule(lr){10-10} \cmidrule(lr){11-11} \cmidrule(lr){12-12}   
                     & \multicolumn{1}{c}{(0.5 sec)} & \multicolumn{1}{c}{(1 sec)} & \multicolumn{1}{c}{(10 sec)} & \multicolumn{1}{c}{(60 sec)} & \multicolumn{1}{c}{(300 sec)} & \multicolumn{1}{c}{(600 sec)} & \multicolumn{1}{c}{(1800 sec)} & \multicolumn{1}{c}{(end)} & \multicolumn{1}{c}{(end)}    & \multicolumn{1}{c}{(end)}   & \multicolumn{1}{c}{(end)}  \\ \midrule 
          PIVIT-BOEM & $\infty$ & $\infty$ & $\infty$ & $      8497.68$ & $      8497.68$ & $      8497.68$ & $      8497.68$ & $      8497.68$ & $        42.88$ sec    & $       6.2243$  & $       0.7413$ \\ 
                 CGC & $      5497.95$ & $      5326.69$ & $      5292.43$ & $      5292.43$ & $      5292.43$ & $      5292.43$ & $      5292.43$ & $      5292.43$ & $         1.69$ sec    & $       2.2497$  & $       0.8074$ \\ 
                  HC & $      5606.70$ & $      5606.70$ & $      5606.70$ & $      5606.70$ & $      5606.70$ & $      5606.70$ & $      5606.70$ & $      5606.70$ & $         0.01$ sec    & $       2.2130$  & $       0.8152$ \\ 
              HC-CGC & $      5203.21$ & $      5203.21$ & $      5203.21$ & $      5203.21$ & $      5203.21$ & $      5203.21$ & $      5203.21$ & $      5203.21$ & $         0.43$ sec    & $       2.0853$  & $       0.8255$ \\ 
              ogm-KL & $      5469.84$ & $      5453.10$ & $      5453.10$ & $      5453.10$ & $      5453.10$ & $      5453.10$ & $      5453.10$ & $      5453.10$ & $         0.91$ sec    & $       2.7002$  & $       0.4325$ \\ 
    CC-Fusion-HC-CGC & $      5206.42$ & $      5190.29$ & $      5190.29$ & $      5190.29$ & $      5190.29$ & $      5190.29$ & $      5190.29$ & $      5190.29$ & $         1.32$ sec    & $       1.8584$  & $       0.8554$ \\ 
     CC-Fusion-HC-MC & $      5208.22$ & $      5184.35$ & $      5179.95$ & $      5179.95$ & $      5179.95$ & $      5179.95$ & $      5179.95$ & $      5179.95$ & $         3.71$ sec    & $       1.8721$  & $       0.8550$ \\ 
    CC-Fusion-WS-CGC & $      5299.91$ & $      5279.80$ & $      5228.45$ & $      5228.45$ & $      5228.45$ & $      5228.45$ & $      5228.45$ & $      5228.45$ & $         1.68$ sec    & $       1.9271$  & $       0.8287$ \\ 
     CC-Fusion-WS-MC & $      5247.15$ & $      5193.00$ & $      5179.29$ & $      5179.29$ & $      5179.29$ & $      5179.29$ & $      5179.29$ & $      5179.29$ & $         7.09$ sec    & $       1.8730$  & $       0.8549$ \\ 
\cmidrule{1-1} 
           MCR-CCFDB & $      5389.80$ & $      5192.68$ & $      5192.68$ & $      5192.68$ & $      5192.68$ & $      5192.68$ & $      5192.68$ & $      5192.68$ & $         0.59$ sec    & $       1.9320$  & $       0.8523$ \\ 
\cmidrule{1-1} 
           MCI-CCIFD & $      5399.74$ & $      5399.74$ & $      5179.29$ & $      5179.29$ & $      5179.29$ & $      5179.29$ & $      5179.29$ & $      5179.29$ & $         1.66$ sec    & $       1.8730$  & $       0.8549$ \\ 
\bottomrule
\end{tabular}
\end{table}

\begin{table}[H]
\scriptsize
\centering
\caption{image-seg (69040.bmp)}
\label{tab:anytimetable-image-seg-69040.bmp}
\begin{tabular}{lrrrrrrrrrrr}
\toprule
           algorithm &                                   \multicolumn{8}{c}{value} & \multicolumn{1}{c}{time}    & \multicolumn{1}{c}{VI}  & \multicolumn{1}{c}{RI} \\  
\cmidrule(lr){2-9}\cmidrule(lr){10-10} \cmidrule(lr){11-11} \cmidrule(lr){12-12}   
                     & \multicolumn{1}{c}{(0.5 sec)} & \multicolumn{1}{c}{(1 sec)} & \multicolumn{1}{c}{(10 sec)} & \multicolumn{1}{c}{(60 sec)} & \multicolumn{1}{c}{(300 sec)} & \multicolumn{1}{c}{(600 sec)} & \multicolumn{1}{c}{(1800 sec)} & \multicolumn{1}{c}{(end)} & \multicolumn{1}{c}{(end)}    & \multicolumn{1}{c}{(end)}   & \multicolumn{1}{c}{(end)}  \\ \midrule 
          PIVIT-BOEM & $\infty$ & $\infty$ & $\infty$ & $\infty$ & $     10448.05$ & $     10448.05$ & $     10448.05$ & $     10448.05$ & $       124.42$ sec    & $       7.9061$  & $       0.3795$ \\ 
                 CGC & $      8253.22$ & $      8241.50$ & $      8135.56$ & $      8046.26$ & $      8046.20$ & $      8046.20$ & $      8046.20$ & $      8046.20$ & $        70.03$ sec    & $       3.3199$  & $       0.4891$ \\ 
                  HC & $      8814.27$ & $      8814.27$ & $      8814.27$ & $      8814.27$ & $      8814.27$ & $      8814.27$ & $      8814.27$ & $      8814.27$ & $         0.01$ sec    & $       3.9397$  & $       0.4196$ \\ 
              HC-CGC & $      8273.73$ & $      8221.44$ & $      8044.29$ & $      8044.29$ & $      8044.29$ & $      8044.29$ & $      8044.29$ & $      8044.29$ & $         7.65$ sec    & $       3.7586$  & $       0.4347$ \\ 
              ogm-KL & $     10138.02$ & $     10138.02$ & $      8254.71$ & $      8254.71$ & $      8254.71$ & $      8254.71$ & $      8254.71$ & $      8254.71$ & $         4.06$ sec    & $       2.1321$  & $       0.5229$ \\ 
    CC-Fusion-HC-CGC & $      8107.08$ & $      8060.11$ & $      8060.11$ & $      8060.11$ & $      8060.11$ & $      8060.11$ & $      8060.11$ & $      8060.11$ & $         1.78$ sec    & $       3.6844$  & $       0.4505$ \\ 
     CC-Fusion-HC-MC & $      8254.25$ & $      8072.69$ & $      7978.40$ & $      7978.40$ & $      7978.40$ & $      7978.40$ & $      7978.40$ & $      7978.40$ & $        16.74$ sec    & $       3.8310$  & $       0.4534$ \\ 
    CC-Fusion-WS-CGC & $      8168.56$ & $      8142.20$ & $      8098.14$ & $      8098.14$ & $      8098.14$ & $      8098.14$ & $      8098.14$ & $      8098.14$ & $         4.23$ sec    & $       3.4037$  & $       0.4752$ \\ 
     CC-Fusion-WS-MC & $      9389.64$ & $      8870.25$ & $      7997.25$ & $      7974.93$ & $      7974.93$ & $      7974.93$ & $      7974.93$ & $      7974.93$ & $        63.21$ sec    & $       4.1772$  & $       0.4165$ \\ 
\cmidrule{1-1} 
           MCR-CCFDB & $      9540.51$ & $      9043.51$ & $      7987.78$ & $      7987.78$ & $      7987.78$ & $      7987.78$ & $      7987.78$ & $      7987.78$ & $         3.15$ sec    & $       4.2102$  & $       0.4231$ \\ 
\cmidrule{1-1} 
           MCI-CCIFD & $      8924.10$ & $      8515.08$ & $      7974.58$ & $      7974.58$ & $      7974.58$ & $      7974.58$ & $      7974.58$ & $      7974.58$ & $         3.65$ sec    & $       4.0533$  & $       0.4314$ \\ 
\bottomrule
\end{tabular}
\end{table}

\begin{table}[H]
\scriptsize
\centering
\caption{image-seg (76053.bmp)}
\label{tab:anytimetable-image-seg-76053.bmp}
\begin{tabular}{lrrrrrrrrrrr}
\toprule
           algorithm &                                   \multicolumn{8}{c}{value} & \multicolumn{1}{c}{time}    & \multicolumn{1}{c}{VI}  & \multicolumn{1}{c}{RI} \\  
\cmidrule(lr){2-9}\cmidrule(lr){10-10} \cmidrule(lr){11-11} \cmidrule(lr){12-12}   
                     & \multicolumn{1}{c}{(0.5 sec)} & \multicolumn{1}{c}{(1 sec)} & \multicolumn{1}{c}{(10 sec)} & \multicolumn{1}{c}{(60 sec)} & \multicolumn{1}{c}{(300 sec)} & \multicolumn{1}{c}{(600 sec)} & \multicolumn{1}{c}{(1800 sec)} & \multicolumn{1}{c}{(end)} & \multicolumn{1}{c}{(end)}    & \multicolumn{1}{c}{(end)}   & \multicolumn{1}{c}{(end)}  \\ \midrule 
          PIVIT-BOEM & $\infty$ & $\infty$ & $\infty$ & $      6245.89$ & $      6245.89$ & $      6245.89$ & $      6245.89$ & $      6245.89$ & $        25.21$ sec    & $       6.2036$  & $       0.5727$ \\ 
                 CGC & $      4540.58$ & $      4540.58$ & $      4540.58$ & $      4540.58$ & $      4540.58$ & $      4540.58$ & $      4540.58$ & $      4540.58$ & $         0.34$ sec    & $       3.6375$  & $       0.5942$ \\ 
                  HC & $      5023.76$ & $      5023.76$ & $      5023.76$ & $      5023.76$ & $      5023.76$ & $      5023.76$ & $      5023.76$ & $      5023.76$ & $         0.00$ sec    & $       3.5673$  & $       0.5585$ \\ 
              HC-CGC & $      4544.14$ & $      4544.14$ & $      4544.14$ & $      4544.14$ & $      4544.14$ & $      4544.14$ & $      4544.14$ & $      4544.14$ & $         0.22$ sec    & $       3.6411$  & $       0.6016$ \\ 
              ogm-KL & $      4805.56$ & $      4805.56$ & $      4805.56$ & $      4805.56$ & $      4805.56$ & $      4805.56$ & $      4805.56$ & $      4805.56$ & $         0.55$ sec    & $       2.9633$  & $       0.5462$ \\ 
    CC-Fusion-HC-CGC & $      4548.82$ & $      4535.11$ & $      4535.11$ & $      4535.11$ & $      4535.11$ & $      4535.11$ & $      4535.11$ & $      4535.11$ & $         1.00$ sec    & $       3.7710$  & $       0.5998$ \\ 
     CC-Fusion-HC-MC & $      4527.71$ & $      4519.00$ & $      4514.99$ & $      4514.99$ & $      4514.99$ & $      4514.99$ & $      4514.99$ & $      4514.99$ & $         8.98$ sec    & $       3.7727$  & $       0.6016$ \\ 
    CC-Fusion-WS-CGC & $      4579.30$ & $      4561.38$ & $      4561.38$ & $      4561.38$ & $      4561.38$ & $      4561.38$ & $      4561.38$ & $      4561.38$ & $         1.32$ sec    & $       3.8064$  & $       0.5810$ \\ 
     CC-Fusion-WS-MC & $      4614.31$ & $      4556.53$ & $      4514.99$ & $      4514.99$ & $      4514.99$ & $      4514.99$ & $      4514.99$ & $      4514.99$ & $        14.18$ sec    & $       3.7727$  & $       0.6016$ \\ 
\cmidrule{1-1} 
           MCR-CCFDB & $      4534.40$ & $      4534.40$ & $      4534.40$ & $      4534.40$ & $      4534.40$ & $      4534.40$ & $      4534.40$ & $      4534.40$ & $         0.37$ sec    & $       3.8046$  & $       0.6028$ \\ 
\cmidrule{1-1} 
           MCI-CCIFD & $      4756.49$ & $      4551.42$ & $      4514.99$ & $      4514.99$ & $      4514.99$ & $      4514.99$ & $      4514.99$ & $      4514.99$ & $         1.50$ sec    & $       3.7727$  & $       0.6016$ \\ 
\bottomrule
\end{tabular}
\end{table}

\begin{table}[H]
\scriptsize
\centering
\caption{image-seg (78004.bmp)}
\label{tab:anytimetable-image-seg-78004.bmp}
\begin{tabular}{lrrrrrrrrrrr}
\toprule
           algorithm &                                   \multicolumn{8}{c}{value} & \multicolumn{1}{c}{time}    & \multicolumn{1}{c}{VI}  & \multicolumn{1}{c}{RI} \\  
\cmidrule(lr){2-9}\cmidrule(lr){10-10} \cmidrule(lr){11-11} \cmidrule(lr){12-12}   
                     & \multicolumn{1}{c}{(0.5 sec)} & \multicolumn{1}{c}{(1 sec)} & \multicolumn{1}{c}{(10 sec)} & \multicolumn{1}{c}{(60 sec)} & \multicolumn{1}{c}{(300 sec)} & \multicolumn{1}{c}{(600 sec)} & \multicolumn{1}{c}{(1800 sec)} & \multicolumn{1}{c}{(end)} & \multicolumn{1}{c}{(end)}    & \multicolumn{1}{c}{(end)}   & \multicolumn{1}{c}{(end)}  \\ \midrule 
          PIVIT-BOEM & $\infty$ & $\infty$ & $      4537.40$ & $      4537.40$ & $      4537.40$ & $      4537.40$ & $      4537.40$ & $      4537.40$ & $         8.82$ sec    & $       4.1380$  & $       0.8733$ \\ 
                 CGC & $      3270.53$ & $      3270.53$ & $      3270.53$ & $      3270.53$ & $      3270.53$ & $      3270.53$ & $      3270.53$ & $      3270.53$ & $         0.04$ sec    & $       1.8387$  & $       0.9292$ \\ 
                  HC & $      3470.35$ & $      3470.35$ & $      3470.35$ & $      3470.35$ & $      3470.35$ & $      3470.35$ & $      3470.35$ & $      3470.35$ & $         0.00$ sec    & $       2.1758$  & $       0.9015$ \\ 
              HC-CGC & $      3256.58$ & $      3256.58$ & $      3256.58$ & $      3256.58$ & $      3256.58$ & $      3256.58$ & $      3256.58$ & $      3256.58$ & $         0.04$ sec    & $       1.8106$  & $       0.9296$ \\ 
              ogm-KL & $      3312.31$ & $      3312.31$ & $      3312.31$ & $      3312.31$ & $      3312.31$ & $      3312.31$ & $      3312.31$ & $      3312.31$ & $         0.14$ sec    & $       2.9369$  & $       0.7800$ \\ 
    CC-Fusion-HC-CGC & $      3254.61$ & $      3254.61$ & $      3254.61$ & $      3254.61$ & $      3254.61$ & $      3254.61$ & $      3254.61$ & $      3254.61$ & $         0.56$ sec    & $       1.8211$  & $       0.9280$ \\ 
     CC-Fusion-HC-MC & $      3260.95$ & $      3254.85$ & $      3254.61$ & $      3254.61$ & $      3254.61$ & $      3254.61$ & $      3254.61$ & $      3254.61$ & $         3.13$ sec    & $       1.8211$  & $       0.9280$ \\ 
    CC-Fusion-WS-CGC & $      3256.22$ & $      3256.22$ & $      3256.22$ & $      3256.22$ & $      3256.22$ & $      3256.22$ & $      3256.22$ & $      3256.22$ & $         0.35$ sec    & $       1.8123$  & $       0.9277$ \\ 
     CC-Fusion-WS-MC & $      3269.53$ & $      3254.83$ & $      3254.61$ & $      3254.61$ & $      3254.61$ & $      3254.61$ & $      3254.61$ & $      3254.61$ & $         4.00$ sec    & $       1.8211$  & $       0.9280$ \\ 
\cmidrule{1-1} 
           MCR-CCFDB & $      3256.19$ & $      3256.19$ & $      3256.19$ & $      3256.19$ & $      3256.19$ & $      3256.19$ & $      3256.19$ & $      3256.19$ & $         0.05$ sec    & $       1.8217$  & $       0.9280$ \\ 
\cmidrule{1-1} 
           MCI-CCIFD & $      3254.61$ & $      3254.61$ & $      3254.61$ & $      3254.61$ & $      3254.61$ & $      3254.61$ & $      3254.61$ & $      3254.61$ & $         0.45$ sec    & $       1.8211$  & $       0.9280$ \\ 
\bottomrule
\end{tabular}
\end{table}

\begin{table}[H]
\scriptsize
\centering
\caption{image-seg (8023.bmp)}
\label{tab:anytimetable-image-seg-8023.bmp}
\begin{tabular}{lrrrrrrrrrrr}
\toprule
           algorithm &                                   \multicolumn{8}{c}{value} & \multicolumn{1}{c}{time}    & \multicolumn{1}{c}{VI}  & \multicolumn{1}{c}{RI} \\  
\cmidrule(lr){2-9}\cmidrule(lr){10-10} \cmidrule(lr){11-11} \cmidrule(lr){12-12}   
                     & \multicolumn{1}{c}{(0.5 sec)} & \multicolumn{1}{c}{(1 sec)} & \multicolumn{1}{c}{(10 sec)} & \multicolumn{1}{c}{(60 sec)} & \multicolumn{1}{c}{(300 sec)} & \multicolumn{1}{c}{(600 sec)} & \multicolumn{1}{c}{(1800 sec)} & \multicolumn{1}{c}{(end)} & \multicolumn{1}{c}{(end)}    & \multicolumn{1}{c}{(end)}   & \multicolumn{1}{c}{(end)}  \\ \midrule 
          PIVIT-BOEM & $\infty$ & $\infty$ & $\infty$ & $      5716.99$ & $      5716.99$ & $      5716.99$ & $      5716.99$ & $      5716.99$ & $        17.25$ sec    & $       6.0632$  & $       0.4745$ \\ 
                 CGC & $      4090.32$ & $      4076.82$ & $      4047.52$ & $      4047.52$ & $      4047.52$ & $      4047.52$ & $      4047.52$ & $      4047.52$ & $         5.40$ sec    & $       1.8680$  & $       0.5566$ \\ 
                  HC & $      4405.95$ & $      4405.95$ & $      4405.95$ & $      4405.95$ & $      4405.95$ & $      4405.95$ & $      4405.95$ & $      4405.95$ & $         0.00$ sec    & $       1.9550$  & $       0.5633$ \\ 
              HC-CGC & $      4109.08$ & $      4067.44$ & $      4038.55$ & $      4038.55$ & $      4038.55$ & $      4038.55$ & $      4038.55$ & $      4038.55$ & $         3.90$ sec    & $       1.9284$  & $       0.5497$ \\ 
              ogm-KL & $      4123.48$ & $      4123.48$ & $      4123.48$ & $      4123.48$ & $      4123.48$ & $      4123.48$ & $      4123.48$ & $      4123.48$ & $         0.60$ sec    & $       1.5005$  & $       0.5548$ \\ 
    CC-Fusion-HC-CGC & $      4038.90$ & $      4038.63$ & $      4038.63$ & $      4038.63$ & $      4038.63$ & $      4038.63$ & $      4038.63$ & $      4038.63$ & $         0.86$ sec    & $       1.9275$  & $       0.5550$ \\ 
     CC-Fusion-HC-MC & $      4044.89$ & $      4031.31$ & $      4026.67$ & $      4026.67$ & $      4026.67$ & $      4026.67$ & $      4026.67$ & $      4026.67$ & $         3.99$ sec    & $       2.0603$  & $       0.5433$ \\ 
    CC-Fusion-WS-CGC & $      4050.31$ & $      4050.31$ & $      4050.31$ & $      4050.31$ & $      4050.31$ & $      4050.31$ & $      4050.31$ & $      4050.31$ & $         0.74$ sec    & $       2.0113$  & $       0.5554$ \\ 
     CC-Fusion-WS-MC & $      4130.27$ & $      4089.65$ & $      4023.62$ & $      4023.62$ & $      4023.62$ & $      4023.62$ & $      4023.62$ & $      4023.62$ & $         9.29$ sec    & $       2.1718$  & $       0.5410$ \\ 
\cmidrule{1-1} 
           MCR-CCFDB & $      4094.33$ & $      4032.75$ & $      4032.75$ & $      4032.75$ & $      4032.75$ & $      4032.75$ & $      4032.75$ & $      4032.75$ & $         0.99$ sec    & $       2.2017$  & $       0.5403$ \\ 
\cmidrule{1-1} 
           MCI-CCIFD & $      4110.19$ & $      4096.95$ & $      4023.38$ & $      4023.38$ & $      4023.38$ & $      4023.38$ & $      4023.38$ & $      4023.38$ & $         3.96$ sec    & $       2.2067$  & $       0.5388$ \\ 
\bottomrule
\end{tabular}
\end{table}

\begin{table}[H]
\scriptsize
\centering
\caption{image-seg (85048.bmp)}
\label{tab:anytimetable-image-seg-85048.bmp}
\begin{tabular}{lrrrrrrrrrrr}
\toprule
           algorithm &                                   \multicolumn{8}{c}{value} & \multicolumn{1}{c}{time}    & \multicolumn{1}{c}{VI}  & \multicolumn{1}{c}{RI} \\  
\cmidrule(lr){2-9}\cmidrule(lr){10-10} \cmidrule(lr){11-11} \cmidrule(lr){12-12}   
                     & \multicolumn{1}{c}{(0.5 sec)} & \multicolumn{1}{c}{(1 sec)} & \multicolumn{1}{c}{(10 sec)} & \multicolumn{1}{c}{(60 sec)} & \multicolumn{1}{c}{(300 sec)} & \multicolumn{1}{c}{(600 sec)} & \multicolumn{1}{c}{(1800 sec)} & \multicolumn{1}{c}{(end)} & \multicolumn{1}{c}{(end)}    & \multicolumn{1}{c}{(end)}   & \multicolumn{1}{c}{(end)}  \\ \midrule 
          PIVIT-BOEM & $\infty$ & $\infty$ & $\infty$ & $      7720.13$ & $      7720.13$ & $      7720.13$ & $      7720.13$ & $      7720.13$ & $        57.98$ sec    & $       5.4287$  & $       0.8908$ \\ 
                 CGC & $      5881.18$ & $      5881.18$ & $      5881.18$ & $      5881.18$ & $      5881.18$ & $      5881.18$ & $      5881.18$ & $      5881.18$ & $         0.25$ sec    & $       3.0630$  & $       0.9128$ \\ 
                  HC & $      6394.69$ & $      6394.69$ & $      6394.69$ & $      6394.69$ & $      6394.69$ & $      6394.69$ & $      6394.69$ & $      6394.69$ & $         0.01$ sec    & $       3.0312$  & $       0.9113$ \\ 
              HC-CGC & $      5879.75$ & $      5879.75$ & $      5879.75$ & $      5879.75$ & $      5879.75$ & $      5879.75$ & $      5879.75$ & $      5879.75$ & $         0.15$ sec    & $       2.9341$  & $       0.9151$ \\ 
              ogm-KL & $      8093.99$ & $      6225.49$ & $      6182.61$ & $      6182.61$ & $      6182.61$ & $      6182.61$ & $      6182.61$ & $      6182.61$ & $         2.02$ sec    & $       3.8576$  & $       0.7150$ \\ 
    CC-Fusion-HC-CGC & $      5865.83$ & $      5861.73$ & $      5861.73$ & $      5861.73$ & $      5861.73$ & $      5861.73$ & $      5861.73$ & $      5861.73$ & $         1.30$ sec    & $       3.1026$  & $       0.9119$ \\ 
     CC-Fusion-HC-MC & $      5862.44$ & $      5855.44$ & $      5852.33$ & $      5852.33$ & $      5852.33$ & $      5852.33$ & $      5852.33$ & $      5852.33$ & $         5.44$ sec    & $       3.1339$  & $       0.9115$ \\ 
    CC-Fusion-WS-CGC & $      5890.54$ & $      5886.23$ & $      5873.79$ & $      5873.79$ & $      5873.79$ & $      5873.79$ & $      5873.79$ & $      5873.79$ & $         2.04$ sec    & $       3.0009$  & $       0.9134$ \\ 
     CC-Fusion-WS-MC & $      5947.52$ & $      5868.52$ & $      5852.33$ & $      5852.33$ & $      5852.33$ & $      5852.33$ & $      5852.33$ & $      5852.33$ & $         9.09$ sec    & $       3.1339$  & $       0.9115$ \\ 
\cmidrule{1-1} 
           MCR-CCFDB & $      5857.10$ & $      5857.10$ & $      5857.10$ & $      5857.10$ & $      5857.10$ & $      5857.10$ & $      5857.10$ & $      5857.10$ & $         0.20$ sec    & $       3.2150$  & $       0.9100$ \\ 
\cmidrule{1-1} 
           MCI-CCIFD & $      5912.25$ & $      5855.81$ & $      5851.38$ & $      5851.38$ & $      5851.38$ & $      5851.38$ & $      5851.38$ & $      5851.38$ & $         1.05$ sec    & $       3.1682$  & $       0.9109$ \\ 
\bottomrule
\end{tabular}
\end{table}

\begin{table}[H]
\scriptsize
\centering
\caption{image-seg (86000.bmp)}
\label{tab:anytimetable-image-seg-86000.bmp}
\begin{tabular}{lrrrrrrrrrrr}
\toprule
           algorithm &                                   \multicolumn{8}{c}{value} & \multicolumn{1}{c}{time}    & \multicolumn{1}{c}{VI}  & \multicolumn{1}{c}{RI} \\  
\cmidrule(lr){2-9}\cmidrule(lr){10-10} \cmidrule(lr){11-11} \cmidrule(lr){12-12}   
                     & \multicolumn{1}{c}{(0.5 sec)} & \multicolumn{1}{c}{(1 sec)} & \multicolumn{1}{c}{(10 sec)} & \multicolumn{1}{c}{(60 sec)} & \multicolumn{1}{c}{(300 sec)} & \multicolumn{1}{c}{(600 sec)} & \multicolumn{1}{c}{(1800 sec)} & \multicolumn{1}{c}{(end)} & \multicolumn{1}{c}{(end)}    & \multicolumn{1}{c}{(end)}   & \multicolumn{1}{c}{(end)}  \\ \midrule 
          PIVIT-BOEM & $\infty$ & $\infty$ & $\infty$ & $      5488.04$ & $      5488.04$ & $      5488.04$ & $      5488.04$ & $      5488.04$ & $        25.86$ sec    & $       4.8788$  & $       0.8206$ \\ 
                 CGC & $      4646.88$ & $      4646.88$ & $      4646.88$ & $      4646.88$ & $      4646.88$ & $      4646.88$ & $      4646.88$ & $      4646.88$ & $         0.11$ sec    & $       3.5697$  & $       0.8312$ \\ 
                  HC & $      4985.07$ & $      4985.07$ & $      4985.07$ & $      4985.07$ & $      4985.07$ & $      4985.07$ & $      4985.07$ & $      4985.07$ & $         0.00$ sec    & $       3.5114$  & $       0.8307$ \\ 
              HC-CGC & $      4644.98$ & $      4644.98$ & $      4644.98$ & $      4644.98$ & $      4644.98$ & $      4644.98$ & $      4644.98$ & $      4644.98$ & $         0.11$ sec    & $       3.6028$  & $       0.8309$ \\ 
              ogm-KL & $      4718.61$ & $      4718.61$ & $      4718.61$ & $      4718.61$ & $      4718.61$ & $      4718.61$ & $      4718.61$ & $      4718.61$ & $         0.58$ sec    & $       3.3422$  & $       0.8198$ \\ 
    CC-Fusion-HC-CGC & $      4637.57$ & $      4637.35$ & $      4637.35$ & $      4637.35$ & $      4637.35$ & $      4637.35$ & $      4637.35$ & $      4637.35$ & $         1.21$ sec    & $       3.5641$  & $       0.8312$ \\ 
     CC-Fusion-HC-MC & $      4649.62$ & $      4640.90$ & $      4633.86$ & $      4633.86$ & $      4633.86$ & $      4633.86$ & $      4633.86$ & $      4633.86$ & $         4.48$ sec    & $       3.5839$  & $       0.8309$ \\ 
    CC-Fusion-WS-CGC & $      4652.11$ & $      4649.03$ & $      4649.03$ & $      4649.03$ & $      4649.03$ & $      4649.03$ & $      4649.03$ & $      4649.03$ & $         0.99$ sec    & $       3.5296$  & $       0.8322$ \\ 
     CC-Fusion-WS-MC & $      4650.60$ & $      4639.73$ & $      4635.75$ & $      4635.75$ & $      4635.75$ & $      4635.75$ & $      4635.75$ & $      4635.75$ & $         4.30$ sec    & $       3.5804$  & $       0.8309$ \\ 
\cmidrule{1-1} 
           MCR-CCFDB & $      4634.68$ & $      4634.68$ & $      4634.68$ & $      4634.68$ & $      4634.68$ & $      4634.68$ & $      4634.68$ & $      4634.68$ & $         0.11$ sec    & $       3.5891$  & $       0.8308$ \\ 
\cmidrule{1-1} 
           MCI-CCIFD & $      4634.99$ & $      4633.86$ & $      4633.86$ & $      4633.86$ & $      4633.86$ & $      4633.86$ & $      4633.86$ & $      4633.86$ & $         0.79$ sec    & $       3.5839$  & $       0.8309$ \\ 
\bottomrule
\end{tabular}
\end{table}

\begin{table}[H]
\scriptsize
\centering
\caption{image-seg (86016.bmp)}
\label{tab:anytimetable-image-seg-86016.bmp}
\begin{tabular}{lrrrrrrrrrrr}
\toprule
           algorithm &                                   \multicolumn{8}{c}{value} & \multicolumn{1}{c}{time}    & \multicolumn{1}{c}{VI}  & \multicolumn{1}{c}{RI} \\  
\cmidrule(lr){2-9}\cmidrule(lr){10-10} \cmidrule(lr){11-11} \cmidrule(lr){12-12}   
                     & \multicolumn{1}{c}{(0.5 sec)} & \multicolumn{1}{c}{(1 sec)} & \multicolumn{1}{c}{(10 sec)} & \multicolumn{1}{c}{(60 sec)} & \multicolumn{1}{c}{(300 sec)} & \multicolumn{1}{c}{(600 sec)} & \multicolumn{1}{c}{(1800 sec)} & \multicolumn{1}{c}{(end)} & \multicolumn{1}{c}{(end)}    & \multicolumn{1}{c}{(end)}   & \multicolumn{1}{c}{(end)}  \\ \midrule 
          PIVIT-BOEM & $\infty$ & $\infty$ & $\infty$ & $\infty$ & $     13541.94$ & $     13541.94$ & $     13541.94$ & $     13541.94$ & $       113.78$ sec    & $       6.1072$  & $       0.7791$ \\ 
                 CGC & $      6624.58$ & $      6624.58$ & $      6624.58$ & $      6624.58$ & $      6624.58$ & $      6624.58$ & $      6624.58$ & $      6624.58$ & $         0.11$ sec    & $       2.5511$  & $       0.4868$ \\ 
                  HC & $      6792.12$ & $      6792.12$ & $      6792.12$ & $      6792.12$ & $      6792.12$ & $      6792.12$ & $      6792.12$ & $      6792.12$ & $         0.01$ sec    & $       2.6682$  & $       0.5046$ \\ 
              HC-CGC & $      6621.36$ & $      6621.36$ & $      6621.36$ & $      6621.36$ & $      6621.36$ & $      6621.36$ & $      6621.36$ & $      6621.36$ & $         0.18$ sec    & $       2.6136$  & $       0.4839$ \\ 
              ogm-KL & $      6661.14$ & $      6661.14$ & $      6661.14$ & $      6661.14$ & $      6661.14$ & $      6661.14$ & $      6661.14$ & $      6661.14$ & $         0.29$ sec    & $       2.6200$  & $       0.4322$ \\ 
    CC-Fusion-HC-CGC & $      6619.37$ & $      6619.37$ & $      6619.37$ & $      6619.37$ & $      6619.37$ & $      6619.37$ & $      6619.37$ & $      6619.37$ & $         0.85$ sec    & $       2.6104$  & $       0.4840$ \\ 
     CC-Fusion-HC-MC & $      6618.85$ & $      6618.85$ & $      6618.85$ & $      6618.85$ & $      6618.85$ & $      6618.85$ & $      6618.85$ & $      6618.85$ & $         2.82$ sec    & $       2.6165$  & $       0.4839$ \\ 
    CC-Fusion-WS-CGC & $      6624.74$ & $      6624.74$ & $      6624.74$ & $      6624.74$ & $      6624.74$ & $      6624.74$ & $      6624.74$ & $      6624.74$ & $         0.43$ sec    & $       2.5833$  & $       0.4845$ \\ 
     CC-Fusion-WS-MC & $      6618.85$ & $      6618.85$ & $      6618.85$ & $      6618.85$ & $      6618.85$ & $      6618.85$ & $      6618.85$ & $      6618.85$ & $         3.44$ sec    & $       2.6165$  & $       0.4839$ \\ 
\cmidrule{1-1} 
           MCR-CCFDB & $      6620.72$ & $      6620.72$ & $      6620.72$ & $      6620.72$ & $      6620.72$ & $      6620.72$ & $      6620.72$ & $      6620.72$ & $         0.14$ sec    & $       2.6183$  & $       0.4839$ \\ 
\cmidrule{1-1} 
           MCI-CCIFD & $      6631.73$ & $      6618.85$ & $      6618.85$ & $      6618.85$ & $      6618.85$ & $      6618.85$ & $      6618.85$ & $      6618.85$ & $         0.64$ sec    & $       2.6165$  & $       0.4839$ \\ 
\bottomrule
\end{tabular}
\end{table}

\begin{table}[H]
\scriptsize
\centering
\caption{image-seg (86068.bmp)}
\label{tab:anytimetable-image-seg-86068.bmp}
\begin{tabular}{lrrrrrrrrrrr}
\toprule
           algorithm &                                   \multicolumn{8}{c}{value} & \multicolumn{1}{c}{time}    & \multicolumn{1}{c}{VI}  & \multicolumn{1}{c}{RI} \\  
\cmidrule(lr){2-9}\cmidrule(lr){10-10} \cmidrule(lr){11-11} \cmidrule(lr){12-12}   
                     & \multicolumn{1}{c}{(0.5 sec)} & \multicolumn{1}{c}{(1 sec)} & \multicolumn{1}{c}{(10 sec)} & \multicolumn{1}{c}{(60 sec)} & \multicolumn{1}{c}{(300 sec)} & \multicolumn{1}{c}{(600 sec)} & \multicolumn{1}{c}{(1800 sec)} & \multicolumn{1}{c}{(end)} & \multicolumn{1}{c}{(end)}    & \multicolumn{1}{c}{(end)}   & \multicolumn{1}{c}{(end)}  \\ \midrule 
          PIVIT-BOEM & $\infty$ & $\infty$ & $\infty$ & $      7942.28$ & $      7942.28$ & $      7942.28$ & $      7942.28$ & $      7942.28$ & $        39.39$ sec    & $       7.3219$  & $       0.2156$ \\ 
                 CGC & $      5282.55$ & $      5271.44$ & $      5235.77$ & $      5235.77$ & $      5235.77$ & $      5235.77$ & $      5235.77$ & $      5235.77$ & $        10.83$ sec    & $       1.7011$  & $       0.6988$ \\ 
                  HC & $      5833.13$ & $      5833.13$ & $      5833.13$ & $      5833.13$ & $      5833.13$ & $      5833.13$ & $      5833.13$ & $      5833.13$ & $         0.01$ sec    & $       2.9679$  & $       0.3717$ \\ 
              HC-CGC & $      5314.36$ & $      5286.88$ & $      5234.18$ & $      5234.18$ & $      5234.18$ & $      5234.18$ & $      5234.18$ & $      5234.18$ & $         9.88$ sec    & $       1.7441$  & $       0.6930$ \\ 
              ogm-KL & $      5330.73$ & $      5324.67$ & $      5324.67$ & $      5324.67$ & $      5324.67$ & $      5324.67$ & $      5324.67$ & $      5324.67$ & $         1.19$ sec    & $       1.3032$  & $       0.7017$ \\ 
    CC-Fusion-HC-CGC & $      5236.94$ & $      5231.95$ & $      5229.37$ & $      5229.37$ & $      5229.37$ & $      5229.37$ & $      5229.37$ & $      5229.37$ & $         2.44$ sec    & $       2.0844$  & $       0.6054$ \\ 
     CC-Fusion-HC-MC & $      5212.38$ & $      5203.99$ & $      5198.87$ & $      5198.87$ & $      5198.87$ & $      5198.87$ & $      5198.87$ & $      5198.87$ & $         3.40$ sec    & $       2.8195$  & $       0.3971$ \\ 
    CC-Fusion-WS-CGC & $      5256.88$ & $      5250.53$ & $      5250.53$ & $      5250.53$ & $      5250.53$ & $      5250.53$ & $      5250.53$ & $      5250.53$ & $         1.32$ sec    & $       1.6953$  & $       0.6973$ \\ 
     CC-Fusion-WS-MC & $      5222.00$ & $      5204.13$ & $      5198.87$ & $      5198.87$ & $      5198.87$ & $      5198.87$ & $      5198.87$ & $      5198.87$ & $         7.56$ sec    & $       2.8195$  & $       0.3971$ \\ 
\cmidrule{1-1} 
           MCR-CCFDB & $      5481.98$ & $      5198.87$ & $      5198.87$ & $      5198.87$ & $      5198.87$ & $      5198.87$ & $      5198.87$ & $      5198.87$ & $         0.99$ sec    & $       2.8195$  & $       0.3971$ \\ 
\cmidrule{1-1} 
           MCI-CCIFD & $      5385.58$ & $      5244.09$ & $      5198.87$ & $      5198.87$ & $      5198.87$ & $      5198.87$ & $      5198.87$ & $      5198.87$ & $         1.41$ sec    & $       2.8195$  & $       0.3971$ \\ 
\bottomrule
\end{tabular}
\end{table}

\begin{table}[H]
\scriptsize
\centering
\caption{image-seg (87046.bmp)}
\label{tab:anytimetable-image-seg-87046.bmp}
\begin{tabular}{lrrrrrrrrrrr}
\toprule
           algorithm &                                   \multicolumn{8}{c}{value} & \multicolumn{1}{c}{time}    & \multicolumn{1}{c}{VI}  & \multicolumn{1}{c}{RI} \\  
\cmidrule(lr){2-9}\cmidrule(lr){10-10} \cmidrule(lr){11-11} \cmidrule(lr){12-12}   
                     & \multicolumn{1}{c}{(0.5 sec)} & \multicolumn{1}{c}{(1 sec)} & \multicolumn{1}{c}{(10 sec)} & \multicolumn{1}{c}{(60 sec)} & \multicolumn{1}{c}{(300 sec)} & \multicolumn{1}{c}{(600 sec)} & \multicolumn{1}{c}{(1800 sec)} & \multicolumn{1}{c}{(end)} & \multicolumn{1}{c}{(end)}    & \multicolumn{1}{c}{(end)}   & \multicolumn{1}{c}{(end)}  \\ \midrule 
          PIVIT-BOEM & $\infty$ & $\infty$ & $\infty$ & $      6290.96$ & $      6290.96$ & $      6290.96$ & $      6290.96$ & $      6290.96$ & $        23.54$ sec    & $       6.2501$  & $       0.6364$ \\ 
                 CGC & $      4340.11$ & $      4334.08$ & $      4334.08$ & $      4334.08$ & $      4334.08$ & $      4334.08$ & $      4334.08$ & $      4334.08$ & $         0.73$ sec    & $       3.5072$  & $       0.6026$ \\ 
                  HC & $      4798.20$ & $      4798.20$ & $      4798.20$ & $      4798.20$ & $      4798.20$ & $      4798.20$ & $      4798.20$ & $      4798.20$ & $         0.00$ sec    & $       3.0094$  & $       0.7185$ \\ 
              HC-CGC & $      4334.44$ & $      4333.16$ & $      4333.16$ & $      4333.16$ & $      4333.16$ & $      4333.16$ & $      4333.16$ & $      4333.16$ & $         0.71$ sec    & $       3.1717$  & $       0.6938$ \\ 
              ogm-KL & $      4563.99$ & $      4557.04$ & $      4557.04$ & $      4557.04$ & $      4557.04$ & $      4557.04$ & $      4557.04$ & $      4557.04$ & $         1.01$ sec    & $       2.8398$  & $       0.5227$ \\ 
    CC-Fusion-HC-CGC & $      4333.35$ & $      4327.38$ & $      4320.32$ & $      4320.32$ & $      4320.32$ & $      4320.32$ & $      4320.32$ & $      4320.32$ & $         2.12$ sec    & $       3.4626$  & $       0.6152$ \\ 
     CC-Fusion-HC-MC & $      4320.27$ & $      4315.53$ & $      4315.53$ & $      4315.53$ & $      4315.53$ & $      4315.53$ & $      4315.53$ & $      4315.53$ & $         2.16$ sec    & $       3.5416$  & $       0.6120$ \\ 
    CC-Fusion-WS-CGC & $      4357.79$ & $      4344.04$ & $      4344.04$ & $      4344.04$ & $      4344.04$ & $      4344.04$ & $      4344.04$ & $      4344.04$ & $         1.01$ sec    & $       3.4823$  & $       0.6196$ \\ 
     CC-Fusion-WS-MC & $      4337.08$ & $      4319.35$ & $      4316.62$ & $      4316.62$ & $      4316.62$ & $      4316.62$ & $      4316.62$ & $      4316.62$ & $         3.24$ sec    & $       3.5261$  & $       0.6172$ \\ 
\cmidrule{1-1} 
           MCR-CCFDB & $      4315.53$ & $      4315.53$ & $      4315.53$ & $      4315.53$ & $      4315.53$ & $      4315.53$ & $      4315.53$ & $      4315.53$ & $         0.40$ sec    & $       3.5416$  & $       0.6120$ \\ 
\cmidrule{1-1} 
           MCI-CCIFD & $      4376.91$ & $      4363.81$ & $      4315.53$ & $      4315.53$ & $      4315.53$ & $      4315.53$ & $      4315.53$ & $      4315.53$ & $         1.40$ sec    & $       3.5416$  & $       0.6120$ \\ 
\bottomrule
\end{tabular}
\end{table}

\begin{table}[H]
\scriptsize
\centering
\caption{image-seg (89072.bmp)}
\label{tab:anytimetable-image-seg-89072.bmp}
\begin{tabular}{lrrrrrrrrrrr}
\toprule
           algorithm &                                   \multicolumn{8}{c}{value} & \multicolumn{1}{c}{time}    & \multicolumn{1}{c}{VI}  & \multicolumn{1}{c}{RI} \\  
\cmidrule(lr){2-9}\cmidrule(lr){10-10} \cmidrule(lr){11-11} \cmidrule(lr){12-12}   
                     & \multicolumn{1}{c}{(0.5 sec)} & \multicolumn{1}{c}{(1 sec)} & \multicolumn{1}{c}{(10 sec)} & \multicolumn{1}{c}{(60 sec)} & \multicolumn{1}{c}{(300 sec)} & \multicolumn{1}{c}{(600 sec)} & \multicolumn{1}{c}{(1800 sec)} & \multicolumn{1}{c}{(end)} & \multicolumn{1}{c}{(end)}    & \multicolumn{1}{c}{(end)}   & \multicolumn{1}{c}{(end)}  \\ \midrule 
          PIVIT-BOEM & $\infty$ & $\infty$ & $\infty$ & $      5364.41$ & $      5364.41$ & $      5364.41$ & $      5364.41$ & $      5364.41$ & $        17.17$ sec    & $       4.9164$  & $       0.8414$ \\ 
                 CGC & $      3950.43$ & $      3950.43$ & $      3950.43$ & $      3950.43$ & $      3950.43$ & $      3950.43$ & $      3950.43$ & $      3950.43$ & $         0.38$ sec    & $       3.2823$  & $       0.8205$ \\ 
                  HC & $      4200.63$ & $      4200.63$ & $      4200.63$ & $      4200.63$ & $      4200.63$ & $      4200.63$ & $      4200.63$ & $      4200.63$ & $         0.00$ sec    & $       3.0533$  & $       0.8437$ \\ 
              HC-CGC & $      3947.25$ & $      3947.25$ & $      3947.25$ & $      3947.25$ & $      3947.25$ & $      3947.25$ & $      3947.25$ & $      3947.25$ & $         0.07$ sec    & $       3.1200$  & $       0.8454$ \\ 
              ogm-KL & $      4195.19$ & $      4171.57$ & $      4171.06$ & $      4171.06$ & $      4171.06$ & $      4171.06$ & $      4171.06$ & $      4171.06$ & $         1.46$ sec    & $       3.9097$  & $       0.6592$ \\ 
    CC-Fusion-HC-CGC & $      3934.71$ & $      3934.17$ & $      3933.75$ & $      3933.75$ & $      3933.75$ & $      3933.75$ & $      3933.75$ & $      3933.75$ & $         1.30$ sec    & $       3.1168$  & $       0.8454$ \\ 
     CC-Fusion-HC-MC & $      3937.40$ & $      3935.13$ & $      3933.75$ & $      3933.75$ & $      3933.75$ & $      3933.75$ & $      3933.75$ & $      3933.75$ & $         2.99$ sec    & $       3.1181$  & $       0.8454$ \\ 
    CC-Fusion-WS-CGC & $      3937.82$ & $      3937.82$ & $      3937.82$ & $      3937.82$ & $      3937.82$ & $      3937.82$ & $      3937.82$ & $      3937.82$ & $         0.54$ sec    & $       3.1132$  & $       0.8453$ \\ 
     CC-Fusion-WS-MC & $      3941.60$ & $      3934.52$ & $      3933.75$ & $      3933.75$ & $      3933.75$ & $      3933.75$ & $      3933.75$ & $      3933.75$ & $         3.14$ sec    & $       3.1181$  & $       0.8454$ \\ 
\cmidrule{1-1} 
           MCR-CCFDB & $      3935.25$ & $      3935.25$ & $      3935.25$ & $      3935.25$ & $      3935.25$ & $      3935.25$ & $      3935.25$ & $      3935.25$ & $         0.10$ sec    & $       3.1188$  & $       0.8454$ \\ 
\cmidrule{1-1} 
           MCI-CCIFD & $      3946.76$ & $      3933.75$ & $      3933.75$ & $      3933.75$ & $      3933.75$ & $      3933.75$ & $      3933.75$ & $      3933.75$ & $         0.57$ sec    & $       3.1181$  & $       0.8454$ \\ 
\bottomrule
\end{tabular}
\end{table}

\begin{table}[H]
\scriptsize
\centering
\caption{image-seg (97033.bmp)}
\label{tab:anytimetable-image-seg-97033.bmp}
\begin{tabular}{lrrrrrrrrrrr}
\toprule
           algorithm &                                   \multicolumn{8}{c}{value} & \multicolumn{1}{c}{time}    & \multicolumn{1}{c}{VI}  & \multicolumn{1}{c}{RI} \\  
\cmidrule(lr){2-9}\cmidrule(lr){10-10} \cmidrule(lr){11-11} \cmidrule(lr){12-12}   
                     & \multicolumn{1}{c}{(0.5 sec)} & \multicolumn{1}{c}{(1 sec)} & \multicolumn{1}{c}{(10 sec)} & \multicolumn{1}{c}{(60 sec)} & \multicolumn{1}{c}{(300 sec)} & \multicolumn{1}{c}{(600 sec)} & \multicolumn{1}{c}{(1800 sec)} & \multicolumn{1}{c}{(end)} & \multicolumn{1}{c}{(end)}    & \multicolumn{1}{c}{(end)}   & \multicolumn{1}{c}{(end)}  \\ \midrule 
          PIVIT-BOEM & $\infty$ & $\infty$ & $\infty$ & $      5862.67$ & $      5862.67$ & $      5862.67$ & $      5862.67$ & $      5862.67$ & $        21.90$ sec    & $       5.1210$  & $       0.7345$ \\ 
                 CGC & $      4337.35$ & $      4337.35$ & $      4337.35$ & $      4337.35$ & $      4337.35$ & $      4337.35$ & $      4337.35$ & $      4337.35$ & $         0.22$ sec    & $       2.4771$  & $       0.7825$ \\ 
                  HC & $      4816.56$ & $      4816.56$ & $      4816.56$ & $      4816.56$ & $      4816.56$ & $      4816.56$ & $      4816.56$ & $      4816.56$ & $         0.00$ sec    & $       2.7490$  & $       0.7650$ \\ 
              HC-CGC & $      4334.12$ & $      4334.12$ & $      4334.12$ & $      4334.12$ & $      4334.12$ & $      4334.12$ & $      4334.12$ & $      4334.12$ & $         0.09$ sec    & $       2.4903$  & $       0.7869$ \\ 
              ogm-KL & $      4553.50$ & $      4553.50$ & $      4553.50$ & $      4553.50$ & $      4553.50$ & $      4553.50$ & $      4553.50$ & $      4553.50$ & $         0.37$ sec    & $       2.6802$  & $       0.7029$ \\ 
    CC-Fusion-HC-CGC & $      4328.51$ & $      4328.51$ & $      4328.51$ & $      4328.51$ & $      4328.51$ & $      4328.51$ & $      4328.51$ & $      4328.51$ & $         0.68$ sec    & $       2.3882$  & $       0.7912$ \\ 
     CC-Fusion-HC-MC & $      4321.87$ & $      4320.94$ & $      4320.69$ & $      4320.69$ & $      4320.69$ & $      4320.69$ & $      4320.69$ & $      4320.69$ & $         2.85$ sec    & $       2.5451$  & $       0.7837$ \\ 
    CC-Fusion-WS-CGC & $      4338.32$ & $      4336.74$ & $      4335.37$ & $      4335.37$ & $      4335.37$ & $      4335.37$ & $      4335.37$ & $      4335.37$ & $         1.56$ sec    & $       2.3734$  & $       0.7902$ \\ 
     CC-Fusion-WS-MC & $      4336.87$ & $      4321.21$ & $      4320.69$ & $      4320.69$ & $      4320.69$ & $      4320.69$ & $      4320.69$ & $      4320.69$ & $         3.49$ sec    & $       2.5451$  & $       0.7837$ \\ 
\cmidrule{1-1} 
           MCR-CCFDB & $      4320.69$ & $      4320.69$ & $      4320.69$ & $      4320.69$ & $      4320.69$ & $      4320.69$ & $      4320.69$ & $      4320.69$ & $         0.18$ sec    & $       2.5451$  & $       0.7837$ \\ 
\cmidrule{1-1} 
           MCI-CCIFD & $      4320.69$ & $      4320.69$ & $      4320.69$ & $      4320.69$ & $      4320.69$ & $      4320.69$ & $      4320.69$ & $      4320.69$ & $         0.37$ sec    & $       2.5451$  & $       0.7837$ \\ 
\bottomrule
\end{tabular}
\end{table}


\subsection{knott-3d-150}
\begin{table}[H]
\scriptsize
\centering
\caption{knott-3d-150 (gm\_knott\_3d\_032)}
\label{tab:anytimetable-knott-3d-150-gm-knott-3d-032}
\begin{tabular}{lrrrrrrrrrrr}
\toprule
           algorithm &                                   \multicolumn{8}{c}{value} & \multicolumn{1}{c}{time}    & \multicolumn{1}{c}{VI}  & \multicolumn{1}{c}{RI} \\  
\cmidrule(lr){2-9}\cmidrule(lr){10-10} \cmidrule(lr){11-11} \cmidrule(lr){12-12}   
                     & \multicolumn{1}{c}{(0.5 sec)} & \multicolumn{1}{c}{(1 sec)} & \multicolumn{1}{c}{(10 sec)} & \multicolumn{1}{c}{(60 sec)} & \multicolumn{1}{c}{(300 sec)} & \multicolumn{1}{c}{(600 sec)} & \multicolumn{1}{c}{(1800 sec)} & \multicolumn{1}{c}{(end)} & \multicolumn{1}{c}{(end)}    & \multicolumn{1}{c}{(end)}   & \multicolumn{1}{c}{(end)}  \\ \midrule 
          PIVIT-BOEM & $\infty$ & $\infty$ & $     -3800.90$ & $     -3800.90$ & $     -3800.90$ & $     -3800.90$ & $     -3800.90$ & $     -3800.90$ & $         1.51$ sec    & $       2.7371$  & $       0.8550$ \\ 
                 CGC & $     -5810.10$ & $     -5810.10$ & $     -5810.10$ & $     -5810.10$ & $     -5810.10$ & $     -5810.10$ & $     -5810.10$ & $     -5810.10$ & $         0.05$ sec    & $       1.0520$  & $       0.9055$ \\ 
                  HC & $     -4917.85$ & $     -4917.85$ & $     -4917.85$ & $     -4917.85$ & $     -4917.85$ & $     -4917.85$ & $     -4917.85$ & $     -4917.85$ & $         0.00$ sec    & $       1.6821$  & $       0.8352$ \\ 
              HC-CGC & $     -5811.47$ & $     -5811.47$ & $     -5811.47$ & $     -5811.47$ & $     -5811.47$ & $     -5811.47$ & $     -5811.47$ & $     -5811.47$ & $         0.03$ sec    & $       1.0508$  & $       0.9055$ \\ 
              ogm-KL & $     -5529.15$ & $     -5529.15$ & $     -5529.15$ & $     -5529.15$ & $     -5529.15$ & $     -5529.15$ & $     -5529.15$ & $     -5529.15$ & $         0.11$ sec    & $       2.6680$  & $       0.7622$ \\ 
    CC-Fusion-HC-CGC & $     -5811.47$ & $     -5811.47$ & $     -5811.47$ & $     -5811.47$ & $     -5811.47$ & $     -5811.47$ & $     -5811.47$ & $     -5811.47$ & $         0.37$ sec    & $       1.0508$  & $       0.9055$ \\ 
     CC-Fusion-HC-MC & $     -5811.47$ & $     -5811.47$ & $     -5811.47$ & $     -5811.47$ & $     -5811.47$ & $     -5811.47$ & $     -5811.47$ & $     -5811.47$ & $         1.51$ sec    & $       1.0508$  & $       0.9055$ \\ 
    CC-Fusion-WS-CGC & $     -5798.93$ & $     -5798.93$ & $     -5798.93$ & $     -5798.93$ & $     -5798.93$ & $     -5798.93$ & $     -5798.93$ & $     -5798.93$ & $         0.26$ sec    & $       1.1435$  & $       0.9022$ \\ 
     CC-Fusion-WS-MC & $     -5766.88$ & $     -5811.01$ & $     -5811.47$ & $     -5811.47$ & $     -5811.47$ & $     -5811.47$ & $     -5811.47$ & $     -5811.47$ & $         3.02$ sec    & $       1.0508$  & $       0.9055$ \\ 
\cmidrule{1-1} 
           MCR-CCFDB & $     -5809.41$ & $     -5809.41$ & $     -5809.41$ & $     -5809.41$ & $     -5809.41$ & $     -5809.41$ & $     -5809.41$ & $     -5809.41$ & $         0.44$ sec    & $       1.1335$  & $       0.9019$ \\ 
\cmidrule{1-1} 
           MCI-CCIFD & $     -5811.47$ & $     -5811.47$ & $     -5811.47$ & $     -5811.47$ & $     -5811.47$ & $     -5811.47$ & $     -5811.47$ & $     -5811.47$ & $         0.67$ sec    & $       1.0508$  & $       0.9055$ \\ 
\bottomrule
\end{tabular}
\end{table}

\begin{table}[H]
\scriptsize
\centering
\caption{knott-3d-150 (gm\_knott\_3d\_033)}
\label{tab:anytimetable-knott-3d-150-gm-knott-3d-033}
\begin{tabular}{lrrrrrrrrrrr}
\toprule
           algorithm &                                   \multicolumn{8}{c}{value} & \multicolumn{1}{c}{time}    & \multicolumn{1}{c}{VI}  & \multicolumn{1}{c}{RI} \\  
\cmidrule(lr){2-9}\cmidrule(lr){10-10} \cmidrule(lr){11-11} \cmidrule(lr){12-12}   
                     & \multicolumn{1}{c}{(0.5 sec)} & \multicolumn{1}{c}{(1 sec)} & \multicolumn{1}{c}{(10 sec)} & \multicolumn{1}{c}{(60 sec)} & \multicolumn{1}{c}{(300 sec)} & \multicolumn{1}{c}{(600 sec)} & \multicolumn{1}{c}{(1800 sec)} & \multicolumn{1}{c}{(end)} & \multicolumn{1}{c}{(end)}    & \multicolumn{1}{c}{(end)}   & \multicolumn{1}{c}{(end)}  \\ \midrule 
          PIVIT-BOEM & $\infty$ & $\infty$ & $      5260.11$ & $      5260.11$ & $      5260.11$ & $      5260.11$ & $      5260.11$ & $      5260.11$ & $         5.45$ sec    & $       4.9486$  & $       0.2371$ \\ 
                 CGC & $     -2545.84$ & $     -2545.84$ & $     -2545.84$ & $     -2545.84$ & $     -2545.84$ & $     -2545.84$ & $     -2545.84$ & $     -2545.84$ & $         0.05$ sec    & $       0.4964$  & $       0.8790$ \\ 
                  HC & $     -1943.62$ & $     -1943.62$ & $     -1943.62$ & $     -1943.62$ & $     -1943.62$ & $     -1943.62$ & $     -1943.62$ & $     -1943.62$ & $         0.01$ sec    & $       0.5591$  & $       0.8613$ \\ 
              HC-CGC & $     -2545.84$ & $     -2545.84$ & $     -2545.84$ & $     -2545.84$ & $     -2545.84$ & $     -2545.84$ & $     -2545.84$ & $     -2545.84$ & $         0.09$ sec    & $       0.4964$  & $       0.8790$ \\ 
              ogm-KL & $     -2487.31$ & $     -2487.31$ & $     -2487.31$ & $     -2487.31$ & $     -2487.31$ & $     -2487.31$ & $     -2487.31$ & $     -2487.31$ & $         0.15$ sec    & $       0.7502$  & $       0.8622$ \\ 
    CC-Fusion-HC-CGC & $     -2545.84$ & $     -2545.84$ & $     -2545.84$ & $     -2545.84$ & $     -2545.84$ & $     -2545.84$ & $     -2545.84$ & $     -2545.84$ & $         0.40$ sec    & $       0.4964$  & $       0.8790$ \\ 
     CC-Fusion-HC-MC & $     -2545.84$ & $     -2545.84$ & $     -2545.84$ & $     -2545.84$ & $     -2545.84$ & $     -2545.84$ & $     -2545.84$ & $     -2545.84$ & $         1.11$ sec    & $       0.4964$  & $       0.8790$ \\ 
    CC-Fusion-WS-CGC & $     -2545.84$ & $     -2545.84$ & $     -2545.84$ & $     -2545.84$ & $     -2545.84$ & $     -2545.84$ & $     -2545.84$ & $     -2545.84$ & $         0.46$ sec    & $       0.4964$  & $       0.8790$ \\ 
     CC-Fusion-WS-MC & $     -2545.84$ & $     -2545.84$ & $     -2545.84$ & $     -2545.84$ & $     -2545.84$ & $     -2545.84$ & $     -2545.84$ & $     -2545.84$ & $         1.86$ sec    & $       0.4964$  & $       0.8790$ \\ 
\cmidrule{1-1} 
           MCR-CCFDB & $     -1917.78$ & $     -2350.25$ & $     -2545.84$ & $     -2545.84$ & $     -2545.84$ & $     -2545.84$ & $     -2545.84$ & $     -2545.84$ & $         1.58$ sec    & $       0.4964$  & $       0.8790$ \\ 
\cmidrule{1-1} 
           MCI-CCIFD & $     -2545.84$ & $     -2545.84$ & $     -2545.84$ & $     -2545.84$ & $     -2545.84$ & $     -2545.84$ & $     -2545.84$ & $     -2545.84$ & $         0.33$ sec    & $       0.4964$  & $       0.8790$ \\ 
\bottomrule
\end{tabular}
\end{table}

\begin{table}[H]
\scriptsize
\centering
\caption{knott-3d-150 (gm\_knott\_3d\_034)}
\label{tab:anytimetable-knott-3d-150-gm-knott-3d-034}
\begin{tabular}{lrrrrrrrrrrr}
\toprule
           algorithm &                                   \multicolumn{8}{c}{value} & \multicolumn{1}{c}{time}    & \multicolumn{1}{c}{VI}  & \multicolumn{1}{c}{RI} \\  
\cmidrule(lr){2-9}\cmidrule(lr){10-10} \cmidrule(lr){11-11} \cmidrule(lr){12-12}   
                     & \multicolumn{1}{c}{(0.5 sec)} & \multicolumn{1}{c}{(1 sec)} & \multicolumn{1}{c}{(10 sec)} & \multicolumn{1}{c}{(60 sec)} & \multicolumn{1}{c}{(300 sec)} & \multicolumn{1}{c}{(600 sec)} & \multicolumn{1}{c}{(1800 sec)} & \multicolumn{1}{c}{(end)} & \multicolumn{1}{c}{(end)}    & \multicolumn{1}{c}{(end)}   & \multicolumn{1}{c}{(end)}  \\ \midrule 
          PIVIT-BOEM & $\infty$ & $\infty$ & $      2834.57$ & $      2834.57$ & $      2834.57$ & $      2834.57$ & $      2834.57$ & $      2834.57$ & $         6.89$ sec    & $       2.8922$  & $       0.8586$ \\ 
                 CGC & $     -4064.87$ & $     -4064.87$ & $     -4064.87$ & $     -4064.87$ & $     -4064.87$ & $     -4064.87$ & $     -4064.87$ & $     -4064.87$ & $         0.08$ sec    & $       0.9066$  & $       0.9106$ \\ 
                  HC & $     -2830.20$ & $     -2830.20$ & $     -2830.20$ & $     -2830.20$ & $     -2830.20$ & $     -2830.20$ & $     -2830.20$ & $     -2830.20$ & $         0.01$ sec    & $       2.0380$  & $       0.6586$ \\ 
              HC-CGC & $     -4062.65$ & $     -4062.65$ & $     -4062.65$ & $     -4062.65$ & $     -4062.65$ & $     -4062.65$ & $     -4062.65$ & $     -4062.65$ & $         0.05$ sec    & $       0.9071$  & $       0.9106$ \\ 
              ogm-KL & $     -3889.86$ & $     -3889.86$ & $     -3889.86$ & $     -3889.86$ & $     -3889.86$ & $     -3889.86$ & $     -3889.86$ & $     -3889.86$ & $         0.23$ sec    & $       2.1895$  & $       0.6956$ \\ 
    CC-Fusion-HC-CGC & $     -3975.79$ & $     -3975.79$ & $     -3975.79$ & $     -3975.79$ & $     -3975.79$ & $     -3975.79$ & $     -3975.79$ & $     -3975.79$ & $         0.67$ sec    & $       1.5758$  & $       0.7397$ \\ 
     CC-Fusion-HC-MC & $     -3972.66$ & $     -3975.79$ & $     -3975.79$ & $     -3975.79$ & $     -3975.79$ & $     -3975.79$ & $     -3975.79$ & $     -3975.79$ & $         2.26$ sec    & $       1.5758$  & $       0.7397$ \\ 
    CC-Fusion-WS-CGC & $     -3974.19$ & $     -3974.19$ & $     -3974.19$ & $     -3974.19$ & $     -3974.19$ & $     -3974.19$ & $     -3974.19$ & $     -3974.19$ & $         0.68$ sec    & $       1.6881$  & $       0.7231$ \\ 
     CC-Fusion-WS-MC & $     -3964.28$ & $     -3975.28$ & $     -3975.79$ & $     -3975.79$ & $     -3975.79$ & $     -3975.79$ & $     -3975.79$ & $     -3975.79$ & $         6.83$ sec    & $       1.5758$  & $       0.7397$ \\ 
\cmidrule{1-1} 
           MCR-CCFDB & $     -4061.36$ & $     -4061.36$ & $     -4061.36$ & $     -4061.36$ & $     -4061.36$ & $     -4061.36$ & $     -4061.36$ & $     -4061.36$ & $         0.49$ sec    & $       0.9069$  & $       0.9106$ \\ 
\cmidrule{1-1} 
           MCI-CCIFD & $     -3957.84$ & $     -4064.87$ & $     -4064.87$ & $     -4064.87$ & $     -4064.87$ & $     -4064.87$ & $     -4064.87$ & $     -4064.87$ & $         0.70$ sec    & $       0.9066$  & $       0.9106$ \\ 
\bottomrule
\end{tabular}
\end{table}

\begin{table}[H]
\scriptsize
\centering
\caption{knott-3d-150 (gm\_knott\_3d\_035)}
\label{tab:anytimetable-knott-3d-150-gm-knott-3d-035}
\begin{tabular}{lrrrrrrrrrrr}
\toprule
           algorithm &                                   \multicolumn{8}{c}{value} & \multicolumn{1}{c}{time}    & \multicolumn{1}{c}{VI}  & \multicolumn{1}{c}{RI} \\  
\cmidrule(lr){2-9}\cmidrule(lr){10-10} \cmidrule(lr){11-11} \cmidrule(lr){12-12}   
                     & \multicolumn{1}{c}{(0.5 sec)} & \multicolumn{1}{c}{(1 sec)} & \multicolumn{1}{c}{(10 sec)} & \multicolumn{1}{c}{(60 sec)} & \multicolumn{1}{c}{(300 sec)} & \multicolumn{1}{c}{(600 sec)} & \multicolumn{1}{c}{(1800 sec)} & \multicolumn{1}{c}{(end)} & \multicolumn{1}{c}{(end)}    & \multicolumn{1}{c}{(end)}   & \multicolumn{1}{c}{(end)}  \\ \midrule 
          PIVIT-BOEM & $\infty$ & $\infty$ & $       880.94$ & $       880.94$ & $       880.94$ & $       880.94$ & $       880.94$ & $       880.94$ & $         4.12$ sec    & $       3.6774$  & $       0.7177$ \\ 
                 CGC & $     -4595.84$ & $     -4595.84$ & $     -4595.84$ & $     -4595.84$ & $     -4595.84$ & $     -4595.84$ & $     -4595.84$ & $     -4595.84$ & $         0.09$ sec    & $       0.7473$  & $       0.9125$ \\ 
                  HC & $     -3756.70$ & $     -3756.70$ & $     -3756.70$ & $     -3756.70$ & $     -3756.70$ & $     -3756.70$ & $     -3756.70$ & $     -3756.70$ & $         0.01$ sec    & $       1.7381$  & $       0.6938$ \\ 
              HC-CGC & $     -4595.84$ & $     -4595.84$ & $     -4595.84$ & $     -4595.84$ & $     -4595.84$ & $     -4595.84$ & $     -4595.84$ & $     -4595.84$ & $         0.05$ sec    & $       0.7473$  & $       0.9125$ \\ 
              ogm-KL & $     -4561.64$ & $     -4561.64$ & $     -4561.64$ & $     -4561.64$ & $     -4561.64$ & $     -4561.64$ & $     -4561.64$ & $     -4561.64$ & $         0.13$ sec    & $       1.6108$  & $       0.8007$ \\ 
    CC-Fusion-HC-CGC & $     -4595.84$ & $     -4595.84$ & $     -4595.84$ & $     -4595.84$ & $     -4595.84$ & $     -4595.84$ & $     -4595.84$ & $     -4595.84$ & $         0.43$ sec    & $       0.7473$  & $       0.9125$ \\ 
     CC-Fusion-HC-MC & $     -4595.84$ & $     -4595.84$ & $     -4595.84$ & $     -4595.84$ & $     -4595.84$ & $     -4595.84$ & $     -4595.84$ & $     -4595.84$ & $         1.18$ sec    & $       0.7473$  & $       0.9125$ \\ 
    CC-Fusion-WS-CGC & $     -4595.04$ & $     -4595.04$ & $     -4595.04$ & $     -4595.04$ & $     -4595.04$ & $     -4595.04$ & $     -4595.04$ & $     -4595.04$ & $         0.29$ sec    & $       0.7473$  & $       0.9125$ \\ 
     CC-Fusion-WS-MC & $     -4595.84$ & $     -4595.84$ & $     -4595.84$ & $     -4595.84$ & $     -4595.84$ & $     -4595.84$ & $     -4595.84$ & $     -4595.84$ & $         1.79$ sec    & $       0.7473$  & $       0.9125$ \\ 
\cmidrule{1-1} 
           MCR-CCFDB & $     -2450.55$ & $     -4595.84$ & $     -4595.84$ & $     -4595.84$ & $     -4595.84$ & $     -4595.84$ & $     -4595.84$ & $     -4595.84$ & $         0.97$ sec    & $       0.7473$  & $       0.9125$ \\ 
\cmidrule{1-1} 
           MCI-CCIFD & $     -4595.84$ & $     -4595.84$ & $     -4595.84$ & $     -4595.84$ & $     -4595.84$ & $     -4595.84$ & $     -4595.84$ & $     -4595.84$ & $         0.22$ sec    & $       0.7473$  & $       0.9125$ \\ 
\bottomrule
\end{tabular}
\end{table}

\begin{table}[H]
\scriptsize
\centering
\caption{knott-3d-150 (gm\_knott\_3d\_036)}
\label{tab:anytimetable-knott-3d-150-gm-knott-3d-036}
\begin{tabular}{lrrrrrrrrrrr}
\toprule
           algorithm &                                   \multicolumn{8}{c}{value} & \multicolumn{1}{c}{time}    & \multicolumn{1}{c}{VI}  & \multicolumn{1}{c}{RI} \\  
\cmidrule(lr){2-9}\cmidrule(lr){10-10} \cmidrule(lr){11-11} \cmidrule(lr){12-12}   
                     & \multicolumn{1}{c}{(0.5 sec)} & \multicolumn{1}{c}{(1 sec)} & \multicolumn{1}{c}{(10 sec)} & \multicolumn{1}{c}{(60 sec)} & \multicolumn{1}{c}{(300 sec)} & \multicolumn{1}{c}{(600 sec)} & \multicolumn{1}{c}{(1800 sec)} & \multicolumn{1}{c}{(end)} & \multicolumn{1}{c}{(end)}    & \multicolumn{1}{c}{(end)}   & \multicolumn{1}{c}{(end)}  \\ \midrule 
          PIVIT-BOEM & $\infty$ & $\infty$ & $     -4015.99$ & $     -4015.99$ & $     -4015.99$ & $     -4015.99$ & $     -4015.99$ & $     -4015.99$ & $         1.06$ sec    & $       2.0915$  & $       0.9387$ \\ 
                 CGC & $     -5164.28$ & $     -5164.28$ & $     -5164.28$ & $     -5164.28$ & $     -5164.28$ & $     -5164.28$ & $     -5164.28$ & $     -5164.28$ & $         0.13$ sec    & $       0.8748$  & $       0.9685$ \\ 
                  HC & $     -4775.56$ & $     -4775.56$ & $     -4775.56$ & $     -4775.56$ & $     -4775.56$ & $     -4775.56$ & $     -4775.56$ & $     -4775.56$ & $         0.00$ sec    & $       1.5805$  & $       0.8975$ \\ 
              HC-CGC & $     -5164.28$ & $     -5164.28$ & $     -5164.28$ & $     -5164.28$ & $     -5164.28$ & $     -5164.28$ & $     -5164.28$ & $     -5164.28$ & $         0.06$ sec    & $       0.8748$  & $       0.9685$ \\ 
              ogm-KL & $     -5031.51$ & $     -5031.51$ & $     -5031.51$ & $     -5031.51$ & $     -5031.51$ & $     -5031.51$ & $     -5031.51$ & $     -5031.51$ & $         0.06$ sec    & $       2.3828$  & $       0.8699$ \\ 
    CC-Fusion-HC-CGC & $     -5198.37$ & $     -5198.37$ & $     -5198.37$ & $     -5198.37$ & $     -5198.37$ & $     -5198.37$ & $     -5198.37$ & $     -5198.37$ & $         0.42$ sec    & $       0.8916$  & $       0.9765$ \\ 
     CC-Fusion-HC-MC & $     -5198.37$ & $     -5198.37$ & $     -5198.37$ & $     -5198.37$ & $     -5198.37$ & $     -5198.37$ & $     -5198.37$ & $     -5198.37$ & $         1.37$ sec    & $       0.8916$  & $       0.9765$ \\ 
    CC-Fusion-WS-CGC & $     -5175.25$ & $     -5175.25$ & $     -5175.25$ & $     -5175.25$ & $     -5175.25$ & $     -5175.25$ & $     -5175.25$ & $     -5175.25$ & $         0.54$ sec    & $       1.0123$  & $       0.9674$ \\ 
     CC-Fusion-WS-MC & $     -5194.89$ & $     -5198.37$ & $     -5198.37$ & $     -5198.37$ & $     -5198.37$ & $     -5198.37$ & $     -5198.37$ & $     -5198.37$ & $         2.92$ sec    & $       0.8916$  & $       0.9765$ \\ 
\cmidrule{1-1} 
           MCR-CCFDB & $     -5195.89$ & $     -5195.89$ & $     -5195.89$ & $     -5195.89$ & $     -5195.89$ & $     -5195.89$ & $     -5195.89$ & $     -5195.89$ & $         0.28$ sec    & $       0.8913$  & $       0.9765$ \\ 
\cmidrule{1-1} 
           MCI-CCIFD & $     -5198.37$ & $     -5198.37$ & $     -5198.37$ & $     -5198.37$ & $     -5198.37$ & $     -5198.37$ & $     -5198.37$ & $     -5198.37$ & $         0.31$ sec    & $       0.8916$  & $       0.9765$ \\ 
\bottomrule
\end{tabular}
\end{table}

\begin{table}[H]
\scriptsize
\centering
\caption{knott-3d-150 (gm\_knott\_3d\_037)}
\label{tab:anytimetable-knott-3d-150-gm-knott-3d-037}
\begin{tabular}{lrrrrrrrrrrr}
\toprule
           algorithm &                                   \multicolumn{8}{c}{value} & \multicolumn{1}{c}{time}    & \multicolumn{1}{c}{VI}  & \multicolumn{1}{c}{RI} \\  
\cmidrule(lr){2-9}\cmidrule(lr){10-10} \cmidrule(lr){11-11} \cmidrule(lr){12-12}   
                     & \multicolumn{1}{c}{(0.5 sec)} & \multicolumn{1}{c}{(1 sec)} & \multicolumn{1}{c}{(10 sec)} & \multicolumn{1}{c}{(60 sec)} & \multicolumn{1}{c}{(300 sec)} & \multicolumn{1}{c}{(600 sec)} & \multicolumn{1}{c}{(1800 sec)} & \multicolumn{1}{c}{(end)} & \multicolumn{1}{c}{(end)}    & \multicolumn{1}{c}{(end)}   & \multicolumn{1}{c}{(end)}  \\ \midrule 
          PIVIT-BOEM & $\infty$ & $\infty$ & $     -2757.98$ & $     -2757.98$ & $     -2757.98$ & $     -2757.98$ & $     -2757.98$ & $     -2757.98$ & $         1.24$ sec    & $       2.6433$  & $       0.8731$ \\ 
                 CGC & $     -4632.24$ & $     -4632.24$ & $     -4632.24$ & $     -4632.24$ & $     -4632.24$ & $     -4632.24$ & $     -4632.24$ & $     -4632.24$ & $         0.08$ sec    & $       1.0475$  & $       0.9561$ \\ 
                  HC & $     -4075.88$ & $     -4075.88$ & $     -4075.88$ & $     -4075.88$ & $     -4075.88$ & $     -4075.88$ & $     -4075.88$ & $     -4075.88$ & $         0.00$ sec    & $       1.5888$  & $       0.8219$ \\ 
              HC-CGC & $     -4638.20$ & $     -4638.20$ & $     -4638.20$ & $     -4638.20$ & $     -4638.20$ & $     -4638.20$ & $     -4638.20$ & $     -4638.20$ & $         0.06$ sec    & $       0.8670$  & $       0.9720$ \\ 
              ogm-KL & $     -4501.88$ & $     -4501.88$ & $     -4501.88$ & $     -4501.88$ & $     -4501.88$ & $     -4501.88$ & $     -4501.88$ & $     -4501.88$ & $         0.05$ sec    & $       2.1186$  & $       0.8513$ \\ 
    CC-Fusion-HC-CGC & $     -4625.96$ & $     -4634.74$ & $     -4634.74$ & $     -4634.74$ & $     -4634.74$ & $     -4634.74$ & $     -4634.74$ & $     -4634.74$ & $         0.92$ sec    & $       0.8466$  & $       0.9721$ \\ 
     CC-Fusion-HC-MC & $     -4635.18$ & $     -4638.99$ & $     -4638.99$ & $     -4638.99$ & $     -4638.99$ & $     -4638.99$ & $     -4638.99$ & $     -4638.99$ & $         3.28$ sec    & $       0.8686$  & $       0.9720$ \\ 
    CC-Fusion-WS-CGC & $     -4617.07$ & $     -4617.07$ & $     -4617.07$ & $     -4617.07$ & $     -4617.07$ & $     -4617.07$ & $     -4617.07$ & $     -4617.07$ & $         0.44$ sec    & $       1.0508$  & $       0.9466$ \\ 
     CC-Fusion-WS-MC & $     -4614.62$ & $     -4633.75$ & $     -4638.99$ & $     -4638.99$ & $     -4638.99$ & $     -4638.99$ & $     -4638.99$ & $     -4638.99$ & $         7.24$ sec    & $       0.8686$  & $       0.9720$ \\ 
\cmidrule{1-1} 
           MCR-CCFDB & $     -4176.13$ & $     -4630.01$ & $     -4630.01$ & $     -4630.01$ & $     -4630.01$ & $     -4630.01$ & $     -4630.01$ & $     -4630.01$ & $         0.60$ sec    & $       0.8744$  & $       0.9719$ \\ 
\cmidrule{1-1} 
           MCI-CCIFD & $     -4491.38$ & $     -4638.99$ & $     -4638.99$ & $     -4638.99$ & $     -4638.99$ & $     -4638.99$ & $     -4638.99$ & $     -4638.99$ & $         0.78$ sec    & $       0.8686$  & $       0.9720$ \\ 
\bottomrule
\end{tabular}
\end{table}

\begin{table}[H]
\scriptsize
\centering
\caption{knott-3d-150 (gm\_knott\_3d\_038)}
\label{tab:anytimetable-knott-3d-150-gm-knott-3d-038}
\begin{tabular}{lrrrrrrrrrrr}
\toprule
           algorithm &                                   \multicolumn{8}{c}{value} & \multicolumn{1}{c}{time}    & \multicolumn{1}{c}{VI}  & \multicolumn{1}{c}{RI} \\  
\cmidrule(lr){2-9}\cmidrule(lr){10-10} \cmidrule(lr){11-11} \cmidrule(lr){12-12}   
                     & \multicolumn{1}{c}{(0.5 sec)} & \multicolumn{1}{c}{(1 sec)} & \multicolumn{1}{c}{(10 sec)} & \multicolumn{1}{c}{(60 sec)} & \multicolumn{1}{c}{(300 sec)} & \multicolumn{1}{c}{(600 sec)} & \multicolumn{1}{c}{(1800 sec)} & \multicolumn{1}{c}{(end)} & \multicolumn{1}{c}{(end)}    & \multicolumn{1}{c}{(end)}   & \multicolumn{1}{c}{(end)}  \\ \midrule 
          PIVIT-BOEM & $\infty$ & $\infty$ & $       208.62$ & $       208.62$ & $       208.62$ & $       208.62$ & $       208.62$ & $       208.62$ & $         2.83$ sec    & $       2.5622$  & $       0.8906$ \\ 
                 CGC & $     -4625.80$ & $     -4625.80$ & $     -4625.80$ & $     -4625.80$ & $     -4625.80$ & $     -4625.80$ & $     -4625.80$ & $     -4625.80$ & $         0.08$ sec    & $       1.0301$  & $       0.8802$ \\ 
                  HC & $     -4248.19$ & $     -4248.19$ & $     -4248.19$ & $     -4248.19$ & $     -4248.19$ & $     -4248.19$ & $     -4248.19$ & $     -4248.19$ & $         0.01$ sec    & $       1.5206$  & $       0.8264$ \\ 
              HC-CGC & $     -4625.80$ & $     -4625.80$ & $     -4625.80$ & $     -4625.80$ & $     -4625.80$ & $     -4625.80$ & $     -4625.80$ & $     -4625.80$ & $         0.06$ sec    & $       1.0301$  & $       0.8802$ \\ 
              ogm-KL & $     -4459.46$ & $     -4459.46$ & $     -4459.46$ & $     -4459.46$ & $     -4459.46$ & $     -4459.46$ & $     -4459.46$ & $     -4459.46$ & $         0.18$ sec    & $       2.0392$  & $       0.7915$ \\ 
    CC-Fusion-HC-CGC & $     -4624.74$ & $     -4624.74$ & $     -4624.74$ & $     -4624.74$ & $     -4624.74$ & $     -4624.74$ & $     -4624.74$ & $     -4624.74$ & $         0.61$ sec    & $       1.0099$  & $       0.8770$ \\ 
     CC-Fusion-HC-MC & $     -4625.80$ & $     -4625.80$ & $     -4625.80$ & $     -4625.80$ & $     -4625.80$ & $     -4625.80$ & $     -4625.80$ & $     -4625.80$ & $         1.12$ sec    & $       1.0301$  & $       0.8802$ \\ 
    CC-Fusion-WS-CGC & $     -4611.47$ & $     -4611.47$ & $     -4611.47$ & $     -4611.47$ & $     -4611.47$ & $     -4611.47$ & $     -4611.47$ & $     -4611.47$ & $         0.51$ sec    & $       1.0298$  & $       0.8802$ \\ 
     CC-Fusion-WS-MC & $     -4625.80$ & $     -4625.80$ & $     -4625.80$ & $     -4625.80$ & $     -4625.80$ & $     -4625.80$ & $     -4625.80$ & $     -4625.80$ & $         1.57$ sec    & $       1.0301$  & $       0.8802$ \\ 
\cmidrule{1-1} 
           MCR-CCFDB & $     -4625.80$ & $     -4625.80$ & $     -4625.80$ & $     -4625.80$ & $     -4625.80$ & $     -4625.80$ & $     -4625.80$ & $     -4625.80$ & $         0.38$ sec    & $       1.0301$  & $       0.8802$ \\ 
\cmidrule{1-1} 
           MCI-CCIFD & $     -4625.80$ & $     -4625.80$ & $     -4625.80$ & $     -4625.80$ & $     -4625.80$ & $     -4625.80$ & $     -4625.80$ & $     -4625.80$ & $         0.20$ sec    & $       1.0301$  & $       0.8802$ \\ 
\bottomrule
\end{tabular}
\end{table}

\begin{table}[H]
\scriptsize
\centering
\caption{knott-3d-150 (gm\_knott\_3d\_039)}
\label{tab:anytimetable-knott-3d-150-gm-knott-3d-039}
\begin{tabular}{lrrrrrrrrrrr}
\toprule
           algorithm &                                   \multicolumn{8}{c}{value} & \multicolumn{1}{c}{time}    & \multicolumn{1}{c}{VI}  & \multicolumn{1}{c}{RI} \\  
\cmidrule(lr){2-9}\cmidrule(lr){10-10} \cmidrule(lr){11-11} \cmidrule(lr){12-12}   
                     & \multicolumn{1}{c}{(0.5 sec)} & \multicolumn{1}{c}{(1 sec)} & \multicolumn{1}{c}{(10 sec)} & \multicolumn{1}{c}{(60 sec)} & \multicolumn{1}{c}{(300 sec)} & \multicolumn{1}{c}{(600 sec)} & \multicolumn{1}{c}{(1800 sec)} & \multicolumn{1}{c}{(end)} & \multicolumn{1}{c}{(end)}    & \multicolumn{1}{c}{(end)}   & \multicolumn{1}{c}{(end)}  \\ \midrule 
          PIVIT-BOEM & $\infty$ & $\infty$ & $     -3709.96$ & $     -3709.96$ & $     -3709.96$ & $     -3709.96$ & $     -3709.96$ & $     -3709.96$ & $         1.02$ sec    & $       2.3966$  & $       0.9100$ \\ 
                 CGC & $     -5092.32$ & $     -5092.32$ & $     -5092.32$ & $     -5092.32$ & $     -5092.32$ & $     -5092.32$ & $     -5092.32$ & $     -5092.32$ & $         0.06$ sec    & $       1.2590$  & $       0.9527$ \\ 
                  HC & $     -4760.80$ & $     -4760.80$ & $     -4760.80$ & $     -4760.80$ & $     -4760.80$ & $     -4760.80$ & $     -4760.80$ & $     -4760.80$ & $         0.00$ sec    & $       1.6742$  & $       0.9167$ \\ 
              HC-CGC & $     -5089.17$ & $     -5089.17$ & $     -5089.17$ & $     -5089.17$ & $     -5089.17$ & $     -5089.17$ & $     -5089.17$ & $     -5089.17$ & $         0.03$ sec    & $       1.2679$  & $       0.9523$ \\ 
              ogm-KL & $     -4992.50$ & $     -4992.50$ & $     -4992.50$ & $     -4992.50$ & $     -4992.50$ & $     -4992.50$ & $     -4992.50$ & $     -4992.50$ & $         0.07$ sec    & $       2.7593$  & $       0.8343$ \\ 
    CC-Fusion-HC-CGC & $     -5083.60$ & $     -5083.60$ & $     -5083.60$ & $     -5083.60$ & $     -5083.60$ & $     -5083.60$ & $     -5083.60$ & $     -5083.60$ & $         0.68$ sec    & $       1.1245$  & $       0.9623$ \\ 
     CC-Fusion-HC-MC & $     -5086.64$ & $     -5087.58$ & $     -5087.58$ & $     -5087.58$ & $     -5087.58$ & $     -5087.58$ & $     -5087.58$ & $     -5087.58$ & $         1.93$ sec    & $       1.0425$  & $       0.9683$ \\ 
    CC-Fusion-WS-CGC & $     -5069.29$ & $     -5069.29$ & $     -5069.29$ & $     -5069.29$ & $     -5069.29$ & $     -5069.29$ & $     -5069.29$ & $     -5069.29$ & $         0.36$ sec    & $       1.2998$  & $       0.9501$ \\ 
     CC-Fusion-WS-MC & $     -5084.84$ & $     -5087.58$ & $     -5087.58$ & $     -5087.58$ & $     -5087.58$ & $     -5087.58$ & $     -5087.58$ & $     -5087.58$ & $         2.65$ sec    & $       1.0425$  & $       0.9683$ \\ 
\cmidrule{1-1} 
           MCR-CCFDB & $     -5087.40$ & $     -5087.40$ & $     -5087.40$ & $     -5087.40$ & $     -5087.40$ & $     -5087.40$ & $     -5087.40$ & $     -5087.40$ & $         0.28$ sec    & $       1.2619$  & $       0.9527$ \\ 
\cmidrule{1-1} 
           MCI-CCIFD & $     -4676.13$ & $     -5092.32$ & $     -5092.32$ & $     -5092.32$ & $     -5092.32$ & $     -5092.32$ & $     -5092.32$ & $     -5092.32$ & $         0.63$ sec    & $       1.2590$  & $       0.9527$ \\ 
\bottomrule
\end{tabular}
\end{table}


\subsection{knott-3d-300}
\begin{table}[H]
\scriptsize
\centering
\caption{knott-3d-300 (gm\_knott\_3d\_072)}
\label{tab:anytimetable-knott-3d-300-gm-knott-3d-072}
\begin{tabular}{lrrrrrrrrrrr}
\toprule
           algorithm &                                   \multicolumn{8}{c}{value} & \multicolumn{1}{c}{time}    & \multicolumn{1}{c}{VI}  & \multicolumn{1}{c}{RI} \\  
\cmidrule(lr){2-9}\cmidrule(lr){10-10} \cmidrule(lr){11-11} \cmidrule(lr){12-12}   
                     & \multicolumn{1}{c}{(0.5 sec)} & \multicolumn{1}{c}{(1 sec)} & \multicolumn{1}{c}{(10 sec)} & \multicolumn{1}{c}{(60 sec)} & \multicolumn{1}{c}{(300 sec)} & \multicolumn{1}{c}{(600 sec)} & \multicolumn{1}{c}{(1800 sec)} & \multicolumn{1}{c}{(end)} & \multicolumn{1}{c}{(end)}    & \multicolumn{1}{c}{(end)}   & \multicolumn{1}{c}{(end)}  \\ \midrule 
          PIVIT-BOEM & $\infty$ & $\infty$ & $\infty$ & $\infty$ & $\infty$ & $\infty$ & $    -15824.36$ & $    -15824.36$ & $       643.37$ sec    & $       3.8676$  & $       0.9434$ \\ 
                 CGC & $         0.00$ & $         0.00$ & $    -32907.64$ & $    -32907.64$ & $    -32907.64$ & $    -32907.64$ & $    -32907.64$ & $    -32907.64$ & $         8.39$ sec    & $       1.6360$  & $       0.9387$ \\ 
                  HC & $    -29619.84$ & $    -29619.84$ & $    -29619.84$ & $    -29619.84$ & $    -29619.84$ & $    -29619.84$ & $    -29619.84$ & $    -29619.84$ & $         0.05$ sec    & $       2.4918$  & $       0.8962$ \\ 
              HC-CGC & $    -32209.61$ & $    -32507.35$ & $    -32863.16$ & $    -32863.16$ & $    -32863.16$ & $    -32863.16$ & $    -32863.16$ & $    -32863.16$ & $         3.19$ sec    & $       1.7597$  & $       0.9372$ \\ 
              ogm-KL & $     -3796.59$ & $     -3796.59$ & $    -30283.52$ & $    -30354.28$ & $    -30354.28$ & $    -30354.28$ & $    -30354.28$ & $    -30354.28$ & $        17.11$ sec    & $       4.8809$  & $       0.7323$ \\ 
    CC-Fusion-HC-CGC & $    -31890.58$ & $    -32029.55$ & $    -32633.04$ & $    -32750.13$ & $    -32750.13$ & $    -32750.13$ & $    -32750.13$ & $    -32750.13$ & $        14.95$ sec    & $       1.7279$  & $       0.9344$ \\ 
     CC-Fusion-HC-MC & $    -32176.66$ & $    -32881.78$ & $    -32916.63$ & $    -32916.63$ & $    -32916.63$ & $    -32916.63$ & $    -32916.63$ & $    -32916.63$ & $         9.94$ sec    & $       1.6595$  & $       0.9389$ \\ 
    CC-Fusion-WS-CGC & $    -29574.18$ & $    -31429.37$ & $    -31638.36$ & $    -31638.36$ & $    -31638.36$ & $    -31638.36$ & $    -31638.36$ & $    -31638.36$ & $        18.60$ sec    & $       2.2600$  & $       0.9069$ \\ 
     CC-Fusion-WS-MC & $    -25944.91$ & $    -30651.55$ & $    -32924.46$ & $    -32924.46$ & $    -32924.46$ & $    -32924.46$ & $    -32924.46$ & $    -32924.46$ & $        29.62$ sec    & $       1.6649$  & $       0.9383$ \\ 
\cmidrule{1-1} 
           MCR-CCFDB & $     -3796.59$ & $     -3796.59$ & $     -3796.59$ & $    -28269.86$ & $    -32987.47$ & $    -32987.47$ & $    -32987.47$ & $    -32987.47$ & $        79.64$ sec    & $       1.5211$  & $       0.9565$ \\ 
\cmidrule{1-1} 
           MCI-CCIFD & $     -3796.59$ & $     -3796.59$ & $    -30703.92$ & $    -32999.85$ & $    -32999.85$ & $    -32999.85$ & $    -32999.85$ & $    -32999.85$ & $        14.25$ sec    & $       1.5202$  & $       0.9564$ \\ 
\bottomrule
\end{tabular}
\end{table}

\begin{table}[H]
\scriptsize
\centering
\caption{knott-3d-300 (gm\_knott\_3d\_073)}
\label{tab:anytimetable-knott-3d-300-gm-knott-3d-073}
\begin{tabular}{lrrrrrrrrrrr}
\toprule
           algorithm &                                   \multicolumn{8}{c}{value} & \multicolumn{1}{c}{time}    & \multicolumn{1}{c}{VI}  & \multicolumn{1}{c}{RI} \\  
\cmidrule(lr){2-9}\cmidrule(lr){10-10} \cmidrule(lr){11-11} \cmidrule(lr){12-12}   
                     & \multicolumn{1}{c}{(0.5 sec)} & \multicolumn{1}{c}{(1 sec)} & \multicolumn{1}{c}{(10 sec)} & \multicolumn{1}{c}{(60 sec)} & \multicolumn{1}{c}{(300 sec)} & \multicolumn{1}{c}{(600 sec)} & \multicolumn{1}{c}{(1800 sec)} & \multicolumn{1}{c}{(end)} & \multicolumn{1}{c}{(end)}    & \multicolumn{1}{c}{(end)}   & \multicolumn{1}{c}{(end)}  \\ \midrule 
          PIVIT-BOEM & $\infty$ & $\infty$ & $\infty$ & $\infty$ & $\infty$ & $\infty$ & $      3494.53$ & $      3494.53$ & $       958.48$ sec    & $       4.7741$  & $       0.8634$ \\ 
                 CGC & $         0.00$ & $         0.00$ & $    -23433.69$ & $    -23433.69$ & $    -23433.69$ & $    -23433.69$ & $    -23433.69$ & $    -23433.69$ & $         1.41$ sec    & $       2.4793$  & $       0.8337$ \\ 
                  HC & $    -21636.83$ & $    -21636.83$ & $    -21636.83$ & $    -21636.83$ & $    -21636.83$ & $    -21636.83$ & $    -21636.83$ & $    -21636.83$ & $         0.06$ sec    & $       1.9538$  & $       0.8070$ \\ 
              HC-CGC & $    -24496.88$ & $    -24828.93$ & $    -25861.17$ & $    -25861.17$ & $    -25861.17$ & $    -25861.17$ & $    -25861.17$ & $    -25861.17$ & $         2.87$ sec    & $       1.4560$  & $       0.8758$ \\ 
              ogm-KL & $     -1556.16$ & $     -1556.16$ & $    -24366.05$ & $    -24368.30$ & $    -24368.30$ & $    -24368.30$ & $    -24368.30$ & $    -24368.30$ & $        13.87$ sec    & $       3.7796$  & $       0.6611$ \\ 
    CC-Fusion-HC-CGC & $    -24660.17$ & $    -25054.07$ & $    -25555.66$ & $    -25555.66$ & $    -25555.66$ & $    -25555.66$ & $    -25555.66$ & $    -25555.66$ & $        12.35$ sec    & $       1.5401$  & $       0.8750$ \\ 
     CC-Fusion-HC-MC & $    -24403.93$ & $    -25111.58$ & $    -25745.16$ & $    -25863.38$ & $    -25863.38$ & $    -25863.38$ & $    -25863.38$ & $    -25863.38$ & $        35.72$ sec    & $       1.4370$  & $       0.8777$ \\ 
    CC-Fusion-WS-CGC & $    -24309.78$ & $    -24309.78$ & $    -24858.58$ & $    -24858.58$ & $    -24858.58$ & $    -24858.58$ & $    -24858.58$ & $    -24858.58$ & $        10.84$ sec    & $       1.8925$  & $       0.8489$ \\ 
     CC-Fusion-WS-MC & $    -21012.24$ & $    -24202.37$ & $    -25800.80$ & $    -25848.08$ & $    -25848.08$ & $    -25848.08$ & $    -25848.08$ & $    -25848.08$ & $        39.36$ sec    & $       1.4371$  & $       0.8777$ \\ 
\cmidrule{1-1} 
           MCR-CCFDB & $     -1556.16$ & $     -1556.16$ & $     -1556.16$ & $    -16078.57$ & $    -25849.06$ & $    -25849.06$ & $    -25849.06$ & $    -25849.06$ & $       107.97$ sec    & $       1.4309$  & $       0.8781$ \\ 
\cmidrule{1-1} 
           MCI-CCIFD & $     -1556.16$ & $     -1556.16$ & $    -25287.81$ & $    -25863.38$ & $    -25863.38$ & $    -25863.38$ & $    -25863.38$ & $    -25863.38$ & $        11.44$ sec    & $       1.4370$  & $       0.8777$ \\ 
\bottomrule
\end{tabular}
\end{table}

\begin{table}[H]
\scriptsize
\centering
\caption{knott-3d-300 (gm\_knott\_3d\_074)}
\label{tab:anytimetable-knott-3d-300-gm-knott-3d-074}
\begin{tabular}{lrrrrrrrrrrr}
\toprule
           algorithm &                                   \multicolumn{8}{c}{value} & \multicolumn{1}{c}{time}    & \multicolumn{1}{c}{VI}  & \multicolumn{1}{c}{RI} \\  
\cmidrule(lr){2-9}\cmidrule(lr){10-10} \cmidrule(lr){11-11} \cmidrule(lr){12-12}   
                     & \multicolumn{1}{c}{(0.5 sec)} & \multicolumn{1}{c}{(1 sec)} & \multicolumn{1}{c}{(10 sec)} & \multicolumn{1}{c}{(60 sec)} & \multicolumn{1}{c}{(300 sec)} & \multicolumn{1}{c}{(600 sec)} & \multicolumn{1}{c}{(1800 sec)} & \multicolumn{1}{c}{(end)} & \multicolumn{1}{c}{(end)}    & \multicolumn{1}{c}{(end)}   & \multicolumn{1}{c}{(end)}  \\ \midrule 
          PIVIT-BOEM & $\infty$ & $\infty$ & $\infty$ & $\infty$ & $\infty$ & $\infty$ & $     13713.62$ & $     13713.62$ & $      1518.59$ sec    & $       4.9831$  & $       0.8227$ \\ 
                 CGC & $         0.00$ & $    -23135.45$ & $    -25689.42$ & $    -25689.42$ & $    -25689.42$ & $    -25689.42$ & $    -25689.42$ & $    -25689.42$ & $         4.69$ sec    & $       1.3949$  & $       0.9111$ \\ 
                  HC & $    -21831.32$ & $    -21831.32$ & $    -21831.32$ & $    -21831.32$ & $    -21831.32$ & $    -21831.32$ & $    -21831.32$ & $    -21831.32$ & $         0.07$ sec    & $       2.1471$  & $       0.7991$ \\ 
              HC-CGC & $    -24285.59$ & $    -25015.57$ & $    -25691.16$ & $    -25691.16$ & $    -25691.16$ & $    -25691.16$ & $    -25691.16$ & $    -25691.16$ & $         2.34$ sec    & $       1.3761$  & $       0.9119$ \\ 
              ogm-KL & $     -1550.70$ & $     -1550.70$ & $    -24026.44$ & $    -24035.85$ & $    -24035.85$ & $    -24035.85$ & $    -24035.85$ & $    -24035.85$ & $        13.30$ sec    & $       3.7416$  & $       0.6808$ \\ 
    CC-Fusion-HC-CGC & $    -25256.68$ & $    -25256.68$ & $    -25430.22$ & $    -25430.22$ & $    -25430.22$ & $    -25430.22$ & $    -25430.22$ & $    -25430.22$ & $         8.98$ sec    & $       1.6724$  & $       0.8892$ \\ 
     CC-Fusion-HC-MC & $    -24739.46$ & $    -25545.78$ & $    -25705.79$ & $    -25705.79$ & $    -25705.79$ & $    -25705.79$ & $    -25705.79$ & $    -25705.79$ & $        16.07$ sec    & $       1.3558$  & $       0.9122$ \\ 
    CC-Fusion-WS-CGC & $    -24604.49$ & $    -24715.52$ & $    -25092.31$ & $    -25092.31$ & $    -25092.31$ & $    -25092.31$ & $    -25092.31$ & $    -25092.31$ & $         7.13$ sec    & $       1.9541$  & $       0.8817$ \\ 
     CC-Fusion-WS-MC & $    -21375.93$ & $    -21375.93$ & $    -25629.38$ & $    -25632.75$ & $    -25632.75$ & $    -25632.75$ & $    -25632.75$ & $    -25632.75$ & $        65.09$ sec    & $       1.5598$  & $       0.8912$ \\ 
\cmidrule{1-1} 
           MCR-CCFDB & $     -1550.70$ & $     -1550.70$ & $     -1550.70$ & $    -14964.28$ & $    -25716.21$ & $    -25716.21$ & $    -25716.21$ & $    -25716.21$ & $       167.49$ sec    & $       1.3924$  & $       0.9099$ \\ 
\cmidrule{1-1} 
           MCI-CCIFD & $     -1550.70$ & $     -1550.70$ & $    -19558.91$ & $    -25721.90$ & $    -25721.90$ & $    -25721.90$ & $    -25721.90$ & $    -25721.90$ & $        14.92$ sec    & $       1.3921$  & $       0.9099$ \\ 
\bottomrule
\end{tabular}
\end{table}

\begin{table}[H]
\scriptsize
\centering
\caption{knott-3d-300 (gm\_knott\_3d\_075)}
\label{tab:anytimetable-knott-3d-300-gm-knott-3d-075}
\begin{tabular}{lrrrrrrrrrrr}
\toprule
           algorithm &                                   \multicolumn{8}{c}{value} & \multicolumn{1}{c}{time}    & \multicolumn{1}{c}{VI}  & \multicolumn{1}{c}{RI} \\  
\cmidrule(lr){2-9}\cmidrule(lr){10-10} \cmidrule(lr){11-11} \cmidrule(lr){12-12}   
                     & \multicolumn{1}{c}{(0.5 sec)} & \multicolumn{1}{c}{(1 sec)} & \multicolumn{1}{c}{(10 sec)} & \multicolumn{1}{c}{(60 sec)} & \multicolumn{1}{c}{(300 sec)} & \multicolumn{1}{c}{(600 sec)} & \multicolumn{1}{c}{(1800 sec)} & \multicolumn{1}{c}{(end)} & \multicolumn{1}{c}{(end)}    & \multicolumn{1}{c}{(end)}   & \multicolumn{1}{c}{(end)}  \\ \midrule 
          PIVIT-BOEM & $\infty$ & $\infty$ & $\infty$ & $\infty$ & $\infty$ & $\infty$ & $      3514.30$ & $      3514.30$ & $      1155.90$ sec    & $       4.2488$  & $       0.9171$ \\ 
                 CGC & $         0.00$ & $         0.00$ & $    -30388.15$ & $    -30388.15$ & $    -30388.15$ & $    -30388.15$ & $    -30388.15$ & $    -30388.15$ & $         5.73$ sec    & $       1.9506$  & $       0.8727$ \\ 
                  HC & $    -28478.32$ & $    -28478.32$ & $    -28478.32$ & $    -28478.32$ & $    -28478.32$ & $    -28478.32$ & $    -28478.32$ & $    -28478.32$ & $         0.07$ sec    & $       2.2776$  & $       0.8459$ \\ 
              HC-CGC & $    -30091.80$ & $    -30355.71$ & $    -30466.35$ & $    -30466.35$ & $    -30466.35$ & $    -30466.35$ & $    -30466.35$ & $    -30466.35$ & $         2.02$ sec    & $       1.7201$  & $       0.8887$ \\ 
              ogm-KL & $     -2669.63$ & $     -2669.63$ & $    -28814.26$ & $    -28815.15$ & $    -28815.15$ & $    -28815.15$ & $    -28815.15$ & $    -28815.15$ & $        14.57$ sec    & $       4.0892$  & $       0.6859$ \\ 
    CC-Fusion-HC-CGC & $    -29972.99$ & $    -30025.17$ & $    -30283.32$ & $    -30283.32$ & $    -30283.32$ & $    -30283.32$ & $    -30283.32$ & $    -30283.32$ & $         5.25$ sec    & $       1.7392$  & $       0.8919$ \\ 
     CC-Fusion-HC-MC & $    -30394.93$ & $    -30470.29$ & $    -30474.44$ & $    -30474.44$ & $    -30474.44$ & $    -30474.44$ & $    -30474.44$ & $    -30474.44$ & $        12.51$ sec    & $       1.6330$  & $       0.8899$ \\ 
    CC-Fusion-WS-CGC & $    -28168.02$ & $    -28488.09$ & $    -29623.75$ & $    -29623.75$ & $    -29623.75$ & $    -29623.75$ & $    -29623.75$ & $    -29623.75$ & $        12.24$ sec    & $       2.0073$  & $       0.8872$ \\ 
     CC-Fusion-WS-MC & $    -25931.41$ & $    -29612.53$ & $    -30478.37$ & $    -30478.37$ & $    -30478.37$ & $    -30478.37$ & $    -30478.37$ & $    -30478.37$ & $        18.53$ sec    & $       1.6331$  & $       0.8901$ \\ 
\cmidrule{1-1} 
           MCR-CCFDB & $     -2669.63$ & $     -2669.63$ & $     -2669.63$ & $    -24694.95$ & $    -30478.37$ & $    -30478.37$ & $    -30478.37$ & $    -30478.37$ & $       102.04$ sec    & $       1.6331$  & $       0.8901$ \\ 
\cmidrule{1-1} 
           MCI-CCIFD & $     -2669.63$ & $     -2669.63$ & $    -30478.37$ & $    -30478.37$ & $    -30478.37$ & $    -30478.37$ & $    -30478.37$ & $    -30478.37$ & $         7.47$ sec    & $       1.6331$  & $       0.8901$ \\ 
\bottomrule
\end{tabular}
\end{table}

\begin{table}[H]
\scriptsize
\centering
\caption{knott-3d-300 (gm\_knott\_3d\_076)}
\label{tab:anytimetable-knott-3d-300-gm-knott-3d-076}
\begin{tabular}{lrrrrrrrrrrr}
\toprule
           algorithm &                                   \multicolumn{8}{c}{value} & \multicolumn{1}{c}{time}    & \multicolumn{1}{c}{VI}  & \multicolumn{1}{c}{RI} \\  
\cmidrule(lr){2-9}\cmidrule(lr){10-10} \cmidrule(lr){11-11} \cmidrule(lr){12-12}   
                     & \multicolumn{1}{c}{(0.5 sec)} & \multicolumn{1}{c}{(1 sec)} & \multicolumn{1}{c}{(10 sec)} & \multicolumn{1}{c}{(60 sec)} & \multicolumn{1}{c}{(300 sec)} & \multicolumn{1}{c}{(600 sec)} & \multicolumn{1}{c}{(1800 sec)} & \multicolumn{1}{c}{(end)} & \multicolumn{1}{c}{(end)}    & \multicolumn{1}{c}{(end)}   & \multicolumn{1}{c}{(end)}  \\ \midrule 
          PIVIT-BOEM & $\infty$ & $\infty$ & $\infty$ & $\infty$ & $\infty$ & $\infty$ & $     -1991.22$ & $     -1991.22$ & $       757.69$ sec    & $       4.2369$  & $       0.9049$ \\ 
                 CGC & $         0.00$ & $         0.00$ & $    -27018.86$ & $    -27018.86$ & $    -27018.86$ & $    -27018.86$ & $    -27018.86$ & $    -27018.86$ & $         3.94$ sec    & $       2.0554$  & $       0.8741$ \\ 
                  HC & $    -24454.42$ & $    -24454.42$ & $    -24454.42$ & $    -24454.42$ & $    -24454.42$ & $    -24454.42$ & $    -24454.42$ & $    -24454.42$ & $         0.06$ sec    & $       2.7923$  & $       0.7837$ \\ 
              HC-CGC & $    -26272.91$ & $    -26895.76$ & $    -27013.01$ & $    -27013.01$ & $    -27013.01$ & $    -27013.01$ & $    -27013.01$ & $    -27013.01$ & $         2.24$ sec    & $       2.1138$  & $       0.8742$ \\ 
              ogm-KL & $     -1597.05$ & $     -1597.05$ & $    -25246.64$ & $    -25246.64$ & $    -25246.64$ & $    -25246.64$ & $    -25246.64$ & $    -25246.64$ & $         9.99$ sec    & $       4.5420$  & $       0.6906$ \\ 
    CC-Fusion-HC-CGC & $    -25443.21$ & $    -25827.16$ & $    -26645.74$ & $    -26645.74$ & $    -26645.74$ & $    -26645.74$ & $    -26645.74$ & $    -26645.74$ & $         9.91$ sec    & $       2.0938$  & $       0.8808$ \\ 
     CC-Fusion-HC-MC & $    -26208.24$ & $    -26610.28$ & $    -27033.48$ & $    -27056.99$ & $    -27056.99$ & $    -27056.99$ & $    -27056.99$ & $    -27056.99$ & $        70.50$ sec    & $       1.9016$  & $       0.8930$ \\ 
    CC-Fusion-WS-CGC & $    -25690.02$ & $    -26025.28$ & $    -26082.69$ & $    -26195.11$ & $    -26195.11$ & $    -26195.11$ & $    -26195.11$ & $    -26195.11$ & $        24.82$ sec    & $       2.4063$  & $       0.8772$ \\ 
     CC-Fusion-WS-MC & $    -23127.04$ & $    -25844.82$ & $    -27050.13$ & $    -27056.71$ & $    -27056.71$ & $    -27056.71$ & $    -27056.71$ & $    -27056.71$ & $        57.05$ sec    & $       1.9016$  & $       0.8930$ \\ 
\cmidrule{1-1} 
           MCR-CCFDB & $     -1597.05$ & $     -1597.05$ & $     -3592.11$ & $    -26879.72$ & $    -27024.74$ & $    -27024.74$ & $    -27024.74$ & $    -27024.74$ & $        70.94$ sec    & $       1.9201$  & $       0.8929$ \\ 
\cmidrule{1-1} 
           MCI-CCIFD & $     -1597.05$ & $     -1597.05$ & $    -26995.91$ & $    -27056.99$ & $    -27056.99$ & $    -27056.99$ & $    -27056.99$ & $    -27056.99$ & $        12.64$ sec    & $       1.9016$  & $       0.8930$ \\ 
\bottomrule
\end{tabular}
\end{table}

\begin{table}[H]
\scriptsize
\centering
\caption{knott-3d-300 (gm\_knott\_3d\_077)}
\label{tab:anytimetable-knott-3d-300-gm-knott-3d-077}
\begin{tabular}{lrrrrrrrrrrr}
\toprule
           algorithm &                                   \multicolumn{8}{c}{value} & \multicolumn{1}{c}{time}    & \multicolumn{1}{c}{VI}  & \multicolumn{1}{c}{RI} \\  
\cmidrule(lr){2-9}\cmidrule(lr){10-10} \cmidrule(lr){11-11} \cmidrule(lr){12-12}   
                     & \multicolumn{1}{c}{(0.5 sec)} & \multicolumn{1}{c}{(1 sec)} & \multicolumn{1}{c}{(10 sec)} & \multicolumn{1}{c}{(60 sec)} & \multicolumn{1}{c}{(300 sec)} & \multicolumn{1}{c}{(600 sec)} & \multicolumn{1}{c}{(1800 sec)} & \multicolumn{1}{c}{(end)} & \multicolumn{1}{c}{(end)}    & \multicolumn{1}{c}{(end)}   & \multicolumn{1}{c}{(end)}  \\ \midrule 
          PIVIT-BOEM & $\infty$ & $\infty$ & $\infty$ & $\infty$ & $\infty$ & $\infty$ & $     -2184.98$ & $     -2184.98$ & $       961.44$ sec    & $       4.1355$  & $       0.9233$ \\ 
                 CGC & $         0.00$ & $         0.00$ & $    -29377.43$ & $    -29377.43$ & $    -29377.43$ & $    -29377.43$ & $    -29377.43$ & $    -29377.43$ & $         4.72$ sec    & $       1.7882$  & $       0.8959$ \\ 
                  HC & $    -27385.48$ & $    -27385.48$ & $    -27385.48$ & $    -27385.48$ & $    -27385.48$ & $    -27385.48$ & $    -27385.48$ & $    -27385.48$ & $         0.06$ sec    & $       2.4407$  & $       0.8374$ \\ 
              HC-CGC & $    -28787.89$ & $    -29167.29$ & $    -29418.86$ & $    -29418.86$ & $    -29418.86$ & $    -29418.86$ & $    -29418.86$ & $    -29418.86$ & $         2.34$ sec    & $       1.9373$  & $       0.8764$ \\ 
              ogm-KL & $     -1550.87$ & $     -1550.87$ & $    -27768.65$ & $    -27819.39$ & $    -27819.39$ & $    -27819.39$ & $    -27819.39$ & $    -27819.39$ & $        17.38$ sec    & $       4.5358$  & $       0.6849$ \\ 
    CC-Fusion-HC-CGC & $    -28593.45$ & $    -28739.32$ & $    -29206.10$ & $    -29206.10$ & $    -29206.10$ & $    -29206.10$ & $    -29206.10$ & $    -29206.10$ & $        12.14$ sec    & $       1.7146$  & $       0.9046$ \\ 
     CC-Fusion-HC-MC & $    -28309.14$ & $    -29389.26$ & $    -29482.24$ & $    -29482.24$ & $    -29482.24$ & $    -29482.24$ & $    -29482.24$ & $    -29482.24$ & $        21.47$ sec    & $       1.6059$  & $       0.9154$ \\ 
    CC-Fusion-WS-CGC & $    -28078.96$ & $    -28232.85$ & $    -28315.02$ & $    -28315.02$ & $    -28315.02$ & $    -28315.02$ & $    -28315.02$ & $    -28315.02$ & $        13.33$ sec    & $       2.3320$  & $       0.8843$ \\ 
     CC-Fusion-WS-MC & $    -23802.33$ & $    -27618.73$ & $    -29478.91$ & $    -29482.24$ & $    -29482.24$ & $    -29482.24$ & $    -29482.24$ & $    -29482.24$ & $        76.25$ sec    & $       1.6059$  & $       0.9154$ \\ 
\cmidrule{1-1} 
           MCR-CCFDB & $     -1550.87$ & $     -1550.87$ & $     -1550.87$ & $    -17822.19$ & $    -29481.84$ & $    -29481.84$ & $    -29481.84$ & $    -29481.84$ & $        94.45$ sec    & $       1.6057$  & $       0.9154$ \\ 
\cmidrule{1-1} 
           MCI-CCIFD & $     -1550.87$ & $     -1550.87$ & $    -28226.71$ & $    -29482.24$ & $    -29482.24$ & $    -29482.24$ & $    -29482.24$ & $    -29482.24$ & $        15.04$ sec    & $       1.6059$  & $       0.9154$ \\ 
\bottomrule
\end{tabular}
\end{table}

\begin{table}[H]
\scriptsize
\centering
\caption{knott-3d-300 (gm\_knott\_3d\_078)}
\label{tab:anytimetable-knott-3d-300-gm-knott-3d-078}
\begin{tabular}{lrrrrrrrrrrr}
\toprule
           algorithm &                                   \multicolumn{8}{c}{value} & \multicolumn{1}{c}{time}    & \multicolumn{1}{c}{VI}  & \multicolumn{1}{c}{RI} \\  
\cmidrule(lr){2-9}\cmidrule(lr){10-10} \cmidrule(lr){11-11} \cmidrule(lr){12-12}   
                     & \multicolumn{1}{c}{(0.5 sec)} & \multicolumn{1}{c}{(1 sec)} & \multicolumn{1}{c}{(10 sec)} & \multicolumn{1}{c}{(60 sec)} & \multicolumn{1}{c}{(300 sec)} & \multicolumn{1}{c}{(600 sec)} & \multicolumn{1}{c}{(1800 sec)} & \multicolumn{1}{c}{(end)} & \multicolumn{1}{c}{(end)}    & \multicolumn{1}{c}{(end)}   & \multicolumn{1}{c}{(end)}  \\ \midrule 
          PIVIT-BOEM & $\infty$ & $\infty$ & $\infty$ & $\infty$ & $\infty$ & $\infty$ & $\infty$ & $     32281.90$ & $      1954.57$ sec    & $       5.3523$  & $       0.8014$ \\ 
                 CGC & $         0.00$ & $    -19330.48$ & $    -20173.44$ & $    -20173.44$ & $    -20173.44$ & $    -20173.44$ & $    -20173.44$ & $    -20173.44$ & $         4.84$ sec    & $       1.9511$  & $       0.7098$ \\ 
                  HC & $    -17629.18$ & $    -17629.18$ & $    -17629.18$ & $    -17629.18$ & $    -17629.18$ & $    -17629.18$ & $    -17629.18$ & $    -17629.18$ & $         0.09$ sec    & $       2.2226$  & $       0.6629$ \\ 
              HC-CGC & $    -18987.65$ & $    -19090.50$ & $    -20174.59$ & $    -20174.59$ & $    -20174.59$ & $    -20174.59$ & $    -20174.59$ & $    -20174.59$ & $         4.29$ sec    & $       1.9433$  & $       0.7096$ \\ 
              ogm-KL & $     -1371.43$ & $     -1371.43$ & $    -19129.67$ & $    -19132.84$ & $    -19132.84$ & $    -19132.84$ & $    -19132.84$ & $    -19132.84$ & $        13.48$ sec    & $       3.1283$  & $       0.6367$ \\ 
    CC-Fusion-HC-CGC & $    -19858.08$ & $    -19876.27$ & $    -20026.57$ & $    -20090.87$ & $    -20090.87$ & $    -20090.87$ & $    -20090.87$ & $    -20090.87$ & $        21.11$ sec    & $       1.8059$  & $       0.7325$ \\ 
     CC-Fusion-HC-MC & $    -20061.19$ & $    -20123.90$ & $    -20211.32$ & $    -20211.32$ & $    -20211.32$ & $    -20211.32$ & $    -20211.32$ & $    -20211.32$ & $        16.91$ sec    & $       1.8338$  & $       0.7346$ \\ 
    CC-Fusion-WS-CGC & $    -19740.30$ & $    -19766.95$ & $    -19969.14$ & $    -19969.14$ & $    -19969.14$ & $    -19969.14$ & $    -19969.14$ & $    -19969.14$ & $        10.56$ sec    & $       2.0234$  & $       0.7087$ \\ 
     CC-Fusion-WS-MC & $    -16853.64$ & $    -16853.64$ & $    -20200.45$ & $    -20211.55$ & $    -20211.55$ & $    -20211.55$ & $    -20211.55$ & $    -20211.55$ & $        66.00$ sec    & $       1.8055$  & $       0.7386$ \\ 
\cmidrule{1-1} 
           MCR-CCFDB & $     -1371.43$ & $     -1371.43$ & $     -1371.43$ & $     -7662.94$ & $    -19737.03$ & $    -20200.94$ & $    -20200.94$ & $    -20200.94$ & $       491.72$ sec    & $       1.8238$  & $       0.7383$ \\ 
\cmidrule{1-1} 
           MCI-CCIFD & $     -1371.43$ & $     -1371.43$ & $    -13717.83$ & $    -20112.37$ & $    -20211.55$ & $    -20211.55$ & $    -20211.55$ & $    -20211.55$ & $       239.34$ sec    & $       1.8055$  & $       0.7386$ \\ 
\bottomrule
\end{tabular}
\end{table}

\begin{table}[H]
\scriptsize
\centering
\caption{knott-3d-300 (gm\_knott\_3d\_079)}
\label{tab:anytimetable-knott-3d-300-gm-knott-3d-079}
\begin{tabular}{lrrrrrrrrrrr}
\toprule
           algorithm &                                   \multicolumn{8}{c}{value} & \multicolumn{1}{c}{time}    & \multicolumn{1}{c}{VI}  & \multicolumn{1}{c}{RI} \\  
\cmidrule(lr){2-9}\cmidrule(lr){10-10} \cmidrule(lr){11-11} \cmidrule(lr){12-12}   
                     & \multicolumn{1}{c}{(0.5 sec)} & \multicolumn{1}{c}{(1 sec)} & \multicolumn{1}{c}{(10 sec)} & \multicolumn{1}{c}{(60 sec)} & \multicolumn{1}{c}{(300 sec)} & \multicolumn{1}{c}{(600 sec)} & \multicolumn{1}{c}{(1800 sec)} & \multicolumn{1}{c}{(end)} & \multicolumn{1}{c}{(end)}    & \multicolumn{1}{c}{(end)}   & \multicolumn{1}{c}{(end)}  \\ \midrule 
          PIVIT-BOEM & $\infty$ & $\infty$ & $\infty$ & $\infty$ & $\infty$ & $    -10443.86$ & $    -10443.86$ & $    -10443.86$ & $       520.72$ sec    & $       4.3903$  & $       0.8570$ \\ 
                 CGC & $         0.00$ & $         0.00$ & $    -26604.18$ & $    -26604.18$ & $    -26604.18$ & $    -26604.18$ & $    -26604.18$ & $    -26604.18$ & $        10.18$ sec    & $       1.8023$  & $       0.8969$ \\ 
                  HC & $    -21925.87$ & $    -21925.87$ & $    -21925.87$ & $    -21925.87$ & $    -21925.87$ & $    -21925.87$ & $    -21925.87$ & $    -21925.87$ & $         0.04$ sec    & $       2.4849$  & $       0.8351$ \\ 
              HC-CGC & $    -25323.89$ & $    -25570.78$ & $    -26586.84$ & $    -26586.84$ & $    -26586.84$ & $    -26586.84$ & $    -26586.84$ & $    -26586.84$ & $         7.51$ sec    & $       1.8026$  & $       0.8966$ \\ 
              ogm-KL & $     -1827.45$ & $     -1827.45$ & $    -24683.01$ & $    -24683.01$ & $    -24683.01$ & $    -24683.01$ & $    -24683.01$ & $    -24683.01$ & $        10.62$ sec    & $       4.3570$  & $       0.7143$ \\ 
    CC-Fusion-HC-CGC & $    -26021.95$ & $    -26021.95$ & $    -26319.42$ & $    -26319.42$ & $    -26319.42$ & $    -26319.42$ & $    -26319.42$ & $    -26319.42$ & $        11.56$ sec    & $       1.8443$  & $       0.9023$ \\ 
     CC-Fusion-HC-MC & $    -25673.82$ & $    -26389.43$ & $    -26606.48$ & $    -26607.98$ & $    -26607.98$ & $    -26607.98$ & $    -26607.98$ & $    -26607.98$ & $        34.54$ sec    & $       1.7859$  & $       0.8976$ \\ 
    CC-Fusion-WS-CGC & $    -25177.61$ & $    -25344.42$ & $    -25817.56$ & $    -25842.37$ & $    -25842.37$ & $    -25842.37$ & $    -25842.37$ & $    -25842.37$ & $        31.74$ sec    & $       2.1993$  & $       0.8817$ \\ 
     CC-Fusion-WS-MC & $     -1827.45$ & $    -20621.47$ & $    -26594.09$ & $    -26607.98$ & $    -26607.98$ & $    -26607.98$ & $    -26607.98$ & $    -26607.98$ & $        62.05$ sec    & $       1.7859$  & $       0.8976$ \\ 
\cmidrule{1-1} 
           MCR-CCFDB & $     -1827.45$ & $     -1827.45$ & $     -4290.04$ & $    -25770.98$ & $    -26578.44$ & $    -26578.44$ & $    -26578.44$ & $    -26578.44$ & $        83.99$ sec    & $       1.7683$  & $       0.8981$ \\ 
\cmidrule{1-1} 
           MCI-CCIFD & $     -1827.45$ & $     -1827.45$ & $    -25902.05$ & $    -26607.98$ & $    -26607.98$ & $    -26607.98$ & $    -26607.98$ & $    -26607.98$ & $        24.85$ sec    & $       1.7859$  & $       0.8976$ \\ 
\bottomrule
\end{tabular}
\end{table}


\subsection{knott-3d-450}
\begin{table}[H]
\scriptsize
\centering
\caption{knott-3d-450 (gm\_knott\_3d\_096)}
\label{tab:anytimetable-knott-3d-450-gm-knott-3d-096}
\begin{tabular}{lrrrrrrrrrrr}
\toprule
           algorithm &                                   \multicolumn{8}{c}{value} & \multicolumn{1}{c}{time}    & \multicolumn{1}{c}{VI}  & \multicolumn{1}{c}{RI} \\  
\cmidrule(lr){2-9}\cmidrule(lr){10-10} \cmidrule(lr){11-11} \cmidrule(lr){12-12}   
                     & \multicolumn{1}{c}{(0.5 sec)} & \multicolumn{1}{c}{(1 sec)} & \multicolumn{1}{c}{(10 sec)} & \multicolumn{1}{c}{(60 sec)} & \multicolumn{1}{c}{(300 sec)} & \multicolumn{1}{c}{(600 sec)} & \multicolumn{1}{c}{(1800 sec)} & \multicolumn{1}{c}{(end)} & \multicolumn{1}{c}{(end)}    & \multicolumn{1}{c}{(end)}   & \multicolumn{1}{c}{(end)}  \\ \midrule 
                 CGC & $         0.00$ & $         0.00$ & $         0.00$ & $    -89582.39$ & $    -89681.90$ & $    -89681.90$ & $    -89681.90$ & $    -89681.90$ & $        84.60$ sec    & $       2.2994$  & $       0.8851$ \\ 
                  HC & $    -77366.26$ & $    -77366.26$ & $    -77366.26$ & $    -77366.26$ & $    -77366.26$ & $    -77366.26$ & $    -77366.26$ & $    -77366.26$ & $         0.28$ sec    & $       3.0903$  & $       0.8035$ \\ 
              HC-CGC & $    -85381.84$ & $    -85407.71$ & $    -87435.50$ & $    -89605.35$ & $    -89605.35$ & $    -89605.35$ & $    -89605.35$ & $    -89605.35$ & $        53.60$ sec    & $       2.3528$  & $       0.8846$ \\ 
              ogm-KL & $     -6504.54$ & $     -6504.54$ & $     -6504.54$ & $    -81992.02$ & $    -83436.56$ & $    -83436.56$ & $    -83436.56$ & $    -83436.56$ & $       187.89$ sec    & $       5.2742$  & $       0.6892$ \\ 
    CC-Fusion-HC-CGC & $    -76149.17$ & $    -80772.94$ & $    -86369.93$ & $    -87442.99$ & $    -87442.99$ & $    -87442.99$ & $    -87442.99$ & $    -87442.99$ & $        84.22$ sec    & $       2.4266$  & $       0.8975$ \\ 
     CC-Fusion-HC-MC & $     -6504.54$ & $    -77741.99$ & $    -89396.66$ & $    -89764.33$ & $    -89810.42$ & $    -89810.42$ & $    -89810.42$ & $    -89810.42$ & $       206.13$ sec    & $       2.0311$  & $       0.9066$ \\ 
    CC-Fusion-WS-CGC & $     -6504.54$ & $    -70751.86$ & $    -83217.04$ & $    -84538.86$ & $    -84538.86$ & $    -84538.86$ & $    -84538.86$ & $    -84538.86$ & $       161.80$ sec    & $       2.6986$  & $       0.9024$ \\ 
     CC-Fusion-WS-MC & $     -6504.54$ & $     -6504.54$ & $    -87957.41$ & $    -89687.58$ & $    -89793.08$ & $    -89828.99$ & $    -89828.99$ & $    -89828.99$ & $       594.71$ sec    & $       2.0285$  & $       0.9068$ \\ 
\cmidrule{1-1} 
           MCR-CCFDB & $     -6504.54$ & $     -6504.54$ & $     -6504.54$ & $     -6504.54$ & $    -13109.24$ & $    -26156.24$ & $    -69829.42$ & $    -69829.42$ & $      1822.72$ sec    & $       3.0765$  & $       0.6926$ \\ 
\cmidrule{1-1} 
           MCI-CCIFD & $     -6504.54$ & $     -6504.54$ & $     -6504.54$ & $    -21343.11$ & $    -89640.23$ & $    -89959.41$ & $    -89959.41$ & $    -89959.41$ & $       337.50$ sec    & $       1.9035$  & $       0.9308$ \\ 
\bottomrule
\end{tabular}
\end{table}

\begin{table}[H]
\scriptsize
\centering
\caption{knott-3d-450 (gm\_knott\_3d\_097)}
\label{tab:anytimetable-knott-3d-450-gm-knott-3d-097}
\begin{tabular}{lrrrrrrrrrrr}
\toprule
           algorithm &                                   \multicolumn{8}{c}{value} & \multicolumn{1}{c}{time}    & \multicolumn{1}{c}{VI}  & \multicolumn{1}{c}{RI} \\  
\cmidrule(lr){2-9}\cmidrule(lr){10-10} \cmidrule(lr){11-11} \cmidrule(lr){12-12}   
                     & \multicolumn{1}{c}{(0.5 sec)} & \multicolumn{1}{c}{(1 sec)} & \multicolumn{1}{c}{(10 sec)} & \multicolumn{1}{c}{(60 sec)} & \multicolumn{1}{c}{(300 sec)} & \multicolumn{1}{c}{(600 sec)} & \multicolumn{1}{c}{(1800 sec)} & \multicolumn{1}{c}{(end)} & \multicolumn{1}{c}{(end)}    & \multicolumn{1}{c}{(end)}   & \multicolumn{1}{c}{(end)}  \\ \midrule 
                 CGC & $         0.00$ & $         0.00$ & $         0.00$ & $    -73194.11$ & $    -73196.32$ & $    -73196.32$ & $    -73196.32$ & $    -73196.32$ & $        73.09$ sec    & $       2.3130$  & $       0.8142$ \\ 
                  HC & $    -65135.76$ & $    -65135.76$ & $    -65135.76$ & $    -65135.76$ & $    -65135.76$ & $    -65135.76$ & $    -65135.76$ & $    -65135.76$ & $         0.35$ sec    & $       2.7417$  & $       0.7889$ \\ 
              HC-CGC & $    -68421.60$ & $    -68432.64$ & $    -71927.20$ & $    -73393.59$ & $    -73393.59$ & $    -73393.59$ & $    -73393.59$ & $    -73393.59$ & $        46.53$ sec    & $       2.0645$  & $       0.8233$ \\ 
              ogm-KL & $     -4626.25$ & $     -4626.25$ & $     -4626.25$ & $    -68048.10$ & $    -69320.32$ & $    -69320.32$ & $    -69320.32$ & $    -69320.32$ & $       190.71$ sec    & $       4.6581$  & $       0.6482$ \\ 
    CC-Fusion-HC-CGC & $    -60813.40$ & $    -67207.78$ & $    -70886.83$ & $    -70914.74$ & $    -71433.42$ & $    -71433.42$ & $    -71433.42$ & $    -71433.42$ & $       145.69$ sec    & $       2.5275$  & $       0.7840$ \\ 
     CC-Fusion-HC-MC & $     -4626.25$ & $    -62666.23$ & $    -73264.79$ & $    -73445.66$ & $    -73445.66$ & $    -73445.66$ & $    -73445.66$ & $    -73445.66$ & $       111.21$ sec    & $       1.9981$  & $       0.8274$ \\ 
    CC-Fusion-WS-CGC & $    -58852.06$ & $    -66211.06$ & $    -69659.84$ & $    -69815.82$ & $    -70797.99$ & $    -70797.99$ & $    -70797.99$ & $    -70797.99$ & $       225.65$ sec    & $       2.6017$  & $       0.8122$ \\ 
     CC-Fusion-WS-MC & $     -4626.25$ & $     -4626.25$ & $    -71860.56$ & $    -73400.97$ & $    -73476.61$ & $    -73476.61$ & $    -73476.61$ & $    -73476.61$ & $       342.34$ sec    & $       1.9975$  & $       0.8270$ \\ 
\cmidrule{1-1} 
           MCR-CCFDB & $     -4626.25$ & $     -4626.25$ & $     -4626.25$ & $     -4626.25$ & $    -10985.99$ & $    -23330.11$ & $    -65180.85$ & $    -65180.85$ & $      1834.93$ sec    & $       2.4361$  & $       0.7286$ \\ 
\cmidrule{1-1} 
           MCI-CCIFD & $     -4626.25$ & $     -4626.25$ & $     -4626.25$ & $    -16559.90$ & $    -73477.55$ & $    -73477.55$ & $    -73477.55$ & $    -73477.55$ & $       268.74$ sec    & $       1.9976$  & $       0.8275$ \\ 
\bottomrule
\end{tabular}
\end{table}

\begin{table}[H]
\scriptsize
\centering
\caption{knott-3d-450 (gm\_knott\_3d\_098)}
\label{tab:anytimetable-knott-3d-450-gm-knott-3d-098}
\begin{tabular}{lrrrrrrrrrrr}
\toprule
           algorithm &                                   \multicolumn{8}{c}{value} & \multicolumn{1}{c}{time}    & \multicolumn{1}{c}{VI}  & \multicolumn{1}{c}{RI} \\  
\cmidrule(lr){2-9}\cmidrule(lr){10-10} \cmidrule(lr){11-11} \cmidrule(lr){12-12}   
                     & \multicolumn{1}{c}{(0.5 sec)} & \multicolumn{1}{c}{(1 sec)} & \multicolumn{1}{c}{(10 sec)} & \multicolumn{1}{c}{(60 sec)} & \multicolumn{1}{c}{(300 sec)} & \multicolumn{1}{c}{(600 sec)} & \multicolumn{1}{c}{(1800 sec)} & \multicolumn{1}{c}{(end)} & \multicolumn{1}{c}{(end)}    & \multicolumn{1}{c}{(end)}   & \multicolumn{1}{c}{(end)}  \\ \midrule 
                 CGC & $         0.00$ & $         0.00$ & $         0.00$ & $    -86331.01$ & $    -86332.22$ & $    -86332.22$ & $    -86332.22$ & $    -86332.22$ & $        71.97$ sec    & $       2.4331$  & $       0.8636$ \\ 
                  HC & $    -76603.26$ & $    -76603.26$ & $    -76603.26$ & $    -76603.26$ & $    -76603.26$ & $    -76603.26$ & $    -76603.26$ & $    -76603.26$ & $         0.28$ sec    & $       2.9603$  & $       0.7963$ \\ 
              HC-CGC & $    -83366.13$ & $    -83378.62$ & $    -85830.24$ & $    -86500.26$ & $    -86500.26$ & $    -86500.26$ & $    -86500.26$ & $    -86500.26$ & $        44.77$ sec    & $       2.0466$  & $       0.8848$ \\ 
              ogm-KL & $     -5807.00$ & $     -5807.00$ & $     -5807.00$ & $    -80778.55$ & $    -81676.37$ & $    -81676.37$ & $    -81676.37$ & $    -81676.37$ & $       172.27$ sec    & $       5.1483$  & $       0.6718$ \\ 
    CC-Fusion-HC-CGC & $    -78628.78$ & $    -78637.88$ & $    -84728.38$ & $    -84964.79$ & $    -84964.79$ & $    -84964.79$ & $    -84964.79$ & $    -84964.79$ & $        50.67$ sec    & $       2.1306$  & $       0.8989$ \\ 
     CC-Fusion-HC-MC & $     -5807.00$ & $    -76495.45$ & $    -86291.60$ & $    -86495.58$ & $    -86495.58$ & $    -86495.58$ & $    -86495.58$ & $    -86495.58$ & $       103.63$ sec    & $       1.9138$  & $       0.9003$ \\ 
    CC-Fusion-WS-CGC & $    -68430.67$ & $    -76884.66$ & $    -81590.47$ & $    -82698.93$ & $    -82698.93$ & $    -82698.93$ & $    -82698.93$ & $    -82698.93$ & $        87.13$ sec    & $       2.8300$  & $       0.8653$ \\ 
     CC-Fusion-WS-MC & $     -5807.00$ & $     -5807.00$ & $    -82556.94$ & $    -86441.85$ & $    -86593.72$ & $    -86593.72$ & $    -86593.72$ & $    -86593.72$ & $       528.88$ sec    & $       1.8569$  & $       0.9054$ \\ 
\cmidrule{1-1} 
           MCR-CCFDB & $     -5807.00$ & $     -5807.00$ & $     -5807.00$ & $     -5807.00$ & $    -10847.89$ & $    -23603.58$ & $    -74529.25$ & $    -74529.25$ & $      1822.77$ sec    & $       2.6551$  & $       0.7548$ \\ 
\cmidrule{1-1} 
           MCI-CCIFD & $     -5807.00$ & $     -5807.00$ & $     -5807.00$ & $    -23550.62$ & $    -86439.15$ & $    -86593.97$ & $    -86593.97$ & $    -86593.97$ & $       328.90$ sec    & $       1.8016$  & $       0.9124$ \\ 
\bottomrule
\end{tabular}
\end{table}

\begin{table}[H]
\scriptsize
\centering
\caption{knott-3d-450 (gm\_knott\_3d\_099)}
\label{tab:anytimetable-knott-3d-450-gm-knott-3d-099}
\begin{tabular}{lrrrrrrrrrrr}
\toprule
           algorithm &                                   \multicolumn{8}{c}{value} & \multicolumn{1}{c}{time}    & \multicolumn{1}{c}{VI}  & \multicolumn{1}{c}{RI} \\  
\cmidrule(lr){2-9}\cmidrule(lr){10-10} \cmidrule(lr){11-11} \cmidrule(lr){12-12}   
                     & \multicolumn{1}{c}{(0.5 sec)} & \multicolumn{1}{c}{(1 sec)} & \multicolumn{1}{c}{(10 sec)} & \multicolumn{1}{c}{(60 sec)} & \multicolumn{1}{c}{(300 sec)} & \multicolumn{1}{c}{(600 sec)} & \multicolumn{1}{c}{(1800 sec)} & \multicolumn{1}{c}{(end)} & \multicolumn{1}{c}{(end)}    & \multicolumn{1}{c}{(end)}   & \multicolumn{1}{c}{(end)}  \\ \midrule 
                 CGC & $         0.00$ & $         0.00$ & $         0.00$ & $    -85921.78$ & $    -86124.21$ & $    -86124.21$ & $    -86124.21$ & $    -86124.21$ & $        85.78$ sec    & $       2.1843$  & $       0.8479$ \\ 
                  HC & $    -80143.71$ & $    -80143.71$ & $    -80143.71$ & $    -80143.71$ & $    -80143.71$ & $    -80143.71$ & $    -80143.71$ & $    -80143.71$ & $         0.33$ sec    & $       2.6392$  & $       0.8125$ \\ 
              HC-CGC & $    -80826.25$ & $    -83419.19$ & $    -84812.40$ & $    -86179.94$ & $    -86179.94$ & $    -86179.94$ & $    -86179.94$ & $    -86179.94$ & $        46.39$ sec    & $       2.0887$  & $       0.8569$ \\ 
              ogm-KL & $     -5800.63$ & $     -5800.63$ & $     -5800.63$ & $    -79899.07$ & $    -81173.62$ & $    -81173.62$ & $    -81173.62$ & $    -81173.62$ & $       185.76$ sec    & $       5.0145$  & $       0.6556$ \\ 
    CC-Fusion-HC-CGC & $    -75188.25$ & $    -75216.40$ & $    -83872.72$ & $    -84504.60$ & $    -84504.60$ & $    -84504.60$ & $    -84504.60$ & $    -84504.60$ & $        54.76$ sec    & $       2.4162$  & $       0.8524$ \\ 
     CC-Fusion-HC-MC & $     -5800.63$ & $    -76667.19$ & $    -86119.72$ & $    -86180.59$ & $    -86180.59$ & $    -86180.59$ & $    -86180.59$ & $    -86180.59$ & $       107.23$ sec    & $       2.0205$  & $       0.8570$ \\ 
    CC-Fusion-WS-CGC & $    -69659.38$ & $    -69659.38$ & $    -82085.79$ & $    -82949.21$ & $    -82949.21$ & $    -82949.21$ & $    -82949.21$ & $    -82949.21$ & $       135.95$ sec    & $       2.7623$  & $       0.8384$ \\ 
     CC-Fusion-WS-MC & $     -5800.63$ & $     -5800.63$ & $    -85511.72$ & $    -86135.24$ & $    -86208.81$ & $    -86238.56$ & $    -86238.56$ & $    -86238.56$ & $       641.25$ sec    & $       2.0254$  & $       0.8578$ \\ 
\cmidrule{1-1} 
           MCR-CCFDB & $     -5800.63$ & $     -5800.63$ & $     -5800.63$ & $     -5800.63$ & $    -11899.23$ & $    -29624.25$ & $    -79620.92$ & $    -79620.92$ & $      1832.70$ sec    & $       2.3841$  & $       0.7664$ \\ 
\cmidrule{1-1} 
           MCI-CCIFD & $     -5800.63$ & $     -5800.63$ & $     -5800.63$ & $    -24095.62$ & $    -85933.44$ & $    -85956.93$ & $    -85956.93$ & $    -85956.93$ & $      1799.70$ sec    & $       1.9844$  & $       0.8588$ \\ 
\bottomrule
\end{tabular}
\end{table}

\begin{table}[H]
\scriptsize
\centering
\caption{knott-3d-450 (gm\_knott\_3d\_100)}
\label{tab:anytimetable-knott-3d-450-gm-knott-3d-100}
\begin{tabular}{lrrrrrrrrrrr}
\toprule
           algorithm &                                   \multicolumn{8}{c}{value} & \multicolumn{1}{c}{time}    & \multicolumn{1}{c}{VI}  & \multicolumn{1}{c}{RI} \\  
\cmidrule(lr){2-9}\cmidrule(lr){10-10} \cmidrule(lr){11-11} \cmidrule(lr){12-12}   
                     & \multicolumn{1}{c}{(0.5 sec)} & \multicolumn{1}{c}{(1 sec)} & \multicolumn{1}{c}{(10 sec)} & \multicolumn{1}{c}{(60 sec)} & \multicolumn{1}{c}{(300 sec)} & \multicolumn{1}{c}{(600 sec)} & \multicolumn{1}{c}{(1800 sec)} & \multicolumn{1}{c}{(end)} & \multicolumn{1}{c}{(end)}    & \multicolumn{1}{c}{(end)}   & \multicolumn{1}{c}{(end)}  \\ \midrule 
                 CGC & $         0.00$ & $         0.00$ & $         0.00$ & $    -76521.06$ & $    -76529.68$ & $    -76529.68$ & $    -76529.68$ & $    -76529.68$ & $        62.24$ sec    & $       2.1392$  & $       0.8930$ \\ 
                  HC & $    -52886.54$ & $    -52886.54$ & $    -52886.54$ & $    -52886.54$ & $    -52886.54$ & $    -52886.54$ & $    -52886.54$ & $    -52886.54$ & $         0.29$ sec    & $       3.2217$  & $       0.7121$ \\ 
              HC-CGC & $    -52959.08$ & $    -69487.57$ & $    -73755.02$ & $    -76462.46$ & $    -76462.46$ & $    -76462.46$ & $    -76462.46$ & $    -76462.46$ & $        46.05$ sec    & $       2.3725$  & $       0.8725$ \\ 
              ogm-KL & $     -4148.97$ & $     -4148.97$ & $     -4148.97$ & $    -68997.05$ & $    -69788.31$ & $    -69788.31$ & $    -69788.31$ & $    -69788.31$ & $       191.74$ sec    & $       5.0973$  & $       0.6205$ \\ 
    CC-Fusion-HC-CGC & $    -60622.39$ & $    -69097.43$ & $    -73051.76$ & $    -74131.11$ & $    -74525.94$ & $    -74525.94$ & $    -74525.94$ & $    -74525.94$ & $       151.95$ sec    & $       2.3949$  & $       0.8941$ \\ 
     CC-Fusion-HC-MC & $     -4148.97$ & $    -64909.33$ & $    -76078.15$ & $    -76584.80$ & $    -76659.12$ & $    -76659.12$ & $    -76659.12$ & $    -76659.12$ & $       215.06$ sec    & $       2.1224$  & $       0.9014$ \\ 
    CC-Fusion-WS-CGC & $     -4148.97$ & $    -58318.15$ & $    -71475.18$ & $    -72523.04$ & $    -72523.04$ & $    -72523.04$ & $    -72523.04$ & $    -72523.04$ & $       131.61$ sec    & $       2.8618$  & $       0.8860$ \\ 
     CC-Fusion-WS-MC & $     -4148.97$ & $     -4148.97$ & $    -73238.42$ & $    -76584.58$ & $    -76672.89$ & $    -76672.89$ & $    -76672.89$ & $    -76672.89$ & $       490.99$ sec    & $       2.1190$  & $       0.9015$ \\ 
\cmidrule{1-1} 
           MCR-CCFDB & $     -4148.97$ & $     -4148.97$ & $     -4148.97$ & $     -4148.97$ & $    -12042.61$ & $    -23872.55$ & $    -56042.54$ & $    -56042.54$ & $      1837.01$ sec    & $       3.6626$  & $       0.5753$ \\ 
\cmidrule{1-1} 
           MCI-CCIFD & $     -4148.97$ & $     -4148.97$ & $     -4148.97$ & $    -27595.65$ & $    -76699.37$ & $    -76699.37$ & $    -76699.37$ & $    -76699.37$ & $       278.73$ sec    & $       2.0840$  & $       0.9055$ \\ 
\bottomrule
\end{tabular}
\end{table}

\begin{table}[H]
\scriptsize
\centering
\caption{knott-3d-450 (gm\_knott\_3d\_101)}
\label{tab:anytimetable-knott-3d-450-gm-knott-3d-101}
\begin{tabular}{lrrrrrrrrrrr}
\toprule
           algorithm &                                   \multicolumn{8}{c}{value} & \multicolumn{1}{c}{time}    & \multicolumn{1}{c}{VI}  & \multicolumn{1}{c}{RI} \\  
\cmidrule(lr){2-9}\cmidrule(lr){10-10} \cmidrule(lr){11-11} \cmidrule(lr){12-12}   
                     & \multicolumn{1}{c}{(0.5 sec)} & \multicolumn{1}{c}{(1 sec)} & \multicolumn{1}{c}{(10 sec)} & \multicolumn{1}{c}{(60 sec)} & \multicolumn{1}{c}{(300 sec)} & \multicolumn{1}{c}{(600 sec)} & \multicolumn{1}{c}{(1800 sec)} & \multicolumn{1}{c}{(end)} & \multicolumn{1}{c}{(end)}    & \multicolumn{1}{c}{(end)}   & \multicolumn{1}{c}{(end)}  \\ \midrule 
                 CGC & $         0.00$ & $         0.00$ & $         0.00$ & $    -74071.58$ & $    -74392.76$ & $    -74392.76$ & $    -74392.76$ & $    -74392.76$ & $       114.08$ sec    & $       2.2969$  & $       0.8156$ \\ 
                  HC & $    -66066.00$ & $    -66066.00$ & $    -66066.00$ & $    -66066.00$ & $    -66066.00$ & $    -66066.00$ & $    -66066.00$ & $    -66066.00$ & $         0.35$ sec    & $       2.8546$  & $       0.7705$ \\ 
              HC-CGC & $    -66495.54$ & $    -70374.37$ & $    -71277.01$ & $    -74384.75$ & $    -74399.79$ & $    -74399.79$ & $    -74399.79$ & $    -74399.79$ & $        86.30$ sec    & $       2.3125$  & $       0.8235$ \\ 
              ogm-KL & $     -4104.14$ & $     -4104.14$ & $     -4104.14$ & $    -68453.16$ & $    -69409.01$ & $    -69409.01$ & $    -69409.01$ & $    -69409.01$ & $       178.30$ sec    & $       4.9624$  & $       0.6187$ \\ 
    CC-Fusion-HC-CGC & $    -63081.79$ & $    -64916.67$ & $    -72397.93$ & $    -72681.24$ & $    -72681.24$ & $    -72681.24$ & $    -72681.24$ & $    -72681.24$ & $        94.15$ sec    & $       2.3526$  & $       0.8420$ \\ 
     CC-Fusion-HC-MC & $     -4104.14$ & $    -65756.83$ & $    -74474.86$ & $    -74509.06$ & $    -74509.06$ & $    -74509.06$ & $    -74509.06$ & $    -74509.06$ & $        90.10$ sec    & $       2.1083$  & $       0.8476$ \\ 
    CC-Fusion-WS-CGC & $     -4104.14$ & $    -60051.49$ & $    -70485.73$ & $    -71364.20$ & $    -71364.20$ & $    -71364.20$ & $    -71364.20$ & $    -71364.20$ & $       129.65$ sec    & $       2.7732$  & $       0.8177$ \\ 
     CC-Fusion-WS-MC & $     -4104.14$ & $     -4104.14$ & $    -70875.67$ & $    -74506.77$ & $    -74529.24$ & $    -74529.24$ & $    -74529.24$ & $    -74529.24$ & $       325.13$ sec    & $       2.1110$  & $       0.8476$ \\ 
\cmidrule{1-1} 
           MCR-CCFDB & $     -4104.14$ & $     -4104.14$ & $     -4104.14$ & $     -4104.14$ & $     -7376.15$ & $    -16040.30$ & $    -59065.52$ & $    -59065.52$ & $      1867.22$ sec    & $       3.1232$  & $       0.5943$ \\ 
\cmidrule{1-1} 
           MCI-CCIFD & $     -4104.14$ & $     -4104.14$ & $     -4104.14$ & $    -15287.07$ & $    -74205.91$ & $    -74529.51$ & $    -74529.51$ & $    -74529.51$ & $       334.60$ sec    & $       2.1110$  & $       0.8476$ \\ 
\bottomrule
\end{tabular}
\end{table}

\begin{table}[H]
\scriptsize
\centering
\caption{knott-3d-450 (gm\_knott\_3d\_102)}
\label{tab:anytimetable-knott-3d-450-gm-knott-3d-102}
\begin{tabular}{lrrrrrrrrrrr}
\toprule
           algorithm &                                   \multicolumn{8}{c}{value} & \multicolumn{1}{c}{time}    & \multicolumn{1}{c}{VI}  & \multicolumn{1}{c}{RI} \\  
\cmidrule(lr){2-9}\cmidrule(lr){10-10} \cmidrule(lr){11-11} \cmidrule(lr){12-12}   
                     & \multicolumn{1}{c}{(0.5 sec)} & \multicolumn{1}{c}{(1 sec)} & \multicolumn{1}{c}{(10 sec)} & \multicolumn{1}{c}{(60 sec)} & \multicolumn{1}{c}{(300 sec)} & \multicolumn{1}{c}{(600 sec)} & \multicolumn{1}{c}{(1800 sec)} & \multicolumn{1}{c}{(end)} & \multicolumn{1}{c}{(end)}    & \multicolumn{1}{c}{(end)}   & \multicolumn{1}{c}{(end)}  \\ \midrule 
                 CGC & $         0.00$ & $         0.00$ & $         0.00$ & $    -65111.15$ & $    -65885.83$ & $    -65885.83$ & $    -65885.83$ & $    -65885.83$ & $       137.87$ sec    & $       2.1590$  & $       0.7896$ \\ 
                  HC & $    -58463.32$ & $    -58463.32$ & $    -58463.32$ & $    -58463.32$ & $    -58463.32$ & $    -58463.32$ & $    -58463.32$ & $    -58463.32$ & $         0.32$ sec    & $       2.7101$  & $       0.7152$ \\ 
              HC-CGC & $    -62469.87$ & $    -62912.76$ & $    -63207.57$ & $    -66183.62$ & $    -66241.33$ & $    -66241.33$ & $    -66241.33$ & $    -66241.33$ & $        95.50$ sec    & $       1.9150$  & $       0.8354$ \\ 
              ogm-KL & $     -3926.11$ & $     -3926.11$ & $     -3926.11$ & $    -60756.42$ & $    -61743.77$ & $    -61743.77$ & $    -61743.77$ & $    -61743.77$ & $       192.87$ sec    & $       4.3688$  & $       0.6020$ \\ 
    CC-Fusion-HC-CGC & $    -60904.68$ & $    -60904.68$ & $    -65091.16$ & $    -65325.61$ & $    -65325.61$ & $    -65325.61$ & $    -65325.61$ & $    -65325.61$ & $        79.07$ sec    & $       2.0283$  & $       0.8417$ \\ 
     CC-Fusion-HC-MC & $     -3926.11$ & $    -47599.28$ & $    -66408.18$ & $    -66477.52$ & $    -66477.52$ & $    -66477.52$ & $    -66477.52$ & $    -66477.52$ & $       106.36$ sec    & $       1.8600$  & $       0.8500$ \\ 
    CC-Fusion-WS-CGC & $    -53782.35$ & $    -59976.55$ & $    -63033.98$ & $    -64007.43$ & $    -64271.44$ & $    -64271.44$ & $    -64271.44$ & $    -64271.44$ & $       143.78$ sec    & $       2.4149$  & $       0.8315$ \\ 
     CC-Fusion-WS-MC & $     -3926.11$ & $     -3926.11$ & $    -63997.53$ & $    -66477.72$ & $    -66482.52$ & $    -66482.52$ & $    -66482.52$ & $    -66482.52$ & $       493.50$ sec    & $       1.8602$  & $       0.8500$ \\ 
\cmidrule{1-1} 
           MCR-CCFDB & $     -3926.11$ & $     -3926.11$ & $     -3926.11$ & $     -3926.11$ & $     -7781.92$ & $    -12748.76$ & $    -46110.61$ & $    -46110.61$ & $      1817.59$ sec    & $       2.7730$  & $       0.5554$ \\ 
\cmidrule{1-1} 
           MCI-CCIFD & $     -3926.11$ & $     -3926.11$ & $     -3926.11$ & $    -13143.57$ & $    -66482.68$ & $    -66482.68$ & $    -66482.68$ & $    -66482.68$ & $       286.48$ sec    & $       1.8507$  & $       0.8508$ \\ 
\bottomrule
\end{tabular}
\end{table}

\begin{table}[H]
\scriptsize
\centering
\caption{knott-3d-450 (gm\_knott\_3d\_103)}
\label{tab:anytimetable-knott-3d-450-gm-knott-3d-103}
\begin{tabular}{lrrrrrrrrrrr}
\toprule
           algorithm &                                   \multicolumn{8}{c}{value} & \multicolumn{1}{c}{time}    & \multicolumn{1}{c}{VI}  & \multicolumn{1}{c}{RI} \\  
\cmidrule(lr){2-9}\cmidrule(lr){10-10} \cmidrule(lr){11-11} \cmidrule(lr){12-12}   
                     & \multicolumn{1}{c}{(0.5 sec)} & \multicolumn{1}{c}{(1 sec)} & \multicolumn{1}{c}{(10 sec)} & \multicolumn{1}{c}{(60 sec)} & \multicolumn{1}{c}{(300 sec)} & \multicolumn{1}{c}{(600 sec)} & \multicolumn{1}{c}{(1800 sec)} & \multicolumn{1}{c}{(end)} & \multicolumn{1}{c}{(end)}    & \multicolumn{1}{c}{(end)}   & \multicolumn{1}{c}{(end)}  \\ \midrule 
                 CGC & $         0.00$ & $         0.00$ & $         0.00$ & $    -71906.81$ & $    -73729.88$ & $    -73729.88$ & $    -73729.88$ & $    -73729.88$ & $       117.21$ sec    & $       2.6585$  & $       0.7639$ \\ 
                  HC & $    -64935.25$ & $    -64935.25$ & $    -64935.25$ & $    -64935.25$ & $    -64935.25$ & $    -64935.25$ & $    -64935.25$ & $    -64935.25$ & $         0.34$ sec    & $       3.1065$  & $       0.6886$ \\ 
              HC-CGC & $    -65666.85$ & $    -70581.15$ & $    -71001.99$ & $    -73688.70$ & $    -73718.40$ & $    -73718.40$ & $    -73718.40$ & $    -73718.40$ & $       100.18$ sec    & $       2.6522$  & $       0.7652$ \\ 
              ogm-KL & $     -4221.22$ & $     -4221.22$ & $     -4221.22$ & $    -68237.46$ & $    -68962.60$ & $    -68962.60$ & $    -68962.60$ & $    -68962.60$ & $       231.02$ sec    & $       4.8927$  & $       0.6210$ \\ 
    CC-Fusion-HC-CGC & $    -63475.47$ & $    -68965.51$ & $    -71532.64$ & $    -72483.88$ & $    -72483.88$ & $    -72483.88$ & $    -72483.88$ & $    -72483.88$ & $        48.73$ sec    & $       2.7707$  & $       0.7657$ \\ 
     CC-Fusion-HC-MC & $     -4221.22$ & $    -65045.76$ & $    -73557.49$ & $    -73731.09$ & $    -73731.09$ & $    -73731.09$ & $    -73731.09$ & $    -73731.09$ & $       122.54$ sec    & $       2.5862$  & $       0.7682$ \\ 
    CC-Fusion-WS-CGC & $    -61634.71$ & $    -61634.71$ & $    -70430.83$ & $    -70697.57$ & $    -71035.03$ & $    -71035.03$ & $    -71035.03$ & $    -71035.03$ & $       225.82$ sec    & $       3.0472$  & $       0.7617$ \\ 
     CC-Fusion-WS-MC & $     -4221.22$ & $     -4221.22$ & $    -72609.52$ & $    -73565.12$ & $    -73763.63$ & $    -73763.90$ & $    -73763.90$ & $    -73763.90$ & $       586.26$ sec    & $       2.5923$  & $       0.7691$ \\ 
\cmidrule{1-1} 
           MCR-CCFDB & $     -4221.22$ & $     -4221.22$ & $     -4221.22$ & $     -4221.22$ & $    -10383.72$ & $    -28085.13$ & $    -70272.34$ & $    -70272.34$ & $      1832.67$ sec    & $       2.8571$  & $       0.7033$ \\ 
\cmidrule{1-1} 
           MCI-CCIFD & $     -4221.22$ & $     -4221.22$ & $     -4221.22$ & $    -24924.00$ & $    -71844.21$ & $    -73425.08$ & $    -73598.74$ & $    -73598.74$ & $      1800.16$ sec    & $       2.2969$  & $       0.8025$ \\ 
\bottomrule
\end{tabular}
\end{table}


\subsection{knott-3d-550}
\begin{table}[H]
\scriptsize
\centering
\caption{knott-3d-550 (gm\_knott\_3d\_112)}
\label{tab:anytimetable-knott-3d-550-gm-knott-3d-112}
\begin{tabular}{lrrrrrrrrr}
\toprule
           algorithm &                                   \multicolumn{8}{c}{value} & \multicolumn{1}{c}{time}   \\  
\cmidrule(lr){2-9}\cmidrule(lr){10-10}   
                     & \multicolumn{1}{c}{(0.5 sec)} & \multicolumn{1}{c}{(1 sec)} & \multicolumn{1}{c}{(10 sec)} & \multicolumn{1}{c}{(60 sec)} & \multicolumn{1}{c}{(300 sec)} & \multicolumn{1}{c}{(600 sec)} & \multicolumn{1}{c}{(1800 sec)} & \multicolumn{1}{c}{(end)} & \multicolumn{1}{c}{(end)}   \\ \midrule 
                 CGC & $         0.00$ & $         0.00$ & $         0.00$ & $         0.00$ & $   -152477.13$ & $   -152486.95$ & $   -152486.95$ & $   -152486.95$ & $       339.69$ sec   \\ 
                  HC & $   -134297.62$ & $   -134297.62$ & $   -134297.62$ & $   -134297.62$ & $   -134297.62$ & $   -134297.62$ & $   -134297.62$ & $   -134297.62$ & $         0.63$ sec   \\ 
              HC-CGC & $   -134297.62$ & $   -134585.84$ & $   -144504.77$ & $   -151570.33$ & $   -152515.41$ & $   -152515.41$ & $   -152515.41$ & $   -152515.41$ & $       208.65$ sec   \\ 
              ogm-KL & $    -11034.52$ & $    -11034.52$ & $    -11034.52$ & $    -11034.52$ & $   -139206.91$ & $   -141245.77$ & $   -141245.77$ & $   -141245.77$ & $       670.59$ sec   \\ 
    CC-Fusion-HC-CGC & $   -126136.98$ & $   -126136.98$ & $   -146470.38$ & $   -147293.44$ & $   -147364.26$ & $   -147364.26$ & $   -147364.26$ & $   -147364.26$ & $       197.83$ sec   \\ 
     CC-Fusion-HC-MC & $    -11034.52$ & $    -11034.52$ & $   -150995.62$ & $   -152708.49$ & $   -152767.04$ & $   -152767.04$ & $   -152767.04$ & $   -152767.04$ & $       306.88$ sec   \\ 
    CC-Fusion-WS-CGC & $    -11034.52$ & $   -123526.43$ & $   -135888.16$ & $   -142503.31$ & $   -144340.09$ & $   -144340.09$ & $   -144340.09$ & $   -144340.09$ & $       839.23$ sec   \\ 
     CC-Fusion-WS-MC & $    -11034.52$ & $    -11034.52$ & $   -130162.40$ & $   -152145.47$ & $   -152840.92$ & $   -152893.34$ & $   -152964.92$ & $   -152964.92$ & $      1745.74$ sec   \\ 
\cmidrule{1-1} 
           MCR-CCFDB & $    -11034.52$ & $    -11034.52$ & $    -11034.52$ & $    -11034.52$ & $    -11034.52$ & $    -11034.52$ & $    -40619.36$ & $    -40619.36$ & $      2031.23$ sec   \\ 
\cmidrule{1-1} 
           MCI-CCIFD & $    -11034.52$ & $    -11034.52$ & $    -11034.52$ & $    -11034.52$ & $    -56285.45$ & $   -105695.57$ & $   -153023.26$ & $   -153023.26$ & $      1115.58$ sec   \\ 
\bottomrule
\end{tabular}
\end{table}

\begin{table}[H]
\scriptsize
\centering
\caption{knott-3d-550 (gm\_knott\_3d\_113)}
\label{tab:anytimetable-knott-3d-550-gm-knott-3d-113}
\begin{tabular}{lrrrrrrrrr}
\toprule
           algorithm &                                   \multicolumn{8}{c}{value} & \multicolumn{1}{c}{time}   \\  
\cmidrule(lr){2-9}\cmidrule(lr){10-10}   
                     & \multicolumn{1}{c}{(0.5 sec)} & \multicolumn{1}{c}{(1 sec)} & \multicolumn{1}{c}{(10 sec)} & \multicolumn{1}{c}{(60 sec)} & \multicolumn{1}{c}{(300 sec)} & \multicolumn{1}{c}{(600 sec)} & \multicolumn{1}{c}{(1800 sec)} & \multicolumn{1}{c}{(end)} & \multicolumn{1}{c}{(end)}   \\ \midrule 
                 CGC & $     -8041.54$ & $     -8041.54$ & $     -8041.54$ & $     -8041.54$ & $   -126456.54$ & $   -134003.54$ & $   -135350.74$ & $   -135350.74$ & $      1280.17$ sec   \\ 
                  HC & $   -114456.46$ & $   -114456.46$ & $   -114456.46$ & $   -114456.46$ & $   -114456.46$ & $   -114456.46$ & $   -114456.46$ & $   -114456.46$ & $         0.70$ sec   \\ 
              HC-CGC & $   -114456.46$ & $   -116141.97$ & $   -127078.51$ & $   -127625.81$ & $   -134243.22$ & $   -135398.71$ & $   -135465.92$ & $   -135465.92$ & $      1300.64$ sec   \\ 
              ogm-KL & $     -8040.83$ & $     -8040.83$ & $     -8040.83$ & $     -8040.83$ & $   -124427.06$ & $   -126464.27$ & $   -126478.02$ & $   -126478.02$ & $       785.83$ sec   \\ 
    CC-Fusion-HC-CGC & $   -112867.34$ & $   -112867.34$ & $   -129428.40$ & $   -131850.04$ & $   -132746.52$ & $   -132746.52$ & $   -132746.52$ & $   -132746.52$ & $       290.91$ sec   \\ 
     CC-Fusion-HC-MC & $     -8040.83$ & $     -8040.83$ & $   -134183.10$ & $   -135307.46$ & $   -135571.24$ & $   -135571.24$ & $   -135571.24$ & $   -135571.24$ & $       249.26$ sec   \\ 
    CC-Fusion-WS-CGC & $     -8040.83$ & $     -8040.83$ & $   -122584.50$ & $   -127581.66$ & $   -129396.51$ & $   -129429.92$ & $   -129429.92$ & $   -129429.92$ & $       602.67$ sec   \\ 
     CC-Fusion-WS-MC & $     -8040.83$ & $     -8040.83$ & $   -116439.48$ & $   -135279.73$ & $   -135555.34$ & $   -135556.46$ & $   -135575.43$ & $   -135575.43$ & $      1801.24$ sec   \\ 
\cmidrule{1-1} 
           MCR-CCFDB & $     -8040.83$ & $     -8040.83$ & $     -8040.83$ & $     -8040.83$ & $     -8040.83$ & $    -19941.08$ & $    -37472.56$ & $    -37472.56$ & $      1908.85$ sec   \\ 
\cmidrule{1-1} 
           MCI-CCIFD & $     -8040.83$ & $     -8040.83$ & $     -8040.83$ & $     -8040.83$ & $    -49055.59$ & $   -112335.39$ & $   -135084.12$ & $   -135084.12$ & $      1800.84$ sec   \\ 
\bottomrule
\end{tabular}
\end{table}

\begin{table}[H]
\scriptsize
\centering
\caption{knott-3d-550 (gm\_knott\_3d\_114)}
\label{tab:anytimetable-knott-3d-550-gm-knott-3d-114}
\begin{tabular}{lrrrrrrrrr}
\toprule
           algorithm &                                   \multicolumn{8}{c}{value} & \multicolumn{1}{c}{time}   \\  
\cmidrule(lr){2-9}\cmidrule(lr){10-10}   
                     & \multicolumn{1}{c}{(0.5 sec)} & \multicolumn{1}{c}{(1 sec)} & \multicolumn{1}{c}{(10 sec)} & \multicolumn{1}{c}{(60 sec)} & \multicolumn{1}{c}{(300 sec)} & \multicolumn{1}{c}{(600 sec)} & \multicolumn{1}{c}{(1800 sec)} & \multicolumn{1}{c}{(end)} & \multicolumn{1}{c}{(end)}   \\ \midrule 
                 CGC & $         0.00$ & $         0.00$ & $         0.00$ & $         0.00$ & $   -149327.93$ & $   -149353.26$ & $   -149353.26$ & $   -149353.26$ & $       341.85$ sec   \\ 
                  HC & $   -133294.75$ & $   -133294.75$ & $   -133294.75$ & $   -133294.75$ & $   -133294.75$ & $   -133294.75$ & $   -133294.75$ & $   -133294.75$ & $         0.63$ sec   \\ 
              HC-CGC & $   -133294.75$ & $   -141562.33$ & $   -142124.17$ & $   -148671.12$ & $   -149296.47$ & $   -149296.47$ & $   -149296.47$ & $   -149296.47$ & $       252.77$ sec   \\ 
              ogm-KL & $     -8472.49$ & $     -8472.49$ & $     -8472.49$ & $     -8472.49$ & $   -139398.79$ & $   -139714.99$ & $   -139743.64$ & $   -139743.64$ & $       723.58$ sec   \\ 
    CC-Fusion-HC-CGC & $   -128277.63$ & $   -128277.63$ & $   -144071.66$ & $   -145274.73$ & $   -145570.04$ & $   -145570.04$ & $   -145570.04$ & $   -145570.04$ & $       194.31$ sec   \\ 
     CC-Fusion-HC-MC & $     -8472.49$ & $     -8472.49$ & $   -148616.13$ & $   -149512.44$ & $   -149601.99$ & $   -149601.99$ & $   -149601.99$ & $   -149601.99$ & $       341.52$ sec   \\ 
    CC-Fusion-WS-CGC & $     -8472.49$ & $   -120557.68$ & $   -137283.09$ & $   -139325.86$ & $   -140712.02$ & $   -140712.02$ & $   -140712.02$ & $   -140712.02$ & $       792.27$ sec   \\ 
     CC-Fusion-WS-MC & $     -8472.49$ & $     -8472.49$ & $     -8472.49$ & $   -148956.34$ & $   -149432.41$ & $   -149465.18$ & $   -149568.91$ & $   -149568.91$ & $      1656.56$ sec   \\ 
\cmidrule{1-1} 
           MCR-CCFDB & $     -8472.49$ & $     -8472.49$ & $     -8472.49$ & $     -8472.49$ & $     -8472.49$ & $    -17831.44$ & $    -37210.59$ & $    -37210.59$ & $      1925.72$ sec   \\ 
\cmidrule{1-1} 
           MCI-CCIFD & $     -8472.49$ & $     -8472.49$ & $     -8472.49$ & $     -8472.49$ & $    -59611.73$ & $   -107097.82$ & $   -149721.26$ & $   -149721.26$ & $      1094.67$ sec   \\ 
\bottomrule
\end{tabular}
\end{table}

\begin{table}[H]
\scriptsize
\centering
\caption{knott-3d-550 (gm\_knott\_3d\_115)}
\label{tab:anytimetable-knott-3d-550-gm-knott-3d-115}
\begin{tabular}{lrrrrrrrrr}
\toprule
           algorithm &                                   \multicolumn{8}{c}{value} & \multicolumn{1}{c}{time}   \\  
\cmidrule(lr){2-9}\cmidrule(lr){10-10}   
                     & \multicolumn{1}{c}{(0.5 sec)} & \multicolumn{1}{c}{(1 sec)} & \multicolumn{1}{c}{(10 sec)} & \multicolumn{1}{c}{(60 sec)} & \multicolumn{1}{c}{(300 sec)} & \multicolumn{1}{c}{(600 sec)} & \multicolumn{1}{c}{(1800 sec)} & \multicolumn{1}{c}{(end)} & \multicolumn{1}{c}{(end)}   \\ \midrule 
                 CGC & $         0.00$ & $         0.00$ & $         0.00$ & $         0.00$ & $   -144474.76$ & $   -149660.22$ & $   -149735.94$ & $   -149735.94$ & $       915.52$ sec   \\ 
                  HC & $   -135677.63$ & $   -135677.63$ & $   -135677.63$ & $   -135677.63$ & $   -135677.63$ & $   -135677.63$ & $   -135677.63$ & $   -135677.63$ & $         0.72$ sec   \\ 
              HC-CGC & $   -135677.63$ & $   -137246.88$ & $   -143943.39$ & $   -146276.56$ & $   -149600.65$ & $   -149696.58$ & $   -149696.58$ & $   -149696.58$ & $       615.10$ sec   \\ 
              ogm-KL & $     -8594.13$ & $     -8594.13$ & $     -8594.13$ & $     -8594.13$ & $   -139924.58$ & $   -140379.54$ & $   -140379.54$ & $   -140379.54$ & $       596.21$ sec   \\ 
    CC-Fusion-HC-CGC & $     -8594.13$ & $   -125467.91$ & $   -142161.19$ & $   -145386.25$ & $   -146750.01$ & $   -146750.01$ & $   -146750.01$ & $   -146750.01$ & $       203.39$ sec   \\ 
     CC-Fusion-HC-MC & $     -8594.13$ & $   -134525.25$ & $   -148449.51$ & $   -149793.41$ & $   -149816.78$ & $   -149816.78$ & $   -149816.78$ & $   -149816.78$ & $       287.65$ sec   \\ 
    CC-Fusion-WS-CGC & $     -8594.13$ & $     -8594.13$ & $   -134376.60$ & $   -141015.28$ & $   -142516.30$ & $   -142866.00$ & $   -142866.00$ & $   -142866.00$ & $       933.21$ sec   \\ 
     CC-Fusion-WS-MC & $     -8594.13$ & $     -8594.13$ & $   -126928.63$ & $   -149077.06$ & $   -149701.29$ & $   -149836.18$ & $   -149872.21$ & $   -149872.21$ & $      1800.17$ sec   \\ 
\cmidrule{1-1} 
           MCR-CCFDB & $     -8594.13$ & $     -8594.13$ & $     -8594.13$ & $     -8594.13$ & $     -8594.13$ & $    -16900.93$ & $    -41127.11$ & $    -41127.11$ & $      1957.96$ sec   \\ 
\cmidrule{1-1} 
           MCI-CCIFD & $     -8594.13$ & $     -8594.13$ & $     -8594.13$ & $     -8594.13$ & $    -71592.84$ & $   -113487.59$ & $   -149560.96$ & $   -149560.96$ & $      1800.95$ sec   \\ 
\bottomrule
\end{tabular}
\end{table}

\begin{table}[H]
\scriptsize
\centering
\caption{knott-3d-550 (gm\_knott\_3d\_116)}
\label{tab:anytimetable-knott-3d-550-gm-knott-3d-116}
\begin{tabular}{lrrrrrrrrr}
\toprule
           algorithm &                                   \multicolumn{8}{c}{value} & \multicolumn{1}{c}{time}   \\  
\cmidrule(lr){2-9}\cmidrule(lr){10-10}   
                     & \multicolumn{1}{c}{(0.5 sec)} & \multicolumn{1}{c}{(1 sec)} & \multicolumn{1}{c}{(10 sec)} & \multicolumn{1}{c}{(60 sec)} & \multicolumn{1}{c}{(300 sec)} & \multicolumn{1}{c}{(600 sec)} & \multicolumn{1}{c}{(1800 sec)} & \multicolumn{1}{c}{(end)} & \multicolumn{1}{c}{(end)}   \\ \midrule 
                 CGC & $         0.00$ & $         0.00$ & $         0.00$ & $    -96662.56$ & $   -130270.48$ & $   -130349.71$ & $   -130349.71$ & $   -130349.71$ & $       466.50$ sec   \\ 
                  HC & $   -112010.05$ & $   -112010.05$ & $   -112010.05$ & $   -112010.05$ & $   -112010.05$ & $   -112010.05$ & $   -112010.05$ & $   -112010.05$ & $         0.77$ sec   \\ 
              HC-CGC & $   -112010.05$ & $   -112505.10$ & $   -122865.35$ & $   -127693.08$ & $   -130260.47$ & $   -130260.47$ & $   -130260.47$ & $   -130260.47$ & $       217.61$ sec   \\ 
              ogm-KL & $     -7670.49$ & $     -7670.49$ & $     -7670.49$ & $     -7670.49$ & $   -118772.02$ & $   -120328.50$ & $   -120328.50$ & $   -120328.50$ & $       590.31$ sec   \\ 
    CC-Fusion-HC-CGC & $     -7670.49$ & $   -105502.05$ & $   -125727.82$ & $   -126227.29$ & $   -126677.91$ & $   -126677.91$ & $   -126677.91$ & $   -126677.91$ & $       241.63$ sec   \\ 
     CC-Fusion-HC-MC & $     -7670.49$ & $     -7670.49$ & $   -129446.26$ & $   -130600.08$ & $   -130654.11$ & $   -130654.11$ & $   -130654.11$ & $   -130654.11$ & $       349.68$ sec   \\ 
    CC-Fusion-WS-CGC & $     -7670.49$ & $   -103691.19$ & $   -120102.83$ & $   -123647.53$ & $   -124152.94$ & $   -124819.77$ & $   -124819.77$ & $   -124819.77$ & $       997.70$ sec   \\ 
     CC-Fusion-WS-MC & $     -7670.49$ & $     -7670.49$ & $   -110212.91$ & $   -130280.49$ & $   -130685.34$ & $   -130757.42$ & $   -130757.67$ & $   -130757.67$ & $      1293.18$ sec   \\ 
\cmidrule{1-1} 
           MCR-CCFDB & $     -7670.49$ & $     -7670.49$ & $     -7670.49$ & $     -7670.49$ & $     -7670.49$ & $     -7670.49$ & $    -36055.47$ & $    -36055.47$ & $      2140.98$ sec   \\ 
\cmidrule{1-1} 
           MCI-CCIFD & $     -7670.49$ & $     -7670.49$ & $     -7670.49$ & $     -7670.49$ & $    -49782.32$ & $    -80334.84$ & $   -130757.67$ & $   -130757.67$ & $      1227.31$ sec   \\ 
\bottomrule
\end{tabular}
\end{table}

\begin{table}[H]
\scriptsize
\centering
\caption{knott-3d-550 (gm\_knott\_3d\_117)}
\label{tab:anytimetable-knott-3d-550-gm-knott-3d-117}
\begin{tabular}{lrrrrrrrrr}
\toprule
           algorithm &                                   \multicolumn{8}{c}{value} & \multicolumn{1}{c}{time}   \\  
\cmidrule(lr){2-9}\cmidrule(lr){10-10}   
                     & \multicolumn{1}{c}{(0.5 sec)} & \multicolumn{1}{c}{(1 sec)} & \multicolumn{1}{c}{(10 sec)} & \multicolumn{1}{c}{(60 sec)} & \multicolumn{1}{c}{(300 sec)} & \multicolumn{1}{c}{(600 sec)} & \multicolumn{1}{c}{(1800 sec)} & \multicolumn{1}{c}{(end)} & \multicolumn{1}{c}{(end)}   \\ \midrule 
                 CGC & $     -7255.99$ & $     -7255.99$ & $     -7255.99$ & $     -7255.99$ & $   -122098.76$ & $   -123160.13$ & $   -123160.50$ & $   -123160.50$ & $       689.15$ sec   \\ 
                  HC & $   -108232.79$ & $   -108232.79$ & $   -108232.79$ & $   -108232.79$ & $   -108232.79$ & $   -108232.79$ & $   -108232.79$ & $   -108232.79$ & $         0.90$ sec   \\ 
              HC-CGC & $   -108232.79$ & $   -108612.16$ & $   -116381.91$ & $   -118156.53$ & $   -123312.66$ & $   -123357.05$ & $   -123357.05$ & $   -123357.05$ & $       596.90$ sec   \\ 
              ogm-KL & $     -7254.87$ & $     -7254.87$ & $     -7254.87$ & $     -7254.87$ & $   -113281.68$ & $   -114609.75$ & $   -114609.75$ & $   -114609.75$ & $       694.62$ sec   \\ 
    CC-Fusion-HC-CGC & $     -7254.87$ & $   -104001.77$ & $   -117466.97$ & $   -119304.80$ & $   -121195.26$ & $   -121195.26$ & $   -121195.26$ & $   -121195.26$ & $       410.41$ sec   \\ 
     CC-Fusion-HC-MC & $     -7254.87$ & $     -7254.87$ & $   -122820.20$ & $   -123432.28$ & $   -123450.53$ & $   -123450.53$ & $   -123450.53$ & $   -123450.53$ & $       247.34$ sec   \\ 
    CC-Fusion-WS-CGC & $     -7254.87$ & $     -7254.87$ & $   -114505.29$ & $   -117994.32$ & $   -118854.72$ & $   -118854.72$ & $   -118854.72$ & $   -118854.72$ & $       535.55$ sec   \\ 
     CC-Fusion-WS-MC & $     -7254.87$ & $     -7254.87$ & $   -106018.18$ & $   -122926.74$ & $   -123444.91$ & $   -123450.47$ & $   -123450.53$ & $   -123450.53$ & $      1606.19$ sec   \\ 
\cmidrule{1-1} 
           MCR-CCFDB & $     -7254.87$ & $     -7254.87$ & $     -7254.87$ & $     -7254.87$ & $     -7254.87$ & $    -13030.60$ & $    -32539.50$ & $    -32539.50$ & $      1990.26$ sec   \\ 
\cmidrule{1-1} 
           MCI-CCIFD & $     -7254.87$ & $     -7254.87$ & $     -7254.87$ & $     -7254.87$ & $    -24077.18$ & $    -66606.80$ & $   -122667.36$ & $   -122667.36$ & $      1802.73$ sec   \\ 
\bottomrule
\end{tabular}
\end{table}

\begin{table}[H]
\scriptsize
\centering
\caption{knott-3d-550 (gm\_knott\_3d\_118)}
\label{tab:anytimetable-knott-3d-550-gm-knott-3d-118}
\begin{tabular}{lrrrrrrrrr}
\toprule
           algorithm &                                   \multicolumn{8}{c}{value} & \multicolumn{1}{c}{time}   \\  
\cmidrule(lr){2-9}\cmidrule(lr){10-10}   
                     & \multicolumn{1}{c}{(0.5 sec)} & \multicolumn{1}{c}{(1 sec)} & \multicolumn{1}{c}{(10 sec)} & \multicolumn{1}{c}{(60 sec)} & \multicolumn{1}{c}{(300 sec)} & \multicolumn{1}{c}{(600 sec)} & \multicolumn{1}{c}{(1800 sec)} & \multicolumn{1}{c}{(end)} & \multicolumn{1}{c}{(end)}   \\ \midrule 
                 CGC & $         0.00$ & $         0.00$ & $         0.00$ & $         0.00$ & $   -122802.40$ & $   -122842.41$ & $   -122842.41$ & $   -122842.41$ & $       448.44$ sec   \\ 
                  HC & $   -111864.41$ & $   -111864.41$ & $   -111864.41$ & $   -111864.41$ & $   -111864.41$ & $   -111864.41$ & $   -111864.41$ & $   -111864.41$ & $         0.81$ sec   \\ 
              HC-CGC & $   -111864.41$ & $   -112960.32$ & $   -118399.72$ & $   -121538.50$ & $   -122946.00$ & $   -122946.00$ & $   -122946.00$ & $   -122946.00$ & $       262.59$ sec   \\ 
              ogm-KL & $     -7370.54$ & $     -7370.54$ & $     -7370.54$ & $     -7370.54$ & $   -113473.15$ & $   -114827.73$ & $   -114827.73$ & $   -114827.73$ & $       648.81$ sec   \\ 
    CC-Fusion-HC-CGC & $     -7370.54$ & $   -105484.64$ & $   -117921.79$ & $   -118783.50$ & $   -120181.48$ & $   -120181.48$ & $   -120181.48$ & $   -120181.48$ & $       190.55$ sec   \\ 
     CC-Fusion-HC-MC & $     -7370.54$ & $     -7370.54$ & $   -122657.25$ & $   -123473.60$ & $   -123487.06$ & $   -123487.06$ & $   -123487.06$ & $   -123487.06$ & $       441.68$ sec   \\ 
    CC-Fusion-WS-CGC & $     -7370.54$ & $     -7370.54$ & $   -112777.05$ & $   -117891.30$ & $   -118440.30$ & $   -118440.30$ & $   -118440.30$ & $   -118440.30$ & $       573.10$ sec   \\ 
     CC-Fusion-WS-MC & $     -7370.54$ & $     -7370.54$ & $   -105655.50$ & $   -123095.73$ & $   -123456.74$ & $   -123528.16$ & $   -123528.18$ & $   -123528.18$ & $      1681.71$ sec   \\ 
\cmidrule{1-1} 
           MCR-CCFDB & $     -7370.54$ & $     -7370.54$ & $     -7370.54$ & $     -7370.54$ & $     -7370.54$ & $     -7370.54$ & $    -28415.43$ & $    -28415.43$ & $      2087.31$ sec   \\ 
\cmidrule{1-1} 
           MCI-CCIFD & $     -7370.54$ & $     -7370.54$ & $     -7370.54$ & $     -7370.54$ & $    -46336.74$ & $    -74333.47$ & $   -123528.33$ & $   -123528.33$ & $      1563.32$ sec   \\ 
\bottomrule
\end{tabular}
\end{table}

\begin{table}[H]
\scriptsize
\centering
\caption{knott-3d-550 (gm\_knott\_3d\_119)}
\label{tab:anytimetable-knott-3d-550-gm-knott-3d-119}
\begin{tabular}{lrrrrrrrrr}
\toprule
           algorithm &                                   \multicolumn{8}{c}{value} & \multicolumn{1}{c}{time}   \\  
\cmidrule(lr){2-9}\cmidrule(lr){10-10}   
                     & \multicolumn{1}{c}{(0.5 sec)} & \multicolumn{1}{c}{(1 sec)} & \multicolumn{1}{c}{(10 sec)} & \multicolumn{1}{c}{(60 sec)} & \multicolumn{1}{c}{(300 sec)} & \multicolumn{1}{c}{(600 sec)} & \multicolumn{1}{c}{(1800 sec)} & \multicolumn{1}{c}{(end)} & \multicolumn{1}{c}{(end)}   \\ \midrule 
                 CGC & $     -7059.66$ & $     -7059.66$ & $     -7059.66$ & $     -7059.66$ & $   -124800.28$ & $   -126228.75$ & $   -126228.85$ & $   -126228.85$ & $       661.72$ sec   \\ 
                  HC & $   -108702.30$ & $   -108702.30$ & $   -108702.30$ & $   -108702.30$ & $   -108702.30$ & $   -108702.30$ & $   -108702.30$ & $   -108702.30$ & $         0.88$ sec   \\ 
              HC-CGC & $   -108702.30$ & $   -109469.07$ & $   -120370.69$ & $   -122209.45$ & $   -126103.25$ & $   -126197.12$ & $   -126197.12$ & $   -126197.12$ & $       552.15$ sec   \\ 
              ogm-KL & $     -7059.24$ & $     -7059.24$ & $     -7059.24$ & $     -7059.24$ & $   -118608.63$ & $   -118648.65$ & $   -118648.65$ & $   -118648.65$ & $       527.58$ sec   \\ 
    CC-Fusion-HC-CGC & $     -7059.24$ & $    -99522.23$ & $   -123069.84$ & $   -123434.65$ & $   -123650.49$ & $   -123650.49$ & $   -123650.49$ & $   -123650.49$ & $       191.40$ sec   \\ 
     CC-Fusion-HC-MC & $     -7059.24$ & $     -7059.24$ & $   -126042.98$ & $   -126304.27$ & $   -126309.59$ & $   -126309.59$ & $   -126309.59$ & $   -126309.59$ & $       394.39$ sec   \\ 
    CC-Fusion-WS-CGC & $     -7059.24$ & $     -7059.24$ & $   -119761.15$ & $   -121866.16$ & $   -121990.05$ & $   -121990.05$ & $   -121990.05$ & $   -121990.05$ & $       491.12$ sec   \\ 
     CC-Fusion-WS-MC & $     -7059.24$ & $     -7059.24$ & $   -108170.19$ & $   -125957.10$ & $   -126252.12$ & $   -126323.35$ & $   -126353.72$ & $   -126353.72$ & $      1647.01$ sec   \\ 
\cmidrule{1-1} 
           MCR-CCFDB & $     -7059.24$ & $     -7059.24$ & $     -7059.24$ & $     -7059.24$ & $     -7059.24$ & $     -7059.24$ & $    -36939.97$ & $    -36939.97$ & $      2036.10$ sec   \\ 
\cmidrule{1-1} 
           MCI-CCIFD & $     -7059.24$ & $     -7059.24$ & $     -7059.24$ & $     -7059.24$ & $    -56268.77$ & $    -80219.43$ & $   -125243.06$ & $   -125243.06$ & $      1834.95$ sec   \\ 
\bottomrule
\end{tabular}
\end{table}


\subsection{seg-3d}
\input{anytimetables/seg-3d-PI.tex}
\subsection{socialnets}
\begin{table}[H]
\scriptsize
\centering
\caption{socialnets (soc-sign-Slashdot081106)}
\label{tab:anytimetable-socialnets-soc-sign-Slashdot081106}
\begin{tabular}{lrrrrrrrrr}
\toprule
           algorithm &                                   \multicolumn{8}{c}{value} & \multicolumn{1}{c}{time}   \\  
\cmidrule(lr){2-9}\cmidrule(lr){10-10}   
                     & \multicolumn{1}{c}{(0.5 sec)} & \multicolumn{1}{c}{(1 sec)} & \multicolumn{1}{c}{(10 sec)} & \multicolumn{1}{c}{(60 sec)} & \multicolumn{1}{c}{(300 sec)} & \multicolumn{1}{c}{(600 sec)} & \multicolumn{1}{c}{(1800 sec)} & \multicolumn{1}{c}{(end)} & \multicolumn{1}{c}{(end)}   \\ \midrule 
                 CGC & $    118309.00$ & $    118309.00$ & $    118309.00$ & $    118309.00$ & $    118309.00$ & $     72015.00$ & $     70292.00$ & $     70291.00$ & $      2623.87$ sec   \\ 
                  HC & $    118309.00$ & $    118309.00$ & $    118309.00$ & $    118309.00$ & $    118309.00$ & $    118309.00$ & $    118309.00$ & $    118309.00$ & $        14.38$ sec   \\ 
              HC-CGC & $    118309.00$ & $    118309.00$ & $    118309.00$ & $    118309.00$ & $    118309.00$ & $     79858.00$ & $     70380.00$ & $     70379.00$ & $      1844.41$ sec   \\ 
              ogm-KL & $    105705.00$ & $    105705.00$ & $    105705.00$ & $    105705.00$ & $    105705.00$ & $    105705.00$ & $    105705.00$ & $     70218.00$ & $      7657.22$ sec   \\ 
    CC-Fusion-HC-CGC & $    105705.00$ & $    105705.00$ & $    103384.00$ & $     93115.00$ & $     81730.00$ & $     78782.00$ & $     75963.00$ & $     75963.00$ & $      1801.25$ sec   \\ 
     CC-Fusion-HC-MC & $    105705.00$ & $    105705.00$ & $    103384.00$ & $     93113.00$ & $     81383.00$ & $     78426.00$ & $     74990.00$ & $     74984.00$ & $      1803.21$ sec   \\ 
    CC-Fusion-WS-CGC & $    105705.00$ & $    105705.00$ & $    105705.00$ & $    105705.00$ & $     88087.00$ & $     79398.00$ & $     73357.00$ & $     72872.00$ & $      2058.64$ sec   \\ 
     CC-Fusion-WS-MC & $\infty$ & $\infty$ & $\infty$ & $\infty$ & $\infty$ & $\infty$ & $\infty$ & $          NaN$ & $          NaN$ sec   \\ 
\cmidrule{1-1} 
           MCR-CCFDB & $    105705.00$ & $    105705.00$ & $    105705.00$ & $    105705.00$ & $    105705.00$ & $    105705.00$ & $    105705.00$ & $    105705.00$ & $      3543.81$ sec   \\ 
\cmidrule{1-1} 
           MCI-CCIFD & $    105705.00$ & $    105705.00$ & $    105705.00$ & $    105705.00$ & $    105705.00$ & $    105705.00$ & $    105705.00$ & $    105705.00$ & $      2565.32$ sec   \\ 
\bottomrule
\end{tabular}
\end{table}

\begin{table}[H]
\scriptsize
\centering
\caption{socialnets (soc-sign-epinions)}
\label{tab:anytimetable-socialnets-soc-sign-epinions}
\begin{tabular}{lrrrrrrrrr}
\toprule
           algorithm &                                   \multicolumn{8}{c}{value} & \multicolumn{1}{c}{time}   \\  
\cmidrule(lr){2-9}\cmidrule(lr){10-10}   
                     & \multicolumn{1}{c}{(0.5 sec)} & \multicolumn{1}{c}{(1 sec)} & \multicolumn{1}{c}{(10 sec)} & \multicolumn{1}{c}{(60 sec)} & \multicolumn{1}{c}{(300 sec)} & \multicolumn{1}{c}{(600 sec)} & \multicolumn{1}{c}{(1800 sec)} & \multicolumn{1}{c}{(end)} & \multicolumn{1}{c}{(end)}   \\ \midrule 
                 CGC & $    120967.00$ & $    120967.00$ & $    120967.00$ & $    120967.00$ & $    120967.00$ & $     56474.00$ & $     51475.00$ & $     51474.00$ & $      4722.44$ sec   \\ 
                  HC & $    120967.00$ & $    120967.00$ & $    120967.00$ & $    120967.00$ & $    120967.00$ & $    120967.00$ & $    120967.00$ & $    120967.00$ & $        26.46$ sec   \\ 
              HC-CGC & $    120967.00$ & $    120967.00$ & $    120967.00$ & $    120310.00$ & $    120310.00$ & $     59242.00$ & $     51452.00$ & $     51451.00$ & $      7572.66$ sec   \\ 
              ogm-KL & $\infty$ & $\infty$ & $\infty$ & $\infty$ & $\infty$ & $\infty$ & $\infty$ & $          NaN$ & $          NaN$ sec   \\ 
    CC-Fusion-HC-CGC & $     84017.00$ & $     84017.00$ & $     84017.00$ & $     74417.00$ & $     62396.00$ & $     58162.00$ & $     54446.00$ & $     54441.00$ & $      1805.22$ sec   \\ 
     CC-Fusion-HC-MC & $     84017.00$ & $     84017.00$ & $     84017.00$ & $     74423.00$ & $     62414.00$ & $     57620.00$ & $     54260.00$ & $     54260.00$ & $      1805.96$ sec   \\ 
    CC-Fusion-WS-CGC & $     84017.00$ & $     84017.00$ & $     84017.00$ & $     84017.00$ & $     56873.00$ & $     53048.00$ & $     51835.00$ & $     51835.00$ & $      1835.15$ sec   \\ 
     CC-Fusion-WS-MC & $\infty$ & $\infty$ & $\infty$ & $\infty$ & $\infty$ & $\infty$ & $\infty$ & $          NaN$ & $          NaN$ sec   \\ 
\cmidrule{1-1} 
           MCR-CCFDB & $     84017.00$ & $     84017.00$ & $     84017.00$ & $     84017.00$ & $     84017.00$ & $     84017.00$ & $     84017.00$ & $     84017.00$ & $      3528.62$ sec   \\ 
\cmidrule{1-1} 
           MCI-CCIFD & $     84017.00$ & $     84017.00$ & $     84017.00$ & $     84017.00$ & $     84017.00$ & $     84017.00$ & $     84017.00$ & $     84017.00$ & $      3372.31$ sec   \\ 
\bottomrule
\end{tabular}
\end{table}


\subsection{normalized socialnets}
\begin{table}[H]
\scriptsize
\centering
\caption{normalizedsocialnets (soc-sign-Slashdot081106-n)}
\label{tab:anytimetable-normalizedsocialnets-soc-sign-Slashdot081106-n}
\begin{tabular}{lrrrrrrrrr}
\toprule
           algorithm &                                   \multicolumn{8}{c}{value} & \multicolumn{1}{c}{time}   \\  
\cmidrule(lr){2-9}\cmidrule(lr){10-10}   
                     & \multicolumn{1}{c}{(0.5 sec)} & \multicolumn{1}{c}{(1 sec)} & \multicolumn{1}{c}{(10 sec)} & \multicolumn{1}{c}{(60 sec)} & \multicolumn{1}{c}{(300 sec)} & \multicolumn{1}{c}{(600 sec)} & \multicolumn{1}{c}{(1800 sec)} & \multicolumn{1}{c}{(end)} & \multicolumn{1}{c}{(end)}   \\ \midrule 
                 CGC & $      6119.38$ & $      6119.38$ & $      6119.38$ & $      6119.38$ & $      6119.38$ & $      6119.38$ & $      6119.38$ & $      3015.97$ & $      2532.15$ sec   \\ 
                  HC & $      6119.38$ & $      6119.38$ & $      4806.04$ & $      4806.04$ & $      4806.04$ & $      4806.04$ & $      4806.04$ & $      4806.04$ & $         8.04$ sec   \\ 
              HC-CGC & $      6119.38$ & $      6119.38$ & $      4806.04$ & $      4806.04$ & $      4806.04$ & $      4806.04$ & $      4806.04$ & $      2951.84$ & $      3164.60$ sec   \\ 
              ogm-KL & $      5097.55$ & $      5097.55$ & $      5097.55$ & $      5097.55$ & $      5097.55$ & $      5097.55$ & $      5097.55$ & $      2893.72$ & $      5525.36$ sec   \\ 
    CC-Fusion-HC-CGC & $      5097.55$ & $      5097.55$ & $      4950.44$ & $      4339.65$ & $      3755.24$ & $      3566.15$ & $      3382.51$ & $      3382.51$ & $      1804.91$ sec   \\ 
     CC-Fusion-HC-MC & $      5097.55$ & $      5097.55$ & $      4950.44$ & $      4339.57$ & $      3735.93$ & $      3535.43$ & $      3334.11$ & $      3333.92$ & $      1805.10$ sec   \\ 
    CC-Fusion-WS-CGC & $      5097.55$ & $      5097.55$ & $      5097.55$ & $      5097.55$ & $      5097.55$ & $      5097.55$ & $      4609.58$ & $      4139.31$ & $      1863.73$ sec   \\ 
     CC-Fusion-WS-MC & $\infty$ & $\infty$ & $\infty$ & $\infty$ & $\infty$ & $\infty$ & $\infty$ & $          NaN$ & $          NaN$ sec   \\ 
\cmidrule{1-1} 
           MCR-CCFDB & $      5097.55$ & $      5097.55$ & $      5097.55$ & $      5097.55$ & $      5097.55$ & $      5097.55$ & $      5097.55$ & $      5097.55$ & $      3705.84$ sec   \\ 
\cmidrule{1-1} 
           MCI-CCIFD & $      5097.55$ & $      5097.55$ & $      5097.55$ & $      5097.55$ & $      5097.55$ & $      5097.55$ & $      5097.55$ & $      5097.55$ & $      2623.51$ sec   \\ 
\bottomrule
\end{tabular}
\end{table}

\begin{table}[H]
\scriptsize
\centering
\caption{normalizedsocialnets (soc-sign-epinions-n)}
\label{tab:anytimetable-normalizedsocialnets-soc-sign-epinions-n}
\begin{tabular}{lrrrrrrrrr}
\toprule
           algorithm &                                   \multicolumn{8}{c}{value} & \multicolumn{1}{c}{time}   \\  
\cmidrule(lr){2-9}\cmidrule(lr){10-10}   
                     & \multicolumn{1}{c}{(0.5 sec)} & \multicolumn{1}{c}{(1 sec)} & \multicolumn{1}{c}{(10 sec)} & \multicolumn{1}{c}{(60 sec)} & \multicolumn{1}{c}{(300 sec)} & \multicolumn{1}{c}{(600 sec)} & \multicolumn{1}{c}{(1800 sec)} & \multicolumn{1}{c}{(end)} & \multicolumn{1}{c}{(end)}   \\ \midrule 
                 CGC & $     10488.76$ & $     10488.76$ & $     10488.76$ & $     10488.76$ & $     10488.76$ & $     10488.76$ & $      2138.66$ & $      2078.23$ & $      3010.56$ sec   \\ 
                  HC & $     10488.76$ & $     10488.76$ & $      9944.79$ & $      9944.79$ & $      9944.79$ & $      9944.79$ & $      9944.79$ & $      9944.79$ & $        12.31$ sec   \\ 
              HC-CGC & $\infty$ & $\infty$ & $\infty$ & $\infty$ & $\infty$ & $\infty$ & $\infty$ & $          NaN$ & $          NaN$ sec   \\ 
              ogm-KL & $\infty$ & $\infty$ & $\infty$ & $\infty$ & $\infty$ & $\infty$ & $\infty$ & $          NaN$ & $          NaN$ sec   \\ 
    CC-Fusion-HC-CGC & $      3166.89$ & $      3166.89$ & $      3166.89$ & $      2971.76$ & $      2659.45$ & $      2547.93$ & $      2439.03$ & $      2439.03$ & $      1803.60$ sec   \\ 
     CC-Fusion-HC-MC & $      3166.89$ & $      3166.89$ & $      3114.15$ & $      2947.83$ & $      2638.25$ & $      2530.63$ & $      2417.37$ & $      2417.37$ & $      1803.39$ sec   \\ 
    CC-Fusion-WS-CGC & $      3166.89$ & $      3166.89$ & $      3166.89$ & $      3166.89$ & $      3166.89$ & $      2993.10$ & $      2736.09$ & $      2644.70$ & $      2173.13$ sec   \\ 
     CC-Fusion-WS-MC & $\infty$ & $\infty$ & $\infty$ & $\infty$ & $\infty$ & $\infty$ & $\infty$ & $          NaN$ & $          NaN$ sec   \\ 
\cmidrule{1-1} 
           MCR-CCFDB & $      3166.89$ & $      3166.89$ & $      3166.89$ & $      3166.89$ & $      3166.89$ & $      3166.89$ & $      3166.89$ & $      3166.89$ & $      4934.37$ sec   \\ 
\cmidrule{1-1} 
           MCI-CCIFD & $      3166.89$ & $      3166.89$ & $      3166.89$ & $      3166.89$ & $      3166.89$ & $      3166.89$ & $      3166.89$ & $      3166.89$ & $      3551.38$ sec   \\ 
\bottomrule
\end{tabular}
\end{table}


\subsection{modularity-clustering}
\begin{table}[H]
\scriptsize
\centering
\caption{modularity-clustering (adjnoun)}
\label{tab:anytimetable-modularity-clustering-adjnoun}
\begin{tabular}{lrrrrrrrrr}
\toprule
           algorithm &                                   \multicolumn{8}{c}{value} & \multicolumn{1}{c}{time}   \\  
\cmidrule(lr){2-9}\cmidrule(lr){10-10}   
                     & \multicolumn{1}{c}{(0.5 sec)} & \multicolumn{1}{c}{(1 sec)} & \multicolumn{1}{c}{(10 sec)} & \multicolumn{1}{c}{(60 sec)} & \multicolumn{1}{c}{(300 sec)} & \multicolumn{1}{c}{(600 sec)} & \multicolumn{1}{c}{(1800 sec)} & \multicolumn{1}{c}{(end)} & \multicolumn{1}{c}{(end)}   \\ \midrule 
          PIVIT-BOEM & $       0.0162$ & $       0.0162$ & $       0.0162$ & $       0.0162$ & $       0.0162$ & $       0.0162$ & $       0.0162$ & $       0.0162$ & $         0.04$ sec   \\ 
                 CGC & $      -0.2582$ & $      -0.2582$ & $      -0.2582$ & $      -0.2582$ & $      -0.2582$ & $      -0.2582$ & $      -0.2582$ & $      -0.2582$ & $         0.37$ sec   \\ 
                  HC & $      -0.1457$ & $      -0.1457$ & $      -0.1457$ & $      -0.1457$ & $      -0.1457$ & $      -0.1457$ & $      -0.1457$ & $      -0.1457$ & $         0.00$ sec   \\ 
              HC-CGC & $      -0.2742$ & $      -0.2742$ & $      -0.2742$ & $      -0.2742$ & $      -0.2742$ & $      -0.2742$ & $      -0.2742$ & $      -0.2742$ & $         0.24$ sec   \\ 
              ogm-KL & $      -0.2980$ & $      -0.2980$ & $      -0.2980$ & $      -0.2980$ & $      -0.2980$ & $      -0.2980$ & $      -0.2980$ & $      -0.2980$ & $         0.01$ sec   \\ 
    CC-Fusion-HC-CGC & $      -0.2559$ & $      -0.2595$ & $      -0.2792$ & $      -0.2792$ & $      -0.2792$ & $      -0.2792$ & $      -0.2792$ & $      -0.2792$ & $         3.47$ sec   \\ 
     CC-Fusion-HC-MC & $      -0.0896$ & $      -0.0896$ & $      -0.0896$ & $      -0.2732$ & $      -0.2948$ & $      -0.2948$ & $      -0.2948$ & $      -0.2948$ & $        77.32$ sec   \\ 
    CC-Fusion-WS-CGC & $      -0.1774$ & $      -0.2015$ & $      -0.2527$ & $      -0.2527$ & $      -0.2527$ & $      -0.2527$ & $      -0.2527$ & $      -0.2527$ & $         2.15$ sec   \\ 
     CC-Fusion-WS-MC & $      -0.1220$ & $      -0.1220$ & $      -0.1220$ & $      -0.2870$ & $      -0.2891$ & $      -0.2891$ & $      -0.2891$ & $      -0.2891$ & $        68.53$ sec   \\ 
\cmidrule{1-1} 
           MCR-CCFDB & $       0.0000$ & $       0.0000$ & $      -0.1707$ & $      -0.1707$ & $      -0.1707$ & $      -0.1707$ & $      -0.1707$ & $      -0.1707$ & $        12.84$ sec   \\ 
\cmidrule{1-1} 
           MCI-CCIFD & $       0.0000$ & $       0.0000$ & $       0.0000$ & $       0.0000$ & $       0.0000$ & $       0.0000$ & $       0.0000$ & $       0.0000$ & $      1800.60$ sec   \\ 
\bottomrule
\end{tabular}
\end{table}

\begin{table}[H]
\scriptsize
\centering
\caption{modularity-clustering (dolphins)}
\label{tab:anytimetable-modularity-clustering-dolphins}
\begin{tabular}{lrrrrrrrrr}
\toprule
           algorithm &                                   \multicolumn{8}{c}{value} & \multicolumn{1}{c}{time}   \\  
\cmidrule(lr){2-9}\cmidrule(lr){10-10}   
                     & \multicolumn{1}{c}{(0.5 sec)} & \multicolumn{1}{c}{(1 sec)} & \multicolumn{1}{c}{(10 sec)} & \multicolumn{1}{c}{(60 sec)} & \multicolumn{1}{c}{(300 sec)} & \multicolumn{1}{c}{(600 sec)} & \multicolumn{1}{c}{(1800 sec)} & \multicolumn{1}{c}{(end)} & \multicolumn{1}{c}{(end)}   \\ \midrule 
          PIVIT-BOEM & $       0.0214$ & $       0.0214$ & $       0.0214$ & $       0.0214$ & $       0.0214$ & $       0.0214$ & $       0.0214$ & $       0.0214$ & $         0.02$ sec   \\ 
                 CGC & $      -0.4974$ & $      -0.4974$ & $      -0.4974$ & $      -0.4974$ & $      -0.4974$ & $      -0.4974$ & $      -0.4974$ & $      -0.4974$ & $         0.05$ sec   \\ 
                  HC & $      -0.2775$ & $      -0.2775$ & $      -0.2775$ & $      -0.2775$ & $      -0.2775$ & $      -0.2775$ & $      -0.2775$ & $      -0.2775$ & $         0.00$ sec   \\ 
              HC-CGC & $      -0.4928$ & $      -0.4928$ & $      -0.4928$ & $      -0.4928$ & $      -0.4928$ & $      -0.4928$ & $      -0.4928$ & $      -0.4928$ & $         0.03$ sec   \\ 
              ogm-KL & $      -0.5268$ & $      -0.5268$ & $      -0.5268$ & $      -0.5268$ & $      -0.5268$ & $      -0.5268$ & $      -0.5268$ & $      -0.5268$ & $         0.00$ sec   \\ 
    CC-Fusion-HC-CGC & $      -0.5148$ & $      -0.5148$ & $      -0.5148$ & $      -0.5148$ & $      -0.5148$ & $      -0.5148$ & $      -0.5148$ & $      -0.5148$ & $         0.24$ sec   \\ 
     CC-Fusion-HC-MC & $      -0.5067$ & $      -0.5265$ & $      -0.5268$ & $      -0.5268$ & $      -0.5268$ & $      -0.5268$ & $      -0.5268$ & $      -0.5268$ & $         1.77$ sec   \\ 
    CC-Fusion-WS-CGC & $      -0.4646$ & $      -0.4646$ & $      -0.4646$ & $      -0.4646$ & $      -0.4646$ & $      -0.4646$ & $      -0.4646$ & $      -0.4646$ & $         0.12$ sec   \\ 
     CC-Fusion-WS-MC & $      -0.4734$ & $      -0.4997$ & $      -0.5246$ & $      -0.5246$ & $      -0.5246$ & $      -0.5246$ & $      -0.5246$ & $      -0.5246$ & $         2.03$ sec   \\ 
\cmidrule{1-1} 
           MCR-CCFDB & $      -0.5192$ & $      -0.5192$ & $      -0.5192$ & $      -0.5192$ & $      -0.5192$ & $      -0.5192$ & $      -0.5192$ & $      -0.5192$ & $         0.32$ sec   \\ 
\cmidrule{1-1} 
           MCI-CCIFD & $       0.0000$ & $       0.0000$ & $      -0.5285$ & $      -0.5285$ & $      -0.5285$ & $      -0.5285$ & $      -0.5285$ & $      -0.5285$ & $         8.06$ sec   \\ 
\bottomrule
\end{tabular}
\end{table}

\begin{table}[H]
\scriptsize
\centering
\caption{modularity-clustering (football)}
\label{tab:anytimetable-modularity-clustering-football}
\begin{tabular}{lrrrrrrrrr}
\toprule
           algorithm &                                   \multicolumn{8}{c}{value} & \multicolumn{1}{c}{time}   \\  
\cmidrule(lr){2-9}\cmidrule(lr){10-10}   
                     & \multicolumn{1}{c}{(0.5 sec)} & \multicolumn{1}{c}{(1 sec)} & \multicolumn{1}{c}{(10 sec)} & \multicolumn{1}{c}{(60 sec)} & \multicolumn{1}{c}{(300 sec)} & \multicolumn{1}{c}{(600 sec)} & \multicolumn{1}{c}{(1800 sec)} & \multicolumn{1}{c}{(end)} & \multicolumn{1}{c}{(end)}   \\ \midrule 
          PIVIT-BOEM & $       0.0088$ & $       0.0088$ & $       0.0088$ & $       0.0088$ & $       0.0088$ & $       0.0088$ & $       0.0088$ & $       0.0088$ & $         0.02$ sec   \\ 
                 CGC & $      -0.5324$ & $      -0.5324$ & $      -0.5324$ & $      -0.5324$ & $      -0.5324$ & $      -0.5324$ & $      -0.5324$ & $      -0.5324$ & $         0.37$ sec   \\ 
                  HC & $      -0.3062$ & $      -0.3062$ & $      -0.3062$ & $      -0.3062$ & $      -0.3062$ & $      -0.3062$ & $      -0.3062$ & $      -0.3062$ & $         0.01$ sec   \\ 
              HC-CGC & $      -0.5687$ & $      -0.5687$ & $      -0.5687$ & $      -0.5687$ & $      -0.5687$ & $      -0.5687$ & $      -0.5687$ & $      -0.5687$ & $         0.12$ sec   \\ 
              ogm-KL & $      -0.6046$ & $      -0.6046$ & $      -0.6046$ & $      -0.6046$ & $      -0.6046$ & $      -0.6046$ & $      -0.6046$ & $      -0.6046$ & $         0.01$ sec   \\ 
    CC-Fusion-HC-CGC & $      -0.5010$ & $      -0.5010$ & $      -0.5010$ & $      -0.5010$ & $      -0.5010$ & $      -0.5010$ & $      -0.5010$ & $      -0.5010$ & $         0.97$ sec   \\ 
     CC-Fusion-HC-MC & $      -0.0969$ & $      -0.0969$ & $      -0.4938$ & $      -0.4938$ & $      -0.4938$ & $      -0.4938$ & $      -0.4938$ & $      -0.4938$ & $         6.71$ sec   \\ 
    CC-Fusion-WS-CGC & $      -0.4580$ & $      -0.4580$ & $      -0.4580$ & $      -0.4580$ & $      -0.4580$ & $      -0.4580$ & $      -0.4580$ & $      -0.4580$ & $         0.64$ sec   \\ 
     CC-Fusion-WS-MC & $      -0.1040$ & $      -0.1040$ & $      -0.5008$ & $      -0.5008$ & $      -0.5008$ & $      -0.5008$ & $      -0.5008$ & $      -0.5008$ & $        10.61$ sec   \\ 
\cmidrule{1-1} 
           MCR-CCFDB & $       0.0000$ & $       0.0000$ & $      -0.5924$ & $      -0.5924$ & $      -0.5924$ & $      -0.5924$ & $      -0.5924$ & $      -0.5924$ & $         6.17$ sec   \\ 
\cmidrule{1-1} 
           MCI-CCIFD & $       0.0000$ & $       0.0000$ & $      -0.6033$ & $      -0.6046$ & $      -0.6046$ & $      -0.6046$ & $      -0.6046$ & $      -0.6046$ & $        12.47$ sec   \\ 
\bottomrule
\end{tabular}
\end{table}

\begin{table}[H]
\scriptsize
\centering
\caption{modularity-clustering (karate)}
\label{tab:anytimetable-modularity-clustering-karate}
\begin{tabular}{lrrrrrrrrr}
\toprule
           algorithm &                                   \multicolumn{8}{c}{value} & \multicolumn{1}{c}{time}   \\  
\cmidrule(lr){2-9}\cmidrule(lr){10-10}   
                     & \multicolumn{1}{c}{(0.5 sec)} & \multicolumn{1}{c}{(1 sec)} & \multicolumn{1}{c}{(10 sec)} & \multicolumn{1}{c}{(60 sec)} & \multicolumn{1}{c}{(300 sec)} & \multicolumn{1}{c}{(600 sec)} & \multicolumn{1}{c}{(1800 sec)} & \multicolumn{1}{c}{(end)} & \multicolumn{1}{c}{(end)}   \\ \midrule 
          PIVIT-BOEM & $       0.0498$ & $       0.0498$ & $       0.0498$ & $       0.0498$ & $       0.0498$ & $       0.0498$ & $       0.0498$ & $       0.0498$ & $         0.01$ sec   \\ 
                 CGC & $      -0.3715$ & $      -0.3715$ & $      -0.3715$ & $      -0.3715$ & $      -0.3715$ & $      -0.3715$ & $      -0.3715$ & $      -0.3715$ & $         0.01$ sec   \\ 
                  HC & $      -0.1362$ & $      -0.1362$ & $      -0.1362$ & $      -0.1362$ & $      -0.1362$ & $      -0.1362$ & $      -0.1362$ & $      -0.1362$ & $         0.00$ sec   \\ 
              HC-CGC & $      -0.3991$ & $      -0.3991$ & $      -0.3991$ & $      -0.3991$ & $      -0.3991$ & $      -0.3991$ & $      -0.3991$ & $      -0.3991$ & $         0.00$ sec   \\ 
              ogm-KL & $      -0.4198$ & $      -0.4198$ & $      -0.4198$ & $      -0.4198$ & $      -0.4198$ & $      -0.4198$ & $      -0.4198$ & $      -0.4198$ & $         0.00$ sec   \\ 
    CC-Fusion-HC-CGC & $      -0.4020$ & $      -0.4020$ & $      -0.4020$ & $      -0.4020$ & $      -0.4020$ & $      -0.4020$ & $      -0.4020$ & $      -0.4020$ & $         0.07$ sec   \\ 
     CC-Fusion-HC-MC & $      -0.4020$ & $      -0.4020$ & $      -0.4020$ & $      -0.4020$ & $      -0.4020$ & $      -0.4020$ & $      -0.4020$ & $      -0.4020$ & $         0.61$ sec   \\ 
    CC-Fusion-WS-CGC & $      -0.4020$ & $      -0.4020$ & $      -0.4020$ & $      -0.4020$ & $      -0.4020$ & $      -0.4020$ & $      -0.4020$ & $      -0.4020$ & $         0.05$ sec   \\ 
     CC-Fusion-WS-MC & $      -0.3153$ & $      -0.3153$ & $      -0.3153$ & $      -0.3153$ & $      -0.3153$ & $      -0.3153$ & $      -0.3153$ & $      -0.3153$ & $         0.49$ sec   \\ 
\cmidrule{1-1} 
           MCR-CCFDB & $      -0.4198$ & $      -0.4198$ & $      -0.4198$ & $      -0.4198$ & $      -0.4198$ & $      -0.4198$ & $      -0.4198$ & $      -0.4198$ & $         0.02$ sec   \\ 
\cmidrule{1-1} 
           MCI-CCIFD & $      -0.4198$ & $      -0.4198$ & $      -0.4198$ & $      -0.4198$ & $      -0.4198$ & $      -0.4198$ & $      -0.4198$ & $      -0.4198$ & $         0.05$ sec   \\ 
\bottomrule
\end{tabular}
\end{table}

\begin{table}[H]
\scriptsize
\centering
\caption{modularity-clustering (lesmis)}
\label{tab:anytimetable-modularity-clustering-lesmis}
\begin{tabular}{lrrrrrrrrr}
\toprule
           algorithm &                                   \multicolumn{8}{c}{value} & \multicolumn{1}{c}{time}   \\  
\cmidrule(lr){2-9}\cmidrule(lr){10-10}   
                     & \multicolumn{1}{c}{(0.5 sec)} & \multicolumn{1}{c}{(1 sec)} & \multicolumn{1}{c}{(10 sec)} & \multicolumn{1}{c}{(60 sec)} & \multicolumn{1}{c}{(300 sec)} & \multicolumn{1}{c}{(600 sec)} & \multicolumn{1}{c}{(1800 sec)} & \multicolumn{1}{c}{(end)} & \multicolumn{1}{c}{(end)}   \\ \midrule 
          PIVIT-BOEM & $       0.0237$ & $       0.0237$ & $       0.0237$ & $       0.0237$ & $       0.0237$ & $       0.0237$ & $       0.0237$ & $       0.0237$ & $         0.01$ sec   \\ 
                 CGC & $      -0.5336$ & $      -0.5336$ & $      -0.5336$ & $      -0.5336$ & $      -0.5336$ & $      -0.5336$ & $      -0.5336$ & $      -0.5336$ & $         0.05$ sec   \\ 
                  HC & $      -0.3014$ & $      -0.3014$ & $      -0.3014$ & $      -0.3014$ & $      -0.3014$ & $      -0.3014$ & $      -0.3014$ & $      -0.3014$ & $         0.00$ sec   \\ 
              HC-CGC & $      -0.5346$ & $      -0.5346$ & $      -0.5346$ & $      -0.5346$ & $      -0.5346$ & $      -0.5346$ & $      -0.5346$ & $      -0.5346$ & $         0.05$ sec   \\ 
              ogm-KL & $      -0.5443$ & $      -0.5443$ & $      -0.5443$ & $      -0.5443$ & $      -0.5443$ & $      -0.5443$ & $      -0.5443$ & $      -0.5443$ & $         0.00$ sec   \\ 
    CC-Fusion-HC-CGC & $      -0.5428$ & $      -0.5428$ & $      -0.5428$ & $      -0.5428$ & $      -0.5428$ & $      -0.5428$ & $      -0.5428$ & $      -0.5428$ & $         0.38$ sec   \\ 
     CC-Fusion-HC-MC & $      -0.4951$ & $      -0.4954$ & $      -0.4954$ & $      -0.4954$ & $      -0.4954$ & $      -0.4954$ & $      -0.4954$ & $      -0.4954$ & $         1.52$ sec   \\ 
    CC-Fusion-WS-CGC & $      -0.5189$ & $      -0.5189$ & $      -0.5189$ & $      -0.5189$ & $      -0.5189$ & $      -0.5189$ & $      -0.5189$ & $      -0.5189$ & $         0.19$ sec   \\ 
     CC-Fusion-WS-MC & $      -0.4859$ & $      -0.5455$ & $      -0.5600$ & $      -0.5600$ & $      -0.5600$ & $      -0.5600$ & $      -0.5600$ & $      -0.5600$ & $         5.74$ sec   \\ 
\cmidrule{1-1} 
           MCR-CCFDB & $      -0.5568$ & $      -0.5568$ & $      -0.5568$ & $      -0.5568$ & $      -0.5568$ & $      -0.5568$ & $      -0.5568$ & $      -0.5568$ & $         0.46$ sec   \\ 
\cmidrule{1-1} 
           MCI-CCIFD & $      -0.5286$ & $      -0.5600$ & $      -0.5600$ & $      -0.5600$ & $      -0.5600$ & $      -0.5600$ & $      -0.5600$ & $      -0.5600$ & $         0.53$ sec   \\ 
\bottomrule
\end{tabular}
\end{table}

\begin{table}[H]
\scriptsize
\centering
\caption{modularity-clustering (polbooks)}
\label{tab:anytimetable-modularity-clustering-polbooks}
\begin{tabular}{lrrrrrrrrr}
\toprule
           algorithm &                                   \multicolumn{8}{c}{value} & \multicolumn{1}{c}{time}   \\  
\cmidrule(lr){2-9}\cmidrule(lr){10-10}   
                     & \multicolumn{1}{c}{(0.5 sec)} & \multicolumn{1}{c}{(1 sec)} & \multicolumn{1}{c}{(10 sec)} & \multicolumn{1}{c}{(60 sec)} & \multicolumn{1}{c}{(300 sec)} & \multicolumn{1}{c}{(600 sec)} & \multicolumn{1}{c}{(1800 sec)} & \multicolumn{1}{c}{(end)} & \multicolumn{1}{c}{(end)}   \\ \midrule 
          PIVIT-BOEM & $       0.0135$ & $       0.0135$ & $       0.0135$ & $       0.0135$ & $       0.0135$ & $       0.0135$ & $       0.0135$ & $       0.0135$ & $         0.02$ sec   \\ 
                 CGC & $      -0.4963$ & $      -0.4963$ & $      -0.4963$ & $      -0.4963$ & $      -0.4963$ & $      -0.4963$ & $      -0.4963$ & $      -0.4963$ & $         0.31$ sec   \\ 
                  HC & $      -0.1952$ & $      -0.1952$ & $      -0.1952$ & $      -0.1952$ & $      -0.1952$ & $      -0.1952$ & $      -0.1952$ & $      -0.1952$ & $         0.00$ sec   \\ 
              HC-CGC & $      -0.4972$ & $      -0.4972$ & $      -0.4972$ & $      -0.4972$ & $      -0.4972$ & $      -0.4972$ & $      -0.4972$ & $      -0.4972$ & $         0.39$ sec   \\ 
              ogm-KL & $      -0.5226$ & $      -0.5226$ & $      -0.5226$ & $      -0.5226$ & $      -0.5226$ & $      -0.5226$ & $      -0.5226$ & $      -0.5226$ & $         0.00$ sec   \\ 
    CC-Fusion-HC-CGC & $      -0.4424$ & $      -0.4424$ & $      -0.4424$ & $      -0.4424$ & $      -0.4424$ & $      -0.4424$ & $      -0.4424$ & $      -0.4424$ & $         0.44$ sec   \\ 
     CC-Fusion-HC-MC & $      -0.1401$ & $      -0.1560$ & $      -0.5221$ & $      -0.5221$ & $      -0.5221$ & $      -0.5221$ & $      -0.5221$ & $      -0.5221$ & $         4.32$ sec   \\ 
    CC-Fusion-WS-CGC & $      -0.4569$ & $      -0.4569$ & $      -0.4569$ & $      -0.4569$ & $      -0.4569$ & $      -0.4569$ & $      -0.4569$ & $      -0.4569$ & $         0.16$ sec   \\ 
     CC-Fusion-WS-MC & $      -0.1596$ & $      -0.3772$ & $      -0.4614$ & $      -0.4614$ & $      -0.4614$ & $      -0.4614$ & $      -0.4614$ & $      -0.4614$ & $         2.21$ sec   \\ 
\cmidrule{1-1} 
           MCR-CCFDB & $       0.0000$ & $       0.0000$ & $      -0.5252$ & $      -0.5252$ & $      -0.5252$ & $      -0.5252$ & $      -0.5252$ & $      -0.5252$ & $         7.80$ sec   \\ 
\cmidrule{1-1} 
           MCI-CCIFD & $       0.0000$ & $       0.0000$ & $       0.0000$ & $       0.0000$ & $      -0.0980$ & $      -0.0980$ & $      -0.0980$ & $      -0.4986$ & $      1800.57$ sec   \\ 
\bottomrule
\end{tabular}
\end{table}




\end{document}
