\documentclass[10pt,letterpaper]{article}

\usepackage{eso-pic}
\usepackage{cvpr}
\usepackage{times}
\usepackage{epsfig}
\usepackage{graphicx}
\usepackage{amsmath}
\usepackage{amssymb}
\usepackage{xspace}
\usepackage{booktabs}
\usepackage{placeins}


% Include other packages here, before hyperref.
\usepackage{my_macros}
\usepackage{paralist}
\usepackage{amsthm}
\usepackage{dirtytalk}
%\usepackage{framed}
\usepackage{caption}
\usepackage{subcaption}
\usepackage{algorithm}
%\usepackage{algorithmic}
\usepackage{algpseudocode}
\usepackage{setspace}
%for tikz
\usepackage{tikz,pgfplots}
\usepackage{pgfplotstable}
\usepackage{filecontents}
%\usepackage{scrextend}
\usepackage[pagebackref=true,breaklinks=true,letterpaper=true,colorlinks,bookmarks=false]{hyperref}

\newcommand{\footlabel}[2]{%
    \addtocounter{footnote}{1}%
    \footnotetext[\thefootnote]{%
        \addtocounter{footnote}{-1}%
        \refstepcounter{footnote}\label{#1}%
        #2%
    }%
    $^{\ref{#1}}$%
}

\newcommand{\footref}[1]{%
    $^{\ref{#1}}$%
}

\usetikzlibrary{arrows,positioning,automata,shadows,fit,shapes}
\usetikzlibrary{arrows,petri,topaths}
\usetikzlibrary{positioning,fit,calc}
\usetikzlibrary{shapes.arrows,chains,decorations.pathreplacing,fadings}
\usetikzlibrary{calc, matrix, backgrounds}

\pgfplotsset{every axis/.append style={
  every axis y label/.style = {at={(ticklabel cs:0.5)}, rotate=90, anchor=south},
  axis x line = {bottom},
  axis y line = {left},
  tick align = outside,
  ymajorgrids = true,
%  legend style = {draw=none, at={(1.05, 0.5)}, anchor=west, font=\small},
  legend style = {font=\tiny},
  legend columns = 1,
  every axis plot/.append style = {line width=1pt},
  label style = {font=\small},
  tick label style={font=\small},
  scaled ticks = false,
}}

% Two Colored Circle Split 
\makeatletter
\tikzset{circle split part fill/.style  args={#1,#2}{%
 alias=tmp@name, 
  postaction={%
    insert path={
     \pgfextra{% 
     \pgfpointdiff{\pgfpointanchor{\pgf@node@name}{center}}%
                  {\pgfpointanchor{\pgf@node@name}{east}}%            
     \pgfmathsetmacro\insiderad{\pgf@x}
      \fill[#1] (\pgf@node@name.base) ([xshift=-\pgflinewidth]\pgf@node@name.east) arc
                          (0:180:\insiderad-\pgflinewidth)--cycle;
      \fill[#2] (\pgf@node@name.base) ([xshift=\pgflinewidth]\pgf@node@name.west)  arc
                           (180:360:\insiderad-\pgflinewidth)--cycle;            
         }}}}}  
 \makeatother  

%\usepackage{tkz-berge}
\DeclareMathOperator*{\argmin}{arg\,min}
\DeclareMathOperator*{\argmax}{arg\,max}
\definecolor{shadecolor}{rgb}{0.01,0.199,0.1}
\usepackage{xargs} 
\newtheorem{theorem}{Theorem}
\newtheorem{remark}{Remark}

% If you comment hyperref and then uncomment it, you should delete
% egpaper.aux before re-running latex.  (Or just hit 'q' on the first latex
% run, let it finish, and you should be clear).
\usepackage{bm}% ändert \boldsymbol
\def\arraystretch{0.8}
\renewcommand{\tabcolsep}{2pt}
\newcommand{\thickline}{2pt}
\newcommand{\scatterplotpath}{./scatterplots/}

\usepackage[colorinlistoftodos,prependcaption,textsize=tiny]{todonotes}
\newcommandx{\unsure}[2][1=]{\todo[linecolor=red,backgroundcolor=red!25,bordercolor=red,#1]{#2}}
\newcommandx{\change}[2][1=]{\todo[linecolor=blue,backgroundcolor=blue!25,bordercolor=blue,#1]{#2}}
\newcommandx{\info}[2][1=]{\todo[linecolor=OliveGreen,backgroundcolor=OliveGreen!25,bordercolor=OliveGreen,#1]{#2}}
\newcommandx{\improvement}[2][1=]{\todo[linecolor=Plum,backgroundcolor=Plum!25,bordercolor=Plum,#1]{#2}}
\newcommandx{\thiswillnotshow}[2][1=]{\todo[disable,#1]{#2}}
\newcommand{\OR}{\textrm{ or }}

% \cvprfinalcopy % *** Uncomment this line for the final submission


\def\cvprPaperID{****} % *** Enter the CVPR Paper ID here
\def\httilde{\mbox{\tt\raisebox{-.5ex}{\symbol{126}}}}

\newcommand{\realplot}[1]{
\begin{tikzpicture}
    \begin{axis}
        \addlegendimage{empty legend}\addlegendentry{Matrix #1}
        \addplot {0};
        \addplot {1};
        \addplot {2};
        \addplot {3};
    \end{axis}
\end{tikzpicture}}  

% \newcommand{\anytimeplot}[1]{
%   \addplot[color=brown,mark=x] table[x=time,y=HC]{data3/knott-3d-300.data};              \addlegendentry{HC}
%   \addplot[color=black,mark=x] table[x=time,y=HC-CGC]{data3/knott-3d-300.data};          \addlegendentry{HC-CGC}
%   \addplot[color=red,mark=x] table[x=time,y=CGC]{data3/knott-3d-300.data};               \addlegendentry{CGC}
%   \addplot[color=gray,mark=o] table[x=time, y=ogm-KL]{data3/knott-3d-300.data};          \addlegendentry{KL}
%   \addplot[color=purple,mark=x] table[x=time, y=MCR-CCFDB]{data3/knott-3d-300.data};     \addlegendentry{MC-R}
%   \addplot[color=blue,mark=x] table[x=time, y=MCI-CCIFD]{data3/knott-3d-300.data};       \addlegendentry{MC-I}
%   \addplot[color=green,mark=x] table[x=time, y=DYNCC-HC-MC]{data3/knott-3d-300.data};    \addlegendentry{Fusion-HC-MC}
%   \addplot[color=cyan,mark=x] table[x=time, y=DYNCC-HC-CGC]{data3/knott-3d-300.data};    \addlegendentry{Fusion-HC-CGC}
%   \addplot[color=orange,mark=x] table[x=time, y=DYNCC-WS-MC]{data3/knott-3d-300.data};  \addlegendentry{Fusion-WS-MC}
%   \addplot[color=orange,mark=x] table[x=time, y=DYNCC-WS-CGC]{data3/knott-3d-300.data};  \addlegendentry{Fusion-WS-CGC}
% }
\newcommand{\anytimeplot}[1]{
  \addplot[color=brown,mark=x] table[x=time,y=HC]{data3/#1.data};              \addlegendentry{HC}
  \addplot[color=black,mark=square] table[x=time,y=HC-CGC]{data3/#1.data};          \addlegendentry{HC-CGC}
  \addplot[color=red,mark=x] table[x=time,y=CGC]{data3/#1.data};               \addlegendentry{CGC}
  \addplot[color=gray,mark=o] table[x=time, y=ogm-KL]{data3/#1.data};          \addlegendentry{KL} 
  \addplot[color=yellow!50!black,mark=square] table[x=time, y=ogm-mcfusion-HC-CF*]{data3/#1.data};  \addlegendentry{Fusion}
  \addplot[color=purple,mark=o] table[x=time, y=MCR-CCFDB]{data3/#1.data};     \addlegendentry{MC-R}
  \addplot[color=blue,mark=o] table[x=time, y=MCI-CCIFD]{data3/#1.data};       \addlegendentry{MC-I}
  \addplot[color=green,mark=o] table[x=time, y=DYNCC-HC-MC]{data3/#1.data};    \addlegendentry{CCFusion-HC-MC}
  \addplot[color=cyan,mark=x] table[x=time, y=DYNCC-HC-CGC]{data3/#1.data};    \addlegendentry{CCFusion-HC-CGC}
  \addplot[color=orange,mark=o] table[x=time, y=DYNCC-WS-MC]{data3/#1.data};  \addlegendentry{CCFusion-WS-MC}
  \addplot[color=pink,mark=x] table[x=time, y=DYNCC-WS-CGC]{data3/#1.data};  \addlegendentry{CCFusion-WS-CGC}
}
%  \addplot[color=yellow!50!black,mark=o] table[x=time, y=ogm-mcfusion-HC-CF*]{data3/#1.data};  \addlegendentry{Fusion}
  %\addplot[color=green!50,mark=o] table[x=time, y=PIVOT*]{data3/#1.data};          \addlegendentry{PIVOT-BOEM}

% Pages are numbered in submission mode, and unnumbered in camera-ready
\ifcvprfinal\pagestyle{empty}\fi
\begin{document}
%%%%%%%%% TITLE
%!TEX root = egpaper_for_review.tex
\newcommand{\Cut }{\mathcal{C}}
%\title{Correlation Clustering with Dynamic Super-Nodes (DySNCC)}
\title{Fusion Moves for Correlation Clustering}

\author{First Author\\
Institution1\\
Institution1 address\\
{\tt\small firstauthor@i1.org}
% For a paper whose authors are all at the same institution,
% omit the following lines up until the closing ``}''.
% Additional authors and addresses can be added with ``\and'',
% just like the second author.
% To save space, use either the email address or home page, not both
\and
Second Author\\
Institution2\\
First line of institution2 address\\
{\tt\small secondauthor@i2.org}
}

\maketitle
%\thispagestyle{empty}


%%%%%%%%% ABSTRACT
\begin{abstract}
\end{abstract}
\section{Anytime Tables}

\section{Anytime Plots}
%!TEX root = egpaper_for_review.tex
%%%%%%%%%%%%%%%%%%%%%%%%%%%%%%%%%%%%%%%%%%%%%%%%%%%%%%%%%%%%%%%%%%%%%%%%%%%%%%%%%%%%%%%%%%%%
\pgfplotsset{every axis legend/.append style={
at={(1.0,1.2)},
anchor=north east}} 
\begin{figure*}[p]
  \begin{subfigure}[b]{0.33\textwidth}
  \centering
  \begin{tikzpicture}
  \begin{semilogxaxis}[  mark size=1pt,
  %restrict y to domain=0:4620,
  xlabel = {runtime},
  xmin = 0,
  xmax = 4100,
  width = 1.0\columnwidth,
  scaled ticks = false,
  every axis legend/.code={\let\addlegendentry\relax} 
  ]
  \addplot[color=yellow,mark=o] table[x=time, y=PIVOT*]{data3/image-seg.data};    \addlegendentry{PIVOT-BOEM}  
  \anytimeplot{image-seg}
  \end{semilogxaxis}
  \end{tikzpicture}
  
  \caption{image-seg}\label{fig:at:image-seg}
  \end{subfigure}
  %%%%%%%%%%%%%%%%%%%%%%%%%%%%%%%%%%%%%%%%%%%%%%%%%%%%%%%%%%%%%%%%%%%%%%%%% 
  \begin{subfigure}[b]{0.33\textwidth}
    \centering
    \begin{tikzpicture}
    \begin{semilogxaxis}[  mark size=1pt,
   % restrict y to domain=-6000:-4400,
    xlabel = {runtime},
    xmin = 0,
    xmax = 100,
    width = 1.0\columnwidth,
    scaled ticks = false,
    every axis legend/.code={\let\addlegendentry\relax} 
    ] 
    \addplot[color=yellow,mark=o] table[x=time, y=PIVOT*]{data3/knott-3d-150.data};          \addlegendentry{PIVOT-BOEM}
    \anytimeplot{knott-3d-150}
  
    \end{semilogxaxis}
    \end{tikzpicture}
    \caption{knott-3d-150}\label{fig:at:knott-150}
  \end{subfigure}
  %%%%%%%%%%%%%%%%%%%%%%%%%%%%%%%%%%%%%%%%%%%%%%%%%%%%%%%%%%%%%%%%%%%%%%%%%%%%%%%%%%%%%%%%%%%%%
  \begin{subfigure}[b]{0.33\textwidth}
  \centering
  \begin{tikzpicture}
  \begin{semilogxaxis}[  mark size=1pt,
  %restrict y to domain=-40000:-24000,
  xlabel = {runtime},
  xmin = 0,
  xmax = 4100,
  width = 1.0\columnwidth,
  scaled ticks = false,
  %legend to name = ledgendPosition,
  legend columns=6,
  every axis legend/.code={\let\addlegendentry\relax} 
  ]  
  \anytimeplot{knott-3d-300}
  \end{semilogxaxis}
  \end{tikzpicture}
  \caption{knott-3d-300}\label{fig:at:knott-300}
  \end{subfigure}
  \newline
  %%%%%%%%%%%%%%%%%%%%%%%%%%%%%%%%%%%%%%%%%%%%%%%%%%%%%%%%%%%%%%%%%%%%%%%%%%%%%%%%%%%%%%%%%%%%%
  \begin{subfigure}[b]{0.33\textwidth}
  \centering
  \begin{tikzpicture}
  \begin{semilogxaxis}[  mark size=1pt,
  %restrict y to domain=-80000:-60000,
  xlabel = {runtime},
  xmin = 0,
  xmax = 4100,
  width = 1.0\columnwidth,
  scaled ticks = false,
  every axis legend/.code={\let\addlegendentry\relax} 
  ] 
  \anytimeplot{knott-3d-450} 
  \end{semilogxaxis}
  \end{tikzpicture}
  \caption{knott-3d-450}\label{fig:at:knott-450}
  \end{subfigure}
  %%%%%%%%%%%%%%%%%%%%%%%%%%%%%%%%%%%%%%%%%%%%%%%%%%%%%%%%%%%%%%%%%%%%%%%%%%%%%%%%%%%%%%%%%%%%%
  \begin{subfigure}[b]{0.33\textwidth}
  \centering
  \begin{tikzpicture}
  \begin{semilogxaxis}[  mark size=1pt,
  %restrict y to domain=-10000000:-100000,
  xlabel = {runtime},
  xmin = 0,
  xmax = 4100,
  width = 1.0\columnwidth,
  scaled ticks = false,
  every axis legend/.code={\let\addlegendentry\relax} 
  ]
  \anytimeplot{knott-3d-550} 
  \end{semilogxaxis}
  \end{tikzpicture}
  \caption{knott-3d-550}\label{fig:at:knott-550}
  \end{subfigure}
  %%%%%%%%%%%%%%%%%%%%%%%%%%%%%%%%%%%%%%%%%%%%%%%%%%%%%%%%%%%%%%%%%%%%%%%%%%%%%%%%%%%%%%%%%%%%%\begin{subfigure}[H]
  \begin{subfigure}[b]{0.33\textwidth}
  \centering
  \begin{tikzpicture}
  \begin{semilogxaxis}[  mark size=1pt,
  %restrict y to domain=-10000000:-100000,
  %restrict y to domain=-10000000:1200000,
  xlabel = {runtime},
  xmin = 0,
  xmax = 4100,
  width = 1.0\columnwidth,
  scaled ticks = false,
  every axis legend/.code={\let\addlegendentry\relax} 
  ] 
  \anytimeplot{seg-3d} 
  \end{semilogxaxis}
  \end{tikzpicture}
  \caption{seg-3d}\label{fig:at:seg3d}
  \end{subfigure}
 %%%%%%%%%%%%%%%%%%%%%%%%%%%%%%%%%%%%%%%%%%%%%%%%%%%%%%%%%%%%%%%%%%%%%%%%%%%%%%%%%%%%%%%%%%%%%
  \begin{subfigure}[b]{0.33\textwidth}
  \centering
  \begin{tikzpicture}
  \begin{semilogxaxis}[  mark size=1pt,
  %restrict y to domain=-10000000:-100000,
  %restrict y to domain=-10000000:100000,
  xlabel = {runtime},
  xmin = 0,
  xmax = 4100,
  width = 1.0\columnwidth,
  scaled ticks = false,
  every axis legend/.code={\let\addlegendentry\relax} 
  ]  
  \anytimeplot{socialnets} 
  \end{semilogxaxis}
  \end{tikzpicture}
  \caption{social nets}\label{fig:at:socialnets}
  \end{subfigure}
 %%%%%%%%%%%%%%%%%%%%%%%%%%%%%%%%%%%%%%%%%%%%%%%%%%%%%%%%%%%%%%%%%%%%%%%%%%%%%%%%%%%%%%%%%%%%%
  \begin{subfigure}[b]{0.33\textwidth}
  \centering
  \begin{tikzpicture}
  \begin{semilogxaxis}[  mark size=1pt,
  %restrict y to domain=-10000000:-100000,
  %restrict y to domain=-10000000:100000,
  xlabel = {runtime},
  xmin = 0,
  xmax = 4100,
  width = 1.0\columnwidth,
  scaled ticks = false,
  every axis legend/.code={\let\addlegendentry\relax} 
  ] 
  \anytimeplot{normalizedsocialnets}
  \end{semilogxaxis}
  \end{tikzpicture}
  \caption{normalized social nets}\label{fig:at:nsocialnets}
  \end{subfigure}
 %%%%%%%%%%%%%%%%%%%%%%%%%%%%%%%%%%%%%%%%%%%%%%%%%%%%%%%%%%%%%%%%%%%%%%%%%%%%%%%%%%%%%%%%%%%%%
  \begin{subfigure}[b]{0.33\textwidth}
  \centering
  \begin{tikzpicture}
  \begin{semilogxaxis}[  mark size=1pt,
  %restrict y to domain=-10000000:-100000,
  %restrict y to domain=-10000000:100000,
  xlabel = {runtime},
  xmin = 0,
  xmax = 4100,
  width = 1.0\columnwidth,
  legend to name = ledgendPosition,
  legend columns=6,
  scaled ticks = false%,
  %every axis legend/.code={\let\addlegendentry\relax} 
  ]
  \addplot[color=yellow,mark=o] table[x=time, y=PIVOT*]{data3/modularity-clustering.data};    \addlegendentry{PIVOT-BOEM} 
  \anytimeplot{modularity-clustering}
  \end{semilogxaxis}
  \end{tikzpicture}
  \caption{modularity clustering}\label{fig:at:modularity}
  \end{subfigure}
 %%%%%%%%%%%%%%%%%%%%%%%%%%%%%%%%%%%%%%%%%%%%%%%%%%%%%%%%%%%%%%%%%%%%%%%%%%%%%%%%%%%%%%%%%%%%%

  \begin{center}
  \hypersetup{linkcolor = black}
  \ref{ledgendPosition}
  \hypersetup{linkcolor = red}
  \end{center}

\end{figure*}
%%%%%%%%%%%%%%%%%%%%%%%%%%%%%%%%%%%%%%%%%%%%%%%%%%%%%%%%%%%%%%%%%%%%%%%%%%%%%%%%%%%%%%%%%%%%%




\end{document}
